\documentclass[11pt,a4paper,titlepage,fleqn]{article}

\usepackage{amsmath}
\usepackage{amsfonts}
\usepackage{amssymb}
\usepackage{commath}
\usepackage{xcolor}
\usepackage{hyperref}
\usepackage[skins,theorems]{tcolorbox}
\usepackage{titlesec}
\usepackage{circuitikz}
\usepackage{pgfplots}
\usepackage{mathtools}
\usepackage[makeroom]{cancel}
\usepackage{mathrsfs}
\usepackage{wrapfig}
\usepackage{subcaption}
\usepackage{floatrow}
\usepackage{esint}
\usepackage{enumitem}
\usepackage{bm}
\usepackage{relsize}
\usepackage{xfrac}
\usepackage{comment}


\usetikzlibrary{arrows.meta}
\usetikzlibrary{patterns}
\usetikzlibrary{decorations.pathmorphing,patterns}
\usetikzlibrary{decorations.markings}
\usetikzlibrary{backgrounds}
\usetikzlibrary{shapes.misc}
\usetikzlibrary{shapes.multipart}

\tikzset{cross/.style={cross out, draw,
        minimum size=2*(#1-\pgflinewidth),
        inner sep=0pt, outer sep=0pt}}
\tikzset{
    mark position/.style args={#1(#2)}{
        postaction={
            decorate,
            decoration={
                markings,
                mark=at position #1 with \coordinate (#2);
            }
        }
    }
}

\usepackage[left=2cm,right=2cm,top=2cm,bottom=2cm]{geometry}

\usepackage[no-math]{fontspec}
\setmainfont{Times New Roman}
\setsansfont{Arial}
%\newfontfamily\greekfont[Script=Greek]{Linux Libertine O}
%\newfontfamily\greekfontsf[Script=Greek]{Linux Libertine O}
\usepackage{polyglossia}
\newfontfamily\greekfont[Script=Greek]{Times New Roman}
\newfontfamily\greekfontsf[Script=Greek]{Arial}
\newfontfamily\greekfonttt[Script=Greek]{Latin Modern Mono}
%\usepackage[greek]{babel}
\setdefaultlanguage{greek}
\setotherlanguage{english}
\newcommand{\textlatin}[1]{#1}
%\newcommand{\mathlarger}{}

%\usepackage[utf8]{inputenc}
%\usepackage[greek]{babel}


\usetikzlibrary{arrows.meta}
\usetikzlibrary{calc}
%\usepackage{tkz-euclide} % loads  TikZ and tkz-base
%\usetkzobj{angles} % important you want to use angles

\newlist{enumparen}{enumerate}{1}
\setlist[enumparen]{label=(\arabic*)}
\newlist{enumpar}{enumerate}{1}
\setlist[enumpar]{label=\arabic*)}

\newlist{enumgreek}{enumerate}{1}
\setlist[enumgreek]{label=\alph*.}
\newlist{enumgreekparen}{enumerate}{1}
\setlist[enumgreekparen]{label=(\alph*)}
\newlist{enumgreekpar}{enumerate}{1}
\setlist[enumgreekpar]{label=\alph*)}


\newlist{enumroman}{enumerate}{1}
\setlist[enumroman]{label=(\roman*)}

\newlist{enumlatin}{enumerate}{1}
\setlist[enumlatin]{label=(\alph*)}

\newlist{invitemize}{itemize}{1}
\setlist[invitemize]{noitemsep,label=}



\makeatletter
\let\anw@true\anw@false

%\newcommand{\attnboxed}[1]{\textcolor{red}{\fbox{\normalcolor\m@th$\displaystyle#1$}}}
\makeatother
\tcbset{highlight math style={enhanced,colframe=red,colback=white,%
        arc=0pt,boxrule=1pt,shrink tight,boxsep=1.5mm,extrude by=0.5mm}}
\newcommand{\attnboxed}[1]{\tcbhighmath[colback=red!5!white,drop fuzzy shadow,arc=0mm]{#1}}
\titleformat{\section}{\bf\Large}{Κεφάλαιο \thesection}{1em}{}
\newtcolorbox{attnbox}[1]{colback=red!5!white,%
    colframe=red!75!black,fonttitle=\bfseries,title=#1}
\newtcolorbox{infobox}[1]{colback=blue!5!white,%
    colframe=blue!75!black,fonttitle=\bfseries,title=#1}

\renewcommand{\arg}{\mathrm{Arg}\, }
\renewcommand{\Re}{\mathrm{Re}}
\renewcommand{\Im}{\mathrm{Im}}
\newcommand{\sinc}{\;\mathrm{sinc}\!}

\newif\ifhidetikz
\hidetikzfalse
%\hidetikztrue   % <---- comment/uncomment that line

\ifhidetikz

\let\oldtikzpicture\tikzpicture
\let\oldendtikzpicture\endtikzpicture

\renewenvironment{tikzpicture}{
    \tiny
    \tt
    \color{blue}
    \newcommand{\draw}{\textit{draw}}
    \newcommand{\filldraw}{\textit{filldraw}}
    %\newcommand{\x}{\textit{x}}
    %\newcommand{\p}{\textit{x}}
    \newcommand{\x1}{\textit{x1}}
    \newcommand{\y1}{\textit{y1}}
    \newcommand{\p1}{\textit{p1}}
}{
}
\newenvironment{axis}{
    \newcommand{\addplot}{\textit{addplot}}
}{
}
\fi

\newtcbtheorem[number within=section]{theorem}{Θ.}%
{colback=green!5,colframe=green!35!black,colbacktitle=green!35!black,fonttitle=\bfseries,enhanced,attach boxed title to top left={yshift=-8mm,xshift=-7mm},width=.9\textwidth,arc=.7mm}{th}
\newtcbtheorem[number within=section]{defn}{Ορισμός}%
{colback=blue!5,colframe=cyan!35!black,colbacktitle=blue!35!black,fonttitle=\bfseries,enhanced,attach boxed title to top left={yshift=-2mm,xshift=-2mm}}{def}
\newtcbtheorem[number within=section]{exercise}{Άσκηση}%
{colback=gray!3,colframe=gray!35!black,colbacktitle=gray!35!black,fonttitle=\bfseries,enhanced,attach boxed title to top left={yshift=-2mm,xshift=-2mm}}{exc}




\title{Θεωρία Σημάτων και Γραμμικών Συστημάτων - Σημειώσεις}
\date{2016}
\author{\textlatin{\csuse{no\greek @numbers}\selectlanguage{english} \url{https://github.com/kongr45gpen/ece-notes}}}



\begin{document}
	Την Τρίτη μάθημα 8:30 χωρίς διάλειμμα

	Σήμα - σύστημα
	\begin{align*}
	\boxed{\quad \underbrace{g}_{\mathclap{\text{εξαρτημένη}}} = f(\underbrace{t}_{\mathclap{\text{ανεξάρτητη}}}) \quad} \qquad g=f(\vec r, t) \qquad \vec E (\vec r, t)
	\end{align*}

	\paragraph{Αναλογικό}
	\begin{invitemize}
		\item Αν \( t \) συνεχής \( \in \mathbb R  \)
		\item και \( y \) συνεχής \( \in \mathbb R \)
	\end{invitemize}
	\begin{tikzpicture}
		\draw[->] (0,-1) -- (0,2);
		\draw[->] (-2,0) -- (2,0);

		\draw [blue, very thick]
		plot [smooth, tension=1, domain=-2:2, samples=9] (\x,{1+rand/2}) node[below] {$g(x,y)$};
	\end{tikzpicture}

	\paragraph{Διακριτού χρόνου / Διακριτό (discrete)}
	\begin{invitemize}
		\item \( t \) διακριτό \( \rightarrow \mathbb Z,\ n \in \mathbb Z \)
		\item \( g \) συνεχής \( \in \mathbb R  \)
	\end{invitemize}

	\begin{tikzpicture}[scale=0.7]
		\draw[->,gray] (0,0) -- (0,4.5);
		\draw[->,gray] (-6,0) -- (6,0);

		\foreach \x in {-5,...,5} {
			\draw (\x,0) node[below] {$\x$};
			\filldraw[black] (\x,0) -- (\x,{(1+rand)*2}) circle (1.5pt);
		};

		\draw (0,-1) node[below] {7\textsuperscript{ο} εξάμηνο};
	\end{tikzpicture}

	\paragraph{Κβαντισμένο}
	\begin{invitemize}
		\item \( n \in \mathbb Z \)
		\item \( g \) διακριτή
	\end{invitemize}
	\begin{tikzpicture}
	\draw[->,gray] (0,0) -- (0,4.5);
	\draw[->,gray] (-4,0) -- (4,0);

	\draw[gray,dashed] (-4,1) -- (4,1);
	\draw[gray,dashed] (-4,2) -- (4,2);
	\draw[gray,dashed] (-4,3) -- (4,3);

	\filldraw[black] (-3,0) node[below] {$-3$} -- (-3,1) circle (1.5pt);
	\filldraw[black] (-2,0) node[below] {$-2$} -- (-2,2) circle (1.5pt);
	\filldraw[black] (-1,0) node[below] {$-1$} -- (-1,1) circle (1.5pt);
	\filldraw[black] ( 0,0) node[below] {$0$} -- (0,0) circle (1.5pt);
	\filldraw[black] ( 1,0) node[below] {$1$} -- (1,1.5) circle (1.5pt);
	\filldraw[black] ( 2,0) node[below] {$2$} -- (2,3) circle (1.5pt);
	\filldraw[black] ( 3,0) node[below] {$3$} -- (3,2) circle (1.5pt);
	\draw (1,1) node[cross=10pt,red!50!black] {};
	\end{tikzpicture}

	\paragraph{Στοχαστικό}
	Περιέχει και τις τρεις κατηγορίες

	\subsection{Σύστημα}
	\begin{tikzpicture}[scale=0.8]
		\draw (-1,1) rectangle (3,-1) node[midway] {system};

		\draw[->] (-5,0) -- (-1,0) node[midway,above] {$x(t)$} node[midway,below] {input signal};
		\draw[->] (3,0) -- (7,0) node[midway,above] {$y(t)$} node[midway,below] {output signal};
	\end{tikzpicture}

	\subsection{Περιοδικά σήματα}
	Αν \( \exists T \in \mathbb R : \forall t \in \mathbb R \quad x(t)=x(t+T) \)
	τότε \( x(t) \) \textbf{περιοδικό σήμα} με περίοδο \( T \).

	Ή θα είναι 0, ή θα συνεχιστεί για πάντα.

	\[
	\int_{-\sfrac{T}{2}}^{\sfrac{T}{2}} x(t) \dif t =
	\int_{t_0-\sfrac{T}{2}}^{t_0+\sfrac{T}{2}} x(t) \dif t\ \forall t
	\]

	Η σύνθεση μιας συνάρτησης με μια περιοδική συνάρτηση είναι περιοδική;
	\subparagraph{Απόδ.}
		Έστω \(g\) μία περιοδική συνάρτηση:
		\begin{align*}
		\big(f\circ g\big)(x) &= f\left(g(x)\right) = f \left( g(x+T) \right) =
		\\ &= \big(f\circ g\big)(x+T)
		\end{align*}

	\subsection{Συμμετρίες}
	\begin{itemize}
	    \item Αν \( x(t) = x(-t) \ \forall t \) τότε η \( x(t) \) λέγεται \textbf{άρτια συνάρτηση} (even function).

	    \item Αν \( x(t) = -x(t) \ \forall t \) τότε η \( x(t) \) λέγεται \textbf{περιττή συνάρτηση} (odd function).
	\end{itemize}

	\paragraph{}
	\( \forall x(t) \quad \exists\ x_0(t), x_e(t): x(t) = x_e(t)+x_0(t) \)
	\subparagraph{Απόδ.}
	\begin{align*}
		x_e(t) &= \frac{x(t)+x(-t)}{2} \\
		x_o(t) &= \frac{x(t)-x(-t)}{2}
	\end{align*}

	\paragraph{}
	\begin{align*}
		x{\underbrace{_e}_{\mathclap{\text{άρτια}}}}y_e &= z_e\\
		x_oy_o &=z_e \\
		x_ey_0 &=z_0 \\
		\int_{-A}^A x_0(t)\dif t &= 0 \\
		\int_{-\infty}^\infty x_0(t)\dif t &= ? \text{ (εξαρτάται)} \\
		\lim_{A\to \infty} \int_{-A}^A x_0(t)\dif t &= 0 \quad \text{(principal Cauchy value)}
	\end{align*}

	\subparagraph{Χαρακτηριστικά σήματα}
	\begin{enumpar}
		\item \textbf{Εκθετικό σήμα}
		\[
		x(t) = ce^{at}\quad a \in \mathbb R \quad c > 0
		\]
		\begin{tikzpicture}
		\draw[->,gray] (0,0) -- (0,4);
		\draw[->,gray] (-2,0) -- (4,0) node[below right] {$t$};

		\draw (1,3) node {$c > 0$};

		\draw[xscale=3,domain=-0.7:1.3,smooth,variable=\x,blue] plot ({\x},{exp(\x)});
		\draw[xscale=3,domain=1.3:-1.2,dashed,smooth,variable=\x,blue] plot ({\x},{exp(-\x)});
		\draw (4,1) node[anchor=north east,blue] {$a < 0$};
		\draw (4.5,3) node[anchor=north east,blue] {$a > 0$};
		\end{tikzpicture}
		\[
		x(t) = ce^{(\sigma t+j\omega)t} = ce^{\sigma t}e^{j\omega t}
		= ce^{\sigma t} \left[
		\cos(\omega t)+j\sin(\omega t)
		\right]
		\]

		\item \textbf{(Συν)ημιτονοειδή σήματα}
		\[
		x(t) = A \cos(\omega t\pm \phi)
		= a \mathrm{Re} \left\lbrace e^{j(\omega t +\phi)} \right\rbrace
		= A \frac{e^{j(\omega t \pm \phi)} + e^{-j(\omega t\pm \phi)}}{2}
		\]
		\begin{tikzpicture}[scale=1.3]
		\draw[->,gray] (0,-2) -- (0,2);
		\draw[->,gray] (-2,0) -- (4,0) node[below right] {$t$};


		\draw[scale=1,domain=0:4,samples=200,smooth,variable=\x,blue,thick] plot ({\x},{sin((\x r)*20)*exp(-\x)});
		\end{tikzpicture}

		\item \textbf{Δέλτα Dirac} \( \bm{ \delta(t)} \)

		\begin{tikzpicture}[scale=1.3]
		\draw[->,gray] (0,-2) -- (0,2) node[black,below right] {$\delta(t)$};
		\draw[->,gray] (-1,0) -- (1,0) node[below right] {$t$};
		\draw (0,0) node[below left] {$0$};

		\draw[very thick,blue,->] (0,0) -- (0,1);
		\end{tikzpicture}
		\begin{tikzpicture}[scale=1.3]
		\filldraw[fill=green!20] (-0.5,0) rectangle (0.5,1) node[above right] {$\sfrac{1}{T}$}
			node[midway,right] {$1$};

		\draw[->,gray] (0,-2) -- (0,2);
		\draw[->,gray] (-1.5,0) -- (2,0);

		\draw (current bounding box.east) node[above] {$\lim_{T\to0} P_T(t)=\delta(t)$};
		\end{tikzpicture}
		\begin{tikzpicture}[scale=1.3]
		\filldraw[scale=1,domain=-2:2,samples=200,smooth,variable=\x,fill=green!20]
			plot ({\x},{exp(-\x*\x)})
			node[above right] {$G_\sigma(t) = \frac{1}{\sqrt{2\pi}\sigma}e^{-\frac{x^2}{\sigma^2}} $}
			node[midway,above left] {$1$};

		\draw[->,gray] (0,-2) -- (0,2);
		\draw[->,gray] (-2,0) -- (2,0);

		\draw[<->,gray] (0.7,{exp(-pow(0.7,2))}) -- (-0.7,{exp(-pow(0.7,2))})
			node[above left] {$\sigma$};

		\draw (current bounding box.east) node[below left] {$\lim_{T\to0} G_\sigma(t)=\delta(t)$};
		\end{tikzpicture}
	    \paragraph{Ορ.}
	    \[
	    \int_{-\infty}^\infty f(t)\mathrm \delta(t) \dif t = f(0)\ \forall f(t)
	    \]
	    \paragraph{}
	    \begin{align*}
	    \int_{-\infty}^\infty \delta(t)\dif t &=1 \\
	    \int_{-\infty}^\infty f(t)\delta(t-\tau) \dif t &= f(\tau) \\
	    \Aboxed{\int_{-\infty}^\infty f(\tau) \delta(t-\tau)\dif \tau &= f(t)} \\
	    \int_{-\infty}^\infty f(t)\delta(t-\tau)\dif \tau &= f(t)
	    \end{align*}
	    \paragraph{Ιδιότητες της \( \bm{ \delta(t)} \)}
	    \begin{enumerate}
	    	\item \textbf{Κλιμάκωση}
	    	\[
	    	a \in \mathbb R: \delta(at) = \frac{1}{|a|}\delta(t)
	    	\]
	    	\subparagraph{Απόδ.}
	    	\[
	    	\underbrace{\int_{-\infty}^\infty \phi(t)\boxed{\delta(at)}\dif t}_{\mathclap{\begin{matrix}
	    			at=\xi\\\dif t = \frac{\dif \xi}{a}
	    			\end{matrix}}}
	    =
	    	\int_{-\infty_{(a)}}^{\infty_{(a)}} \phi\left( \frac{\xi}{a} \right) \delta(\xi)\frac{\dif \xi}{a}
	    	= \frac{1}{|a|}\int_{-\infty}^\infty \frac{\phi\left(\frac{\xi}{a} \right)}{|a|}\delta(\xi)\dif\xi = \frac{\phi(0)}{|a|}
	    	= \int_{-\infty}^\infty \phi(t)\boxed{\frac{\delta(t)}{|a|}}\dif t
	    	\]
	    	\item \(
	    	f(t)\delta(t)=f(0)\delta(t)
	    	\)
	    	\item \(
	    	f(t)\delta(t-\xi)=f(g)\delta(t-\xi)
	    	\)
    	\end{enumerate}
	\end{enumpar}

	\paragraph{}
	\begin{tikzpicture}[scale=0.8]
	\draw (-1,1) rectangle (3,-1) node[midway] {system};

	\draw[->] (-5,0) -- (-1,0) node[midway,above] {$x(t)$} node[midway,below] {input signal};
	\draw[->] (3,0) -- (7,0) node[midway,above] {$y(t)$} node[midway,below] {output signal};
	\end{tikzpicture}
	\begin{gather*}
		y(t) = \mathscr{L}\left\lbrace x(t) \right\rbrace \\
		\forall x_1(t)\ x_n(t)\\
		y_1(t) = \mathscr{L}\left\lbrace x_1(t) \right\rbrace \\
		y_2(t) = \mathscr{L}\left\lbrace x_2(t) \right\rbrace
	\end{gather*}

	Για const \( a_1, a_2 \in \mathbb R  \)
	\begin{align*}
	x(t) &= a_1x_1(t)+a_2x_2(t) \\
	y(t) &= \mathscr{L}\left\lbrace x(t) \right\rbrace \\
	\intertext{ανν}
	y(t) = a_1y_1(t)+a_2y_2(t) \\
	\intertext{τότε}
	\mathscr{L}: \text{ γραμμικό σύστημα}
	\end{align*}

	\begin{itemize}
		\item \(g(t) = \mathscr{L}\left\lbrace x(t) \right\rbrace\)

		\( x'(t)=x(t-\tau) \)

		ανν \( y'(t) = \mathscr{L} \left\lbrace x'(t) \right\rbrace
		= \mathscr{L} \left\lbrace x(t-\tau)^2 \right\rbrace = y(t-\tau)
		 \)

		 τότε το σύστημα \( \mathscr L \) είναι αμετάβλητο κατά τη μετατόπιση.

	\end{itemize}

	\paragraph{}

    \begin{tikzpicture}[scale=0.8]
    \draw (-1,-1) rectangle (3,1) node[above right] {$\mathscr L\left\lbrace\right\rbrace$} node[midway] {γραμμικό \& ΑΚΜ};


    \draw[->] (-5,0) -- (-1,0) node[midway,above] {$\delta(t)$} node[midway,below] {input signal};
    \draw[->] (3,0) -- (14,0) node[midway,above] {$\mathrm h(t)$}
	    node[midway,below] {κρουστική απόκριση (impulse response)};
    \end{tikzpicture}

    Υποστηρίζω ότι ένα γραμμικό \& ΑΚΜ σύστημα περιγράφεται πλήρως από την κρουστική
    απόκριση \( h(t) \).
    \subparagraph{Απόδ.} Από παραπάνω, γνωρίζουμε ότι
    \(
    x(t) = \int_{-\infty}^\infty x(t)\delta(t-\tau)\dif t
    \)
    \begin{align*}
    y(t) &= \mathscr L \left\lbrace y(t) \right\rbrace =
    \mathscr L \left\lbrace \int_{-\infty}^{\infty} x(t)\delta(t-\tau) \dif \tau
     \right\rbrace
     \\ & \overset{\text{linearity}}{=} \int_{-\infty}^{\infty} \mathscr L
     \left\lbrace x(\tau)\delta(t-\tau) \right\rbrace
     \\ &= \int_{-\infty}^{\infty} x(\tau) \mathscr L \left\lbrace
     \delta(t-\tau)
      \right\rbrace \dif \tau
      \\ & \underset{TSI}{\overset{\text{ΑΚΜ}}{=}}
       \int_{-\infty}^{\infty} x(\tau) h(t-\tau)\dif \tau
       \\ y(t) &= \int_{-\infty}^{\infty}
        x(\tau)
        \underbrace{h(t-\tau)}_{\mathclap{\text{linear time-shift invariant}}}
        \dif\tau
    \end{align*}

    \paragraph{}

    \begin{itemize}
    	\item \( \delta(t)=\delta(-t) \) άρτια συνάρτηση
    	\item \( \delta^{(n)}(t) = \od[n]{}{t}\delta(t) \), για την οποία
    αποδεικνύεται ότι:
        \[
            \int_{-\infty}^{\infty} \delta^{(n)}(t)\phi(t)\dif t
            = (-1)^n\left.\phi^{(n)}(t)\right|_{t=0}
        \]
    \end{itemize}

    \subsubsection{Βηματική Συνάρτηση (Unit Step Function)}
    \[
    \mathrm u(t) = \begin{cases}
    1 \quad & t > 0 \\
    0 \quad & t < 0
    \end{cases}
    \]
    \[
    \int_{-\infty}^{\infty} \mathrm u(t)\phi(t)\dif t =
    \mathscr N_{\mathrm u}\left\lbrace \phi(t) \right\rbrace =
    \int_0^\infty \underbrace{\phi(t)}_{\mathclap{\text{number}}}\dif t
    \]
    \begin{tikzpicture}
    \begin{axis}[%
    xlabel=$t$
    ,ylabel=$\mathrm u(t)$
    ,axis lines = center
    ,ymax=1.5
    ,ytick={0,1}
    ,xtick={0,1}
    ]
    \addplot+[const plot, no marks,ultra thick] coordinates {(-4,0) (0,1) (4,1)};
    %\addplot+[const plot, no marks, thick] coordinates {(0,0) (1,0.25) (2,0.4) (3,0.5) (4,1) (4.49,1)} node[below=1.15cm,pos=.76,black] {$F_y$};
    \end{axis}
    \end{tikzpicture}

    \begin{align*}
    \delta(t) &= \od{}{t} \mathrm u(t)\\
    \mathrm u(t) &= \int_{-\infty}^t \delta(\tau)\dif\tau =
    \int_0^\infty \delta(t-\xi)\dif\xi
    \end{align*}

    \subsubsection{Ράμπα}
    \[
    \mathrm r(t) = \int_{-\infty}^t \mathrm u(\tau)\dif\tau =
    \begin{cases}
    t \quad & t \geq0 \\
    0 \quad & \text{else}
    \end{cases} = t\mathrm u(t)
    \]

    \begin{tikzpicture}
    	\draw[->] (0,-0.5) -- (0,4) node[right] {$\mathrm r(t)$};
    	\draw[->] (-2,0) -- (4,0) node[below] {$t$};

    	\draw (0,0) node[below right] {$0$};

    	\draw[very thick,blue] (-2,0) -- (0,0) -- (3.5,3.5);
    \end{tikzpicture}

    \[
    \mathrm u(t) =\od{}{t} \mathrm r(t)
    \]

    \subsubsection{Ορθογωνικός παλμός (Rectangular Pulse function)}
    \[
    \mathrm p_a(t) = \begin{cases}
    1 &\quad |t| < a \\
    0 &\quad |t| > a
    \end{cases}
    \]

    \begin{tikzpicture}
    \draw[->] (0,-0.5) -- (0,3) node[right] {$\mathrm r(t)$};
    \draw[->] (-2,0) -- (2,0) node[below] {$t$};

    \draw[very thick,blue]
	    (-1,0) node[below,black] {$-a$} --
	    (-1,1.5) node[above left, black] {$1$} --
	    (1,1.5) --
	    (1,0) node[below,black] {$a$};
    \end{tikzpicture}

    \subparagraph{}

    \begin{tikzpicture}
    \draw[->] (0,-0.5) -- (0,2);
    \draw[->] (-2,0) -- (3,0) node[below] {$t$};

    \draw[very thick,blue]
    (-3,0) --
    (-1,0) node[below,black] {$-a$} --
    (-1,1) --
    (3,1) node[above right] {$\mathrm u(t+a)$};
    \end{tikzpicture}
    \begin{tikzpicture}
    \draw[->] (0,-0.5) -- (0,2);
    \draw[->] (-2,0) -- (3,0) node[below] {$t$};

    \draw[very thick,blue]
    (-3,0) --
    (1,0) node[below,black] {$a$} --
    (1,1) --
    (3,1) node[above right] {$\mathrm u(t-a)$};
    \end{tikzpicture}

    \begin{align*}
    \mathrm p_a(t) =& \mathrm u(t+a) - \mathrm u(t-a) \\
    \od{}{t} \mathrm p_a(t) =& \delta(t+a)-\delta(t-a)
    \end{align*}

    \begin{tikzpicture}
    \draw[->] (0,-0.5) -- (0,3) node[right] {$\od{}{t}\mathrm p_a(t)$};
    \draw[->] (-2,0) -- (2,0) node[below] {$t$};

    \draw[very thick,blue,->]
    (-1,0) node[below,black] {$-a$} --
    (-1,1.5) node[above] {$\delta(t+a)$};

    \draw[very thick,blue,->]
    (1,0) node[above,black] {$a$} --
    (1,-1.5) node[below] {$-\delta(t-a)$};
    \end{tikzpicture}

    \subsubsection{Τριγωνικός Παλμός (Triangular Pulse function)}
    \[
    \mathrm p_{\mathrm{tr},a} = \begin{cases}
    1-\frac{|t|}{a} &\quad |t|<a \\
    0 &\quad |t|>a
    \end{cases}
    \]
    \begin{tikzpicture}[scale=1.2]
    \draw[->] (0,-0.5) -- (0,2) node[right] {$\mathrm p_{\mathrm{tr},a}(t)$};
    \draw[->] (-2,0) -- (2,0) node[below] {$t$};

    \draw[very thick,blue]
    (-2,0) --
    (-1,0) node[below,black] {$-a$} --
    (0,1.5) node[above left, black] {$1$} --
    (1,0) node[below,black] {$a$} --
    (2,0);
    \end{tikzpicture}
    \[
    p_{\mathrm{tr},a}(t) = \frac{1}{a} \left[
        \mathrm r(t+a) + \mathrm r(t-a) - 2\mathrm r(t)
    \right]
    \]
    %TODO Rekanos Graph 16

    \subsection{Χαρακτηριστικά Μεγέθη}
    \begin{enumpar}
    	\item \textbf{Μέση τιμή (Mean Value)}
    	\[
    	\overline{x(t)} = \frac{1}{t_2-t_1} \int_{t_1}^{t_2}x(t)\dif t
    	\]

    	Αν περιοδική τότε \begin{align*}
    	\bar x(t) &= \frac{1}{T} = \int_0^T x(t)\dif t \\
    	&= \frac{1}{T} \int_{t_0}^{t_0+T} x(t)\dif t
    	\end{align*}
    	\item \textbf{Ενεργός τιμή (Root Mean Square Value)}
    	\[
    	\overline{\overline{x(t)}} = \left[
    	    \frac{1}{t_2-t_1}\int_{t_1}^{t_2} x^2(t)\dif t
    	\right]^{\sfrac{1}{2}}
    	\]
    	Αν ημιτονοειδές σήμα \(\bar{\bar{x}}(t) = \frac{x_{\max}}{\sqrt{2}}
    	\)

    	\item \textbf{Ενέργεια - Ισχύς}
    	\begin{itemize}
    		\item Στιγμιαία ισχύς (Instant power)
    		\[
    		p(t) = x^2(t)
    		\]
    		\item Μέση ισχύς (Mean power)
    		\[
    		\overline{p(t)} = \frac{1}{t_2-t_1}\int_{t_1}^{t_2}x^2(t)\dif t
    		= \left( \overline{\overline{x(t)}} \right)^2
    		\]
    		\item Ενέργεια (Energy)
    		\[
    		W = \int_{t_1}^{t_2} p(t) \dif t =
    		\int_{t_1}^{t_2} x^2(t)\dif t = (t_2-t_1)\left(
    		\overline{\overline{x(t)}}
    		\right)^2
    		\]
    	\end{itemize}

    	\textbf{Σήματα} \( \begin{dcases}
    	\text{\textbf{Σήμα ενέργειας} αν} \lim_{T\to \infty}W< \infty
    	\\ \\
    	\text{\textbf{Σήμα ισχύος} αν} \lim_{T\to \infty}\overline{p(t)}>0
	    \\ \text{Υπάρχουν και σήματα που δεν είναι ούτε ενέργειας, ούτε ισχύος.}
    	\end{dcases}
    	\)
    \end{enumpar}
    
    \subsection{Συνέλιξη}
    \[ x(t) = \int_{-\infty}^{\infty}x(t)\delta(t-\tau)d\tau \]
    %TODO Rekanos Graph 09 (again)
    \[ h(t)=\mathscr L\left\lbrace \delta(t) \right\rbrace \]
    \[ y(t) = \int_{-\infty}^{\infty} x(\tau)h(t-\tau)\dif\tau 
    = \underbrace{x(t)}_{\mathclap{\text{είσοδος}}} 
    \underbrace{*}_{\text{συνέλιξη}}
    \underbrace{h(t) }_{\mathclap{\text{κρουστική απόκριση}}}\]
    
    \paragraph{Συνέλιξη - Convolution}
    \[
    z(t)=x(t)*y(t)=\int_{-\infty}^{\infty} x(\tau)y(t-\tau)\dif\tau
    \]
    
    \paragraph{}
    \begin{itemize}
    	\item \(x(t)*y(t)=y(t)*x(t)\) \textbf{Αντιμεταθετική}
    	\subparagraph{}
    	\[
    	\int_{-\infty}^{\infty} x(\tau)y(t-\tau)\dif\tau
    	= \int_{-\infty}^{\infty} x(t-\lambda)y(\lambda)[-\dif\lambda]
    	= \int_{-\infty}^{\infty} y(\lambda)x(t-\lambda)\dif\lambda
    	= y(t)*x(t)
    	\]
    	\item \( x_1(t)*\left[x_2(t)*x_3(t)\right] = 
    	\left[x_1(t)*x_2(t)\right]*x_3(t) \) \textbf{Προσεταιριστική}
    \end{itemize}
    
    \paragraph{Παρ.}
    \begin{align*}
    f_1(t) &= 2(1-t)\left[ \mathrm u(t)-u(t-1) \right] \\
    f_2(t) &= \mathrm u(t) - \mathrm u(t-\tau)
    \end{align*}
    
    \subparagraph{Γραφική μέθοδος υπολογισμού συνέλιξης}
    %TODO Rekanos Graph 17
    %TODO Rekanos Graph 18
    %TODO Rekanos Graph 19
    %TODO Rekanos Graph 20
    %TODO Rekanos Graph 21
    \subparagraph{}
    Παρατηρώ ότι: \[
    \int_{-\infty}^{\infty} f(t,\tau) \mathrm u(t-\xi)\mathrm u(\phi-\tau)\dif\tau
    = \int_{\xi}^\phi f(t,\tau)\dif\tau\mathrm u(\phi-\xi)
    \]
    
    \begin{align*}
    f(t) &= \int_{-\infty}^{\infty}
    \underbrace{2(1-\tau)}_{\mathclap{x(\tau)}}\left[\mathrm u(\tau)-\mathrm u(\tau+1) \right]\left[
    \mathrm u(t-\tau)-\mathrm u(t-\tau-2)
    \right] \dif\tau \\
    &= \int_{-\infty}^{\infty} x(\tau) \left[
    \mathrm u(\tau)\mathrm u(t-\tau)-\mathrm u(\tau-1)\mathrm u(t-\tau)
    -\mathrm u(\tau)\mathrm u(t-\tau-2)+\mathrm u(\tau-1)\mathrm u(t-\tau-2)
    \right]\dif\tau \\
    &= \int_0^t x(\tau)\dif\tau\mathrm u(t)
    -\int_1^x x(\tau)\dif\tau \mathrm u(t-1)
    -\int_0^{t-2}x(\tau)\dif\tau \mathrm u(t-2)
    +\int_1^{t-2}x(\tau)\dif\tau\mathrm u(t-3)
    \\ &= (2t-t^2)\mathrm u(t) - \left[2t-t^2-1\right]\mathrm u(t-1)
    - \left[ 2(t-2)-(t-2)^2 \right] \mathrm u(t-2)+
    \left[
    2(t-2)-(t-2)^2-1
    \right] \mathrm u(t-3)
    \end{align*}
    
    \paragraph{Ex}
    \begin{align*}
    f_1(t) &= e^t\mathrm u(-t) \\
    f_2(t) &= \mathrm u(t+2)-u(t+1) \\
    f &= f_1*f_2
    \end{align*}
    \begin{align*}
    f &= \int_{-\infty}^{\infty}  e^\tau \mathrm u(-\tau)
    \mathrm u\left( -(t-\tau)+2 \right)\dif\tau
    \\&= \int_{-\infty}^{\infty} e^\tau \mathrm u(-\tau)
    \mathrm u(\tau-t+2)\dif\tau
    \\&= \int_{t-2}^0 e^\tau\dif\tau\mathrm u(t-2)
    \\&= \left. e^\tau \right|_{t-2}^0\mathrm u(2-t)
    \\&=\left[ 1-e^{t-2} \right]\mathrm u(2-t)
    \end{align*}
    




\end{document}
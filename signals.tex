\documentclass[11pt,a4paper,titlepage,fleqn]{article}
%\usepackage[utf8]{inputenc}
%\usepackage[greek]{babel}
\usepackage{amsmath}
\usepackage{amsfonts}
\usepackage{amssymb}
\usepackage{commath}
\usepackage{xcolor}
\usepackage{hyperref}
\usepackage[skins,theorems]{tcolorbox}
\usepackage{titlesec}
\usepackage{circuitikz}
\usepackage{pgfplots}
\usepackage{mathtools}
\usepackage[makeroom]{cancel}
\usepackage{mathrsfs}
\usepackage{wrapfig}
\usepackage{subcaption}
\usepackage{floatrow}
\usepackage{esint}
\usepackage{enumitem}
\usepackage{xfrac}
\usepackage{bm}

\usetikzlibrary{arrows.meta}
\usetikzlibrary{patterns}
\usetikzlibrary{decorations.pathmorphing,patterns}
\usetikzlibrary{decorations.markings}
\usetikzlibrary{backgrounds}
\usetikzlibrary{shapes.misc}
\tikzset{cross/.style={cross out, draw, minimum size=2*(#1-\pgflinewidth), inner sep=0pt, outer sep=0pt},
	%default radius will be 1pt. 
	cross/.default={1pt}}

\usepackage[left=2cm,right=2cm,top=2cm,bottom=2cm]{geometry}

\usepackage[no-math]{fontspec}
\setmainfont{Times New Roman}
\setsansfont{Arial}
%\newfontfamily\greekfont[Script=Greek]{Linux Libertine O}
%\newfontfamily\greekfontsf[Script=Greek]{Linux Libertine O}
\usepackage{polyglossia}
\newfontfamily\greekfont[Script=Greek]{Times New Roman}
\newfontfamily\greekfontsf[Script=Greek]{Arial}
\newfontfamily\greekfonttt[Script=Greek]{Latin Modern Mono}
%\usepackage[greek]{babel}
\setdefaultlanguage{greek}
\setotherlanguage{english}
\newcommand{\textlatin}[1]{#1}

%\usepackage[utf8]{inputenc}
%\usepackage[greek]{babel}


\usetikzlibrary{arrows.meta}
\usetikzlibrary{calc}
%\usepackage{tkz-euclide} % loads  TikZ and tkz-base
%\usetkzobj{angles} % important you want to use angles

\newlist{enumparen}{enumerate}{1}
\setlist[enumparen]{label=(\arabic*)}
\newlist{enumpar}{enumerate}{1}
\setlist[enumpar]{label=\arabic*)}

\newlist{enumgreek}{enumerate}{1}
\setlist[enumgreek]{label=\alph*.}
\newlist{enumgreekparen}{enumerate}{1}
\setlist[enumgreekparen]{label=(\alph*)}
\newlist{enumgreekpar}{enumerate}{1}
\setlist[enumgreekpar]{label=\alph*)}


\newlist{enumroman}{enumerate}{1}
\setlist[enumroman]{label=(\@alph*)}

\newlist{enumlatin}{enumerate}{1}
\setlist[enumlatin]{label=(\alph*)}

\newlist{invitemize}{itemize}{1}
\setlist[invitemize]{noitemsep,label=}



\makeatletter
\let\anw@true\anw@false

%\newcommand{\attnboxed}[1]{\textcolor{red}{\fbox{\normalcolor\m@th$\displaystyle#1$}}}
\makeatother
\tcbset{highlight math style={enhanced,colframe=red,colback=white,%
		arc=0pt,boxrule=1pt,shrink tight,boxsep=1.5mm,extrude by=0.5mm}}
\newcommand{\attnboxed}[1]{\tcbhighmath[colback=red!5!white,drop fuzzy shadow,arc=0mm]{#1}}
\titleformat{\section}{\bf\Large}{Κεφάλαιο \thesection}{1em}{}
\newtcolorbox{attnbox}[1]{colback=red!5!white,%
	colframe=red!75!black,fonttitle=\bfseries,title=#1}
\newtcolorbox{infobox}[1]{colback=blue!5!white,%
	colframe=blue!75!black,fonttitle=\bfseries,title=#1}


\title{Θεωρία Σημάτων και Γραμμικών Συστημάτων - Σημειώσεις}
\date{2016}
\author{\textlatin{\csuse{no\greek @numbers}\selectlanguage{english} \url{https://github.com/kongr45gpen/ece-notes}}}



\begin{document}
	Την Τρίτη μάθημα 8:30 χωρίς διάλειμμα
	
	Σήμα - σύστημα
	\begin{align*}
	\boxed{\quad \underbrace{g}_{\mathclap{\text{εξαρτημένη}}} = f(\underbrace{t}_{\mathclap{\text{ανεξάρτητη}}}) \quad} \qquad g=f(\vec r, t) \qquad \vec E (\vec r, t)
	\end{align*} 
	
	\paragraph{Αναλογικό}
	\begin{invitemize}
		\item Αν \( t \) συνεχής \( \in \mathbb R  \)
		\item και \( y \) συνεχής \( \in \mathbb R \)
	\end{invitemize}
	\begin{tikzpicture}
		\draw[->] (0,-1) -- (0,2);
		\draw[->] (-2,0) -- (2,0);
		
		\draw [blue, very thick]
		plot [smooth, tension=1, domain=-2:2, samples=9] (\x,{1+rand/2}) node[below] {$g(x,y)$}; 
	\end{tikzpicture}
	
	\paragraph{Διακριτού χρόνου / Διακριτό (discrete)}
	\begin{invitemize}
		\item \( t \) διακριτό \( \rightarrow \mathbb Z,\ n \in \mathbb Z \)
		\item \( g \) συνεχής \( \in \mathbb R  \)
	\end{invitemize}
	
	\begin{tikzpicture}[scale=0.7]
		\draw[->,gray] (0,0) -- (0,4.5);
		\draw[->,gray] (-6,0) -- (6,0);
		
		\foreach \x in {-5,...,5} {
			\draw (\x,0) node[below] {$\x$};
			\filldraw[black] (\x,0) -- (\x,{(1+rand)*2}) circle (1.5pt);
		};
		
		\draw (0,-1) node[below] {7\textsuperscript{ο} εξάμηνο};
	\end{tikzpicture}
	
	\paragraph{Κβαντισμένο}
	\begin{invitemize}
		\item \( n \in \mathbb Z \)
		\item \( g \) διακριτή
	\end{invitemize}
	\begin{tikzpicture}
	\draw[->,gray] (0,0) -- (0,4.5);
	\draw[->,gray] (-4,0) -- (4,0);
	
	\draw[gray,dashed] (-4,1) -- (4,1);
	\draw[gray,dashed] (-4,2) -- (4,2);
	\draw[gray,dashed] (-4,3) -- (4,3);
	
	\filldraw[black] (-3,0) node[below] {$-3$} -- (-3,1) circle (1.5pt);
	\filldraw[black] (-2,0) node[below] {$-2$} -- (-2,2) circle (1.5pt);
	\filldraw[black] (-1,0) node[below] {$-1$} -- (-1,1) circle (1.5pt);
	\filldraw[black] ( 0,0) node[below] {$0$} -- (0,0) circle (1.5pt);
	\filldraw[black] ( 1,0) node[below] {$1$} -- (1,1.5) circle (1.5pt);
	\filldraw[black] ( 2,0) node[below] {$2$} -- (2,3) circle (1.5pt);
	\filldraw[black] ( 3,0) node[below] {$3$} -- (3,2) circle (1.5pt);
	\draw (1,1) node[cross=10pt,red!50!black] {};
	\end{tikzpicture}
	
	\paragraph{Στοχαστικό}
	Περιέχει και τις τρεις κατηγορίες
	
	\subsection{Σύστημα}
	\begin{tikzpicture}[scale=0.8]
		\draw (-1,1) rectangle (3,-1) node[midway] {system};
		
		\draw[->] (-5,0) -- (-1,0) node[midway,above] {$x(t)$} node[midway,below] {input signal};
		\draw[->] (3,0) -- (7,0) node[midway,above] {$y(t)$} node[midway,below] {output signal};
	\end{tikzpicture}
	
	\subsection{Περιοδικά σήματα}
	Αν \( \exists T \in \mathbb R : \forall t \in \mathbb R \quad x(t)=x(t+T) \)
	τότε \( x(t) \) \textbf{περιοδικό σήμα} με περίοδο \( T \).
	
	Ή θα είναι 0, ή θα συνεχιστεί για πάντα.
	
	\[
	\int_{-\sfrac{T}{2}}^{\sfrac{T}{2}} x(t) \dif t =
	\int_{t_0-\sfrac{T}{2}}^{t_0+\sfrac{T}{2}} x(t) \dif t\ \forall t
	\]
	
	Η σύνθεση μιας συνάρτησης με μια περιοδική συνάρτηση είναι περιοδική;
	\subparagraph{Απόδ.}
		Έστω \(g\) μία περιοδική συνάρτηση:
		\begin{align*}
		\big(f\circ g\big)(x) &= f\left(g(x)\right) = f \left( g(x+T) \right) =
		\\ &= \big(f\circ g\big)(x+T)
		\end{align*}
	
	\subsection{Συμμετρίες}
	\begin{itemize}
	    \item Αν \( x(t) = x(-t) \ \forall t \) τότε η \( x(t) \) λέγεται \textbf{άρτια συνάρτηση} (even function).
	    
	    \item Αν \( x(t) = -x(t) \ \forall t \) τότε η \( x(t) \) λέγεται \textbf{περιττή συνάρτηση} (odd function).
	\end{itemize}
	
	\paragraph{}
	\( \forall x(t) \quad \exists\ x_0(t), x_e(t): x(t) = x_e(t)+x_0(t) \)
	\subparagraph{Απόδ.}
	\begin{align*}
		x_e(t) &= \frac{x(t)+x(-t)}{2} \\
		x_o(t) &= \frac{x(t)-x(-t)}{2}
	\end{align*}
	
	\paragraph{}
	\begin{align*}
		x{\underbrace{_e}_{\mathclap{\text{άρτια}}}}y_e &= z_e\\
		x_oy_o &=z_e \\
		x_ey_0 &=z_0 \\
		\int_{-A}^A x_0(t)\dif t &= 0 \\
		\int_{-\infty}^\infty x_0(t)\dif t &= ? \text{ (εξαρτάται)} \\
		\lim_{A\to \infty} \int_{-A}^A x_0(t)\dif t &= 0 \quad \text{(principal Cauchy value)}
	\end{align*}
	
	\subparagraph{Χαρακτηριστικά σήματα}
	\begin{enumpar}
		\item \textbf{Εκθετικό σήμα}
		\[
		x(t) = ce^{at}\quad a \in \mathbb R \quad c > 0
		\]
		\begin{tikzpicture}
		\draw[->,gray] (0,0) -- (0,4);
		\draw[->,gray] (-2,0) -- (4,0) node[below right] {$t$};
		
		\draw (1,3) node {$c > 0$};
		
		\draw[xscale=3,domain=-0.7:1.3,smooth,variable=\x,blue] plot ({\x},{exp(\x)});
		\draw[xscale=3,domain=1.3:-1.2,dashed,smooth,variable=\x,blue] plot ({\x},{exp(-\x)});
		\draw (4,1) node[anchor=north east,blue] {$a < 0$};
		\draw (4.5,3) node[anchor=north east,blue] {$a > 0$};
		\end{tikzpicture}
		\[
		x(t) = ce^{(\sigma t+j\omega)t} = ce^{\sigma t}e^{j\omega t}
		= ce^{\sigma t} \left[
		\cos(\omega t)+j\sin(\omega t)
		\right]
		\]

		\item \textbf{(Συν)ημιτονοειδή σήματα}
		\[
		x(t) = A \cos(\omega t\pm \phi)
		= a \mathrm{Re} \left\lbrace e^{j(\omega t +\phi)} \right\rbrace 
		= A \frac{e^{j(\omega t \pm \phi)} + e^{-j(\omega t\pm \phi)}}{2}
		\]
		\item \textbf{Δέλτα Dirac} \( \bm{ \delta(t)} \)
		\begin{tikzpicture}[scale=1.3]
		\draw[->,gray] (0,-2) -- (0,2);
		\draw[->,gray] (-2,0) -- (4,0) node[below right] {$t$};
		
		
		\draw[scale=1,domain=0:4,samples=200,smooth,variable=\x,blue,thick] plot ({\x},{sin((\x r)*20)*exp(-\x)});
		\end{tikzpicture}
	    \paragraph{Ορ.}
	    \[
	    \int_{-\infty}^\infty f(t)\mathrm \delta(t) \dif t = f(0)\ \forall f(t)
	    \]
	    \paragraph{}
	    \begin{align*}
	    \int_{-\infty}^\infty \delta(t)\dif t &=1 \\
	    \int_{-\infty}^\infty f(t)\delta(t-\tau) \dif t &= f(\tau) \\
	    \Aboxed{\int_{-\infty}^\infty f(\tau) \delta(t-\tau)\dif \tau &= f(t)} \\
	    \int_{-\infty}^\infty f(t)\delta(t-\tau)\dif \tau &= f(t)
	    \end{align*}
	    \paragraph{Ιδιότητες της \( \bm{ \delta(t)} \)}
	    \begin{enumerate} %TODO enumparen?
	    	\item \textbf{Κλιμάκωση}
	    	\[
	    	a \in \mathbb R: \delta(at) = \frac{1}{|a|}\delta(t)
	    	\]
	    	\subparagraph{Απόδ.}
	    	\[
	    	\underbrace{\int_{-\infty}^\infty \phi(t)\boxed{\delta(at)}\dif t}_{\mathclap{\begin{matrix}
	    			at=\xi\\\dif t = \frac{\dif \xi}{a}
	    			\end{matrix}}}
	    =
	    	\int_{-\infty_{(a)}}^{\infty_{(a)}} \phi\left( \frac{\xi}{a} \right) \delta(\xi)\frac{\dif \xi}{a}
	    	= \frac{1}{|a|}\int_{-\infty}^\infty \frac{\phi\left(\frac{\xi}{a} \right)}{|a|}\delta(\xi)\dif\xi = \frac{\phi(0)}{|a|}
	    	= \int_{-\infty}^\infty \phi(t)\boxed{\frac{\delta(t)}{|a|}}\dif t
	    	\]
	    	\item \(
	    	f(t)\delta(t)=f(0)\delta(t)
	    	\)
	    	\item \(
	    	f(t)\delta(t-\xi)=f(g)\delta(t-\xi)
	    	\)
    	\end{enumerate}
	\end{enumpar}

	
\end{document}
\documentclass[11pt,a4paper,notitlepage,fleqn,final]{article}

\usepackage{amsmath}
\usepackage{amsfonts}
\usepackage{amssymb}
\usepackage{commath}
\usepackage{xcolor}
\usepackage{hyperref}
\usepackage[skins,theorems]{tcolorbox}
\usepackage{titlesec}
\usepackage{circuitikz}
\usepackage{pgfplots}
\usepackage{mathtools}
\usepackage[makeroom]{cancel}
\usepackage{mathrsfs}
\usepackage{wrapfig}
\usepackage{subcaption}
\usepackage{floatrow}
\usepackage{esint}
\usepackage{enumitem}
\usepackage{bm}
\usepackage{relsize}
\usepackage{xfrac}
\usepackage{comment}


\usetikzlibrary{arrows.meta}
\usetikzlibrary{patterns}
\usetikzlibrary{decorations.pathmorphing,patterns}
\usetikzlibrary{decorations.markings}
\usetikzlibrary{backgrounds}
\usetikzlibrary{shapes.misc}
\usetikzlibrary{shapes.multipart}

\tikzset{cross/.style={cross out, draw,
        minimum size=2*(#1-\pgflinewidth),
        inner sep=0pt, outer sep=0pt}}
\tikzset{
    mark position/.style args={#1(#2)}{
        postaction={
            decorate,
            decoration={
                markings,
                mark=at position #1 with \coordinate (#2);
            }
        }
    }
}

\usepackage[left=2cm,right=2cm,top=2cm,bottom=2cm]{geometry}

\usepackage[no-math]{fontspec}
\setmainfont{Times New Roman}
\setsansfont{Arial}
%\newfontfamily\greekfont[Script=Greek]{Linux Libertine O}
%\newfontfamily\greekfontsf[Script=Greek]{Linux Libertine O}
\usepackage{polyglossia}
\newfontfamily\greekfont[Script=Greek]{Times New Roman}
\newfontfamily\greekfontsf[Script=Greek]{Arial}
\newfontfamily\greekfonttt[Script=Greek]{Latin Modern Mono}
%\usepackage[greek]{babel}
\setdefaultlanguage{greek}
\setotherlanguage{english}
\newcommand{\textlatin}[1]{#1}
%\newcommand{\mathlarger}{}

%\usepackage[utf8]{inputenc}
%\usepackage[greek]{babel}


\usetikzlibrary{arrows.meta}
\usetikzlibrary{calc}
%\usepackage{tkz-euclide} % loads  TikZ and tkz-base
%\usetkzobj{angles} % important you want to use angles

\newlist{enumparen}{enumerate}{1}
\setlist[enumparen]{label=(\arabic*)}
\newlist{enumpar}{enumerate}{1}
\setlist[enumpar]{label=\arabic*)}

\newlist{enumgreek}{enumerate}{1}
\setlist[enumgreek]{label=\alph*.}
\newlist{enumgreekparen}{enumerate}{1}
\setlist[enumgreekparen]{label=(\alph*)}
\newlist{enumgreekpar}{enumerate}{1}
\setlist[enumgreekpar]{label=\alph*)}


\newlist{enumroman}{enumerate}{1}
\setlist[enumroman]{label=(\roman*)}

\newlist{enumlatin}{enumerate}{1}
\setlist[enumlatin]{label=(\alph*)}

\newlist{invitemize}{itemize}{1}
\setlist[invitemize]{noitemsep,label=}



\makeatletter
\let\anw@true\anw@false

%\newcommand{\attnboxed}[1]{\textcolor{red}{\fbox{\normalcolor\m@th$\displaystyle#1$}}}
\makeatother
\tcbset{highlight math style={enhanced,colframe=red,colback=white,%
        arc=0pt,boxrule=1pt,shrink tight,boxsep=1.5mm,extrude by=0.5mm}}
\newcommand{\attnboxed}[1]{\tcbhighmath[colback=red!5!white,drop fuzzy shadow,arc=0mm]{#1}}
\titleformat{\section}{\bf\Large}{Κεφάλαιο \thesection}{1em}{}
\newtcolorbox{attnbox}[1]{colback=red!5!white,%
    colframe=red!75!black,fonttitle=\bfseries,title=#1}
\newtcolorbox{infobox}[1]{colback=blue!5!white,%
    colframe=blue!75!black,fonttitle=\bfseries,title=#1}

\renewcommand{\arg}{\mathrm{Arg}\, }
\renewcommand{\Re}{\mathrm{Re}}
\renewcommand{\Im}{\mathrm{Im}}
\newcommand{\sinc}{\;\mathrm{sinc}\!}

\newif\ifhidetikz
\hidetikzfalse
%\hidetikztrue   % <---- comment/uncomment that line

\ifhidetikz

\let\oldtikzpicture\tikzpicture
\let\oldendtikzpicture\endtikzpicture

\renewenvironment{tikzpicture}{
    \tiny
    \tt
    \color{blue}
    \newcommand{\draw}{\textit{draw}}
    \newcommand{\filldraw}{\textit{filldraw}}
    %\newcommand{\x}{\textit{x}}
    %\newcommand{\p}{\textit{x}}
    \newcommand{\x1}{\textit{x1}}
    \newcommand{\y1}{\textit{y1}}
    \newcommand{\p1}{\textit{p1}}
}{
}
\newenvironment{axis}{
    \newcommand{\addplot}{\textit{addplot}}
}{
}
\fi

\newtcbtheorem[number within=section]{theorem}{Θ.}%
{colback=green!5,colframe=green!35!black,colbacktitle=green!35!black,fonttitle=\bfseries,enhanced,attach boxed title to top left={yshift=-8mm,xshift=-7mm},width=.9\textwidth,arc=.7mm}{th}
\newtcbtheorem[number within=section]{defn}{Ορισμός}%
{colback=blue!5,colframe=cyan!35!black,colbacktitle=blue!35!black,fonttitle=\bfseries,enhanced,attach boxed title to top left={yshift=-2mm,xshift=-2mm}}{def}
\newtcbtheorem[number within=section]{exercise}{Άσκηση}%
{colback=gray!3,colframe=gray!35!black,colbacktitle=gray!35!black,fonttitle=\bfseries,enhanced,attach boxed title to top left={yshift=-2mm,xshift=-2mm}}{exc}




\title{Αριθμητική Ανάλυση
	\\
	{
	\normalsize Σημειώσεις από τις παραδόσεις
	}}
\date{2017
	\\
	{
	\small Τελευταία ενημέρωση: \today
	}
	}
\author{
	Για τον κώδικα σε \LaTeX, ενημερώσεις και προτάσεις:
\\
 \url{https://github.com/kongr45gpen/ece-notes}}

\setmainfont{Linux Libertine O}
\setsansfont{Ubuntu}
%\newfontfamily\greekfont[Script=Greek]{Linux Libertine O}
%\newfontfamily\greekfontsf[Script=Greek]{Linux Libertine O}
\usepackage{polyglossia}
\newfontfamily\greekfont[Script=Greek,Scale=0.95]{GFS Artemisia}


\begin{document}
	\maketitle

	\tableofcontents

	\vspace{50pt}

	\textbf{Αριθμητική ανάλυση - Numerical Analysis}
	
	Μάθημα 4 ώρες την εβδομάδα - δεν υπάρχει διάκριση μεταξύ θεωρίας και ασκήσεων.
	
	Και τα δύο βιβλία προτείνονται, το μάθημα γίνεται περισσότερο με βάση το βιβλίο του κ.
	Πιτσούλη, του κ. Δούγαλη είναι περισσότερο μαθηματικό.
	
	Στις εξετάσεις δεν θα υπάρχει τυπολόγιο/βιβλίο, αλλά θα δίνονται τύποι που χρειάζονται
	στα θέματα. Απαραίτητο το κομπιουτεράκι.
	
	\section{Εισαγωγή}
	Η αριθμητική ανάλυση μάς δίνει \textit{προσεγγιστικές} λύσεις σε μοντέλα και μαθηματικά
	προβλήματα.
	
	Σε δύσκολα προβλήματα, ζητάμε:
	\begin{itemize}
		\item \textbf{Ακρίβεια} αποτελέσματος
		\item \textbf{Ταχύτητα} υπολογισμού
	\end{itemize}
	
	Θα δούμε τα εξής προβλήματα:
	\begin{itemize}
		\item \textbf{Επίλυση εξισώσεων}
		\paragraph{Παράδειγμα}
		Ένας πελάτης θέλει να καταθέσει \( P \) € για \( N \) χρόνια στην τράπεζα, και εγώ
		του λέω ότι θα του επιστρέψω \( A \) € από την κατάθεση. Ο πελάτης όμως ενδιαφέρεται
		για το ετήσιο επιτόκιο \( R \).
		
		Προκύπτει μια εξίσωση της μορφής:
		\begin{gather*}
		A = P + P \left( 1 + \frac{R}{12} \right) + \dots \\
		f(R) = \frac{P}{\sfrac{R}{12}} \left[
		    \left(1+\frac{R}{12}\right)^N -1
		\right] = 0, \quad R = ?
		\end{gather*}
		
		Θυμόμαστε ότι μπορούμε να χρησιμοποιήσουμε το θεώρημα Bolzano για να βρούμε ότι
		υπάρχει μια τουλάχιστον λύση μέσα σε ένα διάστημα, οπότε μπορούμε να "φανταστούμε"
		έναν αριθμό κοντά στη λύση, και να κλείνουμε συνεχώς ένα διάστημα γύρω από αυτήν (το
		διάστημα στις εξετάσεις θα δίνεται, π.χ. \textit{βρείτε μία λύση στο διάστημα
			\( [2.5,\ 3.5] \) με ακρίβεια \( 10^{-5} \)}), αν και αυτό δεν θα γίνεται στον πραγματικό κόσμο.
		\item \textbf{Παρεμβολή}
		
		Σε έναν σταθμό διοδίων μετράω πόσα αυτοκίνητα περνάν το κάθε λεπτό (π.χ. το
		11\textsuperscript{ο} λεπτό περνάν 4, το 12\textsuperscript{ο} περνάν 7, κλπ.)

		\begin{center}
		\begin{tikzpicture}[scale=1]
		\matrix (values) [matrix of nodes,
		nodes={align=center,text width=0.7cm}
		] {
			11 & 12 & 13 & 14 & 15 \\
			4 & 7 & 11 & 1 & 22 \\
		};
		\draw (values-1-1.south west)--(values-1-5.south east);
		\draw (values-1-1.north west) -- (values-2-1.south west);
		\foreach \x in {1,2,...,5}
		\draw (values-1-\x.north east) -- (values-2-\x.south east);
		
		\draw[thick,->] (4,0) -- ++(1.7,0);
		
		
		\begin{scope}[xshift=7cm,yshift=-1cm]
		\draw (-0.5,0) -- (2,0);
		\draw (0,-0.5) -- (0,2);
		
		\begin{scope}[yshift=2mm,xshift=3mm,yscale=.5,xscale=.5]
		
		\coordinate (A) at (0,1);
		\coordinate (B) at (1,2);
		\coordinate (C) at (2,3);
		\coordinate (D) at (3,0.3);
		\coordinate (E) at (4,4);
		
		\draw[thick,blue]
		plot [smooth] coordinates { (A) (B) (C) (D) (E) };
		
		\foreach \p in {(A),(B),(C),(D),(E)}
		\filldraw \p circle (2pt);
		
		\end{scope}
		\end{scope}
		
		\end{tikzpicture}
		\end{center}
		
		Θέλω να βρω ένα πολυώνυμο που να συνδέει όλα τα σημεία μεταξύ τους (θα αποδείξουμε
		ότι τέτοιο πολυώνυμο πάντα υπάρχει), ή ένα πολυώνυμο (αρκετά χαμηλού βαθμού, ώστε
		να μην γίνονται πολλές πράξεις) με αρκετά καλή προσέγγιση.
		
		Με αυτόν τον τρόπο θα μπορώ να προσεγγίσω τιμές που δεν γνωρίζω, π.χ. αν πήγα για
		καφέ στο 13\textsuperscript{ο} λεπτό
		\item \textbf{Προσέγγιση}
		\item \textbf{Αριθμητική Γραμμική Άλγεβρα}
		
		Exact και προσεγγιστικές λύσεις συστημάτων πολλών αγνώστων.
		\item \textbf{Ολοκλήρωση}
		\item \textbf{Υπολογισμός ιδιοτιμών \& ιδιοδιανυσμάτων}
		\item \textbf{Παραγοντοποίηση πινάκων σε γινόμενο πινάκων}
		
		Βολεύει κυρίως για την επίλυση συστημάτων.
		\item \textbf{Επίλυση κανονικών διαφορικών εξισώσεων}
		\item \textbf{Βελτιστοποίηση}
		
		Για παράδειγμα, να πρέπει να ελαχιστοποιήσω μια συνάρτηση τη στιγμή που πρέπει να
		τηρούνται κάποιες συνθήκες.
	\end{itemize}
	
	\subsection{Ακρίβεια vs Ταχύτητα}
	
	Απόλυτο Σφάλμα: \( \displaystyle
	\left| X_t - X_c \right|
	 \) \quad (απόσταση της λύσης που βρήκα από την πραγματική)
	 
	\vspace{5pt}
	
	Σχετικό Σφάλμα: \( \displaystyle
	\frac{\left|X_t - X_c\right|}{\left|X_t\right|}
	 \)
	
	Επειδή δεν θα γνωρίζουμε την πραγματική λύση \( X_t \), θα βρίσκουμε το μέγιστο σφάλμα.
	
	Για παράδειγμα, σε υπολογιστές έχουμε σφάλματα στρογγύλευσης και αποκοπής:
	\[
	\begin{array}{ll}
		0.66666 & \\
		0.66 & \leftarrow \text{αποκοπή} \\
		0.67 & \leftarrow \text{στρογγύλευση}
	\end{array}
	\]
	
	\section{Επίλυση Εξισώσεων}
	\begin{center}
		\( \displaystyle \mathlarger{
		f(x): \qquad \text{βρείτε $\bar x$ έτσι ώστε \( f(\bar x) = 0 \).}}
		\)
	\end{center}
	
	Η επίλυση είναι εύκολη όταν η \( f \) είναι πολυώνυμο μέχρι 2\textsuperscript{ου} βαθμού,
	όχι όμως όταν είναι μεγαλύτερου, ή όταν έχει κι άλλους όρους (π.χ. εκθετικούς) μέσα.
	
	\begin{itemize}
		\item Δημιουργούμε μια ακολουθία \( x_1, \dots, x_n \) προσέγγισης της \( \bar x \).
		\item Σε κάθε βήμα κάνουμε έλεγχο σύγκλισης για να δούμε πόσο κοντά είμαστε.
	\end{itemize}
	
	\subsection{Μέθοδος Διχοτόμησης}
	Θα χρησιμοποιήσω θεώρημα Bolzano (\( \mathsmaller{f(a) \cdot f(b) < 0} \))
	
	\begin{center}
		\begin{tikzpicture}[scale=1.9]
		\draw (0,-0.5) -- (0,2);
		\draw (-1,0) -- (3,0);
		
		\draw[thick, orange,
		mark position=0.5(c),
		mark position=0.3(a),
		mark position=0.8(b),
		mark position={0.55}(c1),
		mark position={(0.55/2+0.8/2)}(c2),
		mark position={(0.55/4+0.8/4+0.55/2)}(c3)
		] plot[smooth,tension=0.7]
		coordinates {(-0.5,0.7) (0.5,1.5)  (2.2,-0.7) (3.2,-0.9)};
		\draw[orange!90!black] (c) node[above] {$f$};
		
		\draw[dashed] (a) -- (a |- 0,0) node[below] {$a$};
		\draw[dashed] (b) -- (b |- 0,0) node[above] {$b$};
		\draw[thick,gray!50!blue] (c1) -- (c1 |- 0,0) node[below] {$c_1$};
		\draw[thick,gray!35!blue] (c2) -- (c2 |- 0,0) node[above] {$c_2$};
		\draw[thick,gray!20!blue!60!green] (c3) -- (c3 |- 0,0) node[below] {$\mathsmaller{c_3}$};
		
		\draw (1,-1.5) node {$\displaystyle c_1 = \frac{a+b}{2}$};
		\draw (1,-2) node {$\displaystyle c_2 = \frac{c_1+b}{2}$};
		\end{tikzpicture}
	\end{center}
	
	Ξεκινάω από μία αρχική προσέγγιση/διάστημα (η μέθοδος διχοτόμησης δεν δίνει λύση
	συγκεκριμένη, αλλά διάστημα - τη λύση την παίρνω σαν το μέσο του διαστήματος).
	
	Συνέχεια κόβω το διάστημα στη μέση, έτσι ώστε \( f(\text{των άκρων}) \) να είναι
	ετερόσημα, και παίρνω συνεχώς ένα μικρότερο διάστημα.
	
	Σταματάω όταν είναι επιτυχής ο έλεγχος σύγκλισης, π.χ
	\( \mathlarger{\left|c_{k+1}-c_k\right| < \epsilon \rightarrow 10^{-5}} \), δηλαδή
	το διάστημα στο οποίο βρίσκεται η ρίζα είναι αρκετά μικρό.
	
	Θα διαπιστώσουμε ότι, αν και αυτή η μέθοδος λειτουργεί πάντα, είναι αρκετά αργή.
	
	\paragraph{Παράδειγμα}
	Να βρεθεί ρίζα της εξίσωσης
	\( \mathlarger{
	\mathlarger{f(x) = x^3+x-1} }
	 \) στο διάστημα \( [0,1] \) με ακρίβεια \( 10^{-3} \).
	
	\begin{gather*}
		f(0) \cdot f(1) < 0 \\
		k=1,\ a_1=0,\ b_1=1,\ m=\frac{a_1+b_1}{2} = 0.5 \\
		\quad f(m) = -0.375 < 0 \\
		\quad f(0.5) \cdot f(1) < 0
	\end{gather*}
	Άρα η λύση βρίσκεται μεταξύ \( 0.5 \) και \( 1 \).
	
	Για το επόμενο βήμα:
	\begin{gather*}
		k = 2,\ m_2 = \frac{0.5+1}{2} = 0.75 \\
		\quad f(m_2) = 0.172 > 0 \\
		\quad f(m_1)f(m_2) < 0 \\[3ex]
		k=3,\ m_3 = \frac{0.5+0.75}{2} = 0.625 \\
		\quad f(m_3) = -0.131 \\
		\quad f(m_2)f(m_3) < 0 \\[3ex]
		\mathlarger{\mathlarger{\vdots}}
	\end{gather*}
	
	\subsubsection{Σύγκλιση}
	Όπου \( r \) και \( m_n \) η πραγματική και προσεγγιστική λύση αντίστοιχα:
	\begin{align*}
		\left|r - m_n\right| \leq \left(\frac{1}{2}\right)^n (b-a)
	\end{align*}
	επειδή στη \( n- \)οστή επανάληψη έχω κόψει το διάστημα στα 2 σε \( n \) φορές.
	
	Για σφάλμα \( 10^{-5} \), θέλουμε \( |r-m_n| = 10^{-5} \):
	\begin{align*}
		&10^{-5} \leq \left(\frac{1}{2}\right)^n (1-0) \implies
		\\ \implies & n \simeq 16.667 \implies \mathlarger{n = 17}
	\end{align*}
	
	\subsection{Μέθοδος Χορδής ή Τέμνουσας}
	Σαν τη μέθοδο Bolzano, αλλά δεν παίρνουμε το μέσο του διαστήματος, αλλά κάποια άλλη τιμή.
	
	Παρατηρείται ότι αυτή η μέθοδος συγκλίνει γρηγορότερα.
	
	\begin{tikzpicture}[scale=2]
	\draw (0,-0.2) -- (0,2);
	\draw (-0.5,0) -- (3,0);
	
	\draw[dashed] (2.7,1.7) node [right] {$\left(b_1,f(b_1)\right)$}
	-- (2.7,0);
	
	\draw[very thick, blue!50!magenta] plot[smooth,tension=1]
	coordinates {(0.5,-0.4) (2,0.3) (2.7,1.7)};
	\draw[blue!50!magenta] (2.11,0.2) node {$f$};
	
	\draw[dashed] (0.5,-0.4) node[below left] {$\left(a_1,f(a_1)\right)$} -- (0.5,0) node[above] {$a_1$};
	\draw[cyan,thick] (0.5,-0.4) -- (2.7,1.7);
	\draw[dashed] (0.93,-0.3) node[below,yshift=-2mm] {$\left(a_2,f(a_2)\right)$} -- (0.93,0) node[above] {$a_2$};
	\draw [cyan,thick] (0.93,-0.3) -- (2.7,1.7);
	\draw[dashed] (1.2,0) -- (1.2,-0.2);
	\draw [cyan,thick] (1.2,-0.2) -- (2.7,1.7);
	\end{tikzpicture}
	
	Πρέπει να βρούμε τη ρίζα εντός του διαστήματος \( (a_1,b_1) \).
	\begin{itemize}
		\item Παίρνουμε τη χορδή που ενώνει τα \( a_1, b_1 \).
		\item Η χορδή αυτή τέμνει τον άξονα των \( x \) στο σημείο \( a_ 2\),
		επομένως η ρίζα βρίσκεται μεταξύ των \( a_2, b_1 \).
		\item Παίρνουμε τη χορδή που ενώνει τα \( a_2, b_2 \)
		\item Η χορδή αυτή τέμνει τον άξονα των \( x \) στο σημείο \( a_3 \),
		επομένως η ρίζα βρίσκεται μεταξύ των \( a_3,b_1 \)
		\item \( \cdots \)
	\end{itemize}
	
	Παρατηρούμε ότι η μέθοδος χορδής λειτουργεί για
	κυρτές συναρτήσεις.
	
	\begin{infobox}{Κυρτές συναρτήσεις}
		Όταν λέμε ότι μια συνάρτηση είναι κυρτή,
		εννοούμε ότι η καμπύλη της είναι κυρτή,
		δηλαδή για δύο σημεία της, η χορδή που τα ενώνει
		δεν τέμνει κάποιο σημείο της \( f \). Αντιθέτως,
		η συνάρτηση λέγεται μη κυρτή.
	\end{infobox}
	
	\paragraph{}
	\( a_1,b_1 \) όπου \( f(a_1)f(b_1) < 0 \) for
	\( k=1,2,\dots \)
	\begin{align*}
		& \left(a, f(a)\right) \qquad \left(b,f(b)\right)
		\\
		y-f(b) &= \frac{f(b)-f(a)}{(b-a)}(x-b) \\
		\text{Σημείο τομής με άξονα } x:
		\Aboxed{x &= \frac{af(b)-bf(a)}{f(b)-f(a)}}
		\\
		c &= \frac{a_kf(b_k)=b_kf(a_k)}{f(b_k)-f(a_k)}
		\\ f: \quad f(a_k)f(c) < 0 &\implies
		a_{k+1} = a_k, \quad b_{k+1}=c \\
		\text{else} &\implies 
		a_{k+1} = c, \quad b_{k+1} = b_k
	\end{align*}
	
	Αν και η μέθοδος χορδής είναι αργή, συνεχίζει να
	είναι γρηγορότερη από τη μέθοδο διχοτόμησης.
	
	\subsection{Μέθοδος Μεταβαλλόμενης Χορδής}
	\begin{tikzpicture}[scale=2]
	\draw (0,-0.2) -- (0,2);
	\draw (-0.5,0) -- (3,0);
	
	\draw[dashed] (2.7,1.7) node [right] {$\left(b_1,f(b_1)\right)$}-- (2.7,0);
	
	\draw[very thick, blue!50!magenta] plot[smooth,tension=1]
	coordinates {(0.5,-0.4) (2,0.3) (2.7,1.7)};
	
	\draw[dashed] (0.5,-0.4) -- (0.5,0) node[above] {$a_1$};
	\draw[draw=cyan,thick] (0.5,-0.4) -- (2.7,1.7);
	\draw[dashed] (0.93,-0.3) -- (0.93,0) node[above,yshift=1mm] {$a_2$};
	\draw [draw=cyan,thick] (0.93,-0.3) -- (2.7,0.85)
	node[right]  {$ \left(b_1,\frac{f(b_1)}{2}\right)$};
	\draw (1.4,0) node[above] {$a_3$} -- (1.4,-0.12);
	\end{tikzpicture}
	
	Όπως η μέθοδος χορδής, αλλά μεταβάλλουμε την κλίση
	της χορδής γρηγορότερα, ώστε να φτάσει πιο κοντά
	στη ρίζα.
	
	Αυτό μπορούμε να το πετύχουμε π.χ. θεωρώντας
	ως 2\textsuperscript{ο} σημείο το \( \left(b_1,\frac{f(b_1)}{2}\right) \)
	αντί για το \( \left( b_1,f(b_1) \right) \), όπως φαίνεται στο
	σχήμα. Αντί για να διαιρέσουμε με 2, μπορούμε να επιλέξουμε έναν
	άλλον αριθμό, π.χ. \( 4 \) για συναρτήσεις που έχουν πιο κάθετες
	χορδές.
	
	\subsection{Μέθοδος Newton}
	Αν και η μέθοδος Newton δεν συγκλίνει πάντα και απαιτεί παραγωγισιμότητα,
	είναι πολύ πιο γρήγορη από τις προηγούμενες μεθόδους, και χρησιμοποιείται
	πολύ πιο συχνά.
	
	Παίρνουμε το ανάπτυγμα Taylor της \( f \):
	\begin{gather*}
		f(x) = f(x_0) + (x-x_0)f'(x_0) \\
		f(x_0) + (x-x_0) f'(x_0) = 0 \\
		\boxed{x = x_0 - \frac{f(x_0)}{f'(x_0)}}
	\end{gather*}
	
	\paragraph{}
	for \( \kappa = 1,2,\dots \)
	\[
	x_{\kappa+1} = x_\kappa - \frac{f(x_1)}{f'(x_1)} \ 
	\rightarrow \text{ΕΠΑΝΑΛΗΠΤΙΚΗ}
	\]
	\begin{gather*}
	\mathlarger{\text{if }} \left|x_{\kappa+1}-x_\kappa\right|<\epsilon  \
	\rightarrow \text{ΚΡΙΤΗΡΙΟ ΤΕΡΜΑΤΙΣΜΟΥ} \\
	\mathlarger{\text{stop}}, \text{ αλλιώς } \kappa = \kappa+1
	\end{gather*}
	
	Η μέθοδος Newton έχει αρκετά περίπλοκες συνθήκες
	σύγκλισης που δεν θα μελετήσουμε.
	
	\subsection{Μέθοδος Σταθερού Σημείου}
	\begin{defn}{}{Σταθερό Σημείο}
		\[
		\begin{array}{ll}
		f(x) & \quad \text{το $\hat{x}$ σταθερό σημείο}
		\\
		\text{ανν} & \quad f(\hat x) = \hat x
		\end{array}
		\]
	\end{defn}
	
	\( \bar x \) της \( f(x) \) στο \( (a,b) \)
	αν κατασκευάσω \( g(x) \):
	\[
	\bar x = g(\bar x) \iff f(\bar x) = 0
	\]
	
	\paragraph{Παράδειγμα}
	\[
	\mathlarger{f(x) = x^2-x-2}
	\]
	
	Μπορούμε να θέσουμε:
	\begin{gather*}
		g(x) = x^2 - 2 \\
		g(x) = \sqrt{x+2} \\
		g(x) = 1 - \frac{2}{x}
	\end{gather*}
	
	Αντί να λύσω την \( f \), βρίσκω υποψήφιες λύσεις
	(δύσκολες επαναληπτικές λύσεις θα δίνονται στις
	εξετάσεις)
	
	\paragraph{Μέθοδος}
	for \(k=1,2,\dots \qquad \left(x_0,g(x),\epsilon\right) \)
	\[
	\mathlarger{\mathlarger{x_i = g\left( x_{i-1} \right)}}
	\]
	\[
	\begin{array}{rl}
		\mathlarger{\text{if }}
		\left|x_i-g(x_i)\right| < \epsilon
		& \rightarrow \text{ΚΡΙΤΗΡΙΟ ΤΕΡΜΑΤΙΣΜΟΥ}
	\end{array}
	\]
	
	\paragraph{Προϋποθέσεις σύγκλισης}
	\begin{enumerate}
		\item Για αρχικό \( x_0 \), τα \( x_1,x_2,\dots \) να
		είναι υπολογίσιμα στην \( g(x) \)
		\subparagraph{π.χ.}
		\( g(x) = -\sqrt{x} \quad \) για \( x_0 > 0 \)
		\begin{gather*}
			x_1 = g(x_0) = -\sqrt{x_0} \\
			x_2 = g(x_1) = -\sqrt{x_1}
		\end{gather*}
		\item \( x_1,x_2,\dots  \) να συγκλίνουν σε ένα
		\( \bar x \)
		\item Το σημείο σύγκλισης \( \gamma \) να είναι σταθερό
		σημείο της \( g(x) \).
	\end{enumerate}
	
	Τα παραπάνω μετασχηματίζονται σε 3 κριτήρια σύγκλισης:
	\paragraph{Κριτήρια Σύγκλισης}
	\begin{enumerate}
		\item Υπάρχει \( [a,b] \) στο οποίο ορίζεται η
		\( g(x) \) και \( g(x) \in [a,b] \), δηλαδή:
		\[
		g: [a,b] \to [a,b]
		\]
		\item \( g(x) \) συνεχής στο \( [a,b] \)
		\item \( g(x) \) παραγωγίσιμη και να υπάρχει \(k<1\):
		\[
		\forall x \in [a,b], \
		\left|g'(x)\right| \leq k
		\]
	\end{enumerate}
	
	\paragraph{Θεώρημα}
	Αν ισχύουν οι 3 παραπάνω προϋποθέσεις, τότε στο διάστημα
	\( [a,b] \) υπάρχει ακριβώς ένα σταθερό σημείο \(\gamma\)
	της \(g\), και η μέθοδος σταθερού σημείου συγκλίνει σε αυτό
	το \( \gamma \).
	
	\subsection{Ασκήσεις}
	\paragraph{Άσκηση}
	\[
	\mathlarger{x^3+2x-1=0}
	\]
	Να αποδειχθεί ότι έχει μια μόνο ρίζα στο \( \left[0,\frac{1}{2}\right] \) και
	να προσεγγιστεί με τη μέθοδο σταθερού σημείου.
	\subparagraph{Λύση}
	\[ \left.
	\begin{array}{l}
	f(0) = -1,\ f\left(\frac{1}{2}\right) = \frac{1}{8} \implies
	f(0)f\left(\frac{1}{2}\right) < 0 \\
	f'() = 3x^2 + 2 > 0
	\end{array} \right\rbrace \text{μοναδικότητα}
	\]
	\begin{gather*}
	g(x) = x \iff f(x) = 0 \\
	g(x) = \frac{1}{2} (1-x^3) \text{ (περιμένω να δουλέψει, δηλαδή να ικανοποιούνται
		τα κριτήρια σύγκλισης)}
	\end{gather*}
	\begin{align*}
		&\mathrm Y_1 \ \checkmark \\
		&\mathrm Y_2 \ \checkmark \\
		&\mathrm Y_3:\ g'(x) = \frac{-3}{2} x^2 \\
		&\ \left|g'(x)\right| = \frac{3}{2} x^2 \leq \frac{3}{2}\left(\frac{1}{2}\right)^2
		= \frac{3}{8} < 1 \\
		\text{Άρα: } & \mathrm Y_3 \ \checkmark
	\end{align*}
	(όπου \( \mathrm Y \) τα κριτήρια σύγκλισης)
	
	Επομένως, η \( g(x) \) που επέλεξα ικανοποιεί τη σύγκλιση.
	
	Επιλέγω: \( x_0 = 0 \)
	\begin{align*}
		x_1 &= g(x_0) = g(0) = \frac{1}{2} \\
		x_2 &= g(x_1) = g\left(\frac{1}{2}\right) = \frac{1}{2}\left(
		1-\left(\frac{1}{2}\right)^3
		\right) = 0.4375 \\
		x_3 &= g(x_2) = \frac{1}{2} \left(1 - 0.4375^3\right) = 0.4581298 \\
		x_4 &= g(x_3) = \frac{1}{2} \left(1-0.4581298^3\right) = 0.451923191
	\end{align*}
	\paragraph{Άσκηση}
	Να υπολογιστή ένας τύπος με τη μέθοδο Newton που θα βρίσκει την τετραγωνική ρίζα θετικού
	αριθμού.
	\subparagraph{Λύση}
	\begin{gather*}
		x = \sqrt{a} \implies x^2=a \\[0.3ex] \\
		f(x) = x^2-a \\
		f'(x) = 2x \\
		x_{n+1} = x_n - \frac{f(x_n)}{f'(x_n)} = x_n - \frac{x_n^2-a}{2x_n}
		= \frac{1}{2} \left( x_n + \frac{a}{x_n} \right)
	\end{gather*}
	\subparagraph{Παράδειγμα}
	\[
	\mathlarger{\sqrt{3} = ?, \qquad \text{με } x_0 = 1.5}
	\]
	\begin{gather*}
		x_1 = \frac{1}{2} \left(x_0+\frac{3}{x_0}\right)
		= \frac{1}{2} \left(1.5+\frac{3}{1.5}\right) = 1.75 \\
		x_2 = \frac{1}{2} \left(x_1+\frac{3}{x_1}\right) = 1.7321428 \\
		\intertext{Υπολογισμός με ακρίβεια 3 δεκαδικών ψηφίων (δύο διαδοχικές ρίζες
			να απέχουν \( 0.5\cdot 10^{-3} \implies \))
			(σφάλμα) \(\to 0.5\cdot 10^{-3}\)  }
		x_3 = \frac{1}{2} \left( x_2 + \frac{3}{x_2} \right) = 1.7320528 \\
		\left|x_3-x_2\right| < 0.5\cdot 10^{-3} \text{ (ικανοποίηση κριτηρίου τερματισμού)}
	\end{gather*}
	
	Άρα:
	\[
	\boxed{x_3 = 1.7320528}
	\]
	\paragraph{Άσκηση}
	\[
	\mathlarger{x^3+2x^2+10x-20=0}
	\]
	Να προσεγγιστεί μια ρίζα της παραπάνω εξίσωσης (π.χ. για 3 επαναλήψεις)
	\subparagraph{Λύση}
	\begin{gather*}
		x_{n+1} = x_n - \frac{f(x_n)}{f'(x_n)} = \frac{2x_n^3+2x_n^2+20}{3x_n^2+4x_n+10} \\
		x_0 = 1 \\
		x_1 = \frac{2\cdot 1^3 + 2\cdot 1^2 + 20}{3\cdot 1^2 + 4\cdot 1 + 10}
		= \dots = 1.41764706 \\
		x_2 = \frac{2x_1^3+2x_1^2+20}{3x_1^2+4x_1+20} = 1.369336471 \\
		x_3 = \frac{2x_2^3+2x_2^2+20}{3x_2^2+4x_2+10} = 1.368908108
	\end{gather*}
	\paragraph{Άσκηση}
	Να βρεθεί μια λύση της εξίσωσης
	\[
	\mathlarger{x^3+x+1}
	\]
	στο διάστημα \( (0,1) \)
	με ακρίβεια 2 δεκαδικών ψηφίων χρησιμοποιώντας τη μέθοδο
	χορδής, και να συγκριθεί με τη μέθοδο Newton.
	\subparagraph{Λύση}
	\begin{align*}
	x_{n+1} &= x_n - f(x_n)\frac{x_n-x_{n-1}}{f(x_n)-f(x_{n-1})}
	\\
	x_2 &= x_1 - f(x_1) \frac{x_1-x_0}{f(x_1)-f(x_0)}
	= 1-1\frac{1}{2} = 0.5
	\\
	x_3 &= x_2 - f(x_2) \frac{x_2-x_1}{f(x_2)-f(x_1)} = 0.636
	\\
	x_4 &= x_3 - f(x_3) \frac{x_3-x_2}{f(x_3)-f(x_2)} = 0.606
	\\
	\vdots
	\\ x_6 &= \dots = 0.687
	\\ x_7 &= \dots = 0.682
	\end{align*}
	Άρα με ακρίβεια 2 δεκαδικών ψηφίων, \( \rho=0.68 \), και
	χρησιμοποιήσαμε 7 επαναλήψεις.
	\subparagraph{Λύση με μέθοδο Newton}
	\begin{align*}
		x_{n+1} &= x_n - \frac{f(x_n)}{f'(x_n)} \\
		x_{n+1} &= x_n - \frac{x_n^3+x_n-1}{3x_n^2+1} \\
		x_{n+1} &= \frac{2x_n^3+1}{3x_n^2+1} \\
		\intertext{\(x_0 = 0.5\)}
		x_1 &= \frac{2\cdot 0.5^2 + 1}{3\cdot 0.5^2+1}
		= 0.714 \\
		x_2 &= 0.683 \\
		x_3 &= 0.682
	\end{align*}
	
	\paragraph{Άσκηση}
	Να βρεθεί μια λύση της:
	\[
	\mathlarger{f(x) = x^{10} -1} \qquad (0,1.3)
	\]
	
	\begin{tikzpicture}[scale=.7]
		\begin{axis}[samples=30,very thick,domain=0:1.3,no marks]
		\addplot+{x^10-1};
		\end{axis}
	\end{tikzpicture}
	
	\subparagraph{Λύση με διχοτόμηση}
	Με τη μέθοδο της διχοτόμησης έχουμε:
	\[
	\begin{array}{c|c|c}
	a_k & b_k & c  \\ \hline
	0 & 1.3 & 0.65 \\
	0.65 & 1.3 & 0.975 \\
	0.975 & 1.3 & 1.1375 \\
	0.975 & 1.1375 & 1.05625 \\
	0.975 & 1.05625 & 1.015625
	\end{array}
	\]
	
	\subparagraph{Λύση με μέθοδο χορδής}
	Με τη μέθοδο της χορδής έχουμε (εφαρμόζοντας τον τύπο
	όπως και στην προηγούμενη άσκηση):
	\[
	\begin{array}{c|c|c}
	a_k & b_k & c  \\ \hline
	0 & 1.3 & 0.09430 \\
	0.09430 & 1.3 & 0.18176 \\
	0.18176 & 1.3 & 0.26287 \\
	0.26287 & 1.3 & 0.33811 \\
	0.33811 & 1.3 & 0.40708
	\end{array}
	\]
	
	Παρατηρούμε ότι σε αυτήν την περίπτωση, η μέθοδος χορδής
	είναι αρκετά πιο αργή. Για να το αποτρέψουμε αυτό, μπορούμε
	να χρησιμοποιήσουμε μεταβαλλόμενη χορδή.
	
	\subparagraph{Λύση με μέθοδο Newton}
	\[
	x_{i+1} = x_i - \frac{x_i^{10}-1}{10x_i^9}
	\]
	
	\[
	\begin{array}{c|c}
	i & x_i \\ \hline
	0 & 0.5 \\
	1 & 51.65 \\
	2 & 46.485 \\
	3 & 41.8365 \\
	4 & 37.65285 \\
	\vdots & \vdots \\
	40 & 1.002316
	\end{array}
	\]
	
	\paragraph{Άσκηση}
	Να βρεθεί μία ρίζα της εξίσωσης
	\[
	\mathlarger{f(x) = x^3-2x-5}
	\]
	στο διάστημα \( [2,3] \) χρησιμοποιώντας τη μέθοδο της
	διχοτόμησης, με ακρίβεια \( \epsilon = 10^{-6} \).
	\subparagraph{Λύση}
	\[
	\begin{array}{r|c|c|c}
	i & a_k & b_k & c \\ \hline
	1 & 2 & 3 & 2.5 \\
	2 & 2 & 2.5 & 2.25 \\
	3 & 2 & 2.25 & 2.125 \\
	\vdots & \vdots & \vdots & \vdots \\
	19 & & & 2.0945530
	\end{array}
	\]
	
	
\end{document}

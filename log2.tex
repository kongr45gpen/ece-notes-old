\documentclass[11pt,a4paper]{article}
\usepackage[utf8]{inputenc}
\usepackage[greek]{babel}
\usepackage{amsmath}
\usepackage{amsfonts}
\usepackage{amssymb}
\usepackage{commath}
\usepackage{xcolor}
\usepackage[left=2cm,right=2cm,top=2cm,bottom=2cm]{geometry}


\begin{document}

\setcounter{section}{-2}

\section{Λογισμός II}

\begin{itemize}
\item 2 ώρες Ζάχαρης (3.5 μον.)
\item 4 ώρες εγώ (6.5 μον.)

\textlatin{http://users.auth.gr/natreas}
\end{itemize}

\paragraph{}

\begin{itemize}
\item Ρασσιάς Θ.
\item Κωνσταντινίδου Μ.
\item Ξένος
\item Σημειώσεις
\end{itemize}

\section{Κεφάλαιο 0: Διανυσματικές συναρτήσεις \\ Καμπύλες στο χώρο}

\paragraph{Ορ.}
Μία συνάρτηση \( \mathbf{r}: A \subseteq \mathbb R \rightarrow \mathbb R ^ n \) απαρτίζεται από:

(α). το πεδίο ορισμού της \(A\) που είναι υποσύνολο της πραγματικής ευθείας και

(β). έναν τύπο έτσι ώστε σε κάθε πραγματικό αριθμό \(t \in A\) αντιστοιχεί \textbf{ΜΟΝΑΔΙΚΟ ΔΙΑΝΥΣΜΑ} \( \mathbf{r}(t) \) στο (διανυσματικό) χώρο \( \mathbb R ^ n \) δηλαδή:

\[
A \subseteq \mathbb R \rightarrow \mathbb R ^ n: \mathbf{r}(t) = \left( f_1(t), \cdots, f_n(t) \right)
\]

όπου \(f_1: A \le \mathbb R \rightarrow \mathbb R\) \textbf{συνήθεις} πραγματικές συναρτήσεις.

\paragraph{}
Πεδίο ορισμού διανυσματικής συνάρτησης είναι εκείνο το υποσύνολο του \( \mathbb R \) για όλα τα σημεία του οποίου ο τύπος της συνάρτησης \textbf{ΕΧΕΙ ΝΟΗΜΑ}.

\emph{Πρακτικά}, αν
\[ \mathbf{r}(t) = \left( f_1(t), \cdots, f_n(t) \right), \]
τότε το πεδίο ορισμού της \( \mathbf{r} \) προκύπτει από τη \textbf{συναλήθευση} των πεδίων ορισμού ΟΛΩΝ των συναρτήσεων \( f_1,\cdots,f_n\).

\emph{π.χ.}
\[ \mathbf r (t) = \left( \ln t, \sqrt{1-t^2} \right) 
\quad \leftarrow \text{διανυσματική συνάρτηση πραγματικής μεταβλητής}
\]
Πρέπει
\[
\begin{cases}
t > 0 \text{ (λόγω λογαρίθμου)} \\
\quad \text{και} \\
1-t^2 > 0 \text{ (λόγω ρίζας)}
\end{cases}
\]
Άρα Π.Ο. της \(\mathbf r\) είναι το \((0,1]\).

\subsubsection{Όριο και συνέχεια διανυσματικών συναρτήσεων}
\paragraph{Θ.} Έστω \(\mathbf r: A \subseteq \mathbb R \rightarrow \mathbb R ^ n\), \( \mathbf r(t) =  \left( f_1(t), \cdots, f_n(t) \right) \) διανυσματική συνάρτηση και \(t_0\) είναι σημείο συσσώρευσης (σ.σ.) του \(A\).
Τότε:

\[ \lim _{ t \to t_0 } \mathbf r(t) = \vec{a} = \left( a_1, \cdots, a_n \right) \iff
\begin{cases}
 \lim_{ t \to t_0 } f_1(t) = a_1 \\
 \vdots \\
 \lim_{ t \to t_0 } f_n(t) = a_n \\
\end{cases}
\]

Επίσης, αν \(τ_0 \in A\) είναι και σ.σ. του \(Α\), τότε:

\[ \mathbf r \text{ συνεχής στο } τ_0 \iff
f_1,f_2,\cdots,f_n  \text{ συνεχείς στο } τ_0
\]

δηλ.
\[ \lim _{ t \to t_0 } \mathbf r(t) = \mathbf r (t_0) \iff
\begin{cases}
 \lim_{ t \to t_0 } f_1(t) = f_1(t_0) \\
 \vdots \\
 \lim_{ t \to t_0 } f_n(t) = f_n(t_0) \\
\end{cases}
\]

\subsection{Καμπύλες στον \(\mathbb R^n\)}
\paragraph{Ορ.}
Έστω \(α,β \in \mathbb R\) με \(α<β\). Κάθε \textbf{ΣΥΝΕΧΗΣ} διανυσματική συνάρτηση:
\[
\mathbf{\gamma}: [a,b] \rightarrow \mathbb R^n: \mathbf r_\gamma(t) = \left( f_1(t),f_2(t),\cdots,f_n(t) \right)
\]
καλείται καμπύλη στο χώρο \( \mathbb R ^n \) (και το γράφημά της καλείται ΙΧΝΟΣ της \(\mathbf \gamma\)).


\subsubsection{Γράφημα}
(\(\cdots\) Graph 1)

Έστω \( \gamma: [a,b] \rightarrow \mathbb R ^ n \) καμπύλη.
\begin{itemize}
\item Η \(\gamma\) θα καλείται \textbf{ΑΠΛΗ} αν είναι 1-1, δηλ. \( \forall t \in (a,b) \) με \(t_1 \neq t_2 \implies \mathbf r(t_1) \neq \mathbf r(t_2)\) (δηλ. ΔΕΝ αυτοτέμνεται).

\item Η \(\gamma\) καλείται \textbf{ΑΝΟΙΚΤΗ}, αν \[\mathbf r(a) \neq \mathbf r(b),\] αλλιώς \textbf{ΚΛΕΙΣΤΗ}.

\item Όλες οι καμπύλες \(\gamma: [a,b] \rightarrow \mathbb R ^ n\) \[\mathbf r_\gamma(t) = \left( f_1(t),f_2(t),\cdots,f_n(t) \right)\] λέμε ότι είναι καμπύλες σε \textbf{ΠΑΡΑΜΕΤΡΙΚΗ} μορφή και οι
\(
\begin{cases}
x_1 = f_1(t) \\
x_2 = f_2(t) \\
\quad \vdots \\
x_n = f_n(t)
\end{cases}
\) καλούνται παραμετρικές εξισώσεις της \(\gamma\).

\item Δύο καμπύλες μπορεί να έχουν το \textbf{ΙΔΙΟ ΙΧΝΟΣ}.
\paragraph{π.χ.}
\begin{alignat*}{2}
&\mathbf r_{\gamma_1}(t) = ( \cos t,  && \sin t )  \quad t \in [0, 2 \pi ) \\
&\mathbf r_{\gamma_2}(t) = ( \cos t, - && \sin t )  \quad t \in [0, 2 \pi )
\end{alignat*}
Δηλαδή το ίχνος είναι το ίδιο ΑΛΛΑ αλλάζει η ΦΟΡΑ ΔΙΑΓΡΑΦΗΣ ή ο προσανατολισμός.

\emph{Έτσι}, σε κάθε καμπύλη \(\gamma\) σε παραμετρική μορφή αντιστοιχεί με φυσικό τρόπο ένας \textbf{ΠΡΟΣΑΝΑΤΟΛΙΣΜΟΣ} (ή ΦΟΡΑ ΔΙΑΓΡΑΦΗΣ), πάντα προς την κατεύθυνση αύξησης των \(\gamma\).

\item Έστω \(\gamma_1: [a,b] \rightarrow \mathbb R^n, \gamma_2: [b,c] \rightarrow \mathbb R^n\) καμπύλες.

Καλώ \textbf{ΑΝΤΙΘΕΤΗ} της \(\gamma_1\), συμβολικά \(- \gamma_1\), την καμπύλη που έχει ίδιο ΙΧΝΟΣ με τη \(\gamma_1\) αλλά \textbf{αντίθετη} φορά διαφραφής.

\[ -{\gamma_1} : [a,b] \rightarrow \mathbb R^n : \mathbf r_{-\gamma_1}(t) - \mathbf r_{\gamma_1}(a+b-t)
\]

\item Αν \(\mathbf r_{\gamma_1}(b) = r_{\gamma_2}(b)\), ορίζω την καμπύλη \( \gamma_1+\gamma_2 \) ως εξής:
\[
\gamma_1+\gamma_2: [a,c] \rightarrow \mathbb R^n:
\mathbf r_{\gamma_1+\gamma_2}(t) =
\begin{cases}
\mathbf r_{\gamma_1}, \quad t \in [a,b] \\
\mathbf r_{\gamma_2}, \quad t \in (b, c]
\end{cases}
\]

\item Έστω \( \phi: [c,d] \rightarrow [a,b] \) \textbf{συνεχής} και γνησίως \textbf{μονότονη} συνάρτηση. Τότε η \textbf{σύνθεση}: \[ \mathbf \gamma_1 \circ \phi: [c, d] \rightarrow \mathbb R ^n \]
είναι καμπύλη που καλείται \textbf{ΙΣΟΔΥΝΑΜΗ} της \(\gamma_1\) και έχει το \textbf{ΙΔΙΟ ΙΧΝΟΣ} με τη \(\gamma_1\).

\item Αν \(\phi\) γν. αύξουσα, τότε η σύνθεση έχει και ίδιο προσανατολισμό, αλλιώς αντίθετο προσανατολισμό σε σχέση με τη \(\gamma_1\).
\end{itemize}

\subsection{Παραδείγματα καμπύλων σε παραμετρική μορφή}
\begin{itemize}
\item Έστω \(A,B \in \mathbb R^n\), το \(\overrightarrow{AB}\) παραμετροποιείται ως:
\begin{align*}
\mathbf r(t) &= (\text{αρχή}) + (\text{πέρας}-\text{αρχή}), \quad {\color{red}t \in [0,1]} \\
&= (a_1, \cdots, a_n) + t \left( (b_1, \cdots, b_n) - (a_1, \cdots, a_n) \right) \\
&= \left( a_1 + t (b_1 - a_1 ), \cdots, a_n + t (b_n - a_n) \right), \quad {\color{red}t \in [0,1]}
\end{align*}
επειδή για κάθε σημείο \(X \in \overrightarrow{AB}\):
\[
\overrightarrow{OX} = \overrightarrow{OA} + \overrightarrow{AX}
= \overrightarrow{OA} + t \overrightarrow{AB}
\]

\item Κύκλος \( (x-a)^2 + (y-b)^2 = R^2\) στο χώρο \(\mathbb R^2\):
\[\mathbf r(t) = \left( f_1(t), f_2(t) \right), \quad t \in [a,b] \quad \leftarrow \text{ παραμετροποίηση γενικά} \]
\emph{Ειδικότερα}
\[
\mathbf r(t) = \left( a+R \cos t, b+ R \sin t \right), \quad t \in [0, 2 \pi)
\]
με θετική φορά διαγραφής (αντιωρολογιακή).

Ή:
\[
\mathbf r(t) = \left( a+R \cos t, b - R \sin t \right), \quad t \in [0, 2 \pi)
\] με αρνητική φορά διαγραφής.

\item Συνάρτηση \(y=f(x)\) πραγματική όπου \(x \in [a,b]\)
\[\mathbf r(t) = \left( x(t), y(t) \right), \quad t \in [a,b] \quad \leftarrow \text{ παραμετροποίηση γενικά} \]
\emph{Ειδικότερα}
\[
\mathbf r(t) = \left( t, f(t) \right) \quad t \in [a, b)
\]

\item Έλλειψη \( \left( \frac{x-a}{A} \right) ^ 2 + \left( \frac{y-b}{B} \right) ^ 2 = 1 \quad \left( \mathbb R^2 \right) \)

\[
\begin{cases}
\frac{x-a}{A} = \cos{t}
\frac{y-b}{B} = \sin{t}
\end{cases}, \text{ τότε}
\begin{cases}
\frac{x}{A} = a+ A \cos{t}
\frac{y}{B} = b+ B \sin{t}
\end{cases}
\]
Έτσι \(\mathbf r (t) = (x, y) = (a + A \cos t, b + B \sin t), \quad t \in [0, 2 \pi) \) 

\item Υπερβολή \( \left( \frac{x-a}{A} \right) ^ 2 - \left( \frac{y-b}{B} \right) ^ 2 = 1 \quad \left( \mathbb R^2 \right) \)

\[
\mathbf r (t) = (a + A \cosh t, b + B \sinh t), \quad t \in [0, 2 \pi)
\]

\end{itemize}


\end{document}
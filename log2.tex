\documentclass[11pt,a4paper,titlepage,draft]{article}
\usepackage{amsmath}
\usepackage{amsfonts}
\usepackage{amssymb}
\usepackage{amsthm}
\usepackage{commath}
\usepackage{xcolor}
\usepackage{hyperref}
\usepackage[skins,theorems]{tcolorbox}
\usepackage{titlesec}
\usepackage{wrapfig}
\usepackage{mathtools}
\usepackage{pgfplots}
\usepackage{subcaption}
\usepackage{float}
\usepackage{relsize}
\usepackage[inline]{enumitem}
\usepackage{cancel}
\usepackage{esint}
\usepackage{graphicx}
\usepackage{caption}
\usepackage{extarrows}

\usepackage[no-math]{fontspec}
\setmainfont{Times New Roman}
\setsansfont{Arial}
%\newfontfamily\greekfont[Script=Greek]{Linux Libertine O}
%\newfontfamily\greekfontsf[Script=Greek]{Linux Libertine O}
\usepackage{polyglossia}
\newfontfamily\greekfont[Script=Greek]{Times New Roman}
\newfontfamily\greekfontsf[Script=Greek]{Arial}
\newfontfamily\greekfonttt[Script=Greek]{Latin Modern Mono}
%\usepackage[greek]{babel}
\setdefaultlanguage{greek}
\setotherlanguage{english}
\newcommand{\textlatin}[1]{#1}

%\usepackage[utf8]{inputenc}
%\usepackage[greek]{babel}


\usetikzlibrary{arrows.meta}
\usetikzlibrary{calc}
%\usepackage{tkz-euclide} % loads  TikZ and tkz-base
%\usetkzobj{angles} % important you want to use angles

\newlist{enumparen}{enumerate}{1}
\setlist[enumparen]{label=(\arabic*)}

\newlist{enumlatin}{enumerate}{1}
\setlist[enumlatin]{label=(\alph*)}

\usepackage[left=2cm,right=2cm,top=2cm,bottom=2cm]{geometry}

\makeatletter
%\newcommand{\attnboxed}[1]{\textcolor{red}{\fbox{\normalcolor\m@th$\displaystyle#1$}}}
\makeatother
\tcbuselibrary{breakable}
\tcbset{highlight math style={enhanced,colframe=red,colback=white,%
  arc=0pt,boxrule=1pt,shrink tight,boxsep=1.5mm,extrude by=0.5mm}}
\newcommand{\attnboxed}[1]{\tcbhighmath[colback=red!5!white,drop fuzzy shadow,arc=0mm]{#1}}
\titleformat{\section}{\bf\Large}{Κεφάλαιο \thesection}{1em}{}
\newtcolorbox{attnbox}[1]{breakable,colback=red!5!white,%
  colframe=red!75!black,fonttitle=\bfseries,title=#1}
\newtcolorbox{infobox}[1]{colback=blue!5!white,%
  colframe=blue!75!black,fonttitle=\bfseries,title=#1}
\newtcolorbox{tbox}[1]{colback=green!5!white,%
  colframe=green!75!black,fonttitle=\bfseries,title=Θεώρημα #1}


\title{Σημειώσεις Λογισμού ΙΙ}
\date{2016, Εαρινό εξάμηνο}
\author{\textlatin{\url{https://github.com/kongr45gpen/ece-notes}}}



\begin{document}

\maketitle

\tableofcontents

\newpage

\part{Ατρέας}

\setcounter{section}{-1}

\begin{itemize}
\item 2 ώρες Ζάχαρης (3.5 μον.)
\item 4 ώρες εγώ (6.5 μον.)

\textlatin{http://users.auth.gr/natreas}
\end{itemize}

\paragraph{}

\begin{itemize}
\item Ρασσιάς Θ.
\item Κωνσταντινίδου Μ.
\item Ξένος
\item Σημειώσεις
\end{itemize}

\section[Διανυσματικές συναρτήσεις \& καμπύλες στο χώρο]{Διανυσματικές συναρτήσεις \\ Καμπύλες στο χώρο}

\paragraph{Ορ.}
Μία συνάρτηση \( \mathbf{r}: A \subseteq \mathbb R \rightarrow \mathbb R ^ n \) απαρτίζεται από:

(α). το πεδίο ορισμού της \(A\) που είναι υποσύνολο της πραγματικής ευθείας και

(β). έναν τύπο έτσι ώστε σε κάθε πραγματικό αριθμό \(t \in A\) αντιστοιχεί \textbf{ΜΟΝΑΔΙΚΟ ΔΙΑΝΥΣΜΑ} \( \mathbf{r}(t) \) στο (διανυσματικό) χώρο \( \mathbb R ^ n \) δηλαδή:

\[
A \subseteq \mathbb R \rightarrow \mathbb R ^ n: \mathbf{r}(t) = \left( f_1(t), \dots, f_n(t) \right)
\]

όπου \(f_1: A \subseteq \mathbb R \rightarrow \mathbb R\) \textbf{συνήθεις} πραγματικές συναρτήσεις.

\paragraph{}
Πεδίο ορισμού διανυσματικής συνάρτησης είναι εκείνο το υποσύνολο του \( \mathbb R \) για όλα τα σημεία του οποίου ο τύπος της συνάρτησης \textbf{ΕΧΕΙ ΝΟΗΜΑ}.

\emph{Πρακτικά}, αν
\[ \mathbf{r}(t) = \left( f_1(t), \dots, f_n(t) \right), \]
τότε το πεδίο ορισμού της \( \mathbf{r} \) προκύπτει από τη \textbf{συναλήθευση} των πεδίων ορισμού ΟΛΩΝ των συναρτήσεων \( f_1,\dots,f_n\).

\emph{π.χ.}
\[ \mathbf r (t) = \left( \ln t, \sqrt{1-t^2} \right) 
\quad \leftarrow \text{διανυσματική συνάρτηση πραγματικής μεταβλητής}
\]
Πρέπει
\[
\begin{cases}
t > 0 \text{ (λόγω λογαρίθμου)} \\
\quad \text{και} \\
1-t^2 > 0 \text{ (λόγω ρίζας)}
\end{cases}
\]
Άρα Π.Ο. της \(\mathbf r\) είναι το \((0,1]\).

\subsubsection{Όριο και συνέχεια διανυσματικών συναρτήσεων}
\paragraph{Θ.} Έστω \(\mathbf r: A \subseteq \mathbb R \rightarrow \mathbb R ^ n\), \( \mathbf r(t) =  \left( f_1(t), \dots, f_n(t) \right) \) διανυσματική συνάρτηση και \(t_0\) είναι σημείο συσσώρευσης (σ.σ.) του \(A\).
Τότε:

\[ \lim _{ t \to t_0 } \mathbf r(t) = \vec{a} = \left( a_1, \dots, a_n \right) \iff
\begin{cases}
 \lim_{ t \to t_0 } f_1(t) = a_1 \\
 \quad \vdots \\
 \lim_{ t \to t_0 } f_n(t) = a_n \\
\end{cases}
\]

Επίσης, αν \(τ_0 \in A\) είναι και σ.σ. του \(Α\), τότε:

\[ \mathbf r \text{ συνεχής στο } τ_0 \iff
f_1,f_2,\dots,f_n  \text{ συνεχείς στο } τ_0
\]

δηλ.
\[ \lim _{ t \to t_0 } \mathbf r(t) = \mathbf r (t_0) \iff
\begin{cases}
 \lim_{ t \to t_0 } f_1(t) = f_1(t_0) \\
 \quad \vdots \\
 \lim_{ t \to t_0 } f_n(t) = f_n(t_0) \\
\end{cases}
\]

\subsection{Καμπύλες στον \(\mathbb R^n\)}
\paragraph{Ορ.}
Έστω \(α,β \in \mathbb R\) με \(α<β\). Κάθε \textbf{ΣΥΝΕΧΗΣ} διανυσματική συνάρτηση:
\[
\mathbf{\gamma}: \attnboxed{[a,b]}
\rightarrow \mathbb R^n: \mathbf r_\gamma(t) = \left( f_1(t),f_2(t),\dots,f_n(t) \right)
\]
καλείται καμπύλη στο χώρο \( \mathbb R ^n \) (και το γράφημά της καλείται ΙΧΝΟΣ της \(\mathbf \gamma\)).


\begin{tikzpicture}
\draw[gray, thick] (-2.5,-2.5) -- (2.5,2.5);
\node[above right, rotate=45] at (-2.5,-2.5) {\(+\infty\)};
\node[above left, rotate=45] at (2.5,2.5) {\(-\infty\)};

\filldraw[black] (0.5,0.5) circle (2pt) node[anchor=west] {\(a\)};

\def\away{7}

\draw[->] (xyz cs:x=\away) -- (xyz cs:x=\away+2.5);
\draw[->] (xyz cs:y=0,x=\away) -- (xyz cs:y=2.5,x=\away);
\draw[->] (xyz cs:z=0,x=\away) -- (xyz cs:z=5,x=\away);

\draw[-{>[scale=2.5,width=3]}, gray] (0.5,0.5) to [bend left=90] (\away+1,3);

\coordinate (A) at (\away+1,2);
\coordinate (B) at (\away+2,1,-0.5);
\coordinate (C) at (\away+3.5,3.5);

\draw [blue, very thick] plot [smooth, tension=2] coordinates { (A) (B) (C) };

\draw[->, very thick] (\away,0) -- (A) node[anchor=east] {\(\mathbf r(a)\)};
\draw[->, very thick] (\away,0) -- (B) node[anchor=west] {\(\mathbf r(t)\)};
\draw[->, very thick] (\away,0) -- (C) node[anchor=east] {\(\mathbf r(b)\)};

\end{tikzpicture}



\subsubsection*{}

Έστω \( \gamma: [a,b] \rightarrow \mathbb R ^ n \) καμπύλη.
\begin{itemize}
\item Η \(\gamma\) θα καλείται \textbf{ΑΠΛΗ} αν είναι 1-1, δηλ. \( \forall t \in (a,b) \) με \(t_1 \neq t_2 \implies \mathbf r(t_1) \neq \mathbf r(t_2)\) (δηλ. ΔΕΝ αυτοτέμνεται).

\item Η \(\gamma\) καλείται \textbf{ΑΝΟΙΚΤΗ}, αν \[\mathbf r(a) \neq \mathbf r(b),\] αλλιώς \textbf{ΚΛΕΙΣΤΗ}.

\item Όλες οι καμπύλες \(\gamma: [a,b] \rightarrow \mathbb R ^ n\) \[\mathbf r_\gamma(t) = \left( f_1(t),f_2(t),\dots,f_n(t) \right)\] λέμε ότι είναι καμπύλες σε \textbf{ΠΑΡΑΜΕΤΡΙΚΗ} μορφή και οι
\(
\begin{cases}
x_1 = f_1(t) \\
x_2 = f_2(t) \\
\quad \vdots \\
x_n = f_n(t)
\end{cases}
\) καλούνται παραμετρικές εξισώσεις της \(\gamma\).

\item Δύο καμπύλες μπορεί να έχουν το \textbf{ΙΔΙΟ ΙΧΝΟΣ}.
\paragraph{π.χ.}
\begin{alignat*}{2}
&\mathbf r_{\gamma_1}(t) = ( \cos t,  && \sin t )  \quad t \in [0, 2 \pi ) \\
&\mathbf r_{\gamma_2}(t) = ( \cos t, - && \sin t )  \quad t \in [0, 2 \pi )
\end{alignat*}
Δηλαδή το ίχνος είναι το ίδιο ΑΛΛΑ αλλάζει η ΦΟΡΑ ΔΙΑΓΡΑΦΗΣ ή ο προσανατολισμός.

\emph{Έτσι}, σε κάθε καμπύλη \(\gamma\) σε παραμετρική μορφή αντιστοιχεί με φυσικό τρόπο ένας \textbf{ΠΡΟΣΑΝΑΤΟΛΙΣΜΟΣ} (ή ΦΟΡΑ ΔΙΑΓΡΑΦΗΣ), πάντα προς την κατεύθυνση αύξησης των \(\gamma\).

\item Έστω \(\gamma_1: [a,b] \rightarrow \mathbb R^n, \gamma_2: [b,c] \rightarrow \mathbb R^n\) καμπύλες.

Καλώ \textbf{ΑΝΤΙΘΕΤΗ} της \(\gamma_1\), συμβολικά \(- \gamma_1\), την καμπύλη που έχει ίδιο ΙΧΝΟΣ με τη \(\gamma_1\) αλλά \textbf{αντίθετη} φορά διαγραφής.

\[ -{\gamma_1} : [a,b] \rightarrow \mathbb R^n : \mathbf r_{-\gamma_1}(t) - \mathbf r_{\gamma_1}(a+b-t)
\]

\item Αν \(\mathbf r_{\gamma_1}(b) = r_{\gamma_2}(b)\), ορίζω την καμπύλη \( \gamma_1+\gamma_2 \) ως εξής:
\[
\gamma_1+\gamma_2: [a,c] \rightarrow \mathbb R^n:
\mathbf r_{\gamma_1+\gamma_2}(t) =
\begin{cases}
\mathbf r_{\gamma_1}, \quad t \in [a,b] \\
\mathbf r_{\gamma_2}, \quad t \in (b, c]
\end{cases}
\]

\item Έστω \( \phi: [c,d] \rightarrow [a,b] \) \textbf{συνεχής} και γνησίως \textbf{μονότονη} συνάρτηση. Τότε η \textbf{σύνθεση}: \[ \mathbf \gamma_1 \circ \phi: [c, d] \rightarrow \mathbb R ^n \]
είναι καμπύλη που καλείται \textbf{ΙΣΟΔΥΝΑΜΗ} της \(\gamma_1\) και έχει το \textbf{ΙΔΙΟ ΙΧΝΟΣ} με τη \(\gamma_1\).

\item Αν \(\phi\) γν. αύξουσα, τότε η σύνθεση έχει και ίδιο προσανατολισμό, αλλιώς αντίθετο προσανατολισμό σε σχέση με τη \(\gamma_1\).
\end{itemize}

\subsection{Παραδείγματα καμπύλων σε παραμετρική μορφή}
\begin{itemize}
\item Έστω \(A,B \in \mathbb R^n\), το \(\overrightarrow{AB}\) παραμετροποιείται ως:
\begin{align*}
\mathbf r(t) &= (\text{αρχή}) + (\text{πέρας}-\text{αρχή}), \quad \attnboxed{t \in [0,1]} \\
&= (a_1, \dots, a_n) + t \left( (b_1, \dots, b_n) - (a_1, \dots, a_n) \right) \\
&= \left( a_1 + t (b_1 - a_1 ), \dots, a_n + t (b_n - a_n) \right), \quad \attnboxed{t \in [0,1]}
\end{align*}
επειδή για κάθε σημείο \(X \in \overrightarrow{AB}\):
\[
\overrightarrow{OX} = \overrightarrow{OA} + \overrightarrow{AX}
= \overrightarrow{OA} + t \overrightarrow{AB}
\]

\item Κύκλος \( (x-a)^2 + (y-b)^2 = R^2\) στο χώρο \(\mathbb R^2\):
\[\mathbf r(t) = \left( f_1(t), f_2(t) \right), \quad t \in [a,b] \quad \leftarrow \text{ παραμετροποίηση γενικά} \]
\emph{Ειδικότερα}
\[
\mathbf r(t) = \left( a+R \cos t, b+ R \sin t \right), \quad t \in [0, 2 \pi)
\]
με θετική φορά διαγραφής (αντιωρολογιακή).

Ή:
\[
\mathbf r(t) = \left( a+R \cos t, b - R \sin t \right), \quad t \in [0, 2 \pi)
\] με αρνητική φορά διαγραφής.

\item Συνάρτηση \(y=f(x)\) πραγματική όπου \(x \in [a,b]\)
\[\mathbf r(t) = \left( x(t), y(t) \right), \quad t \in [a,b] \quad \leftarrow \text{ παραμετροποίηση γενικά} \]
\emph{Ειδικότερα}
\[
\mathbf r(t) = \left( t, f(t) \right) \quad t \in [a, b)
\]

\item Έλλειψη \( \left( \frac{x-a}{A} \right) ^ 2 + \left( \frac{y-b}{B} \right) ^ 2 = 1 \quad \left( \mathbb R^2 \right) \)

\[
\begin{cases}
\frac{x-a}{A} &= \cos t \\
\frac{y-b}{B} &= \sin t
\end{cases}, \text{ τότε}
\begin{cases}
\frac{x}{A} = a+ A \cos{t} \\
\frac{y}{B} = b+ B \sin{t}
\end{cases}
\]
Έτσι \(\mathbf r (t) = (x, y) = (a + A \cos t, b + B \sin t), \quad t \in [0, 2 \pi) \) 

\item Υπερβολή \( \left( \frac{x-a}{A} \right) ^ 2 - \left( \frac{y-b}{B} \right) ^ 2 = 1 \quad \left( \mathbb R^2 \right) \)

\[
\mathbf r (t) = (a + A \cosh t, b + B \sinh t), \quad t \in [0, 2 \pi)
\]

\end{itemize}

\subsection{Παράγωγος διανυσματικών συναρτήσεων μίας μεταβλητής}
\paragraph{Ορ.}
Έστω \(\mathbf r: A \subset \mathbb R \rightarrow  \mathbb R^n : \mathbf r(t) = \left( f_1(t), \dots, f_n(t) \right), \quad t_0 \in A \) είναι σ.σ. του \(Α\). Θα λέμε ότι η \(\mathbf r\) παραγωγίσιμη στο \(t_0\) αν υπάρχει το όριο:
\[
\lim _{t \to t_0} \frac{\mathbf r(t)- \mathbf{r}(t_0)}{t-t_0}
\]
ή ισοδύναμα
\[
\lim _{h \to 0} \frac{\mathbf r(t_0+h)- \mathbf{r}(t_0)}{h}
\]
το οποίο είναι \textbf{ΔΙΑΝΥΣΜΑ} που συμβολίζουμε με \(\mathbf r' (t_0)\) ή \(\frac{\dif \mathbf r}{\dif t}\).

\begin{itemize}
\item Αν η \(\mathbf r\) παραγωγίσιμη σε κάθε σημείο του \(Α\), λέμε ότι είναι παραγωγίσιμη στο \(Α\).
\end{itemize}


\begin{tbox}{}
\(\mathbf{r}\text{ παραγωγίσιμη στο }A \iff f_1,\dots,f_n \text{ παραγ. στο } A\) και
\[
\mathbf r'(t) = \left( f_1'(t),\dots, f_n'(t) \right)
\]
\end{tbox}

\subsubsection{Γεωμετρική ερμηνεία}
Έστω \(h > 0\)

\begin{tikzpicture}
\draw[gray, thick] (-2.5,-2.5) -- (2.5,2.5);
\node[above right, rotate=45] at (-2.5,-2.5) {\(+\infty\)};
\node[above left, rotate=45] at (2.5,2.5) {\(-\infty\)};

\filldraw[black] (0.5,0.5) circle (2pt) node[anchor=west] {\(t_0\)};
\filldraw[black] (-1,-1) circle (2pt) node[anchor=west] {\(t_0+h\)};


\def\away{7}

\draw[->] (xyz cs:x=\away) -- (xyz cs:x=\away+2.5);
\draw[->] (xyz cs:y=0,x=\away) -- (xyz cs:y=2.5,x=\away);
\draw[->] (xyz cs:z=0,x=\away) -- (xyz cs:z=5,x=\away);

\draw[-{>[scale=2.5,width=3]}, gray] (0.5,0.5) to [bend left=90] (\away+1,3);

\coordinate (A) at (\away+0.2,1);
\coordinate (B) at (\away+1.5,3.2);
\coordinate (C) at (\away+4,4);

\draw [blue, very thick] plot [smooth, tension=1] coordinates { (A) (B) (C) };

%\draw[->, very thick] (\away,0) -- (A) node[anchor=east] {\(\mathbf r(a)\)};
\draw[->, very thick] (\away,0) -- (B) node[anchor=west] {\(\mathbf r(t_0)\)};
\draw[->, very thick] (\away,0) -- (C) node[anchor=east] {\(\mathbf r(t_0+h)\)};

\draw[->] (B) -- (C);


\end{tikzpicture}

Το \( \mathbf r'(t) \) έχει τη διεύθυνση της εφαπτόμενης ευθείας της \( \mathbf r\) στο σημείο \( \mathbf r(τ_0)\) και φορά τη φορά της κίνησης. Μπορεί να αναπαριστά π.χ. την ταχύτητα ενός υλικού σημείου.

\subsubsection[Εξίσωση εφαπτομένης]{Εξίσωση (Διανυσματική) εφαπτόμενης ευθείας καμπύλης \(\gamma: [a,b] \rightarrow  \mathbb R ^n = \left(f_1(t),\dots,f_n(t) \right)\) παραγωγίσιμης στο σημείο \(t_0\)}

{\small Αρκεί να ξέρω σημείο της ευθείας και διάνυσμα παράλληλο στην ευθεία}
\begin{attnbox}{Προσοχή!!!}
\[
 \mathbf r_{\text{εφ}}(t) =  \mathbf r(t_0) + \lambda \cdot  \mathbf r'(t_0), \quad \lambda \in  \mathbb R 
 \]
 \end{attnbox}
 
\subsubsection{Ιδιότητες παραγώγου}
Έστω \( \mathbf r_1, \mathbf r_2:[a,b] \rightarrow  \mathbb R ^n\) παραγωγ. καμπύλες, τότε:

\begin{itemize}
\item \( \big( \mathbf r_1 \cdot  \mathbf r_2 \big) ' (t) =
 \mathbf r_1' \cdot  \mathbf r_2(t) +  \mathbf r_1 \cdot  \mathbf r_2'(t)
 \)
\item \( \big( \mathbf r_1 \times  \mathbf r_2 \big) ' (t) =
 \mathbf r_1' \times  \mathbf r_2(t) +  \mathbf r_1 \times  \mathbf r_2'(t)
 \)
 \item \begin{align*} 
 [ \mathbf r_1, \mathbf r_2, \mathbf r_3 ]'(t) = 
 &[ \mathbf r_1', \mathbf r_2, \mathbf r_3 ] + \\
 &[ \mathbf r_1, \mathbf r_2',\mathbf r_3 ] + \\
 &[ \mathbf r_1, \mathbf r_2, \mathbf r_3' ]
\end{align*}
\end{itemize}

\subsection{}
\begin{infobox}{Πρόταση}
Έστω \( \gamma: [a,b] \rightarrow \mathbb R^n: \mathbf r_\gamma (t) = \left( f_1(t),\dots,f_n(t) \right) \) είναι μια παραγωγίσιμη καμπύλη στο \( [a,b] \). \emph{Τότε}:
\[ \left| \mathbf r_\gamma (t) \right| = c = \text{σταθερά } \forall t \iff
\mathbf r_\gamma (t) \bot \mathbf r'_\gamma (t) \ \forall t \]
\tcblower
\begin{proof}
\begin{align*}
|\mathbf r_\gamma(t)| = c \iff \\
|\mathbf r_\gamma(t)|^2 = c^2 \iff \\
\mathbf r_\gamma (t) \cdot \mathbf r_\gamma (t) = c^2 \iff \\
(\mathbf r_\gamma \cdot \mathbf r_\gamma)'(t) = 0 \iff \\
\mathbf r'_\gamma \cdot \mathbf r_\gamma + \mathbf r_\gamma \cdot \mathbf r'_\gamma = 0
\iff \\
\mathbf r'_\gamma \cdot \mathbf r_\gamma = 0 \iff \\
\mathbf r_\gamma \bot \mathbf r'_\gamma
\end{align*}
\end{proof}

\vspace{-20pt}

\begin{center}
	\(\mathbb R^2 \)
\begin{tikzpicture}[scale=0.7]
\draw[gray, thick] (-2.5,-2.5) -- (2.5,2.5);
\node[above right, rotate=45] at (-2.5,-2.5) {\(+\infty\)};
\node[above left, rotate=45] at (2.5,2.5) {\(-\infty\)};

\filldraw[black] (0.5,0.5) circle (2pt) node[anchor=west] {\(t_0\)};


\def\away{7}

\draw[-{>[scale=2.5,width=3]}, gray] (0.5,0.5) to [bend left=90] (\away,3);

\draw[dashed] (\away,0) circle (3);

\draw (\away-4.22,0.37) -- (\away-0.58,3.44);
\draw (\away+0.15,3.53) -- (\away+3.69,1.1);

\draw (\away,0) -- (\away-1.93,2.3);
\draw (\away,0) -- (\away+1.78,2.41);


\end{tikzpicture}
\( \left|\mathbf r_\gamma(t)\right| = c \ \forall t \) σημαίνει κύκλος
\end{center}

\begin{center}
	\(\mathbb R^3 \)
		\begin{tikzpicture}[scale=0.7]
		\draw[-{>[scale=2.5,width=3]}, gray] (0.5,0.5) to [bend left=90] (6,4);
		\draw[gray, thick] (-2.5,-2.5) -- (2.5,2.5);
		\node[above right, rotate=45] at (-2.5,-2.5) {\(+\infty\)};
		\node[above left, rotate=45] at (2.5,2.5) {\(-\infty\)};
		
		\filldraw[black] (0.5,0.5) circle (2pt) node[anchor=west] {\(t_0\)};
		\begin{scope}[shift={(7,0)},scale=2]
		\draw[dashed] (-1,0) arc (180:360:1cm and 0.5cm);
		\draw[dashed] (-1,0) arc (180:0:1cm and 0.5cm);
		\draw[dashed] (0,1) arc (90:270:0.5cm and 1cm);
		\draw[dashed] (0,1) arc (90:-90:0.5cm and 1cm);
		\draw (0,0) circle (1cm);
		%\shade[ball color=blue!10!white,opacity=0.20] (0,0) circle (1cm);
		
		\draw[->] (0,0) -- (0,1.5) node[above right] {$z$};
		\draw[->] (0,0) -- (1.5,0) node[above right] {$y$};
		\draw[->] (0,0) -- (-1,-1) node[below right] {$x$};
		\end{scope}
		
		
		\end{tikzpicture}
	\( \left|\mathbf r_\gamma(t)\right| = c \ \forall t \) σημαίνει καμπύλη πάνω στη σφαίρα κέντρου \( O(0,0) \) και ακτίνας \(c\)
\end{center}



\end{infobox}

\subsection{Διαφορικό καμπύλης}
Ορίζω \(\dif   \mathbf r _ \gamma (t) = \mathbf r '_ \gamma (t) \dif t\), \textbf{διάνυσμα} πάνω στην εφαπτόμενη καμπύλη στο σημείο \(t\) και φορά που καθορίζεται από το πρόσημο του \(\dif t\).

\begin{center}

\begin{tikzpicture}[scale=1]
\draw[-{>[scale=2.5,width=3]}, gray] (0.5,0.5) to [bend left=90] (6,4);
\draw[gray, thick] (-2.5,-2.5) -- (2.5,2.5);
\node[above right, rotate=45] at (-2.5,-2.5) {\(+\infty\)};
\node[above left, rotate=45] at (2.5,2.5) {\(-\infty\)};

\filldraw[black] (0.5,0.5) circle (2pt) node[anchor=west] {\(t\)};
\filldraw[black] (-0.5,-0.5) circle (2pt) node[anchor=west] {\(t+\mathrm d\, t\)};	
\begin{scope}[shift={(6,-2)},scale=1.5]
\draw[->] (0,-0.5) -- (0,3) node[above right] {$y$};
\draw[->] (-0.5,0) -- (3,0) node[above right] {$x$};

\draw[very thick,blue] plot [smooth] coordinates {
	(2.4,2.5) (1, 2.1) (0.5,2.5)
};		

\draw[->] (0,0) -- (2.4,2.5) node[below,midway,sloped] {$\mathbf r_\gamma(t)$} ;
\draw[->] (0,0) -- (0.5,2.5) node[below,midway,sloped] {$\mathbf r_\gamma(t+\mathrm d\, t)$} ;
\draw[->,gray] (2.4,2.5) -- (1.5,841/250) node[above] {$\mathrm d\, \mathbf r_\gamma(t) = \mathbf r'_\gamma(t)\mathrm d\, t$};
\draw[->] (2.2,2.5) -- (0.7,2.5) node[above,midway,sloped] {$\mathbf r_\gamma(t+\mathrm d\, t) - \mathbf r_\gamma(t)$};

\draw (2.8,1.2) node {$\big(\mathrm d\, t > 0\big)$};

\end{scope}
\end{tikzpicture}
\end{center}

Για \(\dif t \to 0\), δηλ. κοντά στο \(t\) ισχύει:
\[ \mathbf r _\gamma (t+dt) - \mathbf r _\gamma (t) \approx d\mathbf r_\gamma (t) \]


\begin{center}
	\begin{tikzpicture}[scale=3]
	
	
	\draw[very thick,blue!80] plot [smooth,tension=1.2] coordinates {
		(0,0) (0.2,0.5) (0.5,0.7) (1,1) (2,1) (2.3,1.1) (2.6,1) (3,1.05) (3.5,0.8)
	};		
	
	\draw (0,0) -- (0.2,0.5) -- (0.5,0.7) -- (1,1) -- (2,1) -- (2.3,1.1) -- (2.6,1) -- (3,1.05) -- (3.5,0.8);
	\filldraw[black] (0,0) circle(0.4pt);
	\filldraw[black] (0.2,0.5) circle(0.4pt);
	\filldraw[black] (0.5,0.7) circle(0.4pt);
	\filldraw[black] (1,1) circle(0.4pt);
	\filldraw[black] (2,1) circle(0.4pt);
	\filldraw[black] (2.3,1.1) circle(0.4pt);
	\filldraw[black] (2.6,1) circle(0.4pt);
	\filldraw[black] (3,1.05) circle(0.4pt);
	\filldraw[black] (3.5,0.8) circle(0.4pt);
	
	\end{tikzpicture}
\end{center}

\subsubsection{}
\paragraph{Ορ.}
Έστω \(\gamma: [a,b] \to \mathbb R ^n\) καμπύλη σε παραμετρική μορφή.
\begin{enumerate}
\item Η \(\gamma\) καλείται \textbf{ΟΜΑΛΗ} αν είναι παραγωγίσιμη και \( \mathbf r' _ \gamma \neq \mathbf 0 \ \forall t \)
\item Η \( \gamma \) καλείται \textbf{ΛΕΙΑ} αν είναι ΟΜΑΛΗ και η \( \mathbf r _ \gamma \) είναι \textbf{ΣΥΝΕΧΗΣ} συνάρτηση στο \([a,b]\). Αν η \( \mathbf r '_ \gamma \) είναι τμηματικά συνεχής στο \([a,b]\), τότε η \( \mathbf r _ \gamma \) καλείται τμηματικά λεία.
\end{enumerate}

\paragraph{Ορ.}
Έστω \(\mathbf r:[a,b] \to  \mathbb R ^n: \mathbf r _\gamma(t) = 
\left( f_1(t), \dots, f_n(t) \right) \) είναι μια καμπύλη. Ορίζουμε το ορισμένο ολοκλήρωμα της \(\mathbf r_\gamma\) ως εξής:
\[
\underbrace{\int _a^b  \mathbf r _\gamma (t) \dif t}_{\text{διάνυσμα στον }\mathbb R^n}
=
\left(
\int _a^b f_1(t) \dif t + \int _a^b f_2(t) \dif t + \dots
+ \int _a^b f_n(t) \dif t
\right)
\]

Επίσης, υπάρχει παραγωγίσιμη καμπύλη \(\mathbf q:[a,b] \to  \mathbb R ^n:\)
\[
\mathbf q'(t) = \mathbf r_\gamma(t) \quad \forall t
\]
Η \(\mathbf q\) καλείται αντιπαράγωγος της \( \gamma \). Το σύνολο \(
 \left\lbrace \mathbf q(t) + \overline{\mathbf c}: \mathbf{q}
 \text { μια αντιπαράγωγος της }  \mathbf r _ \gamma \text{ και } \mathbf c \in  \mathbb R ^n \text{ σταθερά}  \right\rbrace
 \) καλείται ΑΟΡΙΣΤΟ ΟΛΟΚΛΗΡΩΜΑ της \( \mathbf r _ \gamma \), συμβολικά \(
 \int  \mathbf r _ \gamma (t) \dif t \)
 
Πράγματι,
\begin{align*}
\int  \mathbf r _\gamma (t) \dif t 
&= \left( \int f_1(t) \dif t , \int f_2(t) \dif t, \dots, \int f_n(t) \dif t \right) \\
&= \left( q_1(t)+\overbrace{c_1}^{c_1 \in \mathbb R \text{ αυθαίρετη σταθερά}}, q_2(t)+\overbrace{c_2}^{c_2 \in \mathbb R \text{ αυθαίρετη σταθερά}}, \dots, q_n(t)+\overbrace{c_n}^{c_n \in \mathbb R} \right) \\
&= \left( q_1(t), q_2(t), \dots, q_n(t) \right) + (c_1,\dots,c_n) = 
\mathbf q(t) + \underbrace{\mathbf c}_{\text{αυθαίρετο διάνυσμα}} \in \mathbb R^n
\end{align*}

\subsection{Συμπέρασμα}
\begin{attnbox}{}
Η καμπυλότητα και η στρέψη είναι εκτός ύλης
\end{attnbox}

\subsection{Ασκήσεις}
\paragraph{Άσκ. 1}
Αν \(\gamma:[a,b] \to  \mathbb R ^n: \mathbf r^n (t)
= \left( f_1(t),\dots,f_n(t) \right) \) είναι ΟΜΑΛΗ καμπύλη, ΝΔΟ:
\[
\mathbf r_\gamma (t) \cdot \mathbf r' _\gamma(t) = |\mathbf r _\gamma (t)| \cdot
\left( |\mathbf r_\gamma(t)| \right)'
\]
\begin{proof}
\begin{align*}
 \mathbf r_\gamma(t) \cdot   \mathbf r'_\gamma(t)
 &= \left( f_1(t),\dots ,f_n(t) \right) \cdot \left( f_1'(t),\dots,f'_n(t) \right) \\
&= f_1(t)f'_1(t)+f_2(t)f'2(t)+\dots+f_n(t)f'_n(t)
\end{align*}
\begin{align*}
 |\mathbf r _\gamma (t)| \cdot \left( |\mathbf r_\gamma(t)| \right)'
 &= \sqrt{f_1^2(t) + \dots + f_n^2(t)} + \left( \sqrt{f_1^2(t) + \dots + f_n^2(t)} \right)' \\
 &= \sqrt{f_1^2(t) + \dots + f_n^2(t)} + \frac{1}{2} \frac{2f_1(t)f'1(t)+\dots+2f_n(t)=f'_n(t)}{\sqrt{f_1^2(t) + \dots + f_n^2(t)}} \\
 &= f_1(t)f'_1(t)+f_2(t)f'2(t)+\dots+f_n(t)f'_n(t)
\end{align*}
\end{proof}

\paragraph{Άσκ. 2}
Αν \( \mathbf r_\gamma(t) = t \mathbf i+2 \mathbf j + (t^2-3) \boldsymbol{\kappa}\), υπολογίστε την εξίσωση της εφαπτόμενης ευθείας της καμπύλης στο σημείο \((2,2,1)\).
\begin{proof}[Απάντηση]
\[
\begin{cases}
\mathbf i &= (1,0,0) \\
\mathbf j &= (0,1,0) \\
\boldsymbol{\kappa} &= (0,0,1) \\
\end{cases}
\]
\[ \mathbf r_\gamma(t) = \left( t,2,t^2-3 \right) \]
\begin{align*}
\mathbf r_\text{εφ}(t) &= 
\left( \text{σημείο ευθείας γνωστό} \right) +
t \left( \text{διάνυσμα παράλληλο στην ευθεία} \right), \quad & t \in  \mathbb R 
\end{align*}
\begin{itemize}
\item Γνωστό σημείο το \(M=(2,2,1)\). Έτσι
\((2,2,1) = (t,2,t^2-3) \implies
\begin{cases}
t&=2\\2&=2\\t^2-3&=1
\end{cases}
\implies t=2
\)
\item Διάνυσμα γνωστό παράλληλο στην εφαπτομένη στο \((2,2,1)\) είναι το
\( \mathbf r'_\gamma(2)\).
\begin{align*}
 \mathbf r'_\gamma(t) &= (1,0,2t) \\
 \mathbf r'_\gamma(2) &= (1,0,4)
\end{align*}
και τελικά:
\begin{align*}
\boxed{
\mathbf r_\text{εφ}(t) = (2,2,1) + t(1,0,4), \quad t \in \mathbb R
}\\
\text{διανυσματική εξίσωση εφαπτόμενης ευθείας}
\end{align*}
\end{itemize}

\end{proof}

\paragraph{Ασκ. 3}[Απάντηση]
Αν \(\mathbf v(t) = \left( 2 \cos t, -t \sin (t^2), 2t \right) \) είναι η ταχύτητα του υλικού σημείου, βρείτε την εξίσωση κίνησης \(\mathbf r_\gamma \), αν \( \mathbf r_\gamma (0) = \mathbf i + 3 \mathbf k\)
\begin{proof}
Είναι γνωστό ότι \(\mathbf v(t) = \mathbf r'_\gamma(t) \implies \int \mathbf r'_\gamma(t) \dif t = \int \mathbf v(t) \dif t \implies \mathbf r_\gamma(t) = \int \mathbf v(t) \dif t = \left( \int 2 \cos t \dif t, \int - t \sin (t^2) \dif t, \int 2t \dif t \right) = \left( 2 \sin t+ c_1, \frac{\cos (t^2)}{2} + c_2, t^2+c_3 \right)\), όπου \(c_1,c_2,c_3 \in \mathbb R \) αυθαίρετες σταθερές (ΑΝΕΞΑΡΤΗΤΕΣ μεταξύ τους).

\[
(1,0,3) \overset{\text{υποθ.}}{=}  \mathbf r_\gamma(0) = 
\left( c_1, \frac{1}{2} + c_2, c_3 \right) \implies
\begin{cases}
c_1&=1 \\
c_2+\frac{1}{2}&=0 \\
c_3&=3
\end{cases}
\implies
\begin{cases}
c_1&=1 \\
c_2&=-\frac{1}{2}=0 \\
c_3&=3
\end{cases}
\]

Τελικά:
\[
\mathbf r_\gamma(t) = 
\left(
2 \sin t +1, \frac{\cos (t^2)-1}{2}, t^2+3
\right)
\]

\end{proof}


\paragraph{Άσκ. 4}
Να παραμετροποιηθούν οι καμπύλες:
\begin{enumerate}
\item \( \left\lbrace \underbrace{x^2+z^2=a^2}_{\text{άπειρος κύλινδρος}}, \ \underbrace{2x+3y+7=1}_{\text{επίπεδο}}  \right\rbrace\)
\item \( \left\lbrace \underbrace{x^2+y^2+z^2=a^2}_{\text{σφαίρα}}, \ \underbrace{2x+3y+z =1}_\text{επίπεδο} \right\rbrace \)
\end{enumerate}

\begin{proof}[Απάντηση]

\begin{enumerate}
\item Η καμπύλη προκύπτει ως τομή "άπειρου" κυλίνδρου \(x^2+z^2=a^2\) και επιπέδου \(2x+3y+z=1\).

\paragraph{Ενδεικτικό σχήμα}
Έστω π.χ. ότι αυτή η καμπύλη είναι η τομή κυλίνδρου και επιπέδου.\\
\begin{center}
		\begin{tikzpicture}[scale=1.5]
		
		\draw[-{>[scale=2.5,width=3]}] (-2,1) node[above] {$x^2+z^2=a^2$} to [bend right=45] (-0.9,0);
		
		\draw[->,black!60] (0,0) -- (0,1.5) node[above right] {$y$};
		\draw[->,black!60] (0,0) -- (1.5,0) node[above right] {$x$};
		\draw[->,black!60] (0,0) -- (-1,-1) node[below right] {$z$};
		
		\draw (0,0) circle(0.7 and 0.3);
		\draw[dashed] (0,1) circle(0.7 and 0.3);
		\draw[dashed] (0,-1) circle(0.7 and 0.3);
		
		\draw (0.7,-1.5) -- (0.7,1.5);
		\draw (-0.7,-1.5) -- (-0.7,1.5);
		\end{tikzpicture}
\end{center}

\begin{infobox}{Παραμετροποίηση καμπύλης}
Γενικά:
\[
\mathbf r_\gamma(t) = \left( x(t), y(t), z(t) \right), \quad t \in [a,b]
\]
\end{infobox}
\[
 \underbrace{\mathbf r_\text{προβολής}(t)}_{\mathclap{\text{πάντοτε είναι ο κύκλος } 
  \left\lbrace x^2+z^2=a^2,y=0  \right\rbrace}} = \left( x(t), 0 , z(t) \right)
\]
Αλλά ο κύκλος \( \left\lbrace  x^2+z^2=a^2,y=0  \right\rbrace \) μπορεί να παραμετροποιηθεί ως εξής:
\[\mathbf r_\text{προβολής}(t) = \left( a \cos t, 0 , a \sin t \right), \quad t \in [0, 2\pi)\]
Έτσι
\[
 \mathbf r_\gamma(t) = \left( a \cos t, y(t), a \sin t \right), t \in [0, 2\pi)
\]
όπου \(2x+3y+z=1 \implies y = \frac{1-z-2x}{3} = \frac{1-a \sin t -2 a \cos t}{3}\).

Τελικά:
\[
 \mathbf r_\gamma(t) = \left( a \cos t, \frac{1-a \sin t - 2 a \cos t}{3}, a \sin t \right), \quad t \in [0, 2\pi)
\]


\item 
\end{enumerate}

Έχουμε \textbf{ΤΟΜΗ} σφαίρας και επιπέδου που είναι \textbf{ΠΑΝΤΑ κύκλος}.
	\subparagraph{Ενδεικτικό σχήμα}
\begin{center}

		\begin{tikzpicture}[scale=1]
		
		\draw[->,black!60] (0,0) -- (0,1.5) node[above right] {$z$};
		\draw[->,black!60] (0,0) -- (1.5,0) node[above right] {$y$};
		\draw[->,black!60] (0,0) -- (-1,-1) node[below right] {$x$};
		
		\draw (-1,0) arc (180:360:1cm and 0.5cm);
		\draw (-1,0) arc (180:0:1cm and 0.5cm);
		\draw[dashed] (0,1) arc (90:270:0.5cm and 1cm);
		\draw[dashed] (0,1) arc (90:-90:0.5cm and 1cm);
		\draw (0,0) circle (1cm);
		\end{tikzpicture}
\end{center}
Θέτω \(z=t\), οπότε \(x^2+y^2=a^2-t^2\), \(y=2x\), \(x^2=\frac{a^2-t^2}{5}\) και προχωρώ λύνοντας \(2 \times 2\) σύστημα

\emph{ή}

\[
\begin{cases}
x^2+y^2+z^2&=a^2 \\
y&=2x
\end{cases}
\iff
\begin{cases}
x^2+(2x)^2+z^2&=a^2 \\
y&=2x
\end{cases}
\iff
\begin{cases}
5x^2+z^2&=a^2 \\
y&=2x
\end{cases}
\]
Οι παραστάσεις είναι ισοδύναμες, δηλαδή η καμπύλη εκφράζεται ως τομή κυλίνδρου και επιπέδου, και ανάγομαι στο ερώτημα (α). Έτσι:
\[
 \mathbf r_\gamma(t) = \left( x(t),y(t),z(t) \right), t \in [a,b]
\]
\[
 \underbrace{\mathbf r_\text{προβολής}(t)}_{\mathclap{\text{είναι πάντα η έλλειψη } 
  \left\lbrace 5x^2+z^2=a^2,y=0  \right\rbrace}} = \left( x(t), 0 , z(t) \right)
\]

Άρα:
\[
\mathbf r_\text{προβολής}(t) = \left( \frac{a}{\sqrt{5}} \cos t,0,a \sin t \right),\quad t \in [0,2\pi)
\]
και έτσι
\[
 \mathbf r_\gamma(t) =  \left( \frac{a}{\sqrt{5}} \cos t,y(t),a \sin t \right)
 \] με \(y(t)=2x(t)=\frac{2a}{\sqrt 5} \cos t\)
 
Τελικά:
\[
 \mathbf r_\gamma(t) = \left( \frac{a}{\sqrt 5} \cos t, \frac{2a}{\sqrt 5} \cos t, a \sin t \right), \quad t \in [0, 2\pi)
\]
\end{proof}




\section{Διπλά Ολοκληρώματα}
\begin{attnbox}{}
Όποιος δεν κατάλαβε τα διπλά ολοκληρώματα, ας μην πάει παρακάτω.
\end{attnbox}

Έστω \(f: R \subset  \mathbb R ^2 \to  \mathbb R: \ z = f(x,y)\) είναι μια συνάρτηση δύο μεταβλητών ΦΡΑΓΜΕΝΗ επί της \textbf{ορθογώνιας περιοχής}:
\[R =  \left\lbrace (x,y): a \leq x \leq b, \ c \leq y \leq d  \right\rbrace\]

\begin{center}
		\begin{tikzpicture}[scale=2]
		
		\draw[->] (0,0) -- (0,2) node[above right] {$z$};
		\draw[->] (0,0) -- (2.2,0) node[above right] {$y$};
		\draw[->] (0,0) -- (-1.2,-1.2) node[below right] {$x$};
		
		\draw (-0.1,-0.1) -- (0,-0.1) -- (0.1,0);
		\filldraw (0, -0.05) circle(0.05pt);
		
		\draw (0.1,0) -- (0.1,0.1) -- (0,0.1);
		\filldraw (0.05,0.05) circle(0.05pt);
		
		\draw (0,0.1) -- (-0.1,0) -- (-0.1,-0.1);
		\filldraw (-0.05,0) circle(0.05pt);
		
		\foreach \i in {0,...,5} {
			\draw (1+\i*0.15,0) -- ++(45:-2cm);
			\draw (-0.4-\i*0.15/1.5,-0.4-\i*0.15/1.5) -- ++(0:2.2cm);
		}
		
		\draw (1,0) node[above] {$c$};
		\draw (1+5*0.15,0) node[above] {$d$};
		
		\draw(-0.4,-0.4) node[above left] {$a$};
		\draw(-0.4-5*0.15/1.5,-0.4-5*0.15/1.5) node[above left] {$b$};
		
		
		\end{tikzpicture}
\end{center}

\begin{enumparen}
\item
Διαμερίζω το ορθογώνιο \(R\) μέσω διαμέρισης: \[
\Delta = \left( \Delta_x, \Delta_y \right)
\]
όπου \(
\begin{cases}
\Delta_x &=  \left\lbrace a=x_1<x_2< \cdots < x_N = b  \right\rbrace \\
\Delta_y &=   \left\lbrace c=y_1 < y_2 < \cdots < y_N = d  \right\rbrace
\end{cases}
\) σε στοιχειώδη ορθογώνια \(\Omega_{n,\kappa}\) με εβαδόν:
\[ Ε_{n,\kappa } = \left( x_{n+1} - x_n \right) \left( y_{k+1} - y_k \right) \]
\(n=1,\dots,N-1,\ k=1,\dots,M-1\)

\item
Έστω \((\widetilde{x_n}, \widetilde{y_n}) \in \Omega_{n,\kappa} \) είναι ΤΥΧΑΙΟ σημείο του \(\Omega_{n,\kappa}\). Ορίζω τις τιμές \(f(\widetilde{x_n}, \widetilde{y_n})\).

\item
Ορίζω το \textbf{άθροισμα}:
\[
S_{N,M,f} := 
\sum_{n=1}^{N-1} \sum_{k=1}^{M-1} \underbrace{f(\widetilde{x_n}, \widetilde{y_n}) \cdot E_{n_k}}_{\text{παριστάνει τον "όγκο"}}
 \]
 
\item
Έστω \(|\Delta| = \max  \left\lbrace \delta_{n, \kappa}:
\begin{cases}
n=1,\dots,N_1\\
k=1,\dots,M-1
\end{cases}
  \right\rbrace\)
(όπου 
\(\delta_{n,k}= \sqrt{ \left(x_{n+1} - x_n \right) ^2 +  \left(y_{n+1} - y_n \right) ^2 }\)
είναι το μήκος της διαγωνίου του ορθογωνίου \(\Omega_{n,k}\) ) είναι το ΜΕΓΙΣΤΟ ΠΛΑΤΟΣ της διαμέρισης \(\Delta\). Αν
\[
\lim_{|\Delta | \to 0} S_{\Delta, f} = 
\lim_{|\Delta | \to 0} \sum_{n=1}^{N-1} \sum_{k=1}^{M-1} f(\tilde{x_n}, \tilde{y_n}) \cdot E_{n_k}
= \lambda \in  \mathbb R 
\]
\textbf{ανεξάρτητα} της επιλογής της διαμέρισης \(\Delta\) και ανεξάρτητα της επιλογής των σημείων \( ( \widetilde{x_n},\widetilde{y_n}) \in \Omega_{n, \kappa } \). 

Τότε λέμε ότι η \(f\) είναι ολοκληρώσιμη κατά \textlatin{Riemann} στην ορθογώνια περιοχή \(R\) και γράφουμε:

\[
\mathlarger{\mathlarger{
 \iint_R f(x,y) \dif x \dif y = \lambda \in \mathbb R 
}} 
 \]
\end{enumparen}



Προσεγγιστικά,
\[ \iint_R f(x,y) \dif x \dif y \approx \sum \sum f(x_n, y_k) \left( x_{n+1}-x_n \right) \left( y_{n+1}-y_n \right) \]


\subsection{Γενίκευση ορισμού σε μη ορθογώνια χωρία}
Έστω \(f: T \subset \mathbb R^2 \to \mathbb R\) ΦΡΑΓΜΕΝΗ συνάρτηση πάνω σε \textbf{ΦΡΑΓΜΕΝΟ χωρίο \(T\)} με το \textbf{σύνορο αυτού \(\partial T\)} να είναι σύνολο ΑΜΕΛΕΗΤΟΥ ΕΜΒΑΔΟΥ.

Έστω \(T \subset R\), όπου \(R\) είναι οποιοδήποτε \textbf{ορθογώνιο χωρίο} που καλύπτει το \(T\).

Ορίζω την επέκταση της \(f\) στο ορθογώνιο χωρίο \(R\) ως εξής:
\[
g(x,y) = 
\begin{cases}
f(x,y),&(x,y) \in T \\
0,& (x,y) \in R-T
\end{cases}
\]

Αν η \(g: R \subset \mathbb R^2 \to \mathbb R\) είναι ολοκληρώσιμη επί του \textbf{ορθογωνίου} \(R\), τότε λέμε ότι η \(f\) είναι ολοκληρώσιμη επί του \(T\) και:
\[
\iint_T f(x,y) \dif x \dif y =
\iint_R g(x,y) \dif x \dif y
\]

\begin{attnbox}{Θεώρημα 1}
Έστω \(f:T \in  \mathbb R ^2 \to  \mathbb R \) είναι \boxed{\textbf{συνεχής}} συνάρτηση επί ΦΡΑΓΜΕΝΟΥ χωρίου \(T\) (το σύνορο \(\partial T\) του οποίου είναι σύνορο αμελητέου εμβαδού) \textbf{ΕΚΤΟΣ ΕΝΔΕΧΟΜΕΝΩΣ από ένα σύνολο σημείων αμελητέου εμβαδού}. Τότε η \(f\) είναι ολοκληρώσιμη επί του \(T\).

\paragraph{ΣΗΜΕΙΩΣΗ} Σύνολα αμελητέου εμβαδού:
\begin{enumparen}
\item Το πολύ αριθμήσιμο πλήθος σημείων
\item Τμηματικά λείες καμπύλες πεπερασμένου μήκος (ή το πολύ αριθμήσιμη ένωση τέτοιων)
\end{enumparen}


\end{attnbox}

\begin{infobox}{Αριθμήσιμο σύνολο \(A\)}
Υπάρχει 1-1 αντιστοιχία του συνόλου φυσικών \(\mathbb N\) με το \(A\).
\[
\mathbb N =  \left\lbrace 1,2,3\dots \right\rbrace
\]

\textit{π.χ}
\begin{itemize}
\item Το \([a,b] \subset \mathbb R\) είναι υπεραριθμήσιμο
\item Το \(\mathbb Z\) είναι αριθμήσιμο
\item Το \(\mathbb Q\) είναι αριθμήσιμο
\end{itemize}
\end{infobox}

\textit{π.χ.}
Έστω \( T =  \left\lbrace (x,y): x^2+y^2 \leq 1  \right\rbrace\) (μοναδιαίος κυκλικός δίσκος)

\begin{itemize}
\item \(f:T \to \mathbb{R}: f(x,y)=x^2+x^2y\)

συνεχής στο \(T\), άρα ολοκληρώσιμη
\item \(f:T\to \mathbb R : f(x,y) = \begin{cases}1,&x=y\\3,&T- \left\lbrace (x,x):|x| \leq 1 \right\rbrace \end{cases}\) %TODO Atreas graph 01

ολοκληρώσιμη διότι είναι συνεχής στο δίσκο \(T\) εκτός από ένα σύνολο σημείων αμελητέου εμβαδού (το σύνολο \( \left\lbrace (x,x): |x|<1 \right\rbrace\))

\item \(f(x,y) = \begin{cases}1 \ & (x,y): x,y \text{ ρητός}
\\
0& (x,y): x,y \text{ άρρητος}
\end{cases},\ (x,y) \in [0,1]^2 = [0,1] \times [0,1]
\)

δεν είναι ολοκληρώσιμη
\end{itemize}

\subsection{Ιδιότητες}

Ισχύουν οι γνωστές ιδιότητες του ολοκληρώματος.

Έστω \(f,g: T \in  \mathbb R ^2 \to \mathbb R \) ολοκληρώσιμες σε φραγμένο χωρίο \(T\) (με σύνορο αμελητέου εμβαδού). Τότε:
\begin{enumparen}
\item \(af \pm bg\) ολοκλ. επί του \(T\) και
\[
\iint_T \big( af \pm bg \big) (x,y) \dif x \dif y
=
a\iint_T f(x,y) \dif x \dif y
\pm
b\iint_T g(x,y) \dif x \dif y
\]
\item \(f\cdot g, \ \frac{f}{g} (g \neq 0),\ |f|\) ολοκλ. επί του \(T\) και
\[
\left|
\iint_T f(x,y) \dif x \dif y
\right|
\leq
\iint_T \left| f(x,y) \right| \dif x \dif y
\]
\item Αν \(f(x,y) \leq g(x,y) \  \forall (x,y)\), τότε:
\[
\iint_T f(x,y) \dif x \dif y
\leq
\iint_T g(x,y) \dif x \dif y
\]
\item Αν \(T\) χωρίο αμελητέου εμβαδού, τότε:
\[
\iint_T f(x,y) \dif x \dif y = 0
\]
\item Αν \(T = T_1 \cup T_2\) και \(T_1 \cap T_2 = \emptyset\) (ή σύνολο αμελητέου εμβαδού), τότε:
\[
\iint_T f(x,y) \dif x \dif y = 
\iint_{T_1} f(x,y) +
\iint_{T_2} f(x,y)
\]
\item Αν \(m \leq f(x,y) \leq M\), τότε:
\[
mE(T) \leq \iint_T f(x,y) \dif x \dif y \leq M E(T)
\]
όπου \(E(T) = \) εμβαδόν χωρίου \(T\)
\item
\textbf{Θεώρημα μέσης τιμής:}
Αν \(m \leq f(x,y) \leq M\) και \(g(x,y) \geq 0 \ \forall (x,y) \in T\), τότε υπάρχει \(\mu \in [m,M]\):
\[
\iint_T f(x,y) g(x,y) \dif x \dif y =
\mu \iint_T g(x,y) \dif x \dif y
\]

Αν επιπλέον \(f\) \textbf{συνεχής} επί του \(T\) και το \(T\) \textbf{συνεκτικό}, υπάρχει \((x_0,y_0) \in T\):
\[
\iint_T f(x,y) \dif x \dif y = f(x_0,y_0) \iint_T g(x,y) \dif x \dif y
\]
\end{enumparen}

\subsection{Υπολογισμός (πρακτικός) Διπλών Ολοκληρωμάτων}
\subsubsection{Σε Ορθογώνια Χωρία}
\begin{attnbox}{Θεώρημα 2 (\textlatin{Fubini})}
Έστω \(f: R \subset  \mathbb R ^2 \to  \mathbb R \) συνεχής επί ορθογωνίου χωρίου \(R =  \left\lbrace (x,y): a \leq x \leq b, \ c \leq y \leq d  \right\rbrace\).
Τότε και οι συναρτήσεις \(g(x) = \int_c^d f(x,y) \dif y\) και \(h(y) = \int_a^bf(x,y) \dif x\) είναι συνεχείς επί των \([a,b]\) και \([c,d]\) αντίστοιχα, και:
\[
\iint_R f(x,y) \dif x \dif y =
\int_a^b g(x) \dif x =
\int_c^d h(y) \dif y
\]

Με άλλα λόγια:
\begin{equation} \label{eq:fubini}
\boxed{
\iint_R f(x,y) \dif x \dif y
=
\int_a^b \left(
\int_c^d f(x,y) \dif y
\right) \dif x =
\int_c^d \left(
\int_a^b f(x,y) \dif x
\right) \dif y
}
\end{equation}

\paragraph{Σημείωση}
\begin{enumparen}
\item Ξεκινάω να ολοκληρώνω ως προς όποια μεταβλητή θέλω, ΠΑΝΤΑ ΑΠΟ ΜΕΣΑ ΠΡΟΣ ΤΑ ΕΞΩ.
Κάθε φορά ολοκληρώνω ως προς μία μεταβλητή, ΚΡΑΤΩΝΤΑΣ τις υπόλοιπες ΣΤΑΘΕΡΕΣ.
\end{enumparen}
\end{attnbox}

\paragraph{\textit{π.χ.}}

Ποιο το διπλό ολοκήρωμα της:
\[
f(x,y)=x^2+y^2
\]
επί του χωρίου \(T =  \left\lbrace (x,y): 0 \leq x \leq 1, \ 2 \leq y \leq 3  \right\rbrace\)?

\[
I = 
\underbrace{
\int_2^3
\int_0^1
}_{\mathclap{\text{εξειδίκευση ορίων}}}
(x^2+y^2) \underbrace{\dif x \dif y}_{\mathclap{\text{τυχαία επιλογή}}}
=
\int_2^3 \left. \frac{x^3}{3}+y^2x \right|^1_0 \dif y
=
\int_2^3 \left( \frac{1}{3}+y^2 \right) \dif y
= \frac{1}{3}y + \left. \frac{y^3}{3} \right|_2^3 = \dots
\]

\subsubsection{Σε μη ορθογώνια χωρία}
Έστω \(T \subset  \mathbb R ^2\) ΦΡΑΓΜΕΝΟ χωρίο (με σύνορο αμελητέου εμβαδού). Τότε,
\begin{enumlatin}
\item Το \(T\) καλείται \textbf{κανονικό ως προς \(\mathbf y\)}, αν \(T\) είναι \textbf{συνεκτικό} και \boxed{\text{κάθε}} ευθεία \(\parallel \mathbf{y'y}\) \textbf{ΕΝΤΟΣ του \(\mathbf T\)} τέμνει το σύνορο του \(T\) \textbf{ακριβώς σε ΔΥΟ ΣΗΜΕΙΑ}.

\begin{figure}
	\centering
	\begin{subfigure}[b]{0.35\textwidth}
		\begin{tikzpicture}
		\draw[->] (-1,-0.75) -- (-1,2);
		\draw[->] (-1.25,-0.5) -- (4,-0.5);
		
		\begin{scope}[scale=0.5]
		\filldraw[thick,fill=green!20] plot [smooth cycle,tension=1] coordinates {
			(0,0) (0.5,2) (3,0.3) (5.5,2) (6,0) (3,-0.3)
		};
		\draw[thick] (0.5,2) -- (0.5,-0.3);
		\draw[thick] (3,0.3) -- (3,-0.3);
		\draw[thick] (6,0) -- (6,1.8);
		\draw[thick] (4,0.7) -- (4,-0.3);
		
		
		\foreach \p in {
			(0.5,2),  (0.5,-0.3),
			(3,0.3), (3,-0.3), (6,0),(6,1.8), (4,0.7), (4,-0.3)
		} {
		\filldraw \p circle(2pt);
		
		\draw (2,2) node {$T$};
	}
	
	\end{scope}
	\end{tikzpicture}
		\caption{Κανονικό ως προς \(y\)}
	\end{subfigure}
	\quad %add desired spacing between images, e. g. ~, \quad, \qquad, \hfill etc. 
	%(or a blank line to force the subfigure onto a new line)
	\begin{subfigure}[b]{0.31\textwidth}
				\begin{tikzpicture}[scale=2]
				\draw[->] (0,-0.25) -- (0,2);
				\draw[->] (-0.25,0) -- (2,0);
				
				\begin{scope}[shift={(-0.5,-0.7)},scale=1.4]
				\filldraw[fill=green!20] plot [smooth cycle,tension=1.5] coordinates {
					(1.4, 1) (1,1) (1,1.5) (1.5, 2)   (1.2,1.3)
				};
				\draw (1.7,2) node {$T$};
				
				\draw[thin,dashed] (1,0.5) -- (1,2);
				\draw[thick] (1,1) -- (1,1.5);
				
				\draw[thin,dashed] (1.3,0.5) -- (1.3,2.2);
				\draw[thick,red] (1.3,0.9) -- (1.3,1.05);
				\draw[thick,red] (1.3,1.64) -- (1.3,1.93);
				
				
				\foreach \p in {
					(1,1),(1,1.5)
				} {
				\filldraw \p circle(1pt);
			}
			
			\foreach \p in {
				(1.3,0.9), (1.3,1.05),
				(1.3,1.64), (1.3,1.93)
			} {
			\filldraw[fill=red!80] \p circle(1pt);
		}
		
		\end{scope}
		\end{tikzpicture}
		\caption{Μη κανονικό ως προς \(y\)}
	\end{subfigure}
	\\[0.5em]
	\begin{subfigure}[b]{0.31\textwidth}
		\begin{tikzpicture}[scale=2]
		\draw[->] (0,-0.25) -- (0,2);
		\draw[->] (-0.25,0) -- (2,0);
		
		\begin{scope}[]
		\filldraw[thick,fill=green!20] (1,1) circle(0.75);
		\filldraw[thick,fill=white] (1,1) circle(0.45);
		\draw (1.8,1.8) node {$T$};
		\end{scope}
		\end{tikzpicture}
\caption{Μη κανονικό}
\end{subfigure}
\quad 
\begin{subfigure}[b]{0.31\textwidth}
\begin{tikzpicture}[scale=2]
\draw[->] (0,-0.25) -- (0,2);
\draw[->] (-0.25,0) -- (2,0);

\begin{scope}[shift={(1.2,0.5)}]

\filldraw[fill=green!20] plot [smooth cycle,tension=1.5] coordinates {
	(-0.5,0) (-0.8,0) (0,0.7) (0.8,0) (0.5,0) (0,0.4)
};

\draw[thick] (-1,0) -- (1,0);


\foreach \p in {
	(-0.5,0), (-0.8,0), (0.5,0), (0.8,0)
} {
\fill[black] \p circle(1pt);
}

\end{scope}

\end{tikzpicture}
	\caption{Μη κανονικό ως προς \(x\) \\ Κανονικό ως προς \(y\)}
\end{subfigure}
\\[0.3em]
	\begin{subfigure}[b]{0.33\textwidth}
		\begin{tikzpicture}[scale=2.2]
		\draw[->] (0,-0.25) -- (0,2);
		\draw[->] (-0.25,0) -- (2,0);
		
		\begin{scope}[shift={(1.2,0.8)},scale=0.7]
		
		\filldraw[fill=green!20] plot [smooth cycle,tension=0.7] coordinates {
			(-0.3,0.1)  (-0.8,0.9) 
			%(-0.5,1.4)
			(0.5,1) (0.3,0) (-0.8,-0.8) 
			%(-0.5, -0.4)
		};
		
		\draw[thick] (-0.82,0.8)  -- (0.6,0.8);
		
		\foreach \p in {
			(-0.82,0.8)  ,(0.6,0.8)
		} {
		\fill[black] \p circle(1.5pt);
	}
	
	\end{scope}
	
	\end{tikzpicture}
		\caption{Κανονικό ως προς \(x\) \\ Μη κανονικό ως προς \(y\)}
	\end{subfigure}
	~
	\begin{subfigure}[b]{0.25\textwidth}
		\begin{tikzpicture}[scale=1.5]
		\draw[->] (0,-0.25) -- (0,2);
		\draw[->] (-0.25,0) -- (2,0);
		
		\begin{scope}[shift={(1,1)},scale=0.7]
		
		\filldraw[fill=green!20,thick] (0,0) circle(1);
		
		\foreach \p in {
			30,45,60,75,90,105,120,135,150
		} {
		\draw (\p:1) -- (-\p:1);
	}
	
	\end{scope}
	
	\end{tikzpicture}
		\caption{Κανονικό}
	\end{subfigure}
	~
	\begin{subfigure}[b]{0.25\textwidth}
		\begin{tikzpicture}[scale=1.5]
		\draw[->] (0,-0.25) -- (0,2);
		\draw[->] (-0.25,0) -- (2,0);
		
		\begin{scope}[shift={(1,1)},scale=0.7]
		
		\filldraw[thick,fill=green!20] plot [smooth cycle,tension=1.2] coordinates {
			(-1,1) (0.5,0.4) (1,-1) (-0.5,-0.5)
		};
		
		\draw (-1.15,0.7) -- (0.2,0.7);
		\draw (-1.05,0.2) -- (0.7,0.2);
		
		\draw(0.5,0.4) -- (0.5,-1.15);
		\draw (-0.3,1) -- (-0.3,-0.7);
		
		
		\end{scope}
		\end{tikzpicture}
	\caption{Κανονικό}
\end{subfigure}
\end{figure}

\item Το \(T\) καλείται \textbf{κανονικό ως προς \(\mathbf x\)}, αν \(T\) είναι \textbf{συνεκτικό} και \boxed{\text{κάθε}} ευθεία \(\parallel \mathbf{x'x}\) \textbf{ΕΝΤΟΣ του \(\mathbf T\)} τέμνει το σύνορο του \(T\) \textbf{ακριβώς σε ΔΥΟ ΣΗΜΕΙΑ}.
\item Το \(T\) καλείται κανονικό αν είναι κανονικό ως προς \(x\) \boxed{ΚΑΙ} ως προς \(y\).
\end{enumlatin}

\begin{itemize}
\item Ένα χωρίο:
\begin{itemize}
\item μπορεί να ΜΗΝ είναι ΚΑΝΟΝΙΚΟ
\item μπορεί να είναι κανονικό ως προς \(y\) αλλά όχι ως προς \(x\), και αντιστρόφως.
\end{itemize}
\item Αν \(T\) μη κανονικό, το σπάω σε ΕΝΩΣΗ κανονικών (ως προς \(x\) ή $y$) χωρίων.
\end{itemize}

\begin{attnbox}{Θεώρημα 3 (\textlatin{Fubini})}
Έστω \(f: T \subset  \mathbb R ^2 \to  \mathbb R \) συνεχής επί φραγμένου χωρίου \(T\) (με σύνορο αμελητέου εμβαδού).
\begin{enumerate}
\item Αν \(T\) είναι κανονικό ως προς \(y\) χωρίο της μορφής:
\[
T =  \left\lbrace (x,y):\ a \leq x \leq b, \ g(x) \leq y \leq h(x)  \right\rbrace
\]
όπου \(g,h: [a,b] \to  \mathbb R \) συνεχείς συναρτήσεις, τότε:
\[
\iint_T f(x,y) \dif x \dif y =
\int_a^b \left(
\int_{g(x)}^{h(x)} f(x,y) \dif y
\right) \dif x
\]
\item Αν \(T\) είναι κανονικό ως προς \(x\) χωρίο της μορφής:
\[
T =  \left\lbrace (x,y):\ c \leq y \leq d, \ \kappa (x) \leq x \leq \lambda(x)  \right\rbrace
\]
όπου \(\kappa,\lambda: [a,b] \to  \mathbb R \) συνεχείς συναρτήσεις, τότε:
\[
\iint_T f(x,y) \dif x \dif y =
\int_c^d \left(
\int_{\kappa(x)}^{\lambda(x)} f(x,y) \dif x
\right) \dif y
\]
\item Αν \(T\) κανονικό, τότε:
\begin{align*}
\iint_T f(x,y)\dif x \dif y
&=
\int_a^b \left(
\int_{g(x)}^{h(x)} f(x,y) \dif y
\right) \dif x \\
&=
\int_c^d \left(
\int_{\kappa(x)}^{\lambda(x)} f(x,y) \dif x
\right) \dif y
\end{align*}
\end{enumerate}
\end{attnbox}

\begin{figure}[h]
	\centering
\begin{subfigure}[b]{0.45\textwidth}
	\begin{tikzpicture}[scale=3]
	\draw[->] (0,-0.25) -- (0,2) node[right] {$y$};
	\draw[->] (-0.25,0) -- (2,0) node[below right] {$x$};
	
	\begin{scope}[shift={(0.5,0.5)},scale=0.7]
	
	
	\fill[fill=green!20] plot[smooth cycle,tension=0.1] coordinates {
		(0,0) (0,1) (0.3,1.3) (0.6,1) (0.9,1.3) (1.2,1) (1.2,0) (0.9,-0.3) (0.6,0) (0.3,0.3)
	};
	
	\draw[dashed] (0,-0.5/0.7) node[below] {$a$} -- (0,2);
	\draw[dashed] (1.2,-0.5/0.7) node[below] {$b$} -- (1.2,2);
	
	\draw[thick] plot [smooth] coordinates {
		(0,1) (0.3,1.3) (0.6,1) (0.9,1.3) (1.2,1)
	};
	%\draw[thick] plot coordinates {(0,0) (0,1)};
	%\draw[thick] plot coordinates {(1.2,0) (1.2,1)};
	\draw[thick] plot [smooth] coordinates {
		(0,0) (0.3,0.3) (0.6,0) (0.9,-0.3) (1.2,0)
	};
	
	\draw (0.6,1.3) node [above] {$h(x)$};
	\draw (0.6,-0.2) node [below] {$g(x)$};
	
	\end{scope}
	\end{tikzpicture}
\end{subfigure}
~ \begin{subfigure}[b]{0.45\textwidth}
	\begin{tikzpicture}[scale=3]
	\draw[->] (0,-0.25) -- (0,1.8) node[right] {$y$};
	\draw[->] (-0.25,0) -- (2,0) node[below right] {$x$};
	
	\begin{scope}[shift={(0.5,1.3)},rotate=-90,scale=0.7]
	
	
	\fill[fill=green!20] plot[smooth cycle,tension=0.1] coordinates {
		(0,0) (0,1) (0.3,1.3) (0.6,1) (0.9,1.3) (1.2,1) (1.2,0) (0.9,-0.3) (0.6,0) (0.3,0.3)
	};
	
	\draw[dashed] (0,-0.5/0.7) node[left] {$d$} -- (0,2);
	\draw[dashed] (1.2,-0.5/0.7) node[left] {$c$} -- (1.2,2);
	
	\draw[thick] plot [smooth] coordinates {
		(0,1) (0.3,1.3) (0.6,1) (0.9,1.3) (1.2,1)
	};
	%\draw[thick] plot coordinates {(0,0) (0,1)};
	%\draw[thick] plot coordinates {(1.2,0) (1.2,1)};
	\draw[thick] plot [smooth] coordinates {
		(0,0) (0.3,0.3) (0.6,0) (0.9,-0.3) (1.2,0)
	};
	
	\draw (0.6,1.3) node [right] {$\lambda (x)$};
	\draw (0.6,-0.1) node [left] {$\kappa (x)$};
	
	
	\end{scope}
	
	
	\end{tikzpicture}
\end{subfigure}
\end{figure}


\subsection{Εφαρμογές διπλού ολοκληρώματος}
\begin{enumparen}

\item
\textbf{ΟΓΚΟΣ}:
Έστω \(
\boxed{f=f(x,y) \geq 0}
\ \forall(x,y) \in T
\),
όπου \(T \in  \mathbb R ^2\) φραγμένο σύνολο με σύνορο αμελητέου εμβαδού. Τότε, αν \(f\) ολοκληρώσιμη επί του \(T\), έχουμε:
\[
V = \iint_T f(x,y) \dif x \dif y
\]
όπου \(V\) είναι ο όγκος του \boxed{\text{στερεού}} μεταξύ της επιφάνειας \(z=f(x,y)\) του χωρίου \(T\) και της κυλινδρικής επιφάνειας με \textbf{οδηγό καμπύλη} την καμπύλη του συνόρου του χωρίου \(T\) και οι γενέτειρες \( \parallel zz'\).

\item
Έστω \(\rho = \rho (x,y)\) είναι συνεχής πυκνότητα μάζας/φορτίου επί φραγμένου χωρίου \(T\) με σύνορο αμελητέου εμβαδού.

Τότε \(\iint_T \rho(x,y) \dif x \dif \phi = \) Συνολική μάζα/φορτίο επί του επιπέδου χωρίου \(T\).

\[
\iint_T \rho(x,y) \dif x \dif y  \approx
\sum_{k=1}^\mu
\sum_{n=1}^M \rho (x_k,y_k) \cdot (x_{k+1}-x_k)(\phi_{k+1}-y_k)
\]

\item
Αν \(f(x,y)=1)\ \forall (x,y) \in T\), όπου \(T\) φραγμένο χωρίο με σύνορο αμελητέου εμβαδού, τότε:
\[
\iint_T 1 \dif x \dif y = \text{Συνολικό εμβαδό του χωρίου }T
\]
\end{enumparen}

\subsection{Αλλαγή μεταβλητής στα διπλά ολοκληρώματα}

\begin{attnbox}{Θεώρημα 4}
Έστω \(\mathbb F:D \subset \mathbb R^2 \to G \subset \mathbb R^2:\)
\[
\mathbb F(u,v) = \left( x(u,v),\ y(u,v) \right)
\]
είναι διανυσματικό πεδίο, παραγωγίσιμο και αντιστρέψιμο, δηλαδή:
\[
\frac{D(x,y)}{D(u,v)}
=
\left|
\begin{matrix}
x_u & x_v \\
y_u & y_v
\end{matrix}
\right| \neq 0\ \forall (u,v) \in D
\]

Αν \(f:G \to  \mathbb R \ f=f(x,y)\) συνεχής επί του \(G\), τότε:
\[
\iint_G f(x,y) \dif x \dif y
=
\iint_D f \left( x(u,v),\ y(u,v) \right)
\left|
\frac{ D(x,y)}{ D(u,v)}
\right|
\dif u \dif v
\]


\end{attnbox}

\paragraph{Εφαρμογή σε πολικές συντεταγμένες}
\begin{wrapfigure}{r}{0.3\textwidth}
	\begin{tikzpicture}[scale=2]
	\draw[->] (0,-0.25) -- (0,1.5); %node[right] {$y$};
	\draw[->] (-0.25,0) -- (1.5,0); %node[below right] {$x$};
	
	\filldraw[black] (1,1) circle (0.5pt) node[above right] {$M(x,y)$};
	\draw[->] (0,0) -- (1,1) node[midway,above left] {$\mathbf R$};
	\draw [->](0,0) +(0:.2cm) arc (0:45:.2cm) node[midway,right] {$\theta$};
	
	\draw[dashed] (1,1) -- (1,0) node[below] {$x$};
	\draw[dashed] (1,1) -- (0,1) node[left] {$y$};
	
	\end{tikzpicture}
\end{wrapfigure}
\leavevmode



\begin{gather*}
(x,y) \leftrightarrow (\rho, \theta) \\
x = \rho \cos \theta \\
y = \rho \sin \theta
\end{gather*}
\subparagraph{Τότε}
\[
\frac{D(x,y)}{D(\rho,\theta)}
\xlongequal[y=\rho \sin \theta]{x=\rho \cos \theta}
\left|
\begin{matrix}
x_\rho & x_\theta \\
y_\rho & y_\theta
\end{matrix}
\right| = \left|
\begin{matrix}
\cos\theta & -\rho\sin\theta \\
\sin\theta & \rho\sin\theta
\end{matrix}
\right|
= \rho
\]
\subparagraph{Δηλαδή}
\[
\iint_G f(x,y) \dif x \dif y
=
\iint_D f(\rho\cos\theta, \rho\sin\theta) \rho \dif \rho \dif \theta
\]

Μη γνήσια ολοκληρώματα εκτός ύλης.

\subsection{Ασκήσεις}
\paragraph{Άσκηση 1}

Υπολογίστε το:
\[
\iint_T (x^2y + x \cos y) \dif x \dif y
\]
επί του χωρίου:
\[
T =  \left\lbrace (x,y):0 \leq x \leq 1, \ -\pi \leq y \leq \pi  \right\rbrace
\]

\begin{align*}
I&=
\overbrace{\int_{-\pi}^\pi
\overbrace{\int_0^1
(x^2y+x\cos y)}^{\text{τα }x\text{ με τα }\dif x}
}^{\text{τα }y\text{ με τα }\dif y}
\underbrace{\dif x \dif y}_{\mathclap{\text{το επέλεξα τυχαία}}}
\\ \text{(διαδοχική ολοκλήρωση ΠΑΝΤΑ από μέσα προς τα έξω)} &=
\int_{\pi}^\pi
\left.
\frac{x^3}{3}y
+ \frac{x^2}{2} \cos y
\right|_0^1
\dif y
\\ &=
\int_{\pi}^\pi
\left(
\frac{y}{3} + \frac{\cos y}{2}
\right)
\dif y
\\ &=
\left.
\frac{y^2}{6} +
\frac{\sin y}{2}
\right|_{-\pi}^\pi
\\ &= 0
\end{align*}


\paragraph{Άσκηση 2}
Υπολογίστε το \(\iint_T (x^2+y^2) \dif x\dif y\) επί του κλειστού και φραγμένου χωρίου μεταξύ των καμπύλων \(y=x^2\) και \(x=y^2\).

\subparagraph{(α) Υποτυπώδες σχήμα}
%TODO Atreas Graph 01

\subparagraph{Σημεία τομής}
\[
\begin{cases}
y&=x^2\\
x&=y^2
\end{cases}
\implies
\begin{cases}
y&=y^4\\
x&=y^2
\end{cases}
\implies
\begin{cases}
y(1-y^3)&=0\\
x&=y^4
\end{cases}
\implies
\begin{cases}
y&=0\\
x&=0
\end{cases}
\text{ή}
\begin{cases}
y&=1\\
x&=1
\end{cases}
\]

\subparagraph{(β)}
Το χωρίο \(T\) είναι κανονικό ως προς $y$. Πράγματι, τυχαία ευθεία \(\epsilon \parallel y'y\) \textbf{εισέρχεται στο \(\mathbf T\) μέσω της \boxed{\text{ΣΥΝΑΡΤΗΣΗΣ \(\mathbf{y=x^2}\)}} και εξέρχεται από το \(\mathbf T\) μέσω της \boxed{\text{ΣΥΝΑΡΤΗΣΗΣ \(\mathbf{y=\sqrt{x}}\)}}}.

Τότε το \(T\) γράφεται ως:
\[
T =  \left\lbrace 
(x,y):
0 \leq x \leq 1,\
x^2 \leq y \leq \sqrt{x}
 \right\rbrace
\]

\subparagraph{Έτσι:}
\begin{align*}
I &=\iint_T (x^2+y^2) \dif x \dif y
\\ &=
\int_0^1
\int_{x^2}^{\sqrt{x}}
(x^2+y^2)
\underbrace{
\dif y \dif x
}_{\mathclap{\text{αναγκαστικά διότι έχω θεωρήσει ότι το χωρίο κανονικό ως προς $y$}}}
\\ &=
\int_0^1
\left. x^2y + \frac{y^3}{3} \right|_{x^2}^{\sqrt{x}}
\dif x
\\ &=
\int_0^1
\left( x^\frac{5}{2} 
+ \frac{x^\frac{3}{2}}{3}-x^4 -\frac{x^6}{3} \right)
\dif x
\\ &=
\left.
\frac{2}{7}x^{\frac{7}{2}}
+ \frac{2}{15} x^\frac{5}{2}
- \frac{x^5}{5}
- \frac{x^7}{21}
\right|_0^1
\\ &=
\frac{2}{7} + \frac{2}{15}
- \frac{1}{5}
- \frac{1}{21}
\end{align*}

\paragraph{Άσκηση 3}
Υπολογίστε τον όγκο του στερεού μεταξύ της επιφάνειας \(z = f(x,y)=1+xy\), και των επιπέδων \(y=0,\ z=0,\ x+y=1,\ y=x \quad (x,y \geq 0)\).

Γνωρίζω ότι
\[
V = \iint_{\underbrace{\boxed{T}}_?} \underbrace{f(x,y)}_{?} \dif x\dif y
\]

\subparagraph{(Α) Υποτυπώδες σχήμα}
\mbox{}\\
	\begin{figure}[h!]
		\begin{center}
		\begin{tikzpicture}
		\begin{axis}[
		axis lines = middle,
		%hide axis,
		colormap/cool,
		zticklabels={,,},
		yticklabels={,,},
		xticklabels={,,},
		]
		\addplot3[black,variable=t,mesh,domain=-5:5] (t,t, 0);
		\addplot3[black,variable=t,mesh,domain=-5:5] (t,1-t, 0);
		\addplot3[surf,shader=faceted,samples=20] (0,-abs x,abs y);
		\addplot3[surf,shader=faceted,samples=20] (abs y,- abs x,0);
		\addplot3[samples=10,domain=0:3,thick,draw=magenta] (x,-x,0);
		\addplot3[draw=blue,thick,samples=10,domain=0:2,fill=magenta] (x,-2+x,0.1);
		\end{axis}
		\draw[->] (6,3) node[right] {στο πρόχειρο} to[bend right=45] (5,2) ;
		\end{tikzpicture}
	\end{center}
	\end{figure}


\subparagraph{(Β) Προς ολοκλήρωση χωρίο}
\[
V=\iint_T (1+xy) \dif x \dif y
\]
Το χωρίο \(T\) είναι κανονικό (δηλ. και ως προς $x$ και ως προς $y$). Θα εργασθώ επιλέγοντας το χωρίο \(T\) να είναι κανονικό ως προς \(x\).

\begin{figure}[h]
	\centering
	\begin{tikzpicture}[scale=2]
	\fill[fill=green!20] plot coordinates { (0,0) (0.5,0.5) (1,0) } node[above right] {$T$};
	
	
	\draw[->] (0,-0.25) -- (0,1.5);
	\draw[->] (-0.25,0) -- (1.5,0);
	
	\draw[dashed] (0.5,0.5) -- (0,0.5) node[left] {$^1/_2$};
	\draw[dashed] (0.5,0.5) -- (0.5,0) node[below] {$^1/_2$};
	
	\draw[thick] (1,0) -- (0,1) node[above,pos=0.8,sloped] {$x+y=1$};
	\draw[thick] (0,0) -- (45:1.8) node[above,pos=0.8,sloped] {$x=y$};
	
	\end{tikzpicture}
\end{figure}


\textbf{Τότε έχω:}
\begin{align*}
V &= \overbrace{\int_0^\frac{1}{2}}^{\mathclap{\text{προβολή του χωρίου $T$ στον άξονα $y'y$}}} \underbrace{\int_{x_\mathrm{min}}^{x_\mathrm{max}} (1+xy) \dif x}_{\mathclap{\text{κανονικό ως προς $x$,\quad μεταβλητά όρια στο εσωτερικό ολοκλήρωμα}}} \dif y \\
&= \int_0^\frac{1}{2} \int_y^{1-y} (1+xy)\dif x \dif y \\
&= \int_0^\frac{1}{2} x  + \left. \frac{x^2}{2} y \right|_y^{1-y} \dif y \\
&= \int_0^\frac{1}{2}(1-y) + \frac{y}{2}(1-y)^2-y-\frac{y^3}{2} \dif y \\
&= \int_0^\frac{1}{2} \left(
1-y+\frac{y}{2}-y^2+\frac{y^3}{2}-y-\frac{y^3}{2}
 \right) \dif y
 \\ &=
 \int_0^\frac{1}{2} \left(
 1-\frac{3}{2}y-y^2
 \right) \dif y
 \\ &= \left.
 y-\frac{3}{4}y^2-\frac{y^3}{3} \right|_0^\frac{1}{2}
 \\ &= \frac{1}{2}-\frac{3}{16}-\frac{1}{24}
 \\ &= \frac{24-9-2}{18} = \frac{13}{48}
\end{align*}

\begin{center}
	\begin{tikzpicture}[scale=3]
	
	\draw[->] (0,-0.25) -- (0,1);
	\draw[->] (-0.25,0) -- (1,0);
	
	\draw[dashed] (0.5,0.5) -- (0,0.5) node[left] {$^1/_2$};
	\draw[dashed] (0.5,0.5) -- (0.5,0) node[below] {$^1/_2$};
	
	\draw[thick] (1,0) -- (0.5,0.5) node[above,pos=0.5,sloped] {$x+y=1$};
	\draw[thick] (0,0) -- (0.5,0.5) node[above,pos=0.5,sloped] {$x=y$};
	
	\draw[very thick] (0.2,0.2) -- (0.8,0.2);
	
	\end{tikzpicture}
\end{center}


\begin{attnbox}{Σημείωση}
Αν θεωρήσουμε το χωρίο $T$ κανονικό ως προς \(y\), τότε θα είχα:

\begin{center}
	\begin{tikzpicture}[scale=2.5]
	
	\draw[->] (0,-0.25) -- (0,1);
	\draw[->] (-0.25,0) -- (1,0);
	
	\draw[dashed] (0.5,0.5) -- (0,0.5) node[left] {$^1/_2$};
	\draw[dashed] (0.5,0.5) -- (0.5,0) node[below] {$^1/_2$};
	
	\draw[thick] (1,0) -- (0.5,0.5) node[above,pos=0.5,sloped] {$x+y=1$};
	\draw[thick] (0,0) -- (0.5,0.5) node[above,pos=0.5,sloped] {$x=y$};
	
	\draw[very thick] (0.2,0) -- (0.2,0.2);
	\draw[very thick] (0.8,0) -- (0.8,0.2);
	
	
	\end{tikzpicture}
\end{center}

\[
V = \int_0^\frac{1}{2} \int_0^x (1+xy) \dif y \dif x
+ \int_\frac{1}{2}^1 \int_0^{1-x} (1+xy) \dif y \dif x
\]
\end{attnbox}

\paragraph{Άσκηση 4}
Υπολογίστε το διπλό ολοκλήρωμα
\[
I = \int_0^3 \int_y^3 e^{x^2} \dif x \dif y
\]

\subparagraph{(Α)}
Σην άσκηση, όπως είναι γραμμένο το προς ολοκλήρωση χωρίο, έχει θεωρηθεί κανονικό \textbf{ως προς $\mathbf x$}.

\subparagraph{(Β) Ποιό είναι το χωρίο? Σχεδίαση (υποτυπώδης)}

\begin{figure}
	\centering
	\begin{subfigure}[b]{0.4\textwidth}
		\begin{tikzpicture}[scale=1]
		\fill[fill=green!20] plot coordinates { (0,0) (3,3) (3,0) } node[xshift=-1cm,yshift=1cm] {$T$};
		\draw[->] (0,-0.25) -- (0,4);
		\draw[->] (-0.25,0) -- (4,0);
		\draw[thick] (3,0) -- (3,3.2);
		\draw[thick] (0,0) -- (3.2,3.2) node[above,pos=0.5,sloped] {$y=x$};
		\filldraw[black] (0,3) circle (1pt) node[left] {3};
		\filldraw[black] (3,0) circle (1pt) node[below] {3};
		\draw[very thick] (2,2) -- (3,2);
		\end{tikzpicture}
	\end{subfigure}
	\begin{subfigure}[b]{0.4\textwidth}
		\begin{tikzpicture}[scale=1]
		\fill[fill=green!20] plot coordinates { (0,0) (3,3) (3,0) } node[xshift=-1cm,yshift=1cm] {$T$};
		\draw[->] (0,-0.25) -- (0,4);
		\draw[->] (-0.25,0) -- (4,0);
		\draw[thick] (3,0) -- (3,3.2);
		\draw[thick] (0,0) -- (3.2,3.2) node[above,pos=0.5,sloped] {$y=x$};
		\filldraw[black] (0,3) circle (1pt) node[left] {3};
		\filldraw[black] (3,0) circle (1pt) node[below] {3};
		\draw[very thick] (2.5,0) -- (2.5,2.5);
		\end{tikzpicture}
	\end{subfigure}
\end{figure}

Το προς ολοκλήρωση χωρίο είναι το γραμμοσκιασμένο στο σχήμα

\subparagraph{(Γ)}
Θεωρώ το χωρίο αυτό κανονικό ως προς \(y\) (το οποίο ισχύει) και έχω:
\begin{align*}
Ι &= \int_0^3 \int_0^x e^{x^2} \dif y \dif x
\\ &=
\int_0^3 \left. ye^{x^2} \right|_0^x \dif x
\\ &=
\int_0^x xe^{x^2} \dif x
\\ &= \left.
\frac{1}{2}e^{x^2} \right|_0^3
\\ &= \frac{1}{2}\left(
e^2-1
\right)
\end{align*}

\paragraph{Άσκηση 5}
Μετασχηματίστε τα χωρία
\begin{enumparen}
\item \(A =  \left\lbrace (x,y): 1 \leq x^2+y^2 \leq 4 \right\rbrace\)
\item \(B =  \left\lbrace (x,y): 1 \leq x^2+y^2 \leq 18x \right\rbrace\)
\end{enumparen}
σε πολικές συντεταγμένες.

\begin{enumparen}
\item 
\mbox{}
\\
\begin{center}
	\begin{tikzpicture}[scale=1]
	
	\path [thick,draw=black,fill=green!20,even odd rule] (0,0) circle (2) (0,0) circle (1) ;
	
	\filldraw[black] (1,0) circle (1pt) node[below left] {1};
	\filldraw[black] (2,0) circle (1pt) node[below right] {2};
	
	\draw[very thick,orange] (0,0) -- (50:3);
	
	\draw[->] (0,-3) -- (0,3);
	\draw[->] (-3,0) -- (3,0);
	
	\end{tikzpicture}
\end{center}
\[
A' =  \left\lbrace  (\rho,\theta): 1 \leq \rho \leq 2,\ 0 \leq \theta \leq 2\pi\right\rbrace
\]
\item
\[
\begin{cases}
x^2+y^2 =1 \qquad & \text{(κύκλος κέντρου $(0,0)$ ακτ. 1} \\
x^2+y^2 =18x \qquad & \text{(κύκλος κέντρου $(9,0)$ ακτ. 9} 
\end{cases}
\]

\begin{center}
	\begin{tikzpicture}[scale=0.3]
	
	\filldraw[thick,fill=green!20] (9,0) circle (9) node[above] {$\rho = 18\cos\theta$};
	\filldraw[thick,fill=white] (0,0) circle(1) node[above,xshift=-20pt] {$\rho = 1$};;
	%\fill[white] (0,0) circle(1);
	
	\draw[->] (0,-11) -- (0,11);
	\draw[->] (-3,0) -- (20,0);
	
	\end{tikzpicture}
\end{center}

\[
x^2-18x+y^2=0 \implies
x^2-2\cdot 9 x +9^2-9^2+y^2=0
\implies (x-9)^2+y^2=9^2
\]

Περιγράφω τις συνοριακές καμπύλες
\(
\begin{cases}
x^2+y^2 &=1 \\
x^2+y^2 &=18x  
\end{cases}
\)
σε πολικές συντεταγμένες (\(x=\rho\cos\theta,\ y=\rho\sin\theta\))
\begin{itemize}
\item \(\rho^2=1 \implies \boxed{\rho=1}\)
\item \(\rho^2=18\rho\cos\theta \implies \boxed{\rho =18\cos\theta}\)
\end{itemize}

Λύνω το σύστημα \(
\begin{cases}
\rho &=1 \\
\rho &= 18\cos\theta
\end{cases}
\) (μου δίνει τα κοινά σημεία τομής των δύο κύκλων)
\[
18\cos\theta=1\implies\cos\theta=\frac{1}{18}
\implies \theta = \arccos\left( \frac{1}{18}\right)
\]

\[
A' =  \left\lbrace
(\rho,\theta): 1 \leq \rho \leq 18\cos\theta,\
-\arccos\left( \frac{1}{18}\right) \leq \theta \leq \arccos\left( \frac{1}{18}\right)
 \right\rbrace
\]
\end{enumparen}

\paragraph{Άσκηση 6}
Υπολογίστε το \(\iint_t (x^2+y^2)\dif x \dif y\) επί του δίσκου \(T =  \left\lbrace (x,y): x^2+y^2 \leq 1  \right\rbrace\)

Με μετασχηματισμό σε πολικές συντεταγμένες \textbf{έχουμε}:
\begin{align*}
I &= \iint_T (x^2+y^2)\dif x \dif y \\ &=
\int_0^{2\pi} \int_0^1 \rho^2 \cdot \rho\dif\rho\dif\theta \\
&=
2\pi \cdot \left. \frac{\rho^4}{4} \right|_0^1 \\
&= \frac{\pi}{2}
\end{align*}

Αν δεν χρησιμοποιούσα πολικές συντεταγμένες:
\begin{align*}
I&=\int_{-1}^1 \int_{-\sqrt{1-x^2}}^{\sqrt{1-x^2}} (x^2+y^2) \dif y \dif x
\\ &=
\int_{-1}^1x^2y+\left. \frac{y^3}{3} \right|_{-\sqrt{1-x^2}}^{\sqrt{1-x^2}} \dif x
\\ &=
2 \int_{-1}^1 x^2\sqrt{1-x^2} + \frac{\left(\sqrt{1-x^2}\right)^3}{3}
\\ &=^{x=\sin x}
\int_{-\frac{\pi}{2}}^\frac{\pi}{2} \dif x
\left(
\sin^2 y \cdot \cos y + \frac{\cos y}{3}
\right)^3 \cos y \dif y
\\ &=^{\cos 2y = 2 \cos^2 y -1}_{1-2 \cos^2 y}  \dots
\end{align*}


\paragraph{Άσκηση 7}
Υπολογίστε το \(\iint_T \sqrt{R^2-x^2-y^2}\dif x \dif y\) επί του χωρίου
\(T =  \left\lbrace (x,y): x^2+y^2\leq kx \right\rbrace\) \( (R > 0 )\)

\subparagraph{(Α) Σχήμα}
Έχουμε \(x^2+y^2 \leq Rx \implies x^2-Rx+y^2\leq0\implies
\underbrace{x^2-2\cdot\frac{R}{2}\cdot x + \left(\frac{R}{2}\right)^2}
_{\left(x-\frac{R}{2}\right)^2+y^2 \leq \frac{R^2}{4}}
-\left(\frac{R}{2}\right)^2+y^2\leq0\), κυκλικός δίσκος κέντρου \(
\left(
\frac{R}{2},0
\right)
\) και ακτίνας \(\frac{R}{2}\).

Θα πρέπει να μετασχηματισθεί σε πολικές συντεταγμένες
\[
A= \left\lbrace 
(\rho,\theta): -\frac{\pi}{2} \leq \theta \leq \frac{\pi}{2},\
0 \leq \rho \leq R\cos\theta
 \right\rbrace
\]

Τότε:
\begin{align*}
I &= \int_{-\frac{\pi}{2}}^\frac{\pi}{2}
\int_0^{R\cos\theta}
\sqrt{R^2-\rho^2} \cdot \overbrace{\rho \dif \rho \dif \theta}^{\mathrlap{\text{κατ' ευθείαν μπορώ να κάνω την αλλαγή}}}
\\ &=
-\frac{1}{2}
\int_{-\frac{\pi}{2}}^\frac{\pi}{2}
\int_0^{R\cos\theta}
(R^2-\rho^2)^\frac{1}{2}
\dif(R^2-\rho^2) \dif \theta
\\ &=
-\frac{1}{2}\cdot\frac{2}{3}
\int_{-\frac{\pi}{2}}^\frac{\pi}{2}
\left.
\left(
R^2-\rho^2
\right)^{\frac{3}{2}}
\right|_0^{R\cos\theta} \dif\theta
\\ &=
-\frac{R^3}{3}
\int_{-\frac{\pi}{2}}^\frac{\pi}{2}
|\sin^3\theta|
\\ &=
\frac{R^3\pi}{3}
-\frac{2R^2}{3}
\int_0^\frac{\pi}{2}
\sin^3\theta\dif\theta
\\ &=
\frac{\pi R^3}{3}
+\frac{2R^3}{3}
\int_0^\frac{\pi}{2}
\sin^2\theta \dif (\cos\theta)
\\ &=
\frac{2R^3}{3}
\int_0^\frac{\pi}{2}
\left(
1-\cos^2\theta
\right) \dif \cos\theta + \frac{\pi R^3}{3}
\\ &=
\frac{2R^3}{3}
\left.
\left(
\cos\theta-\frac{\cos^3\theta}{3}
\right)
\right|_0^\frac{\pi}{2}
+ \frac{\pi R^3}{3}
\\ &= \frac{\pi R^3}{3} - \frac{4R^3}{9}
\end{align*}


\begin{attnbox}{}
\[
\int_a^b f\left(g(x)\right)
\underbrace{g'(x)\dif x}_{\dif g(x)}
\]
\end{attnbox}





\begin{attnbox}{SOS!}
	\paragraph{Άσκηση}
	Υπολογίστε τον όγκο του φραγμένου στερεού μεταξύ των επιφανειών \(\overbrace{z=3x^2}^{\mathllap{\text{κυλινδρική επιφάνεια γιατί λείπει το $y$}}}\) και \(\underbrace{z=4-x^2-y^2}_{\mathclap{\text{παραβολοειδές}}}\)
	
	\tcblower
	
	\subparagraph{(Α) Υποτυπώδες σχήμα}
	\begin{center}
	\begin{tikzpicture}
	\begin{axis}[
	axis lines = middle,
	%hide axis,
	colormap/cool,
	zticklabels={,,},
	yticklabels={,,},
	xticklabels={,,},
	]
	
	\addplot3[mesh,samples=20,domain=-7:7] {3*x*x+2};
	\addplot3[mesh,domain=-7:7] {90-x^2-y^2};
	
	
	\end{axis}
	\end{tikzpicture}
	\end{center}
	\subparagraph{(Β) Χωρίο ολοκλήρωσης}
	Η \textbf{προβολή} της τομής των δύο επιφανειών είναι:
	\[
	\begin{cases}
	z&=3x^2\\z&=4-x^2-y^2
	\end{cases}
	\implies
	3x^4-x^2-y^2\implies 4x^2+y^2=4 \implies x^2+\frac{y^2}{4}=1 \text{ (έλλειψη)}
	\]
	
	Άρα ολοκληρώνω εντός του χωρίου \(T\).
	\begin{center}
		\begin{tikzpicture}[scale=0.5]
		
		\filldraw[thick,fill=green!20] (0,0) circle(1 and 2) node[above right] {$T$};
		
		\filldraw[black] (1,0) circle (1pt) node[below right] {1};
		\filldraw[black] (-1,0) circle (1pt) node[below left] {-1};
		
		\draw[->] (0,-3) -- (0,3);
		\draw[->] (-2,0) -- (2,0);
		
		\end{tikzpicture}
		
	\end{center}
	
	\subparagraph{(Γ) Τι ολοκληρώνω?}
	\begin{align*}
	V &= \iint_T \big(z_\mathrm{max} - z_\mathrm{min}\big)(x,y)\dif x \dif y\\
	&= \iint_T (4-x^2-y^2)-3x^2 \dif x \dif y
	\end{align*}
	
	\subparagraph{(Δ) Υπολογισμός}
	\[
	V = \iint_T (4-4x^2-y^2)\dif x \dif y
	\]
	
	Θα χρησιμοποιήσω το μετασχηματισμό:
	\[
	\begin{cases}
	x&=\rho\cos\theta\\
	y&=2\rho\sin\theta
	\end{cases}
	\]
	
	
	\begin{align*}
	\left| \frac{D(x,y)}{D(\rho,\theta)} \right| =
	\left|
	\begin{matrix}
	x_\rho&x_\theta\\
	y_\rho&y_\theta
	\end{matrix}
	\right| = \left|
	\begin{matrix}
	\cos\theta&-\rho\sin\theta\\
	2\cos\theta&2\rho\sin\theta
	\end{matrix}
	\right| = 2\rho
	\end{align*}
	και
	\begin{align*}
	x^2+\frac{y^2}{4}=1 \xRightarrow[y=2\rho\sin\theta]{x=\rho\cos\theta}
	\rho^2=1 \implies \boxed{\rho=1}
	\end{align*}
	άρα το χωρίο \(T\) μετασχηματίζεται στο:
	\[
	T' =  \left\lbrace 
	(\rho,\theta):\ 0\leq\rho\leq1,\ 0\leq\theta\leq2\pi
	\right\rbrace
	\]
	\textbf{άρα:}
	\begin{align*}
	V&=\iint_T \left( 4-4x^2-y^2\right)\dif x\dif y\\
	&= \iint_{T'} \left( 4-4\rho^2\cos^2\theta-4\rho^2\sin^2\theta\right)
	\mathbf{2\rho}\dif\rho\dif\theta\\
	&=
	8\int_0^{2\pi} \int_0^1 \left(1-\rho^2\right)\rho\dif\rho\dif\theta \\
	&= 8\cdot2\pi\cdot \int_0^1 \left(\rho-\rho^3\right)\dif\rho = 4\pi
	\end{align*}
\end{attnbox}



\paragraph{Άσκηση}
Υπολογίστε το εμβαδόν μεταξύ των καμπύλων \(\rho=a\sin\theta,\ \rho=a\left(1-\cos\theta\right),\quad a>0\) στο ΑΝΩ ημιεπίπεδο.

\subparagraph{(Α)}
\[
E = \iint_T 1 \dif x \dif y = \iint_{T'} \rho\dif\rho\dif\theta
\]
Εννοείται η άσκηση μπορεί να λυθεί και με τον τύπο του εμβαδού του Λογισμού \textlatin{I}.
%TODO Atreas Graph 03

\textbf{Άρα:}
\begin{itemize}
\item \(a\sin\theta \geq a(1-\cos\theta)\ \forall \theta \in [0,\frac{\pi}{2}]\)
\item \(a\sin\theta\leq a(1-\cos\theta)\ \forall \theta \in [\frac{\pi}{2},\pi]\)
\end{itemize}

\begin{align*}
E=&\int_0^\frac{\pi}{2}\int^{a\sin\theta}_{a(1-\cos\theta)}\rho\dif\rho\dif\theta +
\\
+&\int_\frac{\pi}{2}^\pi\int_{a\sin\theta}^{a(1-\cos\theta)}\rho\dif\rho\dif\theta
\end{align*}


\paragraph{Άσκηση}
Υπολογίστε το \(\iint_D e^{\frac{x-y}{x+y}} \dif x \dif y\) επί του χωρίου του σχήματος:
%TODO Atreas Graph 01

Θα εφαρμόσουμε αλλαγή μεταβλητών:
\[
\begin{cases}
u&=x-y\\v&=x+y
\end{cases}
\implies
\begin{cases}
x&=\frac{u+v}{2}\\
y&=\frac{-u+v}{2}
\end{cases}
\]
άρα
\[
\frac{D(x,y)}{D(u,v)} = \left|
\begin{matrix}
x_u&x_v\\y_u&y_v
\end{matrix}
\right|=\left|
\begin{matrix}
\frac{1}{2}&\frac{1}{2}\\-\frac{1}{2}&\frac{1}{2}
\end{matrix}
\right|=\frac{1}{4}+\frac{1}{4}=\frac{1}{2}
\]

\textbf{Άρα:}
\[
\iint_D e^{\frac{x-y}{x+y}}\dif x \dif y = \iint_\mathbf{D'}e^\frac{u}{v} \cdot \left|
\frac{D(x,y)}{D(u,v)}\right|\dif u \dif v = \iint_{D'} e^\frac{u}{v}\left|\frac{1}{2}\right| \dif u \dif v
= \frac{1}{2} \iint_{D'} e^\frac{u}{v} \dif u \dif v
\]

\subparagraph{Σχεδίαση χωρίου $D'$}
\begin{itemize}
\item \(x+y=1 \implies^{x=\frac{u+v}{2}}_{x=\frac{u+v}{2}} \frac{u+v}{2}+ \frac{-u+v}{2} = 1 \implies \boxed{v=1}\)\\
\item \(x+y=2 \implies \frac{u+v}{2}+ \frac{-u+v}{2} = 2 \implies \boxed{v=2}\)\\
\item \(x=0 \implies \frac{u+v}{2} = 0 \implies \boxed{v=-u}\)
\item \(y=0 \implies \frac{-u+v}{2} = 0 \implies \boxed{v=u}\)
\end{itemize}

\subparagraph{Τελικά}
\begin{align*}
I &= \int_1^2\int_{-v}^v e^\frac{u}{v} \dif u \dif v \\
&= \frac{1}{2} \int_1^2 \left.ve^\frac{u}{v}\right|_{-v}^v \dif v \\
&= \frac{1}{2} \int_1^2 v(e-e^{-1})\dif v \\
&= \frac{1}{4} \left. v^2 \right|_1^2 \cdot \left( e-\frac{1}{e} \right) = \frac{3}{4} \left(
e- \frac{1}{e}
\right)
\end{align*}

\begin{figure}
	\centering
	\begin{subfigure}[b]{0.4\textwidth}
		\begin{tikzpicture}[scale=1]
		
		\filldraw[thick,fill=green!20] (0,0.8) circle (0.8);
		
		%\filldraw[black] (1,0) circle (1pt) node[below right] {1};
		%\filldraw[black] (-1,0) circle (1pt) node[below left] {-1};
		
		\draw[->] (0,-2) -- (0,2);
		\draw[->] (-2,0) -- (2,0);
		
		\end{tikzpicture}
	\end{subfigure}
	\quad 
	\begin{subfigure}[b]{0.4\textwidth}
		\begin{tikzpicture}[scale=1]
		
		\filldraw[thick,fill=green!20] (0,0.8) circle (0.8);
		
		%\filldraw[black] (1,0) circle (1pt) node[below right] {1};
		%\filldraw[black] (-1,0) circle (1pt) node[below left] {-1};
		
		\draw[->] (0,-2) -- (0,2);
		\draw[->] (-2,0) -- (2,0);
		
		\end{tikzpicture}
	\end{subfigure}
\end{figure}



\section{Τριπλά Ολοκληρώματα}
Έστω \(f: R \subset  \mathbb R ^3 \to  \mathbb R \) \textbf{φραγμένη} συνάρτηση επί ορθογωνίου \textbf{παραλληλεπιπέδου}
\[
R =  \left\lbrace (x,y,z): a_1\leq x\leq b_1,\ a_2\leq y\leq b_2,\ a_3\leq z\leq b_3  \right\rbrace
\]
και \(\Delta = (\Delta_x,\Delta_y,\Delta_z)\), όπου \begin{align*}
\Delta_x &=  \left\lbrace a_1=x_0 < x_1 < \dots < x_N = b_1  \right\rbrace \\
\Delta_y &=  \left\lbrace a_2=y_0 < y_1 < \dots < y_N = b_2  \right\rbrace \\
\Delta_z &=  \left\lbrace a_3=z_0 < z_1 < \dots < z_N = b_3  \right\rbrace \\
\end{align*}
διαμέριση του \(R\) σε $Ν\cdot M \cdot K\) "στοιχειώδη" ορθογώνια παραλληλεπίπεδα \(\Omega_{n,m,k}\) όγκου
\[
V_{n,m,k} = (x_{n+1}-x_n)(y_{n+1}-y_n)(z_{n+1}-z_n)
\qquad
\begin{cases}
n&=0,\dots,N-1\\
m&=0,\dots,M-1\\
k&=0,\dots,K-1
\end{cases}
\]

Έστω \((\widetilde{x_n},\widetilde{y_m},\widetilde{z_k})\) είναι τυχαίο σημείο στο ορθογ. παρ/δο \(\Omega_{n,m,k}\) και ορίζω την \(f(\widetilde{x_n},\widetilde{y_m},\widetilde{z_k})\ \forall_{n,m,k}\).

Θεωρώ το άθροισμα:
\[
S(f,\Delta) = \sum_{n=0}^{N-1} \sum_{m=0}^{M-1} \sum_{k=0}^{K-1} f(\widetilde{x_n},\widetilde{y_m},\widetilde{z_k}) V_{n,m,k}
\]

Έστω \(|\Delta| = \max \left\lbrace \delta_{n,m,k}:\ n=0,\dots,N-1,\ m=0,\dots,M-1,\ k=0,\dots,K-1 \right\rbrace\) είναι το πλάτος της έδιαμρισης $\Delta$, όπου:
\[
\delta_{n,m,k} = \max  \left\lbrace |\rho\rho'|:\ \forall \rho,\rho; \in \Omega_{n,m,k} \right\rbrace
\]

Αν \(\lim\limits_{|\Delta|\to0} S(f,\Delta) = \lambda \in \mathbb R\), ανεξάρτητα της διαμέρισης $\Delta$ και της επιλογής των σημείων \((\widetilde{x_n},\widetilde{y_m},\widetilde{z_k}) \in \Omega_{n,m,k}\), τότε λέμε ότι υπάρχει το τριπλό ολοκλήρωμα της \(f\) επί του παραλ/δου \(R\) και γράφουμε:

\[
\iiint_R f(x,y,z) \dif x \dif y \dif z = \lambda \in \mathbb{R}
\]

\paragraph{}

Ο ορισμός γενικεύεται και για κλειστά και φραγμένα στερεά του \( \mathbb R ^3\) ως εξής:

Έστω \(S \subset  \mathbb R ^3 \) κλειστό και φραγμένο στερεό με σύνορο αμελητέου όγκου (π.χ. το σύνορό του αποτελείται από ένωση επιφανειών \(z=f(x,y)\)). Έστω
\[
S \subset \Omega
\]
όπου \(\Omega\) ορθογ. παρ/δο που καλύπτει το \(S\).

Αν $f:S \subset \mathbb R^3 \to  \mathbb R $, ορίζω την επέκτασή της στο \(\Omega\) ως εξής:
\[
g(x,y,z) = \begin{cases}
f(x,y,z),\quad&(x,y,z)\in S\\
0,\quad&(x,y,z)\in \Omega-S
\end{cases}
\]

Αν η \(g:\Omega \subset  \mathbb R ^3\to \mathbb R \) είναι ολοκληρώσιμη επί του \(\Omega\), τότε ορίζουμε:
\[
\iiint_S f(x,y,z)\dif x\dif y\dif z=
\iiint_\Omega g(x,y,z)\dif x\dif y \dif z
\]
και αποδεικνύεται ότι ο ορισμός αυτός \textbf{ΔΕΝ} εξαρτάται απ' την επιλογή του \(\Omega \supset S\).

\begin{attnbox}{Θεώρημα}
Έστω \(f:S\in  \mathbb R ^3\to  \mathbb R \) (όπου $S$ κλειστό και φραγμένο στερεό με σύνορο αμελητέου όγκου) \textbf{ΣΥΝΕΧΗΣ} στο \(S\), \textbf{εκτός ενδεχομένως} από ένα σύνολο σημείων αμελητέου όγκου.

Τότε η \(f\) ολοκληρώσιμη επί του \(S\).
\end{attnbox}

\subsection{Ιδιότητες}
Όπως στα διπλά.

\subsection{Υπολογισμός τριπλών ολοκληρωμάτων}
\paragraph{\boxed{\mathrm A} Σε ορθογώνια παραλ/δα}
\subparagraph{Θ (\textlatin{Fubini})} Έστω \(f:R \subset  \mathbb R ^3\to  \mathbb R \) \textbf{ΣΥΝΕΧΗΣ} επί ορθογ. παρ/δου
\[
R =  \left\lbrace (x,y,z): a_1\leq x\leq b_1,\ a_2\leq y\leq b_2,\ a_3\leq z\leq b_3  \right\rbrace
\]

Έστω \begin{align*}
D_{xy} &= [a_1,b_1]\times[a_2,b_2]\\
D_{yz} &= [a_2,b_2]\times[a_3,b_3]\\
D_{xz} &= [a_1,b_1]\times[a_3,b_3]
\end{align*}
είναι οι ορθογώνιες προβολές του παρ/δου \(R\) στα επίπεδα \(xy,\ yz\) και \(xz\) αντιστοίχως.

Τότε οι συναρτήσεις
\begin{align*}
g(x,y)&=\int_{a_3}^{b_3}f(x,y,z)\dif z,\\
g(y,z)&=\int_{a_1}^{b_1}f(x,y,z)\dif x,\\
g(x,z)&=\int_{a_2}^{b_2}f(x,y,z)\dif y
\end{align*}
είναι \textbf{συνεχείς} επί των προβολών \(D_{xy},D_{yz},D_{xz}\) αντιστοίχως και
\begin{align*}
\iiint_R f(x,y,z)\dif x \dif y \dif z & =
\iint_{D_{xy}} \left(
\int_{a_3}^{b_3} f(x,y,z) \dif z
\right) \dif x \dif y  \qquad
\begin{cases}
\int_{a_1}^{b_1}\int_{a_2}^{b_2}\int_{a_3}^{b_3} \cdots \dif z \dif y \dif x\\
\int_{a_2}^{b_2}\int_{a_1}^{b_1}\int_{a_3}^{b_3} \cdots \dif z \dif x \dif y
\end{cases}
\\
& =
\iint_{D_{yz}} \left(
\int_{a_1}^{b_1} f(x,y,z) \dif x
\right) \dif y \dif z \qquad \begin{cases}
\int_{a_2}^{b_2}\int_{a_3}^{b_3}\int_{a_1}^{b_1} \cdots \dif x \dif z \dif y\\
\int_{a_2}^{b_3}\int_{a_2}^{b_2}\int_{a_1}^{b_1} \cdots \dif x \dif y \dif z
\end{cases}\\
& =
\iint_{D_{xz}} \left(
\int_{a_2}^{b_2} f(x,y,z) \dif z
\right) \dif x \dif y \qquad
\begin{cases}
\int_{a_1}^{b_1}\int_{a_3}^{b_3}\int_{a_2}^{b_2} \cdots \dif y \dif z \dif x\\
\int_{a_3}^{b_3}\int_{a_1}^{b_1}\int_{a_2}^{b_2} \cdots \dif y \dif x \dif z
\end{cases}
\end{align*}

\subparagraph{π.χ.}
Υπολογίστε το \(\iint_S xyz \dif x \dif z \dif y\) επί του στερεού \(S= \left\lbrace (x,y,z): -1 \leq y \leq 1,\ 0\leq x \leq 2,\ -4 \leq z \leq 7  \right\rbrace\).

\begin{align*}
I &= \underbrace{\int_{-4}^7 \underbrace{\int_0^2 \underbrace{\int_{-1}^1 x y z \dif y} \dif x} \dif z}\\
&= \int_{-4}^7 \int_0^2 \left. xz \frac{y^2}{2}\right|_{-1}^1 \dif y \dif z\\
&= \int_{-4}^7 \int_0^2 \left( \frac{xz}{2} - \frac{xz}{2}\right) \dif y \dif z\\
&= 0
\end{align*}

\paragraph{\boxed{\mathrm B} Σε κλειστά και φραγμένα χωρία}
\begin{itemize}
\item 'Εστω \(S \in  \mathbb R ^3\) κλειστό και φραγμένο στερεό με σύνορο αμελητέου όγκου. Θα λέμε ότι το \(S\) είναι κανονικό ως προς \(z\) αν το \(S\) είναι συνεκτικό και ΚΑΘΕ ευθεία \(\parallel z'z\) που διέρεται και από το εσωτερικό του στερεού \(S\) τέμνει το σύνορο του \(S\) \textbf{ακριβώς} σε δύο σημεία ΚΑΙ η προβολή του \(S\) στο \(Oxy\) επίπεδο είναι χωρίο κανονικό ως προς \(x\) \textbf{ή} ως προς \(y\).
\end{itemize}

%TODO Atreas Graph 05


\(S\) κανονικό ως προς \(z\):
\[
S =  \left\lbrace (x,y,z): (x,y) \in D_{xy},\ f(x,y)\leq z \leq g(x,y)  \right\rbrace
\]

\begin{attnbox}{Θεώρημα (\textlatin{Fubini})}
\setcounter{equation}{0}
\(f: S \subset  \mathbb R ^3 \to  \mathbb R \) συνεχής συνάρτηση επί στερεού \(S\) με σύνορο αμελητέου όγκου και \(D_{xy},D_{xz},D_{yz}\) είναι οι ορθογώνιες προβολές του στερεού \(S\) πάνω στα επίπεδα \(Oxy,Oxz\) και \(Oyz\) αντίστοιχα.
\begin{itemize}
\item Αν \(S\) είναι κανονικό ως προς \(z\) στερεό της μορφής
\[
S =  \left\lbrace (x,y,z): (x,y) \in D_{xy},\ A(x,y) \leq z \leq B(x,y) \right\rbrace
\]
τότε
\begin{equation}
\iiint_S f(x,y,z) \dif x \dif y\dif z = \iint_{D_{xy}} \left(
\int_{A(x,y)}^{B(x,y)} f(x,y,z)\dif z
\right) \dif x \dif y
\end{equation}
\item Αν \(S\) είναι κανονικό ως προς \(\mathbf{y}\) στερεό της μορφής
\[
S =  \left\lbrace (x,y,z): (x,z) \in D_{yz}, \Gamma (x,z) \leq y \leq \Delta (x,z)
 \right\rbrace
\]
τότε
\begin{equation}
\iiint_S f(x,y,z)\dif x \dif y \dif z =
\iint_{D_{xz}} \left(
\int_{\Gamma(x,z)}^{\Delta(x,z)} f(x,y,z) \dif y
\right) \dif x \dif z
\end{equation}
\item Αν \(S\) κανονικό ως προς \(x\) στερεό της μορφής
\[
S =  \left\lbrace (x,y,z): (y,z) \in D_{yz},\ K(y,z)\leq x \leq \Lambda (y,z) \right\rbrace
\]
τότε
\begin{equation}
\iiint_S f(x,y,z)\dif x \dif y \dif z = 
\iint_{D_{yz}} \left(
\int_{K(y,z)}^{\Lambda(y,z)}f(x,y,z) \dif x
\right) \dif y \dif z
\end{equation}
\item Αν \(S\) κανονικό, τότε:
\[
\iint_S f(x,y,z)\dif x \dif y \dif z = (1) = (2) = (3)
\]
\end{itemize}
\end{attnbox}

\subsection{Εφαρμογές τριπλού ολοκληρώματος}
\paragraph{1}
Αν \(w=f(x,y,z)=1\ \forall(x,y,z)\in S\) όπου \(S\) κλειστό και φραγμένο στερεό, τότε
\[
\iiint_S 1 \dif x \dif y \dif z = \text{όγκος του στερεού } S
\]

\paragraph{2}
Αν \(\rho =r\rho(x,y,z)\) συνεχής πυκνότητα μάζας/φορτίου επί στερεού \(S\) κλειστού και φραγμένου, τότε
\[
\iiint_S \rho(x,y,z) \dif x \dif y \dif z = \text{συνολική μάζα/φορτίο επί του στερεού }S
\]

\paragraph{3}
Αν \(w=f(x,y,z) \geq 0\ \forall (x,y,z)\in S\) όπου \(w\) είναι μια υπερ-επιφάνεια, τότε το \(\iiint_S f(x,y,z)\dif x \dif y\dif z = \) υπερ-όγκος του υπερ-στερεού που περικλείεται από την \(w\) και το \(Oxyz\)-χώρο.

\subsection{Αλλαγή μεταβλητής}
Ακριβώς όπως στα διπλά ολοκληρώματα.

Ενδιαφέρομαι κυρίως για τις ακόλουθες περιπτώσεις:
\paragraph{(Α) Κυλινδρικές συντεταγμένες}
%TODO Atreas Graph 01

\begin{gather*}
(x,y,z) \leftrightarrow_\text{κυλινδρ.} (\rho,\theta,z)
\\
\begin{cases}
x&=\rho\cos\theta\\
y&=\rho\sin\theta\\
z&=z
\end{cases}
\text{ή}
\begin{cases}
\rho&=\sqrt{x^2+y^2}\\
\tan\theta&=\frac{y}{x}\ (x\neq0)\\
z&=z
\end{cases}
\\
\frac{D(x,y,z)}{D(\rho,\theta,z)}=
\left|
\begin{matrix}
x_\rho & x_\theta & x_z\\
y_\rho & y_\theta & y_z\\
z_\rho & z_\theta & z_z\\
\end{matrix}
\right|
=
\left|
\begin{matrix}
\cos\theta&-\rho\sin\theta&0\\
\sin\theta&\rho\cos\theta&0\\
0&0&1\\
\end{matrix}
\right|
= \left|
\begin{matrix}
\cos\theta & -\rho\sin\theta\\
\sin\theta & \rho\cos\theta
\end{matrix}
\right| = \rho
\\
\dif x \dif y \dif z \leftrightarrow
\boxed{
\rho\dif\rho\dif\theta\dif z
}
\end{gather*}
%TODO Atreas Graph 02




\subsubsection{Μετασχηματισμός σε σφαιρικές συντεταγμένες}
%TODO Atreas Graph 01
\[
(x,y,z) \xleftrightarrow{1-1} (r,\theta,\phi)
\]
όπου \(r=\left|\overrightarrow{OM}\right|\), \newline
\(\theta\): η προσανατολισμένη γωνία μεταξύ της ημιευθείας \(Ox\ (\theta=0)\) και ημιευθείας \(OM\). \((0\leq\theta<2\pi \ \text{ή} -\pi\leq\theta<\pi\), \newline
\(\phi\): η γωνία μεταξύ του ημιάξονα \(Oz\ (\phi=0)\) και της ημιευθείας \(OM\). (\(0\leq\phi\leq\pi\))

Έχουμε:
\[
\boxed{
\begin{array}{rl}
x&=r\cos\theta\sin\phi\\
y&=r\sin\theta\sin\phi\\
z&=r\cos\phi
\end{array}
\leftrightarrow
\begin{array}{rl}
r&=\sqrt{x^2+y^2+z^2}\\
\tan\theta&=\frac{z}{x}\quad (x\neq0)\\
\cos\phi &= \frac{z}{r} = \frac{z}{\sqrt{x^2+y^2+z^2}}
\end{array}
}
\]

\[
\boxed{x^2+y^2+z^2=R^2} \xleftrightarrow{\text{σφαιρικές}} \boxed{r=R}
\]

\textbf{Τότε} \(
\frac{\mathrm D(x,y,z)}{\mathrm D(\rho,\theta,\phi)} =
\left|
\begin{matrix}
x_r&x_\theta&x_\phi\\
y_r&y_\theta&y_\phi\\
z_r&z_\theta&z_\phi\\
\end{matrix}
\right|
= \dots
= \boxed{-r^2\sin\phi}
\)

Άρα \(\dif x \dif y \dif z \rightarrow \left|
\frac{\mathrm D(x,y,z)}{\mathrm D(\rho,\theta,\phi)}
\right|
\dif r \dif\theta\dif\phi = \boxed{
r^2\sin\phi \dif r \dif\theta\dif\phi
}
\)

\subsubsection*{Μη γνήσια τριπλά ολοκληρώματα εκτός ύλης}

\subsection{Ασκήσεις}
\paragraph{Άσκηση}
Υπολογίστε το τριπλό ολοκλήρωμα
\[
I = \iiint_S x^2\dif x \dif z \dif y
\]
όπου \(S\) το στερεό που περικλείεται από τις επιφάνειες \(x=0,\ y=0,\ z=0\) και \(\frac{x}{a}+\frac{y}{b}+\frac{z}{c}=1 \quad (a,b,c\neq0)\).
\subparagraph{\boxed{\mathrm A} Υποτυπώδες σχήμα}
%TODO Atreas Graph 02
\begin{itemize}
\item Είναι στερεό κανονικό ως προς $z$

\[
\left.
\begin{array}{r}
\text{επιφάναια "εισόδου"} \\
\text{επιφάνεια "εξόδου"}
\end{array}
\right\rbrace
\text{τυχαίας ευθείας \((\epsilon) \parallel z'z\) που διέρχεται και από το εσωτερικό του $S$}
\]
\[
\begin{array}{l}
\mathbf{z=0}\\
z=c\left(1-\frac{x}{a}-\frac{y}{b}\right)
\end{array}
\]
\end{itemize}

\subparagraph{\boxed{\mathrm B} Υπολογισμός}
\[
I = \iiint_S x^2\dif x \dif z \dif y \overset{\text{στερεό $S$ κανονικό ως προς $z$}}{=}
\iint_{\underbrace{D_{xy}}_{\mathclap{\text{η προβολή του στερεού στο $Oxy$ επίπεδο}}}} \left(
\int^{c\left(1-\frac{x}{a}-\frac{y}{b}\right)}_0 x^2\dif z
\right) \dif x \dif y
\]

\subparagraph{\boxed{\mathbf \Gamma} Υπολογισμός \(D_{xy}\)}
%TODO Atreas Graph 03

\subparagraph{\boxed{\mathbf \Delta} Τελικός υπολογισμός}
\begin{align*}
I &= \iint_{D_{xy}} \left. x^2z\right|_0^{c\left(1-\frac{x}{a}-\frac{y}{b}\right)} \dif x \dif y
\\ &= \iint_{D_{xy}} x^2 \left(1-\frac{x}{a}-\frac{y}{b}\right) \dif x \dif y
\\ &\xlongequal{\text{εργάζομαι θεωρώντας $D_{xy}$ κανονικό ως προς $x$}}
\int_0^b \int_0^{a\left(1-\frac{y}{b}\right)} x^2\left(1-\frac{x}{a}-\frac{y}{b}\right) 
\mathbf{dx} \dif y
\\ &=
\int_0^b \left(
\int_0^{a\left(1-\frac{y}{b}\right)}
\left(x^2-\frac{x^3}{a}-\frac{x^2y}{b}\right)\dif x
\right)\dif y
\\ &= \dots
\end{align*}

\paragraph{}
\[
\iiint_S f(x,y,z) \dif x \dif y \dif z
\]
\begin{itemize}
\item Σχήμα για το $S$
\item Το $S$ κανονικό ως προς π.χ. $z$, τότε:
\[
\iiint_S = \iint_{\underbrace{D_{xy}}_{\mathclap{
\begin{cases}
D_{xy} \text{ προβολή του $S$ στο $Οxy$ }\\
D_{xy} \text{ κανονικό ως προς $x$ ή $y$? }
\end{cases}
}}} \left(
\int_\text{επιφ. εισόδου}^\text{επιφ. εξόδου}f(x,y,z) \dif z
\right) \dif x \dif y
\]
\item Τότε \(\iint_{D_{xy}} \Big(\quad\Big) \dif x \dif y = \int_a^b \int_{g(x)}^{h(x)} \cdots\)
\end{itemize}

\paragraph{Άσκηση}
Υπολογίστε τον όγκο του στερεού \[
S =  \left\lbrace (x,y,z):\ x^2+y^2+z^2\leq 2az,\ z \geq \sqrt{x^2+y^2}
 \right\rbrace
\]
όπου \(a>0\)

\subparagraph{Θεωρία}
\[
V = \iiint_S 1 \dif x \dif y \dif z
\]

\subparagraph{\boxed{\mathbf A} Υποτυπώδες σχήμα}
\begin{itemize}
\item \begin{align*}
&x^2+y^2+z^2=2az\\
\implies& x^2+y^2+(z^2-2az+a^3)-a^2=0\\
\implies& \boxed{x^2+y^2+(z-a)^2=a^2}
\end{align*}
σφαίρα κέντρου \((0,0,a)\) και ακτίνας $a$
%TODO Atreas Graph 04
\item Η $x^2+y^2+z^2\leq2az$ είναι \textbf{το} εσωτερικό (και το σύνορο) της σφαίρας του σχήματος
\item \begin{align*}
z=\sqrt{x^2+y^2}
\end{align*}
\item Καμπύλη τομής των δύο επιφανειών
\begin{gather*}
\begin{cases}
x^2+y^2+z^2&=2az\\
x&=\sqrt{x^2+y^2}
\end{cases}
\implies
\begin{cases}
x^2+y^2+(x^2+y^2)&=2a\sqrt{x^2+y^2}\\
z&=\sqrt{x^2+y^2}
\end{cases}\\
\implies
\begin{cases}
x^2+y^2&=a\sqrt{x^2+y^2}\\
z&=\sqrt{x^2+y^2}
\end{cases}
\implies
\begin{cases}
\sqrt{x^2+y^2}&=a\\
z&=a
\end{cases}\\
\implies \begin{cases}
x^2+y^2&=a^2\\
z&=a
\end{cases}
\quad \left(
\begin{matrix}
\text{κύκλος } x^2+y^2=a^2\\
\text{στο επίπεδο } z= a
\end{matrix}
\right)
\end{gather*}
\end{itemize}

\subparagraph{\boxed{\mathbf B} Υπολογισμός}
\(S\) κανονικό ως προς $z$. 
\(
\left.
\begin{array}{r}
\text{επιφάναια "εισόδου"} \\
\text{επιφάνεια "εξόδου"}
\end{array}
\right\rbrace
\text{ευθεία \((\epsilon) \parallel z'z\)}
\)
(δεν χρειάζεται να το γράψω)
\begin{align*}
z&=\sqrt{x^2+y^2}&\text{(είσοδος)}\\
z&=a+\sqrt{a^2-x^2-y^2}&\text{(έξοδος)}\\
\end{align*}

οπότε:
\begin{align*}
V=&\iiint_S 1 \dif x \dif y \dif z\\
\overset{\text{θ}}{=}& \iint_{D_{xy}} \left(
\int_{\sqrt{x^2+y^2}}^{a+\sqrt{a^2-x^2-y^2}}1 \dif z
\right) \dif x \dif y \\
=& \iint_{D_{xy}} \left(
a+\sqrt{a^2-x^2-y^2}-\sqrt{x^2+y^2}
\right) \dif x \dif y
\end{align*}

\subparagraph{\boxed{\mathbf \Gamma} \(D_{xy}\): προβολή του $S$ στο $Oxy$ επίπεδο}
%TODO Filled circle with radius $a$ and text to the right:
% D_{xy} =  \left\lbrace (x,y):\ x&2+y^2\leq a^2 \right\rbrace

\subparagraph{\boxed{\mathbf \Delta} Υπολογισμός της (1) με χρήση πολικών συντεταγμένων}
\begin{align*}
I =& \int_0^{2\pi} \int_0^a \left(a+\sqrt{a^2-\rho^2}-\rho\right)\rho\dif\rho\dif\theta\\
=& 2\pi \left(\int_0^a a\rho+\rho\sqrt{a^2-\rho^2}\right)\dif\rho\\
=&\left.2\pi a\frac{\rho^2}{2}\right|_0^a
-\left.2\pi\frac{\rho^3}{3}\right|_0^a -
\frac{2\pi}{2}\int_0^a (a^2-\rho^2)^\frac{1}{2}\dif(a^2-\rho^2)\\
=&\pi a^2 - \frac{2\pi}{3}a^3-\left.\frac{2\pi}{3}(a^2-\rho^2)\right|_0^a\\
=&\frac{\pi a^3}{3}+\frac{2\pi}{3}a^3\\
=&\mathbf{\pi a^3}
\end{align*}


\paragraph{Άσκηση}
Υπολογίστε το \(\iiint_S (x^2+y^2) \dif x \dif y \dif z\) επί του στερεού \(S =  \left\lbrace (x,y,z):\ \mathbf{a^2\leq x^2+y^2+z^2\leq b^2},\ z\geq0\  \right\rbrace\).
\subparagraph{\boxed{\mathrm A} Σχήμα}
%TODO Atreas Graph 05

Η προβολή επί του \(Oxy\) επίπεδο είναι:
%TODO Atreas Graph 06

\subparagraph{\boxed{\mathrm B}}
Θα χρησιμοποιήσουμε μετασχηματισμό σε σφαιρικές συντεταγμένες:
\[
\begin{array}{|ll}
x=r\cos\theta\sin\phi\quad&(0\leq\theta <2\pi)\\
y=r\sin\theta\sin\phi\quad&(0\leq\phi\leq\pi)\\
z=r\cos\phi\quad&\text{στη γενική μορφή}
\end{array}
\]

\subparagraph{\boxed{\mathbf \Gamma} Υπολογισμός}
\begin{align*}
I = &
\int_{\underbrace{0}_{\mathclap{\text{διότι ΔΕΝ υπάρχουν αρνητικά $z$, δηλ. "κάτω" ημισφαίριο}}}}^\frac{\pi}{2}
\int_{\underbrace{0}_{\mathclap{\text{διότι η προβολή $D_{xy}$ είναι όλος ο δίσκος του σχήματος}}}}^{3\pi}
\int_a^{\overbrace{b}^{\mathclap{\text{διότη η επιφάνεια εισόδου είναι η σφαίρα $r=a$ και η επιφάνεια εξόδου η σφαίρα $r=b$}}}}
\left[
(r\cos\theta\sin\phi)^2+(r\sin\theta\sin\phi)^2
\right]\cdot
\mathbf{
r^2\sin\phi\dif r \dif\theta\dif\phi
}\\
=&\int_0^\frac{\pi}{2}
\int_0^\frac{2\pi}{0}
\left(
\int_a^b r^4\sin^3\phi \dif r
\right)\dif\theta\dif\pi\\
=&
\int_0^\frac{\pi}{2}
\int_0^{2\pi}\left.
\sin^3\phi \frac{r^5}{5}
\right|_a^b \dif\theta\dif\phi
\\=&
\frac{b^5-a^5}{5}
\int_0^\frac{\pi}{2}
\left(
\int_0^{2\pi} \sin^3\phi \dif \theta
\right)\dif\phi
\\ = &
\frac{b^5-a^5}{5}
\int_0^\frac{\pi}{2}
\left.\sin^3\phi\cdot\theta\right|_0^{2\pi}\dif\phi
\\=&
\frac{b^5-a^5}{5}2\pi \int_0^\frac{\pi}{2}\sin^2\phi\dif\phi
\\=&
-\frac{2\pi}{5}(b^5-a^5)
\int_0^\frac{\pi}{2}
(1-\cos^2\phi)\dif(\cos\phi)
\\=&
-\frac{2\pi}{5}(b^5-a^5)\cdot\left.\left(
\cos\phi-\frac{\cos^3\phi}{3}
\right)\right|_0^\frac{\pi}{2}
\\=&
\frac{2\pi}{5}(b^5-a^5)\cdot\frac{2}{3}
\\=&
\frac{4\pi}{15}(b^5-a^5)
\end{align*}

\paragraph{Άσκηση}
Υπολογίστε το:
\[
\int_0^R \int_{-\sqrt{R^2-y^2}}^{\sqrt{R^2-y^2}} \int_{-\sqrt{R^2-x^2-y^2}}^{\sqrt{R^2-x^2-y^2}} (x^2+y^2) \dif z \dif x \dif y
\]

Το προς ολοκλήρωση στερεό έχει θεωρηθεί κανονικό ως προς $z$.
%TODO Atreas Graph 07
Ολοκληρώνοντας στο τετράγωνο της επιφάνειες εισόδου \(z=-\sqrt{R^2-x^2-y^2}\) και εξόδου \(z=\sqrt{R^2-x^2-y^2}\), παίρνω \(\boxed{x^2+y^2+z^2}=R^2\).

Αλλά το \(\underbrace{\int_0^R \int_{-\sqrt{R^2-y^2}}^{\sqrt{R^2-y^2}} \Big(\quad \dif z\Big)\dif x \dif y}_{\mathclap{\text{$D_{xy}$: η προβολή του προς ολοκλήρωση στερεού στο $Oxy$ επίπεδο}}}
\).

Το $D_{xy}$ έχει θεωρηθεί κανονικό ως προς $x$.
%TODO Atreas Graph 08

\subparagraph{Τελικά} $S$ είναι το εξής:
%TODO Atreas Graph 09

Εφαρμόζω σφ. συντεταγμένες (για το $S$) και έχω:
\begin{align*}
I=&\int_0^\pi \int_0^\pi \int_0^R (r\cos\theta\sin\phi)^2+(r\sin\theta\sin\phi)^2\cdot r^2\sin\phi\dif r\dif\theta\dif\phi\\
=&\int_0^\pi \int_0^\pi \int_0^R r^4\sin^2\phi\dif r \dif\theta\dif \phi
\\
=& \text{ υπολογισμός όπως στην προηγούμενη άσκηση}
\end{align*}

\paragraph{Άσκηση}
Να υπολογιστεί το τριπλό ολοκλήρωμα:
\[
\int_{-R}^R\int_o^{\sqrt{R^2-y^2}} \int_{x^2+y^2}^{R\sqrt{x^2+y^2}}(z^2z)\dif z\dif x\dif y
\]
σε κυλινδρικές συντεταγμένες (\((x,y,z)\leftrightarrow(\rho,\theta,z)\). Το προς ολοκλήρωση στερεό είναι το:
\[
S =  \left\lbrace (x,y,z):\ -R\leq y \leq R, \ 0 \leq x \leq \sqrt{R^2-y^2},\ x^2+y^2\leq z \leq R\sqrt{x^2+y^2} \right\rbrace
\]

\subparagraph{}
\begin{itemize}
\item \(D_{xy}:=\) η προβολή του $S$ πάνω στο $Oxy$ επίπεδο, την οποία οπωσδήποτε πρέπει να σχεδιάσω.

Τότε ερμηνεύω το \(\underbrace{\int_{-R}^R \int_0^{\sqrt{R^2-y^2}}}_{\mathclap{\text{μαθηματικό μοντέλο του $D_{xy}$}}} \Big(\quad\Big) \dif \mathbf x \dif y\).

$D_{xy}$ κανονικό ως προς $x$:
%TODO Atreas Graph 01

\item \textbf{Κυλινδρικές συντεταγμένες}

\(
\begin{array}{l}
x=\rho\cos\theta\\
y=\rho\sin\theta\\
z=z
\end{array}
\). Έτσι:

\begin{align*}
I=& \int_{-\frac{\pi}{2}}^\frac{\pi}{2}\int_0^R \left(\int_{\rho^2}^{R\cdot\rho} \rho^2\cos^2\theta z \dif z\right)\rho\dif\rho\dif\theta\\
=&\int_{-\frac{\pi}{2}}\int_0^R \left.\rho^2\cos^2\theta\frac{z^2}{2}\right|_{\rho^2}^{R\rho} \dif\rho\dif\theta\\
=&
\int_{-\frac{\pi}{2}} \int_0^R \left(\frac{R^2\rho^2}{2}-\frac{\rho^7}{2}\right) \cos^2\theta \dif\rho\dif\theta
\\=&
\frac{1}{2}\int_{-\frac{\pi}{2}} \cos^2\theta \cdot \left( \frac{R^2\rho^6}{6}-\left. \frac{R^8}{8}\right|_0^R\right) \dif\theta
\\=&
\frac{1}{2}\int_{-\frac{\pi}{2}} \cos^2\theta \dif\theta R^8 \left(
\frac{1}{6}-\frac{1}{8}
\right)
\\=&
\left(\frac{R^8}{12}-\frac{R^8}{16}\right)
\int_{-\frac{\pi}{2}}
\left(
\frac{1+\cos(2\theta)}{2}
\right)\dif\theta
\\=&
\left(
\frac{R^8}{12}-\frac{R^8}{16}
\right)\frac{\pi}{2}
\end{align*}

\begin{attnbox}{Τύποι αποτετραγωνισμού}
\begin{align*}
\cos(2\theta)=&2\cos^2\theta-1\\=&1-2\sin^2\theta
\end{align*}
\end{attnbox}
\end{itemize}


\paragraph{Άσκηση}
Υπολογίστε τον όγκο του στερεού
\[
S =  \left\lbrace 
(x,y,z):\ \frac{x^2}{a^2}+\frac{y^2}{b^2}+\frac{z^2}{c^2}\leq 1
 \right\rbrace
\]
\(
a,b,c \neq 0
\)
\subparagraph{}
\[
V = \iiint_S 1 \dif x \dif y \dif z
\]

Θα χρησιμοποιήσω μετασχηματισμό "τύπου" σφαιρικών συντεταγμένων.

Έχω:
\[
\left|
\begin{array}{l}
x=a\cdot r\cos\overbrace{\theta}^{\mathclap{\text{αζιμουθιακή}}}\sin\overbrace{\phi}^{\mathclap{\text{πολική}}}\\
y=b\cdot r\sin\theta\sin\phi\\
z=c\cdot r\cos\phi
\end{array}
\right.
\]

\textbf{Τότε} \(
\frac{x^2}{a^2}+\frac{y^2}{b^2}+\frac{z^2}{c^2}\leq 1 \implies \boxed{r \leq 1}
\)

\textbf{Επίσης}
\(
\left|
\frac{\mathrm D(x,y,z)}{\mathrm D(r,\theta,\phi)}
\right|
=\dots=
\mathbf{abc}\cdot r^2\sin\phi
\)

\textbf{Τότε}
\begin{align*}
V =& \int_0^\pi \int_0^{2\pi} \int_0^1 1 \cdot abc \cdot r^2 \sin\phi\dif r \dif\theta\dif\phi
\\
=&
abc\cdot 2\pi \cdot \int_0^\pi \sin\phi \dif\phi \int_0^1 r^2\dif r = \boxed{\frac{4\pi}{3}abc}
\end{align*}


\section{Διανυσματικά Πεδία, Διαφορικοί τελεστές}
\subsection{Διανυσματικές συναρτήσεις πολλών μεταβλητών}

\paragraph{Ορ.}
Έστω $n,m>1$. Κάθε απεικόνιση \(\mathbf F\) ή \(\vec{F}:\ A \subseteq \mathbb{R}^n \to \mathbb R^m: \)
\[
\mathbb F(x_1,\dots,x_n)=\left( f_1(x_1,\dots,x_n),\ \dots,\ f_m(x_1,\dots,x_n)\right) 
\]
καλείται διανυσματική συνάρτηση πολλών μεταβλητών.

Στην παραπάνω οι \(f_1,\dots,f_m: A \subseteq \mathbb R^n \to \mathbb R\) είναι \textbf{πραγματικές} συναρτήσεις πολλών μεταβλητών που καλούται \textbf{συνιστώσες} ή \textbf{συντεταγμένες} συναρτήσεις του πεδίου \(\mathbb F \) (ως προς καρτεσιανό πάντα σύστημα συντ/νων).

\paragraph{Πεδίο ορισμού} είναι η συναλήθευση των πεδίων ορισμού των συνιστωσών συναρτήσεων \(f_1,\dots,f_m\)

\paragraph{π.χ}
\[
\underbrace{\mathbf F(x,y,z) }_{\mathbb R^3} = \underbrace{(x,y)}_{\mathbb R^2} \quad \text{προβολή του $P=(x,y,z)$ στο $Oxy$ επίπεδο}
\]

\paragraph{π.χ}
\[
\underbrace{\mathbf F(x,y)}_{\mathclap{\text{2 ανεξάρτητες μεταβλητές}}} = \underbrace{(x+2y,\ x-y,\ x+3y)}_{\mathclap{\text{Διάνυσμα με 3 συν/νες}}} = (x,x,x)+(2y,-y,3y)=
\underbrace{x(1,1,1)+y(2,-1,3)}_{\mathclap{
	\begin{matrix}
	\text{παραμετρική παράσταση}\\
	\text{επιπέδου που διέρχεται}\\
	\text{από το $(1,-2,3)$ και}\\
	\text{παράγεται από τα γραμμ.}\\
	\text{ανεξ. διαν. $(1,1,1)$ κ $(2,-1,3)$}
	\end{matrix}
}
	}
\]

\begin{itemize}
\item \(\mathbf{F}(x,y,z) = \left( x,y,z,\sqrt{1-x^2-y^2-z^2} \right)\)

Πρέπει: \(x^2+y^2+z^2 \leq 1\)

\[
\mathbf F: \text{μοναδιαία σφαιρική μπάλα του } \mathbb{R}^3 \to \mathbb R^4
\]

Γεωμετρικά είναι το τμήμα της υπερσφαίρας
\[
x^2+y^2+z^2+w^2=1,\ \text{για } w \geq 0
\]
\item 
\(\mathbf F(x,y) = \left( x,y,\sqrt{x^2+y^2} \right) \)
\[
z= \sqrt{x^2+y^2} \text{ (το τμήμα κώνου $z^2=x^2+y^2$ για $z \geq 0$)}
\]
\subparagraph{Γενικότερα} Κάθε διαν. συνάρτηση της μορφής \(\mathbf F(x_1,\dots,x_n)=
\left(
x_1,\dots,x_n,f(x_1,\dots,x_n)
\right)
 \), παριστάνει μία υπερ-επιφάνεια στο \(\mathbb R\).
\end{itemize}

Έστω \(
\mathbf F(x_1,\dots,x_n) = \left( 
f_1(x_1,\dots,x_n),\dots,f_m(x_1,\dots,x_n)
\right)
\) διανυσματική συνάρτηση, $P_0$ \textbf{σ.σ.} του πεδίου ορισμού της $\mathbf{F}$ και $\vec{\lambda} = (\lambda_1,\dots,\lambda_m)$. Τότε:

\begin{gather*}
\lim_{P\to P_0} \mathbf F(P) = \vec \lambda \iff\begin{cases}
\lim_{P\to P_0} f_1(x_1,\dots,x_n) &= \lambda_1 \\
\qquad \vdots \\
\lim_{P\to P_0} f_m(x_1,\dots,x_n) &= \lambda_m
\end{cases}
\\
\mathbf F \text{ συνεχής στο } P_0 \iff f_1,\dots,f_m \text{ είναι συνεχείς στο } P_0
\end{gather*}

Έστω ότι υπάρχουν οι μερικές παράγωγοι των συνιστωσών συναρτήσεων:
\[
\pd{f_i(P)}{x_j}, \quad\begin{matrix}
i=1,\dots,m\\ j=1,\dots,n
\end{matrix}
\]
και είναι συνεχείς σε μια περιοχή σημείου $P$, για κάθε $P$ στο πεδίο ορισμού.

Ορίζουμε τη μερική παράγωγο:\[
\pd{\mathbf F(P)}{x_j} = 
\left(
\pd{f_1(P)}{x_j},\dots,\pd{f_m(P)}{x_j}
\right)
\]

Η $\mathbf F$ είναι διαφορίσιμη στο $P$ και ο $m \times n$ \textbf{ΙΑΚΩΒΙΑΝΟΣ} πίνακας \(J_{\mathbf F}=
\left[
\begin{matrix}
\pd{f_1(P)}{x_1} & \cdots & \pd{f_1(P)}{x_n}\\
\vdots & \ddots & \vdots \\
\pd{f_m(P)}{x_1} & \cdots & \pd{f_m(P)}{x_n}
\end{matrix}
\right]
= \left[
\begin{matrix}
\nabla f_1\\
\nabla f_2\\
\vdots \\
\nabla f_m
\end{matrix}
\right]
\)
καλείται \textbf{παράγωγος} της \(\mathbf F \) στο $P$, συμβολικά και $\mathbf F'(P)$ ή $\mathrm D \mathbf F(P)$. Για σημεία $Q$ "κοντά" στο $P$ ισχύει:
\[
\mathbf F(Q) - \mathbf F(P) \approx \underbrace{\mathbf F'(P) \cdot (Q-P)}_{\mathclap{
		\dif \mathbf F_P(Q) \rightarrow \text{ διαφορικό του $\mathbf F$ στο $\mathbb R$}
		}}
\]

\paragraph{π.χ} \(
\mathbf F(x,y) = \big(
x+2y^2,\ x-3y,\ 2x^2+y+1
\big)
\)

\[
\mathbf F'(x,y)= \left[
\begin{matrix}
1&4y\\ 1 & -3 \\ 4x & 1
\end{matrix}
\right]_{3 \times 2}
\]

\subsection{Διανυσματικά πεδία}
\paragraph{Ορ.}
Κάθε διανυσματική συνάρτηση πολλών μεαβλητών της μορφής:
\[
\mathbf F \text{ ή } \vec F: A\subseteq \mathbb R^n \to \mathbb R^n
\]
καλείται διανυσματικό πεδίο.

\paragraph{Ερμηνεία} Σε ΚΑΘΕ ΣΗΜΕΙΟ του χώρου $\mathbb R^b$ ασκείται μια ΔΥΝΑΜΗ με τύπο $
\mathbf F(P)
$

\subsubsection{Παραδείγματα}
\paragraph{Α. Γραμμικά πεδία}
Αν $A_{n \times n}$ είναι πίνακας πραγματικός, τότε:
\[
\mathbf{F} : \mathbb R^n \to R^n: \mathbf F(x_1,\dots,x_n) = A_{n \times n} \cdot \left[
\begin{matrix}
x_1 \\ \vdots \\ x_n
\end{matrix}
\right]_{n\times 1}
\]
\subparagraph{π.χ} \(
\mathbf F(x,y)=(x+y,\ 2x-3y)
\). Είναι γραμμικό?

ΝΑΙ, διότι:
\[
\mathbf F(x,y) = \left(
\begin{matrix}
1&1\\2&-3
\end{matrix} \right) \cdot \left( \begin{matrix}
x\\y
\end{matrix}
\right)
\]


\begin{itemize}
\item \(\mathbf F(x,y,z) = (2x+y+z+1,\ 3x-y-z,\ -x+2y+4) \). Είναι γραμμικό?

ΟΧΙ, διότι γράφεται ως:
\[
\mathbf F(x,y,z) = \left(
\begin{matrix}
2&1&1 \\ 3 & -1 & -1 \\
-1 & 2 & 0
\end{matrix}
\right) \cdot \left(
\begin{matrix}
x\\y\\z
\end{matrix}
\right) + \left(
\begin{matrix}
1 \\ 0 \\ 4
\end{matrix}
\right)
\]

Γίνεται όμως γραμμικό αν μετατεθεί κατά $(1,0,4)$. Τέτοια πεδία λέγονται \textbf{αφφινικά}.
\end{itemize}

\paragraph{Β. Πεδία κλίσεων}
\begin{itemize}
\item Έστω $f: A \subseteq \mathbb R^n \to \mathbb R$ διαφορίσιμο βαθμωτό πεδίο (ή αριθμητικό πεδίο, ή πραγματική συνάρτηση $n$-μεταβλητών).

Τότε το πεδίο
\begin{align*}
\mathbf F: A \subseteq \mathbb R^n \to \mathbb R^n:\ \mathbf F(P) &= \nabla f(P) \\
\text{ή } \mathbf F(x_1,\dots,x_n) &= \left(
\pd{f(x_1,\dots,x_n)}{x_1},\dots,\pd{f(x_1,\dots,x_n)}{x_n}
\right)
\end{align*}
καλείται πεδίο κλίσεων της $f$.

H $f$ καλείται (βαθμωτό) ΔΥΝΑΜΙΚΟ του πεδίου $\mathbf F$.

\subparagraph{π.χ.}
Έστω $\vec r = (x_1,\dots,x_n)$ και $r = |\vec r | = \sqrt{x_1^2 + \dots + x_n^2}$.

Ορίζω συνάρτηση \textbf{ΔΥΝΑΜΙΚΟΥ}
\[
f(\vec r) = \frac{c}{r} \qquad \Big(\;  c \text{ σταθερά} \; \Big)
\]

\textbf{Ορίζω}
\begin{align*}
\mathbf E &= -\nabla f = - \nabla \left( \frac{c}{r}\right) \\
&= -c \nabla \left(\frac{1}{r}\right) = -c \frac{-\nabla r}{r^2} = c \frac{\nabla r}{r^2}\\
\nabla r &= \nabla \left(
\sqrt{x_1^2+\dots+x_n^2}
\right) = \nabla \left(
(x_1^2+\dots+x_n^2)^{\frac{1}{2}}
\right) \\
&= \left(
\left[
(x_1^2+\dots+x_n^2)^\frac{1}{2}
\right]_{x_1},\ \dots,\ \left[
(x_1^2+\dots+x_n^2)^\frac{1}{2}
\right]_{x_n}
\right) \\
&= \left(
\frac{x_1}{\sqrt{x_1^2+\dots+x_n^2}},
\frac{x_2}{\sqrt{x_1^2+\dots+x_n^2}},
\dots,
\frac{x_n}{\sqrt{x_1^2+\dots+x_n^2}}
\right) \\ &=
\frac{1}{\sqrt{x_1^2+\dots+x_n^2}} \big(
x_1,x_2,\dots,x_n
\big) = \frac{\vec r}{r}
\intertext{Άρα}
\Aboxed{
	\mathbf E &= \frac{c}{r^2} \cdot \frac{\vec r}{r}
	}
\end{align*}

Αν $q$ ακίνητο σημειακό φορτίο (μαζί με το πρόσημό του) στην αρχή των αξόνων, τότε:
\[
\mathbf E_q = q\mathbf E
\]
είναι η ένταση του ηλεκτρικού πεδίου που παράγει το φορτίο $q$.

%TODO Atreas Graph 01
\begin{align*}
\mathbf F &= \mathbf Q \cdot \mathbf E_q \\
&= \frac{c\cdot q \cdot Q}{r^2} \cdot \frac{\vec r}{r}
\end{align*}

(Για $c=k=\frac{1}{4\pi \varepsilon_0}$ σταθερά \textlatin{Coulomb}), τότε \(
\mathbf F
\) δύναμη \textlatin{Coulomb}.
\end{itemize}

\paragraph{Γ. Κεντρικά διανυσματικά πεδία}
Έστω $f: \mathbb{R} \to \mathbb{R}$ πραγματική συνάρτηση.

\(\vec r = (x_1,\dots,x_n)\) το διάνυσμα θέσης σημείου $P \in \mathbb R^n$ και $r=|\vec r|$. Τότε το πεδίο $\mathbf F: A \subseteq \mathbb R^n - \left\lbrace (0,\dots,0) \right\rbrace \to \mathbb R^n:$
\[
\boxed{
	\mathbf F(\vec r) = f(r) \cdot \frac{\vec r}{r}
	}
\] καλείται κεντρικό πεδίο διότι \textbf{ο ΦΟΡΕΑΣ} των εικόνων $\mathbb F(\vec r)$ ΤΑΥΤΙΖΕΤΑΙ με το \textbf{ΦΟΡΕΑ} του διανύσματος θέσης $\vec r$.

\[
\left.
\begin{matrix}
\text{το πεδίο \textlatin{Coulomb}}\\
\text{το βαρυτικό πεδίο}
\end{matrix}
\right\rbrace \text{ΕΙΝΑΙ ΚΕΝΤΡΙΚΑ ΠΕΔΙΑ}
\]

\subsubsection{Παράσταση πεδίων}
\paragraph{Α} Με χρήση Η/Υ

\paragraph{Β}
Διακριτοποιώ και σε κάθε σημείο $P$, ζωγραφίζω το διάνυσμα $\mathbf F(P)$ με αφετηρία το $P$.
\subparagraph{π.χ} \(\mathbf F(x,y) = (-y,\ x) \)
\[
\begin{cases}
\mathbf F(1,0) &= (-0,1) = (0,1) \\
\mathbf F(0,1) &= (-1,0) \\
\mathbf F(-1,0) &= (-0,-1) = (0,-1) \\
\mathbf F(0,-1) &= (1,0)
\end{cases}
\]

\paragraph{Γ (Με χρήση διανυσματικών γραμμών)}

\subparagraph{Παραδοχή}
Σε κάθε σημείο $P$, οι τιμές του πεδίου $\mathbf F(P)$ είναι ταχύτητες κατά την κίνηση κάποιου υλικού σημείου στο χώρο. Οι τροχιές (δηλαδή οι καμπύλες) κίνησης καλούνται \textbf{διανυσματικές} γραμμές του πεδίου.

\textbf{Δηλ.} αν $\mathbf r$ είναι καμπύλη κίνησης, τότε:
\begin{align*}
\mathbf{v} &= \mathbf{F}(\mathbf r)
\\ \implies
\mathbf{r}'(t) &= \mathbf{F} \left(
\mathbf r(t)
\right) \quad t \in [a,b]
\end{align*}

π.χ Έστω $F(x,y) = (-y,x)$.

\textbf{Έστω} $\mathbf{r}(t) = \Big(x(t), t(t) \Big)$

\textbf{Τότε λόγω (1) έχω:}
\begin{align*}
\mathbf{r}'(t) &= \mathbf{F}\left(
\mathbf r(t)
\right) \implies
\left(
x'(t),y'(t) \right) = \left(
-y(t),x(t)
\right) \\
&\implies \begin{cases}
x' &= -y\\
y' &= x
\end{cases}
\implies \frac{y'}{x'} = -\frac{x}{y} \implies xx'+yy' = 0
\\ &\implies \boxed{x^2+y^2=c^2} \text{ ($c \in \mathbb{R}$ αυθαίρετη σταθερά)}
\end{align*}

\subsection{Διαφορικοί τελεστές}
Έστω :

\begin{itemize}
\item \(
V_n(E) = \left\lbrace f: E \subseteq \mathbb R ^n \to \mathbb R \right\rbrace
\) ο χώρος όλων των \textbf{διαφορίσιμων} πραγματικών συναρτήσεων πολλών μεταβλητών (ή βαθμωτών ή αριθμητικών πεδίων)

\item \(
\mathbf{V}_{n,m}^{(E)} = \left\lbrace \mathbf F: E \subseteq \mathbb R ^{n} \to \mathbb R ^{m} \right\rbrace
\) ο χώρος των ΔΙΑΦΟΡΙΣΙΜΩΝ διανυσμ. συναρτήσεων πολλώ μεταβλητών και
\item 
\(
\mathbf V_n(E) = \left\lbrace \mathbf F: E \subseteq \mathbb R ^{(n)} \to \mathbb R ^\mathbf{n} \right\rbrace
\) ο χώρος όλων των \textbf{ΔΙΑΦΟΡΙΣΙΜΩΝ διανυσμ. πεδίων}
\end{itemize}

\subsubsection{Διαφορικός τελεστής κλίσης}
\textbf{Συμβολίζω:}
\[
\nabla := \left(
\pd{}{x_1}, \pd{}{x_2}, \dots, \pd{}{x_n}
\right)
\]
τον \textbf{τελεστή} κλίσης που δρα ως εξής:
\begin{itemize}
\item

\begin{gather*}
\nabla: \underbrace{V_n(E)}_{\mathclap{\text{αριθμ. πεδίο διαφορ.}}} \to \underbrace{\mathbf V_n(E)}_{\mathclap{\text{διανυσμ. πεδίο διαφορ.}}}:\\
\nabla f = \left(
\pd{}{x_1}, \dots, \pd{f}{x_n}
\right), \quad E \subseteq \mathbb R ^n
\end{gather*}

\item
\begin{gather*}
\nabla: \underbrace{\mathbf V_n(E)}_{\mathclap{\text{διαν. πεδίο διαφορ.}}}
\to
\underbrace{\mathbf V_{n,n^2}(E)}_{\mathclap{\text{διανυσμ. συνάρτ. διαφορ. $E \subseteq \mathbb R ^n \to \mathbb R ^{n^2}$}}} \\
:
\\
\nabla \mathbf F = \left(
\pd{}{x_1},\dots,\pd{}{x_n}
\right)\mathbf F = \left(
\pd{\mathbf F}{x_1},\dots,\pd{\mathbf F}{x_n}
\right)
\end{gather*}

\subsubsection{Απόκλιση διανυσμ. πεδίου}
Έστω \(\mathbf F: E \subseteq \mathbb R ^n \to \mathbb R ^n \):

\(
\forall P=(x_1,\dots,x_n) \in E
\) έχουμε:
\[
\mathbf F(P) = \left(
f_1(P),\dots,f_n(P)
\right)
\] όπου $f_1,\dots,f_n$ είναι διαφορίσιμα αριθμητικά πεδία.
\end{itemize}

\paragraph{Ορ.}
Καλούμε ΑΠΟΚΛΙΣΗ του πεδίου $\mathbf F$ στο σημείο P, συμβολικά $
\mathrm{div} \mathbf{F}(P)$ να είναι ο \textbf{ΑΡΙΘΜΟΣ}:
\[
\mathrm{div} \mathbf F(P) = \pd{f_1(P)}{x_1}+\pd{f_2(P)}{x_2}+\dots+\pd{f_n(P)}{x_n}
\]

Παρατηρώ ($\bullet$ εσωτερικό γινόμενο) ότι:
\begin{align*}
\mathrm{div}\; \mathbf{F}(P) &= \nabla \bullet \mathbf{F}(P)  =
\\ &=
\left(
\pd{}{x_1},\dots,\pd{}{x_n}
\right) \bullet
\left(
f_1(P),\dots,f_n(P)
\right) \\
&=
\pd{f_1(P)}{x_1} + \dots + \pd{f(P)}{x_n}
\end{align*}

Στο εξής θα χρησιμοποιούμε το συμβολισμό:
\[
\mathrm{div} \, :\overset{\text{συμβολισμός}}{=} \nabla \bullet
\]

Έτσι η \textbf{απόκλιση} μπορεί να θεωρηθεί ως ένας \textbf{διαφορικός τελεστής που δρα ως εξής}:

\[
\nabla \bullet
 :
 \underbrace{\mathbf V_n(E)}_{\mathclap{\text{διαφορίσιμο διαν. πεδίο}}}
 \to
 \underbrace{V_n(E)}_{\mathclap{\text{διαφορίσιμο βαθμωτό πεδίο}}}
 \]
 
 \paragraph{π.χ}
 \(
 \mathbf F(x,y) = (x+2y^2,\ -x^2+y^2)
 \)
 
 \[
 \nabla \bullet \mathbf F(x,y) = \pd{(x+2y^2)}{x}+\pd{(-x^2+y^2)}{y} = \boxed{1+2y}
 \]

\paragraph{}

Η απόκλιση είναι ΓΡΑΜΜΙΚΟΣ τελεστής, δηλαδή:
\[
\nabla \bullet \big(
a\mathbf F +b \mathbf G
\big) = a\nabla\bullet\mathbf F + b\nabla\bullet\mathbf G
\]
και επίσης ισχύει:

Αν \(
f: E \subseteq \mathbb R ^n \to \mathbb R
\) διαφορίσιμο αριθμητικό πεδίο και \(\mathbf F: E \subseteq \mathbb R^n \to \mathbb R^n\) διαφορίσιμο διαν. πεδίο, τότε:
\[
\nabla \bullet (f\mathbf F) = \nabla f\bullet \mathbf F + f \nabla \bullet \mathbf F
\]

\paragraph{Ορ.}
Ένα διανυσμ. πεδίο \(\mathbf F: E \subseteq \mathbb R^n\) με $\nabla \bullet \mathbf F = 0 \quad \forall P \in E$ καλείται \textbf{ασυμπίεστο} ή \textbf{σωληνοειδές}
\subsubsection{Περιστροφή διανυσμ. πεδίου στον $\mathbb R ^2$ και $\mathbb R ^3$}
Έστω \(\mathbb{F}:E \subseteq \mathbb R ^{\mathbb 3} \to R^{\mathbb 3} \).

\(\mathbf F(x,y,z) = \left(
f_1(x,y,z),\ f_2(x,y,z),\ f_3(x,y,z)
\right)  \) διαφορίσιμο διαν. πεδίο επί του \(E\).

\paragraph{Ορ.}
Καλούμε περιστροφή του πεδίου \(\mathbf F \) στο σημείο \(P=(x,y,z) \in E\) να είναι το \underline{διάνυσμα}:
\[
\mathrm{rot}\; \mathbf{F}(P) = \left|
\begin{matrix}
\vec i & \vec j & \vec k\\
\pd{}{x} & \pd{}{y} & \pd{}{z} \\
f_1(P) & f_2(P) & f_3(P)
\end{matrix}
\right| = \left(
\pd{f_3(P)}{}-\pd{f_2(P)}{z},\ \pd{f_1}{z}-\pd{f_3}{x},\ \pd{f_2}{x}-\pd{f_1}{y}
\right)
\]

Παρατηρούμε ότι:
\[
\mathrm{rot}\; := \nabla x
\]
και αυτό το συμβολισμό θα χρησιμοποιούμε στο εξής.

Έτσι η περιστροφή \( \nabla f \) είναι ένας τελεστής:
\[
\nabla x: \underbrace{\mathbf V_B(E)}_{\mathclap{\text{χώρος διαν. πεδίων}}}
\to \underbrace{\mathbf V_b(E)}_{\mathclap{\text{χώρος διαν. πεδίων}}}
\quad \mathbf G = \nabla \times \mathbf F
\]

\begin{itemize}
\item Η περιστροφή είναι γραμμικός τελεστής:
\[
\nabla \times (a\mathbf F + b \mathbf G) = a \nabla \times \mathbf F + b \nabla \times \mathbf G
\]
\item Αν \( f:E \subseteq \mathbb R ^n \to \mathbb R  \) διαφορίσιμο αριθμητικό πεδίο, τότε:
\[
\nabla times (f\mathbf F) = \nabla f \times \mathbf F + f \nabla \times \mathbf F
\]
\end{itemize}

\paragraph{Ορ.}
Αν \[
\nabla \times \mathbf F(P) = 0 \quad \forall P \in E
\]
τότε το πεδίο \( \mathbf F \) καλείται \textbf{ΑΣΤΡΟΒΙΛΟ}

\paragraph{Ορ.}
Αν \( \mathbf{F}(x,y)= \left(
P(x,y),\ Q(x,y)
\right) \) διαφορίσιμο πεδίο επί συνόλου \( E\subseteq \mathbb R \).

Τότε ορίζουμε:
\[
\nabla \times \mathbf F(x,y) = \pd{Q(x,y)}{x} - \pd{P(x,y)}{y}
\]

\paragraph{Ιδιότητες}
\begin{enumparen}
\item Κάθε πεδίο κλίσεων είναι
\[
\text{ΑΣΤΡΟΒΙΛΟ}
\]

Με άλλα λόγια, αν \( f:E \subseteq \mathbb R ^3 \to \mathbb R  \) αριθμητικό πεδίο
με συνεχείς μερικές παραγώγους 2\textsuperscript{ης} τάξης, τότε:
\[
\nabla \times (\nabla f) = \vec 0
\]

\subparagraph{Απόδ.}
Έστω \( f \) όπως παραπάνω, και \(P=(x,y,z)\in E \)
\begin{gather*}
\nabla f = \left( f_x,f_y,f_z \right) \\
\nabla \times (\nabla f) = \left| \begin{matrix}
\vec i & \vec j & \vec k\\
\pd{}{x} & \pd{}{y} & \pd{}{z}\\
f_x & f_y & f_z
\end{matrix}\right|  = \left(
f_{zy} - f_{yz},\ f_{xz} - f_{zx},\ f_{yx}-f_{xy}
\right) = (0,0,0) \quad \forall P \in E
\end{gather*}

\item Έστω \( f: E \subseteq \mathbb R ^n \) αριθμ. πεδίο, και \( P=(x_1,\dots,x_n) \in E \) \\
και υπάρχουν όλες οι μερικές παράγωγοι 2\textsuperscript{ης} τάξης του \(f\).

Τότε ορίζουμε:
\begin{align*}
\nabla \cdot (\nabla f) = \nabla \cdot \nabla f &= \left(
\pd{}{x_1},\dots,\pd{}{x_n}
\right) \bullet \left(
f_{x_1},\dots,f_{x_n}
\right) \\
&=
\pd{f_{x_1}}{x_1} + \pd{f_{x_2}}{x_2} + \dots + \pd{f_{x_n}}{x_n} \\
&=
f_{x_1^2}+f_{x_2^2}+\dots+f_{x_n^2}
\end{align*}

Συμβολίζουμε τον τελεστή
\[
\nabla^2 := \nabla \cdot \nabla
\]
και τον καλούμε τελεστή \textlatin{Laplace} ή Λαπλασιανή. Ισχύει δε:
\[
\nabla^2 : \underbrace{V_n(E)}_{\mathclap{\text{αριθμ. πεδίο}}}
\to \underbrace{V_n(E)}_{\mathclap{\text{αριθμ. πεδίο}}}:\quad
\nabla^2 f = \pd[2]{f}{x_1}+\pd[2]{f}{x_2}+\dots+\pd[2]{f}{x_n}
\]

\item Τα κεντρικά διανυσματικά πεδία είναι \textbf{αστρόβιλα}.
(βλέπε άσκ. παρακαλώ)

\item Έστω \( \mathbf F(x,y,z) = \left(
P(x,y,z),\ G(x,y,z),\ R(x,y,z)
\right) \) διαφορίσιμο πεδίο. Τότε
\[
\mathbf F' = J_{\mathbf F} = \left[
\begin{matrix}
P_x & P_y & P_z \\
Q_x & Q_y & Q_z \\
R_x & R_y & R_z
\end{matrix}
\right]
\]

\begin{align*}
\text{Ίχνος}\; J_{\mathbb F} :&= \text{άθροισμα όλων των στοιχείων της κύριας διαγωνίου του } J_\mathbf{F} \\
&= P_x+Q_y+R_z = \nabla \cdot \mathbb F
\end{align*}

Αν \( \mathbf F \) ασυμπίεστο \(\iff \) Ίχνος του \(J_{\mathbf F}\) ισούται με μηδέν.


\begin{itemize}
\item Επίσης:
\(\mathbf F \) αστρόβιλο \(\iff J_{\mathbf F}\)  ένας συμμετρικός πίνακας
\end{itemize}

\item 
Έστω \(\mathbf F: E \subseteq \mathbb R ^3 \to \mathbb R ^3 \) διαν. πεδίο με συνεχείς μερικές παραγώγους 2\textsuperscript{ης} τάξης. Τότε: \[
\nabla \cdot (\nabla \times \mathbf F ) =0
\]

Έστω \(\mathbf F=(P,Q,R) \). Τότε:
\[
\nabla \times \mathbf F = (R_y-Q_z,\ P_z-R_x,\ Q_x-P_y)
\]
\textbf{Άρα:}
\begin{align*}
\nabla \cdot (\nabla \times \mathbf F) &= \left(
\pd{}{x},\pd{}{y},\pd{}{z}
\right) \cdot (R_y-Q_z,P_z-R_x,Q_x-P_y)
\\ &= R_{yx} - Q_{zx} +R_{zy} - R_{xy} + Q_{xz} -P_{yz} = 0
\end{align*}

\item Έστω \(\mathbf F: E \subseteq \mathbb R ^3 \to \mathbb R ^3 \) διαν. πεδίο με συνεχείς μερικές παραγώγους επί ΚΥΡΤΟΥ χωρίου \(E\), και επιπλέον υποθέτουμε ότι το \(\mathbf F \) Είναι ΑΣΥΜΠΙΕΣΤΟ στο \(E\).

Τότε υπάρχει διαφορίσιμο διαν. πεδίο \(\mathbf G \) έτσι ώστε:
\[
\mathbf F = \nabla \times \mathbf G
\]

Το \(\mathbf G  \) καλείται ΔΙΑΝΥΣΜΑΤΙΚΟ ΔΥΝΑΜΙΚΟ του \(\mathbf F \).

Ένας τύπος για το \( \mathbf G \) είναι:
\[
\mathbf G = - \vec r \times \int_0^1 \mathbf F(t\vec r)\dif\vec r
\]
(\( \vec r \) διάνυσμα θέσης) αλλά δεν είναι μοναδικός. Πράγματι, και το πεδίο:
\[
\mathbf G + \nabla \cdot \mathbf F \quad 
\text{($f:E \subseteq \mathbb R ^n \to \mathbb R $ οποιοδήποτε διαφορίσιμο πεδίο)}
\]
είναι επίσης διανυσματικό δυναμικό του \(\mathbf F \) διότι \( \nabla \times (\mathbf G + \nabla f) = \nabla \times \mathbf G + \cancelto{0}{\nabla \times (\nabla f)} = \nabla \times \mathbf G = \mathbf F \)

\textbf{Ισχύει και το αντίστροφο.}

\begin{align*}
\text{Αστρόβιλο πεδίο } &\xrightarrow[\text{συνθήκη}]{+} \text{έχει βαθμωτό δυναμικό}, 
\exists f: E \mathbb R ^n \to \mathbb R \quad \mathbf F = \nabla f\\
\text{Ασυμπ. πεδίο } &\rightarrow \text{έχει διαν. δυναμ.}, \exists \mathbf G:\ \mathbf F:\nabla \times \mathbf G
\end{align*}

\item Αν \( \mathbf F: e \subseteq \mathbb R ^3 \to \mathbb R ^3 \) διαν. πεδίο με συνεχείς μερικές παραγ. επί κυρτού χωρίου \( E \), τότε υπάρχει διαφορίσιμο αριθμ. πεδίο \(f: E \subseteq \mathbb R ^3 \to \mathbb R  \) και διαφορ. διαν. πεδίο \(\mathbf G: E \subseteq \mathbb R ^3 \to \mathbb R ^3 \):
\[
\mathbf F = \underbrace{\nabla f}_{\mathclap{\text{αστρόβιλο}}} + \underbrace{\nabla \times \mathbf G}_{\mathclap{\text{ασυμπίεστο}}}
\]

\end{enumparen}

\subsection{Ασκήσεις}
\paragraph{Άσκ.}
Έστω \( f: \mathbb R \to \mathbb R  \) είναι παραγωγίσιμη συνάρτηση και \(
u: \mathbb R ^3 \to \mathbb R :\ u=u(x,y,z)
\) έχει συνεχείς μερικές παραγώγους.\\
ΝΔΟ το πεδίο \(\nabla u \times \nabla f(u) \) είναι αστρόβιλο.

\subparagraph{}
\begin{align*}
\nabla u &= (u_x,u_y,u_z)\\
\nabla f(u) &= f'(u) \cdot \nabla u \text{(παραγώγιση σύνθετης συνάρτησης)}
\end{align*}
άρα:
\begin{align*}
\nabla u \times \nabla f(u) &= \nabla u \times f'(u) \cdot \nabla u \\
&= \left|
\begin{matrix}
\vec i & \vec j & \vec k \\
u_x & u_y & u_z \\
f'(u)u_x & f'(u)u_y & f'(u)u_z
\end{matrix}
\right| &= f'(u) \cdot \
\left|
\begin{matrix}
\vec i & \vec j & \vec k \\
u_x & u_y & u_z \\
u_x & u_y & u_z
\end{matrix}
\right| = \vec 0.
\end{align*}

\paragraph{Άσκ.}
Έστω \(\vec r = (x,y,z),\ r = |\vec r|=\sqrt{x^2+y^2+z^2}\)

\( f:\mathbb R ^+ \to \mathbb R  \) παραγωγίσιμη συνάρτηση και

\( \mathbf F(x,y,z) := \mathbf F(\vec r) = \frac{f(r)}{r} \cdot \vec r \)
διανυσμ. πεδίο επί του \( \mathbb R ^3 - \left\lbrace (0,0,0) \right\rbrace \)

\begin{itemize}
\item ΝΔΟ \(
\nabla \cdot \mathbf F = \frac{rf'(r)+2f(r)}{r}
\).

Για ποιες \(f\) το πεδίο \( \mathbf F \) είναι ασυμπίεστο στο \( \mathbb R ^3-\left\lbrace (0,0,0) \right\rbrace \)

\item ΝΔΟ \( \mathbf F \) αστρόβιλο στον \( \mathbf \mathbb R ^3 - \left\lbrace (0,0,0) \right\rbrace \)

\subparagraph{}
Από τύπο θεωρίας έχω:
\begin{align}
\nabla \cdot \left(
\frac{f(r)}{r}\cdot \vec r
\right) = \nabla \left(
\frac{f(r)}{r} \cdot \vec r + \frac{f(r)}{r} \nabla \vec r
\right)
\end{align}

\begin{itemize}
\item \begin{align*}
\nabla \cdot \vec r &= \left(\pd{}{x}.\pd{}{y},\pd{}{z} \right) \cdot (x,y,z) \\
&= \pd{x}{x} + \pd{y}{y} + \pd{z}{z} = 3.
\end{align*}

\item \begin{align*}
\nabla \left(
\frac{f(r)}{r}\right) &\overset{\text{παραγ. πηλίκου}}{=} \frac{\nabla f(r) \cdot r - f(r) \nabla r}{r^2} \\
&= \frac{r \cdot f'(r) \cdot \nabla r - f(r) \nabla r }{r^2} = \frac{rf'(r)-f(r)}{r^2}\nabla r
\end{align*}

\item \begin{align*}
\nabla r &= \nabla \left(
(x^2+y^2+z^2)^\frac{1}{2}
\right) \\ &= \left(
(x^2+y^2+z^2)_x^\frac{1}{2},(x^2+y^2+z^2)_y^\frac{1}{2},(x^2+y^2+z^2)_z^\frac{1}{2}
\right) \\
&= \left(
\frac{x}{\sqrt{x^2+y^2+z^2}}+\frac{y}{\sqrt{x^2+y^2+z^2}}+\frac{z}{\sqrt{x^2+y^2+z^2}}
\right) \\ &=
\frac{\vec r}{r}.
\end{align*}

\end{itemize}
\textbf{Άρα:}
\[
\boxed{
	\nabla \left(
	\frac{f(r)}{r} = \frac{rf'(r)-f(r)}{r^2} \cdot \frac{\vec r}{r}
	\right)
	}
\]

Αντικαθιστώ στην (1) και έχω:
\begin{align*}
\nabla \cdot \mathbf F &= \frac{rf'(r)-f(r)}{r^3} \cdot \vec r \cdot \vec r + \frac{f(r)}{r}\cdot 3 \\
&= \frac{rf'(r)-f(r)}{r^3} \cdot r^2 + 3 \frac{f(r)}{r} \\
&= \frac{rf'(r)+2f(r)}{r}
\end{align*}

\begin{align*}
\nabla \cdot \mathbf F &= 0 \implies  \\
rf'(r) + 2f(r) &= 0 \implies \\
f'(r) + \frac{2}{r}f(r)&= 0
\end{align*}
(ομογενής συνήθης γραμμική δ.ε.)

\textbf{Άρα:}
\[
f(r) = ce^{-\int \frac{2}{r}\dif r} =ce^{-2\ln x} = \frac{c}{r^2}
\]

\begin{itemize}
\item Από θεωρία έχουμε:
\begin{align}
\nabla \times \left(
\frac{f(r)}{r}\cdot \vec r
\right) &= \nabla \left(
\frac{f(r)}{r}\times \vec r + \frac{f(r)}{r}\nabla \times \vec r
\right) 
\nonumber
\\ &=
\frac{rf'(r)-f(r)}{r^2} \cdot \frac{\vec r}{r} \times \vec r + \frac{f(r)}{r} \nabla \times \vec r
\end{align}

\item \[
\nabla \times \vec r = \left|
\begin{matrix}
\vec i & \vec j & \vec k\\
\pd{}{x} & \pd{}{y} &\pd{}{z} \\
x&y&z
\end{matrix}
\right| = (0,0,0) = \vec 0,
\]
άρα αντικαθιστώ στη (2).

Έουμε ότι \(\nabla \times \mathbb F = 0 \) επειδή \( \vec r \times \vec r = \vec 0 \) και \( \nabla \times \vec r = \vec 0 \quad \forall P \in \mathbb R ^3-\left\lbrace (0,0,0) \right\rbrace \)

\paragraph{Άσκ.}
Έστω \( \mathbf F(x,y,z) = \left(P(x,y,z),Q(x,y,z),R(x,y,z)  \right) \) διαν. πεδίου του οποίου οι συνιστώσες \( P,Q,R \) έχουν μερικές παραγώγους 2\textsuperscript{ης} τάξης. \\
ΝΔΟ:

\[
\nabla^2 \mathbf F = \left(
\nabla^2 P,\ \nabla^2 Q,\ \nabla^2 R
\right)
\]

\subparagraph{}
\begin{align*}
\nabla^2 \mathbf F &\overset{\text{εξ' ορισμού}}{=} \nabla \bullet \nabla \mathbf F
\\ &= \left( \pd{}{x},\pd{}{y},\pd{}{z} \right) \cdot \big( \mathbf F_x,\ \mathbf F_y,\ \mathbf F_z \big) \\ &= \pd{\mathbf F_x}{x} + \pd{\mathbf F_y}{y} + \pd{\mathbf F_z}{z}
\\ &= \pd{}{x}(P_x,Q_x,R_x) + \pd{}{x}(P_y,Q_y,R_y) + \pd{}{z}(P_z,Q_z,R_z) \\
&= (P_{xx},Q_{xx},R_{xx})+(P_{yy},Q_{yy},R_{yy})+(P_{zz},Q_{zz},R_{zz})\\
&= (
P_{xx} + P_{yy}+P_{zz},\ Q_{xx} + Q_{yy}+Q_{zz},\ R_{xx} + R_{yy}+R_{zz}
)
\\ &= \left(
\nabla^2 P, \nabla^2 Q,\nabla^2 R
\right)
\end{align*}
\end{itemize}

\end{itemize}

\paragraph{Άσκ.}
Έστω \( u:\mathbb R ^3\to \mathbb R : u = u(x,y,z) \) έχει συνεχείς μερικές παραγώγους 2\textsuperscript{ης} τάξης. ΝΔΟ
\begin{itemize}
\item \( \nabla \cdot (u\nabla u) = |\nabla u|^2 + u \nabla^2 u \)
\item Το πεδίο \( u\nabla u \) είναι αστρόβιλο στον \( \mathbb R ^3 \).
\end{itemize}

\subparagraph{}
\begin{itemize}
\item \( \nabla u = (u_x,u_y,u_z) \), άρα
\begin{align*}
u\nabla u &= u(u_x,u_y,u_z) \\
&= (uu_x,uu_y,uu_z)
\end{align*}

\( \nabla \cdot (u\nabla u) = \left( \pd{}{x},\pd{}{y},\pd{}{z} \right)  \cdot (uu_x,uu_y,uu_z)  = \pd{(uux)}{x} + \pd{uu_y}{y} + \pd{uu_z}{z}
= (u_xu_x+uu_{xx})+(u_yu_y+uu_{yy})+(u_zu_z+uu_{zz}) = (u_x^2+u_y^2+u_z^2)+u \underbrace{(u_{xx}+u_{yy}+u_{zz})}_{\mathclap{\text{Λαπλασιανή της $u$}}}
= |\nabla u|^2 + u\nabla^2 u
\)
\item Αρκεί ΝΔΟ \( \nabla \times (u \nabla u) = \vec 0 \) παντού στο \( \mathbb R  \).
\begin{itemize}
\item Είτε μέσω ορισμού: \[
\nabla \times (u\nabla u) = \left|
\begin{matrix}
\vec i & \vec j & \vec k\\
\pd{}{x} & \pd{}{y} & \pd{}{z} \\
uu_x&uu_y&uu_z
\end{matrix}
\right| = \dots = (0,0,0) = \vec 0
\]
\item Είτε με χρήση της ιδιότητας: \(
\boxed{\nabla \times(f \mathbf F) = \nabla f \times \mathbf F + f\nabla \times \mathbf F}\quad f=u, \quad \mathbf F = \nabla u
\)
\[
\nabla \times (u\nabla u) = \cancelto{\vec 0}{\nabla u \times \nabla u} + \cancelto{0}{u\nabla x (\nabla u)} = \vec 0
\]
\end{itemize}
\end{itemize}

\section{Επικαμπύλια ολοληρώματα}
%TODO

\paragraph{Θ1}
Έστω \( \mathbf F: D \subseteq \mathbb R ^n \to \mathbb R ^n \) συνεχές διανυσμ. πεδίο σε \textbf{τόπο} \( D \), τότε:
\[
\mathbf F \text{ συντηρητικό στο } D \iff \underbrace{\oint_\gamma \mathbf F \cdot \dif \mathbf r = 0}_{\mathclap{\text{για κάθε \textbf{κλειστή}, λεία καμπύλη εντός του} D}}
\]

\textbf{Σημ.}: Στο εξής, αν \( \gamma  \) είναι κλειστή και λεία καμπύλη, γράφουμε \( 
\oint_\gamma \mathbf F \cdot \dif \mathbf r \left(
\text{αντί } \int_\gamma \mathbf F \cdot \dif \mathbf r
\right)
 \)
 \subparagraph{Απόδ.}
 %TODO Atreas Graph 01
 
 Έστω \( A,B \in D \) \textbf{ΤΥΧΑΙΑ} σημεία και έστω \(\gamma_1, \gamma_2\) \textbf{τυχαίες} καμπύλες με αρχή \( A \) κ πέρας \( B \) κ ίχνη όπως στο σχήμα.
 
 Παρατηρούμε ότι \( \gamma = \gamma_1-\gamma_2 \) είναι κλειστή καμπύλη:
 \begin{align*}
 \oint_\gamma \mathbf F \cdot \dif \mathbf r&=
 \oint_{\gamma_1-\gamma_2} \mathbf F \cdot \dif\mathbf r
 =\oint_{\gamma_1}\mathbf F \cdot \dif \mathbf r_{\gamma_1}+
 \oint_{\gamma_2}\mathbf F \cdot \dif \mathbf r_{\gamma_2} \\
 &= \int_{\gamma_1} \mathbf F \cdot \dif \mathbf r_{\gamma_1}
 - \int_{\gamma_2} \mathbf F \cdot \dif \mathbf r_{\gamma_2}
 \end{align*}
 
 Άρα:
 \begin{align} \label{eq:oint1p} \boxed{
 \oint_\gamma \mathbf F \cdot \dif \mathbf r = \int_{\gamma_1} \mathbf F \cdot \dif \mathbf r_{\gamma_1}
 - \int_{\gamma_2} \mathbf F \cdot \dif \mathbf r_{\gamma_2}
}
 \end{align}
 
 Διότι αν \( \mathbf F \) συντηρ. \( \xRightarrow{\text{ορ.}} \int_{\gamma'} \mathbf F \cdot \dif \mathbf r \) ανεξάρτητο του δρόμου \( \implies \text{δεξι μελος } \eqref{eq:oint1p} = 0 \xRightarrow{\eqref{eq:oint1p}} \oint_\gamma \mathbf F \cdot \dif \mathbf r = 0 \).
 
 Αντίστροφα: Αν \( \oint_\gamma \mathbf F \cdot \dif \mathbf r = 0 \implies \text{δεξί μέλος \eqref{eq:oint1p} ισούται με μηδέν } \implies \int \mathbf F \cdot \dif \mathbf r \text{ ανεξάρτητο δρόμου} \implies \mathbf F \) συντηρητικό.
 
\paragraph{Θ2.}
Έστω \( \mathbf F:D \subseteq \mathbb R ^n\to\mathbb R ^n \) συνεχές διαν. πεδίο επί \textbf{τόπου \( D \)}. Τότε:

\(
\mathbf F \text{ συντηρητικό στο } D \iff
\mathbf{F} \text{ είναι πεδίο κλίσεων στο $D$, δηλ. } \mathbf{F} = \nabla f \text{ για κάποια } f:D \subseteq \mathbb R ^n \to \mathbb R \text{ με συνεχείς μερ. παραγ. στο } D
\)

\subparagraph{Απόδ.}
``\( \Leftarrow \)'' Έστω \( \mathbf F \) πεδίο κλίσεων, δηλ. \( \mathbf F = \nabla f \).

\textbf{Τότε} αν \( \gamma \) οποιαδήποτε καμπύλη με παραμετροποίηση \( \mathbf r \), λεία και κλειστή έχουμε:
\begin{align*}
\oint_\gamma\mathbf F \cdot \dif \mathbf r = \oint \nabla f \cdot \dif \mathbf r \overset{\text{ορισμός}}{=} \\
\int_a^b \nabla f \left(\mathbf r(t)\right) \cdot \mathbf r'(t)\dif t = \\
\int_a^b (f \circ \mathbf r)'(t) \dif t = \\
= f\left(\mathbf r(b) \right) - \mathbf{f} \left(\mathbf r(a) \right) = 0
\end{align*},
διότι η \( \gamma \) κλειστή καμπύλη, άρα \( \mathbf r(a) = \mathbf r(b) \). 
Άρα από Θ1 το \( \mathbf F \) συντηρητικό πεδίο.

\subparagraph{Σημείωση:} 
Η συνάρτηση βαθμωτού δυναμικού συντηρητικού πεδίου \( \mathbf F \) βρίσκεται ως εξής:
\[ \boxed{
f(P) - f(A) = \int_A^P \mathbf F \cdot \dif \mathbf r
}
\]
όπου \(  A \in D \) σημείο του \( \mathbf D \) που εσείς επιλέγετε,\\
\( (x_1,\dots,x_n) = P \) \textbf{τυχαίο σημείο του } \( D  \) και:
\[
\int_A^P \mathbf F \cdot \dif \mathbf r \text{ συμβολισμός που δηλώνει ότι το επικαμπύλιο ολοκλ. είναι ανεξάρτητο του δρόμου}
\]

\paragraph{Τελικά}
\(
\mathbf F
\) συντηρητικό σε τόπο \( D \iff \oint_\gamma \mathbf F \cdot \dif \mathbf r = 0 \) για κάθε κλειστή λεία καμπύλη εντός του \( D \)
\[
\iff \mathbf F = \nabla f
\]

\paragraph{\( ^*_* \)Π\( ^*_* \)}
Αν \( D \) είναι \textbf{ΑΠΛΑ ΣΥΝΕΚΤΙΚΟΣ ΤΟΠΟΣ} του \( \mathbb R ^n \) και \( \mathbf F:D \subseteq \mathbf R^n \to \mathbf R^n \) έχει συνεχείς μερικές παραγώγους στο \( \mathbf D \), τότε:
\[
\mathbb F \text{ συντηρητικό στο } D \iff \mathbf F \text{ αστρόβιλο στο } D
\]

\begin{itemize}
\item Αν \( D \) \textbf{ΔΕΝ} είναι \textbf{απλά} συνεκτικό, τότε \textbf{ΔΕΝ} συνεπάγεται κατά ανάγκη ότι ατρόφιβο πεδίο είναι συντηρητικό. Μπορεί ΝΑΙ, μπορεί ΟΧΙ.
\end{itemize}

\subsection{Θεώρημα Green \emph{στο επίπεδο}}
Έστω \( \gamma \) είναι απλή, \textbf{κλειστή} και τμηματικά λεία καμπύλη στο \( \mathbb R^2 \). Τότε αυτή χωρίζει το \(\mathbb R^2 \) σε δύο χωρία, που καλούμε \textbf{εσωτερικά της \( \gamma \)}, και ένα μη φραγμένο \textbf{χωρίο που καλούμε εξωτερικό της \( \gamma \)}.

Επίσης μια τέτοια (\textbf{κλειστή}) καμπύλη λέμε ότι είναι ΘΕΤΙΚΑ ΠΡΟΣΑΝΑΤΟΛΙΣΜΕΝΗ Ανν η φορά διγραφής της είναι η \textbf{αντιωρολογιακή}.

%TODO Atreas Graph 01
\paragraph{Θ (Green)}
Έστω \( \gamma \) είναι μια απλή, \textbf{ΚΛΕΙΣΤΗ}, τμημ. λεία και ΘΕΤΙΚΑ \textbf{ΠΡΟΣΑΝΑΤΟΛΙΣΜΕΝΗ} καμπύλη και
\[
\mathbb F(x,y)= \left(P(x,y),\ Q(x,y) \right)
\]
διανυσματικό πεδίο με ΣΥΝΕΧΕΙΣ ΜΕΡΙΚΕΣ παραγώγους \textbf{ΠΑΝΩ κ στο ΕΣΩΤΕΡΙΚΟ της \( \gamma \)}.

Αν συμβολίσουμε \( \boxed{ \gamma := \partial D } \) (δηλ. η \( \gamma \) είναι το σύνορο του εσωτερικού \( D \) της \( \gamma \)), τότε:
\[
\boxed{\oint_{\partial D} = \iint_D(Q_x-P_y)\dif x\dif y}
\], ή ισοδύναμα:
\[
\boxed{
	\underbrace{\oint_{\partial D}\mathbf F \cdot \dif \mathbf r}_{\text{Το έργο του πεδίου $\mathbf F$ κατά μήκος του συνόρου $\partial D$}}
	\underbrace{=}_{\text{ισούται με}} 
	\underbrace{\iint_D \nabla \times \mathbf F(x,y) \dif x\dif y}_{\text{τη συνολική περιστροφή του πεδίου $\mathbf F$ στο εσωτερικό $D$}}
	}
\]

Υπό τις προϋποθέσεις του Θ. \textlatin{Green}, αν ισχύει:
\[
\nabla \times \mathbf F(x,y) = \vec{0}
\]
πάνω κ στο εσωτερικό της συνοριακής καμπύλης \( \partial D \), τότε
\[
\oint_{\partial D} \mathbf F \cdot \dif \mathbf r
\]

\paragraph{ΠΡΟΣΟΧΗ!!!}
Αν το πεδίο \( \mathbf F \) \textbf{ΔΕΝ} έχει συνεχείς μερικές παραγώγους πάνω κ στο εσωτερικό της \( \partial D \), τότε το παραπάνω θεώρημα ΔΕΝ ισχύει, αλλά έχουμε το ακόλουθο:
\paragraph{Θ (Παραμόρφωσης δρόμου)}
Έστω \( \gamma_1,\gamma_2 \) είναι απλές, κλειστές, τμημ. λείες και ΘΕΤΙΚΑ προσανατολισμένες καμπύλες, έτσι ώστε:

\(\boxed{\text{π.χ η $\gamma_2$ να βρίσκεται στο εσωτερικό της $\gamma_1$}}\)

Αν \( \mathbf F(x,y) = \left(P(x,y),\ Q(x,y)\right) \) έχει ΣΥΝΕΧΕΙΣ ΜΕΡΙΚΕΣ ΠΑΡΑΓΩΓΟΥΣ πάνω σε \textbf{ΦΡΑΓΜΕΝΟ} χωρίο \( R \) με σύνορο
\[
\partial R = \gamma_1 \cup \gamma_2,
\]
τότε: \[
\oint_{\gamma_1} \mathbf F \cdot \dif \mathbf r_{\gamma_1} = \oint_{\gamma_2} \mathbf F \cdot \dif \mathbf r_{\gamma_2} + \iint_R(Q_x-P_y)\dif x\dif y
\]

Γενικεύοντας το Θ. παραμόρφωσης δρόμων για περισσότερες καμπύλες, προκύπτει το:
\paragraph{Γενικευμένο Θ. \textlatin{Green}}
%TODO Atreas Graph 03
\[
\oint_\varGamma \mathbf F \cdot \dif \mathbf r_\varGamma = 
\sum_{j=1}^n
\oint_{\gamma_j} \mathbf F \cdot \dif \mathbf r_{\gamma_j} + \iint R (Q_x-P_y)\dif x\dif y,
\] όπου:
\begin{itemize}
\item \( \varGamma,\gamma_1,\dots,\gamma_n \) απλές, κλειστές, τμ. λείες θετικά προσαν.
\item \( \gamma_1,\dots,\gamma_n \) στο εσωτερικό της \( \varGamma \)
\item κάθε \( \gamma_i \) στο εξωτερικό κάθε άλλης καμπύλης \( \gamma_\mu \),\quad \( \forall \mu=1,\dots,i-1,i+1,\dots,n \)
\item \( \mathbf F=(P,Q) \) έχει συνεχείς μερ. παραγ. επί του γραμμοσκιασμένου χωρίου \( R \)
\end{itemize}

\subparagraph{Εφαρμογή:}
Ξέρουμε ότι \[
\mathrm E(D) = \iint_d 1 \dif x \dif y
\]

Έστω \( \mathbf F=(P,Q) \) έ.ώ:
\[ Q_x-P_y=1 \]
\textbf{π.χ} \( \mathbf F = (0,x) \) \\
ή \( \mathbf F = (-y,0) \),

τότε από Θ. \textlatin{Green} έχω:
\begin{align*}
\oint_{\partial D} \mathbf F \cdot \dif \mathbf r &= \iint_D (Q_x-P_y)\dif x\dif y
\\ &= \iint_D 1\dif x\dif y
\\ &= \mathrm E(D)
\end{align*}

\subsection*{}
Το Θ. Απόκλισης \boxed{\text{στο επίπεδο} \mathbb R ^2} ΕΚΤΟΣ ΥΛΗΣ

Παρασκευή 3-5μμ μάθημα με Ατρέα.

\subsection*{}
Υπολογίστε τη μάζα καλώδίου πυνκότητας \( \rho(x,y,z)=x^2+y^2+z^2 \), που απλώνεται κατά μήκος της τεθλασμένης γραμμής \( P_0 \to P_1 \to P_2 \) με \( P_0=(1,1,1),\ P_1=(2,2,0),\ P_2=(2,2,2) \)

Έχουμε:

\begin{align*}
M &\overset{\text{συμβ.}}{=} \underbrace{\int_\varGamma \rho \dif s}_{\mathclap{\text{επικαμπύλιο 1ου είδους}}} 
\\ &= \int_{\gamma_1+\gamma_2}\rho\dif s
\\ &= \int_{\gamma_1}\rho\dif s + \int_{\gamma_2}\rho\dif s,
\end{align*} 
όπου \( \gamma_1 := \overrightarrow{P_0P_1}, \ \gamma_2 := \overrightarrow{P_1P_2} \)

Γενικά, θυμάμαι ότι \begin{align*}
\int_\varGamma \rho \dif s = \int_a^b \rho\left(\mathbf r(t) \right)\left|\mathbf r'(t)\right| \dif t
\end{align*}
όπου \( \mathbf r = \mathbf r(t) \) η παραμετροποίηση της καμπύλης \( \varGamma \).

\begin{itemize}
\item Για το \( \int_{\gamma_1} \rho\dif s := \int_{\overrightarrow{P_0P_1}} \rho\dif s \)
\item Μια παραμετροποίση του \( \overrightarrow{P_0P_1} \) είναι η
\begin{align*}
\mathbf r_1(t) &= (\text{αρχή}) + t(\text{πέρας} - \text{αρχή}) \quad \forall t\int[0,1]
\\ \mathbf r_1(t) &= \overrightarrow{OP_0} + t (\overrightarrow{OP_1} - \overrightarrow{OP_0})
\\ &= (1,1,1) + t \left((2,2,0)-(1,1,1)\right) \\
  \Aboxed{
	\mathbf r_1(t) &= (1+t,1+t,1-t) \quad \forall t \in [0,1]
	}
\end{align*}
\item \( \mathbf r_1'(t) = (1,1,-1) \)
\item \( \left|\mathbf r_1'(t)\right| = \sqrt{1^2+1^2+(-1)^2} = \sqrt{3} \)
\item Χρειάζομαι το \( \rho\left(\mathbf{r}_1(t)\right) \), δηλ.:
\[
\rho\left(\mathbf{r}_1(t)\right) = (1+t)^2+(1+t)^2+(1-t)^2
\implies \dots \implies \int_{\gamma_1} = \dots
\]
\item Ομοίως για \( \gamma_2 \)
\end{itemize}

\paragraph{Άσκ.}
Υπολογίστε το έργο του πεδίου
\[
\mathbf F(x,y) = \left(x^2-y,\ x+y^2 \right)
\]
επί του τμήματος της παραβολής \( y=x^2+1 \) με αρχή το σημείο \( A=(0,1) \) κ πέρας το σημείο \( B=(3,10) \)

Στο πρόχειρο βλέπω:
\[
\int_\gamma \mathbf F \bullet \dif \mathbf r
= \int \mathbf F\left(\mathbf{r}(t)\right) \cdot \mathbf r'(t)\dif t
\]

Έχω επικαμπύλιο 2\textsuperscript{ου} είδους
\[
W = \int_\gamma \mathbf F \cdot \dif \mathbf r
\]
Θα παραμετροποιήσουμε την καμπύλη \( \gamma \) του τμήματος παραβολής
\[
\mathbf r(t) = \left(t,\ t^2+1\right), \quad t \in [0,3]
\]
\[
\mathbf r'(t) = (1,2t)
\]

Έτσι από τύπο έχω:
\begin{align*}
W &= \int_0^3 \mathbf F \left(\mathbf{r}(t)\right) \cdot \mathbf r'(t) \dif t
\\ &= \int_0^3 \left(
t^2-(t^2+1),\ t+(t^2+1)^2
\right) \underbrace{\cdot}_{\mathclap{\text{εσωτερικό γινόμενο!!!}}} (1,2t) \dif t 
\\ &= 
\int_0^3 \left[
-1+2t\left(t+(t^2+1)^2\right)
\right] \dif t
\\ &= \cdots
\end{align*}

\paragraph{Άσκ.}
Δίνεται το πεδίο \( \mathbf{F}(x,y,z) = (x^2,y^2,z^2) \)
\begin{enumparen}
\item Είναι το \( \mathbf{F} \) συντηρητικό στο πεδίο ορισμού του?
\item Αν ναι, βρείτε τη συνάρτηση βαθμωτού δυναμικού του \( f \), αν \( f(0,0,0)=0 \)
\item Ποιές οι ισοδυναμικές επιφάνειες του \( \mathbf F \)?
\end{enumparen}

\begin{enumparen}
\item Π.Ο. του \( \mathbb F \) είναι το \( \mathbb R ^3 \) (διότι οι συνιστώσες συναρτήσεις είναι πολυώνυμα).

Άρα το πεδίο ορισμού είναι ΑΠΛΑ ΣΥΝΕΚΤΙΚΟ αφού είναι το \( \mathbb R^3 \)

\[
\boxed{
\begin{array}{r}
\mathbf F	\text{ έχει συνεχείς μερικές παραγ.} \\
	\text{σε ΑΠΛΑ συνεκτικό σύνολο}
	\end{array}
	 +
\mathbf F \text{ αστρόβιλο στο } D \implies
\mathbf F \text{ συντηρητικό στο } D
}
\]

Έτσι \( \nabla \times \mathbf F = \left|
\begin{matrix}
\vec i & \vec j & \vec k \\
\pd{}{x} & \pd{}{y} & \pd{}{z} \\
x^2 & y^2 & z^2
\end{matrix} 
\right| = (0,0,0) = \vec 0 \quad \forall (x,y,z) \in \mathbb R^3 \implies \mathbf F \) αστρόβιλο σε απλά συνεκτικό σύνολο, το \( \mathbb{R}^3 \xRightarrow{\text{θεωρία}} \mathbf F \) συντηρητικό στο \( \mathbb R ^3 \)

\item
Αφού \( \mathbf F \) συντηρ. \( \xRightarrow{θ.} \mathbf F \) πεδίο κλίσεων στο \( \mathbb R^3 \), δηλ. \( \mathbf F = \nabla f \).

Έτσι:
\[
f(P) - f(A) = \int_A^P \mathbf F \cdot \dif \mathbf r \quad \text{(τύπος)}
\]
όπου \( A \) σημείο δικής σας επιλογής και \( P=(x,y,z) \) τυχαίο σημείο του \( \mathbb R ^3 \).

Για \( A=(0,0,0),\ P=(x,y,z) \), έχω:
\begin{align*}
f(x,y,z) - \cancelto{0}{f(0,0,0)} &= \int_0^P \mathbf F \cdot \dif \mathbf r \implies \\
\implies f(x,y,z) &= \int_O^P \mathbf F \cdot \dif \mathbf r,\quad O=(0,0,0)
\end{align*}

Εφόσον \( \mathbf F \) συντηρητικό στο \( \mathbb R^3 \), το \( \int_\gamma \mathbf F \cdot \dif r \) είναι ανεξάρτητο του δρόμου, άρα μπορώ να επιλέξω όποιον τύπο καμπύλης εγώ επιθυμώ, αρκεί να έχει αρχή την \(\mathrm O(0,0,0) \) και πέρας \( P=(x,y,z) \). Επιλέγω να εργασθώ στο ευθ. τμήμα \( \overrightarrow{OP} \). Έχω:

\begin{align*}
\mathbf r(t) &= \text{αρχή} + t(\text{πέρας} - \text{αρχή}) \\
&= (0,0,0) + t\left((x,y,z)-(0,0,0)\right) \quad \forall t\in [0,1] \\
\implies \Aboxed{\mathbf r(t) &= (tx,ty,tz) \quad \forall t\in [0,1]} \\
\mathbf r'(t) &= (x,y,z)\quad \forall t \in [0,1],\ \text{οπότε} \\
\mathbf{F} \left(\mathbf{r}(t)\right) &= \left((tx)^2,(ty)^2,(tz)^2\right).\ \\
\intertext{Έτσι}
f(x,y,z) &= \int_0^1 \mathbf{F}\left(\mathbf{r}(t)\right) \cdot \mathbf{r}'(t)\dif t \\
&+ \int_0^1 \left(t^2x^2,t^2y^2,t^2,z^2\right) \underbrace{\cdot}_{\mathclap{\text{εσ. γιν}}}(x,y,z) \dif t \\
&= \int_0^1 (t^2x^2,t^2y^2,t^2z^2)\dif t\\
&= (x^2+y^2+z^2)\int_0^1 t^2\dif t\\
\implies f(x,y,z) &= \frac{x^3+y^3+z^3}{3}
\end{align*}
\item
Για να βρω τις ισοδυναμικές επιφάνειες, \\
Θέτω:
\begin{align*}
f(x,y,z) &= c,\quad c \in \mathbb R \\
\frac{x^3+y^3+z^3}{3} &= c \\
\implies x^3+y^3+z^3 &= c' \quad (c'=3c) %TODO plot?
\end{align*}
\end{enumparen}


\paragraph{Άσκ.}
Υπολογίστε το έργο του πεδίου \( \mathbf F(x,y) = (x^2,xy) \) επί του τριγώνου με πλευρές \( y=1+x,\ y=1-x,\ y=0 \)
%TODO Atreas Graph 05
Εφόσον έχω καμπύλη απλή, κλειστή, τμ. λεία, ΘΕΤΙΚΑ προσανατολισμένη, χρησιμοποιώ Θ. \textlatin{Green} (αν και μπορώ και με ορισμό)
\begin{align*}
W = \int_{\partial D} \mathbf F \cdot \dif \mathbf r &= \iint_D (Q_x-P_y)\dif x\dif y
\\ &=\iint_D y\dif x \dif y = 
\\ &= \int_0^1 \int_{y-1}^{1-y} y\dif x\dif y
\\ &= \int_0^1 \left. xy\right|_{y-1}^{-y+1} \dif y
\\ &= \int_0^1 \left(y(1-y)-y(y-1)\right)\dif y
\\ &= 2\int_0^1(y-y^2)\dif y
\\ &= \left. y^2 - \frac{2y^3}{3} \right|_0^1 = 1-\frac{2}{3} = \frac{1}{3}
\end{align*}

\paragraph{**Άσκ.**}
Δίνεται το πεδίο \( \mathbf{F}(x,y) = \left(
\frac{y}{x^2+y^2},\frac{-x}{x^2+y^2}
\right) \)
\begin{enumparen}
\item ΝΔΟ το \( \mathbf F \) αστρόβιλο στο \( \mathbb R ^2 - \left\lbrace 0,0 \right\rbrace \)
\item Υπολογίστε την κυκλοφορία του \( \mathbf F \) επί του κύκλου \( (x-2)^2+(y-2)^2=1 \) με θατική φορά.
\item Υπολογίστε το έργο του \( \mathbf F \) κατά μήκος οποιασδήποτε κλειστής, λείας καμπύλης \( \gamma \) που περιέχει το \( (0,0) \) με θετική φορά διαγραφής
\item Είναι το \( \mathbf F \) συντηρητικό στο \( \mathbb R ^2 - \left\lbrace (0,0) \right\rbrace \)?
\end{enumparen}

\subparagraph{}
\begin{enumparen}
\item \begin{align*}
\nabla \times \mathbf F &= \pd{}{x}\left(-\frac{x}{x^2+y^2}\right) - \pd{}{y}\left(\frac{y}{x^2+y^2}\right)
\\ &= \frac{-(x^2+y^2)+x\cdot2x}{(x^2+y^2)^2} - \frac{(x^2+y^2)-y\cdot2y}{(x^2+y^2)^2}
\\ &= \frac{x^2+y^2}{(x^2+y^2)^2}-\frac{x^2-y^2}{(x^2+y^2)^2}
\\ &= 0 \quad \forall (x,y) \in \mathbb R ^2 - \left\lbrace (0,0) \right\rbrace \text{όπου στο $(0,0)$ το πεδίο $\mathbf F$ \textbf{ΔΕΝ} ορίζεται καν.}
\end{align*}
Άρα \( \mathbf F \) αστρόβιλο στο \( \mathbb R ^2 - \left\lbrace (0,0) \right\rbrace \)
\item
%TODO Atreas Graph 06
Σκέφτομαι να εφαρμόσω Θ. \textlatin{Green}:
\[
\oint_{\partial D}\mathbf F \cdot \dif \mathbf r = \iint_D \nabla \times \mathbf{F} \dif x \dif y
\]

Παρατηρώ ότι πάνω και στο εσωτερικό του κύκλου \( \partial D \), το πεδίο \( \mathbf F \) έχει συνεχείς μερικές παραγώγους
(πηλίκα πολυωνύμων) και έτσι μπορώ να χρησιμοποιήσω Θ. \textlatin{Green}:
\begin{align*}
\oint_{\partial D}\mathbf F \cdot \dif \mathbf r &= \iint_D \nabla \times \mathbf{F}(x,y) \dif x \dif y
\\ &= 0
\end{align*}
διότι στο ερ. (α) είδαμε ότι \( \nabla \times \mathbf{F}(x,y) = \vec 0 \ \forall (x,y)\in\mathbb R ^2-\left\lbrace (0,0) \right\rbrace  \)
άρα και πάνω κ στο εσωτερικό του κύκλου

\item
%TODO Atreas Graph 07
Εδώ, εφόσον στο ΕΣΩΤΕΡΙΚΟ κάθε τέτοιας καμπύλης υπάρχει τουλάχιστον μία ανωμαλία (εδώ στο σημείο \( (0,0) \)), το θ. \textlatin{Green} \textbf{ΔΕΝ} μπορεί να εφαρμοσθεί.

Χρησιμοποιώ γενικευμένο θ. \textlatin{Green}.

Με κέντρο το \( (0,0) \) και οποιοδήποτε \( \varepsilon > 0 \) (ακτίνα) ορίζω κύκλο \( \gamma: x^2+y^2=\varepsilon^2 \) με θετική φορά (και φυσικά με κάποιο \( \varepsilon > 0 \) ώστε ο δίσκος \( x^2+y^2 \leq \varepsilon^2 \) να είναι εξ' ολοκλήρου μέσα - στο εσωτερικό δηλαδή - στη \( \gamma \)).

Στο γραμμοσκιασμένο χωρίο \( R \) το πεδίο έχει συνεχείς μερικές παραγώγους κ ισχύει το γενικευμένο θ. \textlatin{Green}:
\begin{align*}
\oint_{\partial D} \mathbf F \cdot \dif \mathbf r &=
\oint_{x^2+y^2=\varepsilon^2} \mathbf F \cdot \dif \mathbf r + \cancelto{
	\mathclap{ 0
	\text{ διότι από (α) το $\mathbf F$ αστρόβιλο στο } \mathbb R^2 \supset \mathbb R^2-\left\lbrace (0,0) \right\rbrace }
	}{\iint_R \nabla \times \mathbf{F}(x,y) \dif x \dif y}
	\\ &= \oint_{x^2+y^2=\varepsilon^2} \mathbf F \cdot \dif \mathbf R
\end{align*}
Αλλά μια παραμετροποίηση του κύκλου \( x^2+y^2=\varepsilon^2 \) είναι η:
\begin{align*}
\mathbf r(t) &= \left(
\varepsilon \cos t, \varepsilon \sin t
\right),\quad t\int[0,2\pi)
\\
\mathbf{r}'(t) &= \left(-\varepsilon \sin t , \varepsilon \cos t \right)
\intertext{Άρα:}
\oint_{x^2+y^2=\varepsilon^2}\mathbf{F}\cdot \dif\mathbf r &\overset{\text{τύπος}}{=}
\int_0^{2\pi}\left(
\frac{\varepsilon \sin t}{\varepsilon^2},\frac{-\varepsilon \cos t}{\varepsilon^2}
\right) \underbrace{\bullet }_{\mathclap{\text{εσωτ. γιν}}}
\left(
-\varepsilon \sin t, \varepsilon \cos t
\right) \dif t
\\ &= \int_0^{2\pi} \left( -\sin^2t-\cos^2t \right) \dif t = -2\pi.
\end{align*}
\item
\begin{attnbox}{}
Αστρόβιλο + Απλά συνεκτικό \( \implies \) Συντηρητικό
\end{attnbox}

Αν ήταν συντηρητικό στο \( \mathbb R -\left\lbrace (0,0) \right\rbrace \), θα έπρεπε \textbf{ΓΙΑ ΚΑΘΕ} κλειστή, λεία καμπύλη \( \gamma \) εντός του \( \mathbb R ^2 - \left\lbrace (0,0) \right\rbrace \) να έχουμε:
\[
\oint_{\gamma} \mathbf F \cdot \dif \mathbf r = 0,
\]
άτοπο λόγω ερωτήματος (γ). Δηλ. το \( \mathbf F \) \textbf{ΔΕΝ} είναι συντηρητικό στο \( \mathbb R ^2 - \left\lbrace (0,0) \right\rbrace \)
\end{enumparen}

\paragraph{Άσκ.}
Υπολογίστε το \( \int_\gamma (x^2-y)\dif x + (x+y^2)\dif y \) κατά μήκος του τμήματος έλλειψης \[
\frac{x^2}{a^2}+\frac{y^2}{b^2} = 1,
\]
όπως στο σχήμα.

\begin{attnbox}{}
Συμβολισμός: \( \mathbf F(x,y) = \left(x^2-y,\ x+y^2\right) \)
\end{attnbox}

%TODO Atreas Graph 08
\begin{itemize}
\item Έστω μια παραμετροποίηση του τμήματος της έλλειψης της μορφής:
\[
\mathbf r(t) = \left(
a\cos t,\ -b\sin t
\right), \quad t \in \left[0,\frac{\pi}{2}\right]
\]
Αντικαθιστώ στον τύπο όπου \( \left|\begin{array}{l}
x = a\cos t \\ y = -b\sin t
\end{array}\right. \) και έχω:
\begin{align*}
I &= \int_0^\frac{\pi}{2}
\left[
(a\cos t)^2+b\sin t
\right] \dif (a\cos t) + \left[
a\cos t + (-b\sin t)^2
\right] \dif (-b\sin t)
\\ &= \int_0^\frac{\pi}{2} \left[
-a\left(
a^2\cos^2t+b\sin t
\right) \sin t -b \left(
a\cos t + b^2\sin^2 t
\right) \cos t \dif t
\right]
\\ &= \int_0^\frac{\pi}{2} -a^3\cos^2t\sin t\dif t + \int_0^\frac{\pi}{2} -b^3\sin^2t\cos t\dif t - \int_0^\frac{\pi}{2} ab\dif t
\\ &= \frac{-ab\pi}{2} + a^3\int_0^\frac{\pi}{2} \cos^2t\dif(\cos t) - b^3\int_0^\frac{\pi}{2} \sin^2t\dif(\sin t)
\\ &= -\frac{\pi ab}{2}+\left. a^3\frac{\cos^3t}{3} \right|_0^\frac{\pi}{2} -\left. b^3\frac{\sin^3t}{t} \right|_0^\frac{\pi}{2}
\\ &= -\frac{\pi ab}{2} - \frac{a^3}{3} - \frac{b^3}{3}
\end{align*}
\end{itemize}

\section{Παραμετροποιημένες επιφάνειες, επιφανειακά ολοκληρώματα και εφαρμογές}
Ποιο είναι το εμβαδόν σφαίρας?

Ποιο είναι το εμφαδόν ελλειψοειδούς?

Ποιο είναι το εμβαδόν μιας επιφάνειας καμπυλωτής?

Το κεφάλαιο λοιπόν έρχεται να δώσει απάντηση σε αυτά τα ερωτήματα.

\paragraph{Ορ.}
Έστω \( D \in \mathbb R ^2 \) τόπος του \( \mathbb R ^2 \), και \( \mathbf r: D \subset \mathbb R ^2 \to \mathbb R ^3 : \mathbf r(u,v)=\left(
x(u,v),\ y(u,v),\ z(u,v)
\right) \) ΣΥΝΕΧΗΣ διανυσμ. συνάρτηση δύο μεταβλητών. Τότε καλούμε την εικόνα αυτής \( \mathbf r(D) \) \textbf{ΕΠΙΦΑΝΕΙΑ} \( \varSigma \) σε παραμετρική μορφή.

Αν η \( \mathbf r \) είναι 1-1, τότε η επιφάνεια \( \varSigma \) καλείται ΑΠΛΗ. Στο εξής ασχολούμαστε μόνο με απλές επιφάνειες.

\subparagraph{π.χ.}
\begin{align*}
\mathbf r(u,v) &= \left( u,v,\sqrt{R^2-u^2-v^2} \right) & \text{(άνω ημισφαίριο κέντρου $(0,0,0)$ κ ακτίνας $R>0$)}
\\ \mathbf r(u,v) &= \left( u,v,u^2+v^2 \right)
& \text{(κυκλικό παραβολοειδές)}
\\ \mathbf r(u,v) &= \left( R\cos u, R\sin u, v \right)
& \text{κύλινδρος } \forall u \in [0,2\pi), v \in \mathbb R 
\end{align*}

Έστω \( \varSigma \) είναι μια \boxed{\text{απλή}} επιφάνεια με παραμετροποίηση
\[
\mathbf r : D \subset \mathbb R ^2 \mathbb R ^3 : \mathbf r(u,v) = \left(x(u,v),y(u,v),z(u,v)\right)
\]

Αν \( u=u_0 \) σταθεροποιημένο, τότε η
\[
\mathrm C_{u_0} := \left(x(u_0,v),y(u_0,v),z(u_0,v)\right)
\]
είναι \textbf{καμπύλη} πάνω στην επιφάνεια \( \varSigma \),
και για \( v=v_0 \) σταθεροποιημένο, η
\[
\mathrm C_{v_0} = \left(x(u,v_0),y(u,v_0),z(u_v0)\right)
\]
είναι μια άλλη \textbf{καμπύλη} πάνω στην επιφάνεια \( \varSigma \).
Έτσι, αν \( Q_0 = \mathbf r(u_0,v_0) \in \varSigma \), τότε το \( Q_0 \) ορίζεται ως η τομή των
\textbf{καμπύλων} \( \mathrm C_{u_0} \) και \( \mathrm C_{v_0} \), ακριβώς όπως το σημείο
\( P_0 = (u_0,v_0) \in D  \) ορίζεται ως τομή των ευθειών \( u=u_0 \) και \( v=v_0 \). Για διάφορες
τιμές των \( u,v \) προκύπτει λοιπόν ένα δίκτυο παραμετρικών γραμμών πάνω στην επιφάνεια
\( \varSigma \) και οι \( \dots \) καλούνται ακτινικές γραμμές στην επιφάνεια \( \varSigma \).

%TODO Atreas Graph 09

Έστω \( \varSigma \) είναι μια \boxed{\text{απλή και διαφορίσιμη}} επιφάνεια με παραμετροποίηση
\[
\mathbf r:D \subset \mathbb R ^2 \to \mathbb R ^ 3 : \mathbf r(u,v)=\left(x(u,v),y(u,v),z(u,v)\right)
\]
\textbf{Έστω} \( Q_0  = \mathbf{r}(u_0,v_0) \) σημείο της επιφάνειας \( \varSigma \).
Τότε η εξίσωση του εφαπτόμενου επιπέδου της επιφ. \( \varSigma \) στο σημείο \( Q_0 \) (σε παραμετρική μορφή)
είναι η εξής:
\begin{align*}
\mathbf T(u,v) &= \mathbf{r}(u_0,v_0)+\mathrm J_{\mathbf r}(u_0,v_0)\cdot(P-P_0)
\\ &= \left[\begin{matrix}
x(u_0,v_0) \\ y(u_0,v_0) \\ z(u_0,v_0)
\end{matrix}\right] + \left[ \begin{matrix}
x_u(u_0,v_0) & x_v(u_0,v_0) \\
y_u(u_0,v_0) & y_v(u_0,v_0) \\
z_u(u_0,v_0) & z_v(u_0,v_0)
\end{matrix} \right] \cdot \left[ \begin{matrix}
u-u_0 \\ v-v_0
\end{matrix} \right]
\\ &= \left[\begin{matrix}
x(u_0,v_0)+x_u(u_0,v_0)(u-u_0)+x_v(u_0,v_0)(v-v_0) \\
y(u_0,v_0)+y_u(u_0,v_0)(u-u_0)+y_v(u_0,v_0)(v-v_0) \\
z(u_0,v_0)+z_u(u_0,v_0)(u-u_0)+z_v(u_0,v_0)(v-v_0)
\end{matrix} \right]
\\ &= \mathbf r(u_0,v_0) + \mathbf r_u(u_0,v_0)(u-u_0) + \mathbf r_v(u_0,v_0)(v-v_0) \\
\mathbf T(u,v) &= \mathbf r(u_0,v_0) + \mathbf r_u(u_0,v_0)(u-u_0) + \mathbf r_v(u_0,v_0)(v-v_0)
\end{align*}

(εξίσ. επιπέδου που διέρχεται από το \( Q_0=\mathbf r(u_0,v_0) \) και παράγεται από τα διανύσματα \( \mathbf r_u(u_0,v_0) \) και \( \mathbf r_v(u_0,v_0) \) υπό την προϋπόθεση \( \mathbf r_u \times \mathbf r_v(u_0,v_0) \neq \vec 0 \)

\paragraph{Ορ.}
Αν \( \varSigma \) απλή κ διαφορίσιμη επιφάνεια, καλούμε \textbf{ΚΑΘΕΤΟ} της επιφάνειας σε σημείο της \( Q_0 = \mathbf r(u_0,v_0) \) να είναι το διάνυσμα \( \mathbf n(u,v) = \mathbf r_u \times \mathbf r_v(u_0,v_0) \).

\textbf{Τότε}
\begin{itemize}
\item η \( \varSigma \) καλείται \textbf{ΟΜΑΛΗ}, αν \( \mathbf n(u,v) \neq 0 \) σε κάθε σημείο \( (u,v) \)
\item η \( \varSigma \) καλείται \textbf{ΛΕΙΑ}, αν η \( \varSigma \) είναι \emph{ομαλή},και η \( \mathbf r \) είναι όχι μόνον διαφορίσιμη, αλλά και η παράγωγός της είναι συνεχής συνάρτηση των \( u,v \)
\item η \( \varSigma \) καλείται \textbf{προσανατολίσιμη}, αν είναι \emph{ΟΜΑΛΗ} και η κάθετος αυτής \( \mathbf n(u,v) \) είναι συνεχής συνάρτηση των \( u \) και \( v \)
\end{itemize}

Προφανώς:
\[
\text{ΛΕΙΑ ΕΠΙΦΑΝΕΙΑ } \implies \text{ΠΡΟΣΑΝΑΤΟΛΙΣΙΜΗ}
\]
Στο εξής ασχολούμαστε αποκλειστικά με λείες επιφάνειες.

\paragraph{Ορ.}
Κάθε προσανατολίσιμη επιφάνεια λέμε ότι έχει δύο όψεις. Η μια όψη καθορίζεται απ' την κατεύθυνση της καθέτου \( \mathbf n_0 = \frac{\mathbf n}{|\mathbf n|} \) και η άλλη απ' την κατεύθυνση της \( -\mathbf n_0 = \frac{-\mathbf n}{|\mathbf n|} \) (σε κάθε σημείο της)

H \( \varSigma \) καλείται \textbf{προσανατολισμένη} αν εμείς έχουμε ορίσει στη \( \varSigma \) έναν προσανατολισμό ως θετικό.

\paragraph{Ορ.}
Έστω \( \varSigma \) λεία επιφάνεια. Καλούμε \textbf{σύνορο} της \( \varSigma \) την καμπύλη (ή καμπύλες) ή σημείο (ή σημεία), μέσω των οποίων περνάμε (με συνεχή τρόπο) από τη μια όψη της επιφάνειας \( \varSigma \) στην άλλη όψη. Αν το σύνορο της \( \varSigma \) είναι το \( \emptyset \), η \( \varSigma \) καλείται \boxed{\text{κλειστή}} επιφ., αλλιώς καλείται \boxed{\text{ανοικτή}}.

\paragraph{Ορ.}
Έστω \( \varSigma \) λεία, ανοικτή, προσανατολισμένη επιφάνεια. Θα λέμε ότι το σύνορό της διαγράφεται με τη θετική φορά, αν κινούμενη κατά μήκος τους συνόρου με το κεφάλι μας να δείχνει προς τον προσανατολισμό (που ήδη έχουμε ορίσει), τότε αφήνουμε την επιφάνεια πάντα στο αριστερό μας χέρι.

%TODO Atreas Graph 10

\paragraph{Παραδείγματα}
\begin{itemize}
\item Έστω \( z = f(x,y)  \quad (x,y) \in D \) είναι μια λεία επιφάνεια.

Παραμετροποίηση αυτής:
\begin{align*}
\mathbf r(x,y) &= \left(x,y,f(x,y) \quad (x,y)\in D \right) \\
\mathbf n(x,y) &= \mathbf r_x \times \mathbf r_y = \left|
\begin{matrix}
\vec i & \vec j & \vec k \\
1 & 0 & f_x \\
0 & 1 & f_y
\end{matrix}
\right| = \left(-f_x,-f_y,1\right)
\end{align*}
\subparagraph{Προφανώς} \( \mathbf n(x,y) \neq 0 \ \forall (x,y) \), ομαλή και μάλιστα λεία επιφάνεια, αφού \( f \) λεία (δελ. \( f_x,f_y \) συνεχείς)
\item \textbf{Σφαίρα} \( (x-a)^2+(y-b)^2+(z-c)^2=R^2 \)
\subparagraph{1ος τρόπος}
\[
\mathbf{r}(x,y) = \begin{cases}
\left(x,y,c+\sqrt{R^2-(x-a)^2-(y-b)^2}\right) \quad& \text{(άνω ημισφαίριο)} \\
\left(x,y,c-\sqrt{R^2-(x-a)^2-(y-b)^2}\right) \quad& \text{(κάτω ημισφαίριο)}
\end{cases},\ \forall (x,y): (x-a)^2-(y-b)^2\leq R
\]
\subparagraph{2ος τρόπος}
με χρήση σφαιρικών συντεταγμένων
\[
\mathbf r(\theta,\varphi) = \left(
a+R\cos\theta\sin\varphi,\ b+R\cos\theta\sin\varphi,\ c+R\cos\varphi
\right)
\]
\item Κύλινδρος \( (x-a)^2-(y-b)^2=R^2 , \quad c\leq z \leq d \)
\[
\mathbf r(\theta,z) = \left(
a+R\cos\theta,\ b+R\sin\theta,\ z
\right) \quad \forall \theta \in [0,2\pi],\ c\leq z \leq d
\]
\[
\mathbf n = \mathbf r_\theta \times \mathbf r_z = \cdots
\]
\end{itemize}

\subsection{Εμβαδόν επιφάνειας σε παραμετρική μορφή\\Επιφανεικά ολοκλ. 1ου είδους}
Έστω \( \varSigma \) είναι λεία επιφάνεια με παραμετροποίηση
\[
\mathbf r: D \subset \mathbb R ^2 \to \mathbb R ^3: \mathbf r(u,v) = \left(
x(u,v),y(u,v),z(u,v)
\right)
\]
Ποιο το εμβαδόν της επιφάνειας \( \varSigma \)?

Για απλότητα θεωρούμε \( D \) να είναι ορθογώνιο χωρίο. Θεωρούμε μια διαμέριση \( \Delta \) του ορθογωνίου \( D \) σε \( N \cdot M \) στοιχειώδη ορθογώνια που με τη σειρά της διαμερίζει την επιφ. \( \varSigma \) σε \( N \cdot M \) στοιχειώδη καμπυλόγραμμα παρ/μα όπως στο σχήμα
%TODO Atreas Graph 11

Έστω στοιχειώδες ορθογώνιου του \( D \)
%TODO Atreas Graph 12
Αν η Διαμέριση \( \Delta \) είναι πολύ πυκνή, δηλ. το πλάτος της \( |\Delta| \to 0 \), τότε χωρίς μεγάλο σφάλμα θεωρώ ότι το εμβαδόν στοιχειώδους καμπυλόγραμμου παραλληλογράμμου ισούται με:
\[
E_{\begin{array}{c}
	\text{στοιχειώδους} \\
	\text{καμπύλης} \\
	\text{παρ/μου}
	\end{array}}
	\approx
	E_{\begin{array}{l}
		\text{παρ/μου} \\
		Q_{ij}Q_{ij}^1Q_{ij}'Q_{ij}^2
		\end{array}}
	= \underbrace{\left|
	\overrightarrow{
		Q_{ij}Q_{ij}^1
		} \times \overrightarrow{Q_{ij}Q_{ij}^2}
	\right|}_{{
		\left(
		\begin{array}{l}
		\text{εφαρμογή εξωτ. γινομένου} \\
		\text{βλέπε αναλυτική γεωμετρία}
		\end{array}
		\right)
		}}
\]

\textbf{Αλλά} \( \left( E=|\vec a \times \vec b| \right) \)
\[
\left|
\begin{array}{l}
\overrightarrow{Q_{ij}Q_{ij}^1} \overset{\text{διαφορικό}}{\approx} \mathbf r_u(u_i,v_j) \dif u_i \\
\overrightarrow{Q_{ij}Q_{ij}^2} \approx \mathbf r_v(u_i,v_j)\dif v_i
\end{array}
\right\rbrace
\]

Άρα:
\begin{align*}
E_{
	\begin{array}{c}
	\text{στοιχ. καμπύλ.} \\
	\text{παρ/μου}
	\end{array}
	} &\approx \left|
	\mathbf r_u \times \mathbf r_v(u_i,v_j) 
	\right|\dif u_i \dif v_j
	\\
	 &= \left|
	 \mathbf n(u_i,v_j)\dif u_i \dif v_j
	 \right|
\end{align*}

\textbf{Έτσι}
\[
E \approx \sum_{i=0}^N \sum_{j=0}^M \left|
\mathbf n(u_i,v_j)
\right| \dif u_i\dif v_j
\]

Εφόσον η \underline{\( \mathbf{n}(u,v) \)} είναι και συνεχής συνάρτηση, το παραπάνω διπλό άθροισμα τείνει στον αριθμό:
\[
E = \iint_D \left|
\mathbf n(u,v)
\right| \dif u \dif v
\]
υπό την προϋπόθεση ότι το πλάτος διαμέρισης τείνει στο μηδέν.

Το αποτέλεσμα αυτό γενικεύεται και για μη ορθογώνια χωρία.

\paragraph{Ορ.}
Έστω \( \varSigma \) λεία επιφάνεια με παραμετροποίηση \( r:D \subset \mathbb R ^2 \to \mathbb R ^3 \) και \( f: \varSigma \to \mathbb R  \) είανι \textbf{συνεχές} βαθμωτό πεδίο πάνω στην επιφάνεια \( \varSigma \). Καλούμε επιφανειακό ολοκλήρωμα 1\textsuperscript{ου} είδους του βαθμωτού πεδίου \textbf{πάνω} στην επιφάνεια \( \varSigma \), συμβολικά
\[
\iint_\varSigma f\dif S,
\] (
όπου η ποσότητα
\[
\dif S = \left|
\mathbf n(u,v) \dif u \dif v
\right|
\]
καλείται διαφορικό εμβαδού της επιφ. \( \varSigma \) ) να είναι ο \textbf{ΠΡΑΓΜΑΤΙΚΟΣ ΑΡΙΘΜΟΣ}
\[
\underbrace{\iint_\varSigma f \dif S}_{\mathclap{\text{συμβολισμός}}} = \underbrace{\iint_D f\left(\mathbf{r}(u,v)\right) \left|
\mathbf n(u,v)
\right|  \dif u \dif v}_{\mathclap{\text{διπλό ολοκλήρωμα}}}
\]

\subsubsection{Εφαρμογές}
\begin{itemize}
\item \(
\iint\limits_\varSigma f \dif S  = \begin{array}{l}
\text{συνολική μάζα/φορτίο πάνω στην} \\
\text{επιφάνεια $\varSigma$, αν $f$ πυκνότητα μάζας/φορτίου}
\end{array}
\)
\item \(
\iint\limits_\varSigma f \dif S  = \text{ εμβαδόν της επιφάνειας } \varSigma
\)
\end{itemize}

\subsection{Επιφανειακά ολοκληρώματα\\διανυσματικών πεδίων}
Έστω \( \varSigma \) λεία επιφάνεια με παραμετροποίηση \( \bf r: D \subset \mathbb R ^2 \to \mathbb R ^3 \)
και \( \bf F: \varSigma \to \mathbb R ^3 \) είναι \textbf{ΣΥΝΕΧΕΣ} διανυσματικό πεδίο
επί της επιφ. \( \varSigma \).\\
(π.χ. \( \bf F \) πεδίο ταχυτήτων \( \bf F = \bf v \) ή πυκνότητα ροής \( \bf F =\rho \bf v \)).

Έστω για απλότητα \( D \) ορθογώνιο χωρίο, \( \Delta \) διαμέριση του \( D \) σε π.χ. \( N\cdot M \) το πλήθος στοιχειώδη ορθογώνια,
που μετη σειρά της διαμερίζει την επιφ. \( \varSigma \) σε \( N \cdot M \) το πλήθος στοιχειώδη καμπυλόγραμμα παρ/δα.

Έστω ένα τέλειο στοιχειώδες καμπυλόγραμμο παρ/δο της μορφής
%TODO Atreas Graph 13
Τότε το
\begin{align*}
& \bf F \left(\bf r (u_i,v_j) \right) \underbrace{\cdot}_{\mathclap{\text{εσωτ. γιν}}} \left(
\overrightarrow{Q_{ij}Q_{ij}^1} \times \overrightarrow{Q_{ij}Q_{ij}^2}
\right)
\\ =& \bf F \left(\bf r(u_i,v_j) \right) \cdot \left(
\bf r_u \times \bf r_v(u_i,v_j) \dif u_i \dif v_j
\right)
\\ =& \bf F\left(\bf r(u_i,v_j) \right) \cdot \bf n(u_i,v_j) \dif u_i \dif v_j
\end{align*}
προσεγγίζει τον \textbf{ΟΓΚΟ} που ``ρέει" διαμέσου 

προς την καυτεύθυνση της καθέτου \( \bf n \) (αν \( \bf F \cdot \bf n >0 \)) στο \( (u_i,v_j) \) είτε προς την αντίθετη κατεύθυνση αν \( \bf F \cdot \bf n <0 \) στο \( \left(u_i,v_j\right) \)).
Αν αθροίσω ως προς όλα τα στοιχειώδει καμπυλόγραμμα χωρία παίρνω:
\[
\Phi (\underbrace{\text{\textlatin{flux}}}_{\mathclap{\text{ροή}}})
\approx
\sum_{i=1}^N \sum_{j=1}^M \bf F \left(
\bf r(u_i,v_j)
\right) \cdot \bf n(u_i,v_j)\dif u_i\dif v_j
\]
και εφ' όσον η \( \bf F\left( \bf r(u,v) \right) \cdot \bf n(u,v) \) \textbf{συνεχής} πάνω στη \( \varSigma \), όσο το πλάτος της διαμέρισης \( \Delta \) τείνει στο μηδέν, το παραπάνω διπλό άθροισμα τείνει στον ΠΡΑΓΜΑΤΙΚΟ ΑΡΙΘΜΟ
\[
\iint_D \underbrace{\bf F \left(
\bf r(u,v)
\right) \cdot \bf n(u,v)}_{\mathclap{\text{διπλό ολοκλήρ.}}} \dif u \dif v
\]

\paragraph{Ορ.}
Αν \( \varSigma \) λεία επιφάνεια με παραμετροποίηση \( \bf r:D \subset \mathbb R ^2 \to \mathbb R ^3 \) και κάθετο \( \bf n(u,v) = \bf r_u \times \bf r_v \) και \( \bf F \) συνεχές πεδίο επί της \( \varSigma \), καλούμε επιφανειακό ολοκλ. του πεδίου \( \bf F \) επί της \( \varSigma \) ή \textbf{ΡΟΗ}, συμβολικά:
\[
\Phi = \iint_\varSigma \bf F \cdot \dif \bf S
\]
να είναι ο πραγματικός αριθμός
\[
\Phi = \iint \bf F \cdot \dif \bf S = \iint_D \underbrace{\bf F \left(\bf r(u,v) \right) \cdot n(u,v)}_{\mathclap{\text{διπλό ολοκλ.}}} \dif u \dif v
\]
όπου
\[
\dif \bf S = \bf n (u,v) \dif u \dif v
\]
καλείται διαφορικό της επιφάνειας \( \varSigma \).

\subsubsection{Παρατηρήσεις}
\begin{itemize}
\item \begin{align*}
\iint_\varSigma \bf  F \cdot \dif \bf S &= \iint_D \bf F \left(\bf r(u,v)\right) \cdot \bf n(u,v) \dif u ]dif v
\\ &=
\iint_D \bf F \left(\bf  r(u,v) \right) \cdot \underbrace{\frac{\bf n(u,v)}{\left|
\bf n (u,v)
	\right|
	}}_{\mathclap{\bf n_0}} \underbrace{\left|
	\bf n(u,v)
	\right| \dif u \dif v}_{\mathclap{\dif S}}
\\ &= \iint_\varSigma \underbrace{\bf F \cdot \bf n_0}_{\mathclap{\text{επιφ. ολοκλ. 1\textsuperscript{ου} είδους}}} \dif S
\end{align*}

\item
Το επιφανειακό ολοκλ. διανυσμ. πεδίων \textbf{ΕΞΑΡΤΑΤΑΙ} απ' τον προσανατολισμό της επιφάνειας \( \varSigma \).

\item
\( \bf F = \left(f_1,f_2,f_3\right),\ \bf r(u,v) = \left(
x(u,v),y(u,v),z(u,v)
\right) \)

\( \bf n = \bf r_u \times \bf r_v
= \left|
\begin{matrix}
\vec i & \vec j & \vec k & x_u & y_u & z_u & x_y & y_y & z_y
\end{matrix}
\right| = \left(
\frac{\mathrm D(y,z)}{\mathrm D(u,v)} ,
\frac{\mathrm{D}(z,x)}{\mathrm D(u,v)}.
\frac{\mathrm D(x,y)}{\mathrm D(u,v)}
\right)
 \)
 
 Τότε \begin{align*}
 \iint_\varSigma \bf F \cdot \dif \bf S &=
 \iint_D \left(
 f_1\left(\bf r(u,v)\right),
 f_2\left(\bf r(u,v)\right)
 f_3\left(\bf r(u,v)\right)
 \right) \cdot \left(
 \frac{\mathrm D(y,z)}{\mathrm D(u,v)} ,
 \frac{\mathrm{D}(z,x)}{\mathrm D(u,v)}.
 \frac{\mathrm D(x,y)}{\mathrm D(u,v)}
 \right) \dif u \dif v
 \\ &\overset{\mathclap{\text{συμβ.}}}{=}
 \iint_\varSigma \underbrace{f_1\dif y\dif z+f_2\dif z\dif x + f_3\dif x\dif y}_{\mathclap{\text{συμβολισμός για επιφ. ολοκλ. πεδίου}}}
 \end{align*}
 
 Τέλος, εφόσον \( n_0 = \left(\bf n_0 \cdot \vec i,\bf n_0 \cdot \vec j, \bf n_0 \cdot \vec k \right) \)
 
 Άρα
 \begin{align*}
 \iint_\varSigma \bf F \cdot \dif \bf  S &= \iint_\varSigma \bf F\cdot \bf n_0 \dif S
 \\ &= \iint_\varSigma \left(f_1,f_2,f_3\right)\cdot \left(
 \bf n_0 \cdot \vec i, \bf n_0 \cdot \vec j, \bf n_0 \cdot \vec k
 \right) \dif S
 \\ &= \iint_\varSigma f_1 \cdot \left(\bf n_0 \cdot \vec i \right)
 + \iint_\varSigma f_2 \cdot \left(\bf n_0 \cdot \vec j \right)
 +\iint_\varSigma f_3 \cdot \left(\bf n_0 \cdot \vec k \right)
 \end{align*}
 
 Τα \begin{align*}
 \iint_\varSigma f_1\dif y \dif z =& \iint_\varSigma f_1 \cdot  \left(\bf n_0 \cdot \vec i \right)
 \\
 \iint_\varSigma f_2\dif y \dif z =& \iint_\varSigma f_2 \cdot  \left(\bf n_0 \cdot \vec j \right)
 \\
 \iint_\varSigma f_3\dif y \dif z =& \iint_\varSigma f_3 \cdot  \left(\bf n_0 \cdot \vec k \right)
 \end{align*}
 καλούνται επιφ. ολοκλ. 2\textsuperscript{ου} είδους
\end{itemize}

Έστω \( \varSigma \) απλή, κλειστή, λεία επιφάνεια. Τότε χωρίζει τον \( \mathbb R ^3 \) σε ένα φραγμένο στερεό, \textbf{το εσωτερικό}
της \( \varSigma \), και σε ένα μη φραγμένο στερεό, το εξωτερικό της.
%TODO Atreas Graph 14

Αν \( \varSigma \) κλειστή επιφάνεια, γράφουμε:
\[
\oiint_\varSigma \mathbf F \cdot \dif \mathbf S \quad \left( \text{αντί } \iint_\varSigma \mathbf{F} \cdot \dif \mathbf S \right)
\]

\paragraph{*Θ* Απόκλισης (\textlatin{Gauss})}
Έστω \( \Omega \subset \mathbb R ^3 \) κανονικό και φραγμένο στερεό με σύνορο
\( \partial \Omega \) και είναι απλή, \boxed{\text{ΚΛΕΙΣΤΗ}}, τμημ. λεία
επιφάνεια, με \textbf{ΠΡΟΣΑΝΑΤΟΛΙΣΜΟ ΠΡΟΣ ΤΗΝ ΕΞΩΤΕΡΙΚΗ ΟΨΗ της}. Αν
\[
\mathbf F: \Omega \to \mathbb R ^3
\]
διανυσματικό πεδίο με συνεχείς μερικές παραγώγους στο \( \Omega \) και
στα σύνορα του \( \partial \Omega \), τότε:
\[
\underbrace{\oiint_{\partial \Omega}}_{
	\begin{matrix}
	\text{ροή του πεδίου } \mathbf F\\
	\text{διαμέσου της επιφάνειας } \partial \Omega 
	\end{matrix}
} \underbrace{=}_{\mathclap{\text{ισούται}}} \underbrace{\iiint_\Omega \nabla \cdot \mathbf F(x,y,z)\dif x\dif y\dif z}_{
\begin{matrix}
\text{με τη συνολική} \\
\text{απόκλιση του } \mathbf{F} \\
\text{στο στερεό } \Omega
\end{matrix}
}
\]

* Υπό τις παραπάνω προϋποθέσεις, αν το \( \mathbf F \) είναι ΑΣΥΜΠΙΕΣΤΟ στο \( \Omega \)
και στο σύνορό του \( \partial \Omega \), τότε:
\[
\oiint_{\partial \Omega} \mathbf F \cdot \dif \mathbf S =0
\]

Αν το πεδίο \textbf{τουλάχιστον σ' ένα σημείο τού \( \Omega \)} έχει ανωμαλία (δεν έχει συνεχείς μερικές παραγ.) τότε το θ. \textlatin{Gauss} \textbf{δεν} ισχύει, αλλά παρ' όλα αυτά ισχύει το ακόλουθο:

\paragraph{Θ (Γενικευμένο θεώρ. απόκλισης)}
Αν \( \Omega_0,\ \Omega_1, \dots, \ \Omega_n \) είναι κανονικά φραγμένα στερεά,
έτσι ώστε \( \Omega_1,\dots,\Omega_n \) στο εσωτερικό του \( \Omega_0 \)
και κάθε στερεό \( \Omega_j \) στο εσωτερικό του \( \Omega_i\ \forall i \neq j \), τότε αν \( \partial \Omega_0, \dots, \partial \Omega_n \) είναι απλές, κλειστές, τμημ. λείες επιφάνειες προσανατολισμένες στην εξωτερική όψη, και αν
\( \mathbf F: \Omega_0 - \left( \bigcup_{i=1}^N \Omega_i \right) \to \mathbb R ^3 \) έχει συνεχείς μερικές παραγώγους πάνω στο σύνορο \( \partial \Omega_0 \cup \partial \Omega_1 \cup \cdots \cup \partial \Omega_n \) και στο στερεό
\[ \Omega_0 = \left( \bigcup_{i=1}^N \Omega_i \right) \], τότε
\[
\oiint_{\partial \Omega_0} \mathbf F \cdot \dif \mathbf S =
\sum_{i=1}^n \oiint_{\partial \Omega_i} \mathbf F \cdot \dif \mathbf S
+ \iiint_{\Omega_0 - \left(\bigcup_{i=1}^N \Omega_i \right)} \nabla \cdot \mathbf F(x,y,z)\dif x \dif y \dif z
\]
%TODO Atreas Graph 15

\paragraph{Τύπος \textlatin{Stokes} (Γενίκευση θεωρήματος \textlatin{Green} για τον \( \mathbb R ^3 \)}
	Έστω \( \varSigma \) είναι απλή, \boxed{\text{ΑΝΟΙΚΤΗ}}, τμημ. λεία και
	\textbf{προσανατολισμένη} επιφάνεια με παραμετροποίηση
	\[
	\mathbf r \ \ D \subset \mathbb R ^2\to \mathbb R ^3: \mathbf r(u,v) = \left(
	x(u,v),\ y(u,v),\ z(u,v)
	\right)
	\]
	ώστε η \( \mathbf r \) έχει συνεχείς μερικές παραγώγους 2\textsuperscript{ης} τάξης.
	
	Αν \( \partial \varSigma \) είναι το σύνορο της επιφάνειας \( \varSigma \), και είναι μια απλή, κλειστή, λεία καμπύλη με τη \textbf{ΘΕΤΙΚΗ ΦΟΡΑ ΔΙΑΓΡΑΦΗΣ} και \( \mathbf F: \varSigma \to \mathbb R ^3 \) διανυσματικό πεδίο με \textbf{συνεχείς μερικές παραγώγους} στην επιφ. \( \varSigma \) και στο σύνορό της \( \partial \varSigma \), τότε:
	\[
	\underbrace{\oint_{\partial \varSigma} \mathbf F \cdot \dif \mathbf r }_
	{
		\begin{matrix}
		\text{έργο/κυκλοφορία του πεδίου } \mathbf F\\
		\text{κατά μήκος της } \partial \varSigma
		\end{matrix}
	}
	\underbrace{=}_{\mathclap{\text{ισούται}}} 
	\underbrace{\iint_\varSigma \nabla \times \mathbf F \cdot \dif \mathbf S}_{
		\begin{matrix}
		\text{με τη συνολική} \\
		\text{περιστροφή του } \mathbf F \\
		\text{πάνω στην επιφ. } \varSigma
		\end{matrix}
	}
	\]

\begin{gather*}
\text{ροή } \leftrightarrow \text{ απόκλιση}  \\
\text{έργο/κυκλοφ. } \leftrightarrow \text{ περιστροφή} 
\end{gather*}

\subsubsection{Παρατηρήσεις}
\begin{enumparen}
\item Αν \( \varSigma \) είναι απλή, ΚΛΕΙΣΤΗ, τμ. λεία επιφάνεια, τότε υπό τις προϋποθέσεις του παραπάνω θεωρήματος ισχύει:
\[
\oiint_\varSigma \nabla \times \mathbf F \cdot \dif \mathbf S = 0
\].

Πράγματι, αν \( \varSigma \) επιφάνεια με προσανατολισμό όπως στο Σχήμα και \( E \) επίπεδο που την τέμμνει σε καμπύλη με προσανατολισμό όπως στο σχήμα.

%TODO Atreas Graph 16

Τότε, αν \( \varSigma_1,\varSigma_2 \) επιφάνειες όπως στο σχήμα, με εφαρμογή του τύπου \textlatin{Stokes} έχουμε:
\[
\oint_\gamma \mathbf F \cdot \dif \mathbf r = \iint_{\varSigma_1} \nabla \times \mathbf F \cdot \dif \mathbf S
\]
και 
\[
\oint_{-\gamma} \mathbf F \cdot \dif \mathbf r = \iint_{\varSigma_2} \nabla \times \mathbf F \cdot \dif \mathbf S
\]

\begin{align*}
(+) \quad 0 &= \iint_{\varSigma_1} \nabla \times \mathbf F \cdot \dif \mathbf S + \iint_{\varSigma_2} \nabla \times \mathbf S \cdot \dif \mathbf S \\
&= \iint_\varSigma \nabla \times \mathbf{F} \cdot \dif \mathbf S
\end{align*}

\item
Αν \( \varSigma_1,\varSigma_2 \) απλές, ανοικτές, τμημ. λείες επιφ. με \textbf{ΚΟΙΝΟ προσανατολισμό} και \textbf{ΚΟΙΝΟ σύνορο}, τότε
\[
\iint_{\varSigma_1} \nabla \times \mathbf F \cdot \dif \mathbf S
= \iint_{\varSigma_2} \nabla \times \mathbf F \cdot \dif \mathbf S
\]
\end{enumparen}

%TODO Atreas Graph 17

\paragraph{}
\hrule
\begin{center}{ \Huge \textbf{\textlatin{THE END}}}\end{center}

\subsection{Ασκήσεις}
\paragraph{Άσκ.}
Υπολογίστε το εμβαδόν της επιφάνειας \( z=x^2+y^2 \) για \( 0 \leq z \leq 9 \).
\subparagraph{}
%TODO Plot χονδροειδώς + κύκλος (0,0),R=3, με βελάκι D = \left\lbrace (x,y):x^2+y^2 \right\rbrace

Έχω επιφανειακό ολοκλήρωμα 1\textsuperscript{ου} είδους \[
E = \iint_\varSigma 1 \cdot \underbrace{\dif S}_{\mathclap{\text{διαφορικό εμβαδού επιφ.}}} \overset{\text{τύπος}}{=} \iint_D 1 \underbrace{\left|
\mathbf n(u,v)
\right|\dif u \dif v}_{\dif S}
\]

\( D\! : \) προβολή της επιφάνειας που με ενδιαφέρει στο επίπεδο \( (u,v) \)

\begin{itemize}
\item Μια παραμετροποίηση της επιφάνειας είναι η εξής:
\[
\mathbf r(x,y) = \left(x,\ y,\ x^2+y^2 \right), \ \forall\ (x,y) : \underbrace{x^2+y^2 \leq 9}_{\mathclap{\text{προβολή της επιφάνειας στο $Oxy$ επίπεδο}}}
\]
\item
\begin{align*}
\mathbf n(x,y) &= \mathbf r_x \times \mathbf r_y = \left|
\begin{matrix}
	\vec i & \vec j & \vec k \\
	1      & 0      & 2x     \\
	0      & 1      & 2y
\end{matrix}
\right| = (-2x,-2y,1)
\end{align*}
\item \begin{align*}
\left|
\mathbf n(x,y)
\right| = \sqrt{1+ (2x)^2+(2y)^2 } = \sqrt{1 + 4(x^2+y^2)}
\end{align*}
\end{itemize}

Άρα:
\begin{align*}
E &\overset{\text{τύπος}}{=} 
\iint_{x^2+y^2 \leq 9}
\sqrt{1+4(x^2+y^2)} \dif x \dif y
\\ &\overset{\begin{matrix}
	\text{πολικές} \\
	x = \rho\cos\theta \\ y =\rho\sin\theta
	\end{matrix}}{=} \int_0^{2\pi} \int_0^3 \left(
1+4\rho^2
\right)^{\frac{1}{2}} \rho\dif\rho\dif\theta
\\ &= 2\pi \int_0^3 \left(
1+4\rho^2
\right)^\frac{1}{2} \dif (1+4\rho^2)\frac{1}{8}
\\ &=
\left. \frac{\pi}{4} (1+4\rho^2)^\frac{3}{2}\frac{2}{3}\right|_0^3
\\ &= \frac{\pi}{6}\left(
37^\frac{3}{2}-1
\right)
\end{align*}

\paragraph{Άσκ.}
Υπολογίστε το εμβαδόν σφαίρας \( 
(x-a)^2+(y-b)^2+(z-c)^2=R^2
 \)
 \subparagraph{}
 Αρκεί να υπολογίσω το εμβαδόν του άνω ημισφαιρίου:
 \[
 x^2+y^2+z^2 = R^2 \quad \text{δηλ.}
 \]
 \[
 z=\sqrt{R^2-x^2-y^2}
 \]
 
 Έχω επιφανειακό ολοκλήρωμα 1\textsuperscript{ου} είδους \[
 E = \iint_\varSigma 1 \cdot \underbrace{\dif S}_{\mathclap{\text{διαφορικό εμβαδού επιφ.}}} \overset{\text{τύπος}}{=} \iint_D 1 \underbrace{\left|
 	\mathbf n(u,v)
 	\right|\dif u \dif v}_{\dif S}
 \]
 
 \( D\! : \) προβολή της επιφάνειας που με ενδιαφέρει στο επίπεδο \( (u,v) \)
 
 \begin{itemize}
 \item Μια παραμετροποίηση της επιφάνειας είναι η εξής:
 \[
 \mathbf r(x,y) = \left(x,\ y,\ \sqrt{R^2-x^2-y^2} \right), \ \forall\ (x,y) : \underbrace{x^2+y^2 \leq 9}_{\mathclap{\text{προβολή της επιφάνειας στο $Oxy$ επίπεδο}}}
 \]
 \item \( 
 \mathbf n(x,y) = \mathbf r_x \times \mathbf r_y = \left|\begin{matrix}
 \vec i & \vec j & \vec k \\
 1 & 0 & \frac{x}{\sqrt{R^2-x^2-y^2}} \\
 0 & 1 & \frac{y}{\sqrt{R^2-x^2-y^2}}
 \end{matrix}
 \right| = \left(
 \frac{x}{\sqrt{R^2-x^2-y^2}}, \frac{y}{\sqrt{R^2-x^2-y^2}},1
 \right)
  \)
  \item 
  \( 
  \left|
  \mathbf n(x,y)
  \right| = \sqrt{\frac{x^2}{R^2-x^2-y^2} + \frac{y^2}{R^2-x^2-y^2}+1} = \frac{R}{\sqrt{R^2-x^2-y^2}}
   \)
 \end{itemize}

Έτσι: \begin{align*}
E_{\begin{matrix}
	\text{άνω} \\ \text{ημισφ.}
	\end{matrix}}
&= \iint_{x^2+y^2 \leq R^2} \frac{R}{\sqrt{R^2-x^2-y^2}} \dif x \dif y 
\\ &\overset{\begin{matrix}
	\text{πολικές} \\
	x = \rho\cos\theta \\ y =\rho\sin\theta
	\end{matrix}}{=} \int_0^{2\pi} \int_0^R \frac{R}{\sqrt{R^2-\rho^2}} \rho\dif\rho\dif\theta
\\ &= 2\pi R \int_0^R \left( R^2-\rho^2 \right) \dif \left(R^2-\rho^2\right) \left(
-\frac{1}{2}
\right) \\ &= \left. -2\pi R \left(R^2-\rho^2\right) \right|_0^R
\\ &= 2\pi R^2
 \end{align*}

\textbf{Τελικά}
\[
\mathrm E_{\text{σφαίρας}} = 2 \cdot 2\pi R^2 = 4\pi R^2
\]

\paragraph{Άσκ.}
Υπολογίστε τη ροή του πεδίου \( \mathbf F(x,y,z): \left( 3xy^2,\ xe^z,\ z^3 \right) \) διαμέσου κλειστής επιφάνειας που ορίζεται από τον κύλινδρο \( y^2+z^2=a^2 \) και τα επίπεδα \( x=-1 \) και \( x=2 \).
Θεωρήστε την επιφ. προσανατολισμένη προς την εξωτερική της όψη.

\subparagraph{}
Έχω πεδίο που ορίζεται στο \( \mathbb R ^3 \) με συνεχείς μερικές παραγώγους σ' όλο το \( \mathbb R ^3 \) και εφόσον έχω ροή μπορώ να χρησιμοποιήσω το θεώρ. απόκλισης \textlatin{Gauss}
\[
\oiint_{\partial \Omega} \mathbf F \cdot \dif \mathbf S = \iiint_{\Omega} \nabla \cdot \mathbf F(x,y,z)\dif x\dif y\dif z
\]
όπου
%TODO Atreas Graph 18

\textbf{Έτσι}
\begin{align*}
\Phi &= \iiint_\Omega \pd{}{x} \left(3xy^2\right) + \pd{}{y} \left(xe^z\right) + \pd{}{z}\left(z^2\right) \dif x\dif y\dif z
\\ &= \iiint_\Omega \left(3y^2+0+3z^2 \right) \dif x\dif y\dif z
\\ &= 3\iiint_\Omega (y^2+z^2)\dif x \dif y \dif z
\\ &\underset{\begin{cases}
		y = \rho\cos\theta \\ z =\rho\sin\theta \\ x = x
		\end{cases}}{\overset{\mathclap{\text{κυλινδρικές}}}{=}}
3 \int_1^2 \int_0^{2\pi} \int_0^a \rho^2 \rho\dif\rho\dif\theta
\\ &= 3 \cdot 3 \cdot 2\pi \cdot \int_0^a \rho^3\dif\rho - \left. 18\pi\frac{\rho^4}{4}\right|_0^a
\\ &= \frac{9\pi a^4}{2}
\end{align*}

\paragraph{Άσκ}
\begin{attnbox}{}
Αυτήν την άσκηση να την ξέρουμε σαν θεωρία.
\end{attnbox}
Ακίνητο σημειακό φορτίο \( -q \) στο σημείο \( (0,0,0) \) ορίζει πεδίο έντασης\[
\mathbf E = \frac{-q\vec r}{r^3}, \ \vec r = (x,y,z)
\]
και \( r= \sqrt{x^2+y^2+z^2} \).

\begin{enumparen}
\item ΝΔΟ \( \mathbf E \) ασυμπίεστο στο \( \mathbb R ^3 - \left\lbrace (0,0,0) \right\rbrace \) \\
\item Δείξτε ότι η ροή του πεδίου \( \mathbf E \) διαμέσου οποιασδήποτε κλειστής, απλής, λείας επιφάνειας που περιέχει το \( (0,0,0) \) στο εσωτερικό της και είναι προσανατολισμένη προς την εξωτερική όψη ισούται με
\[
\Phi = -4\pi q
\]
\item ΝΔΟ \( \mathbf E \) συντηρητικό στον \( \mathbb{R}^3 - \left\lbrace (0,0,0) \right\rbrace \)
\end{enumparen}

\subparagraph{}
\begin{enumparen}
\item Έχω Π.Ο του πεδίου \( \mathbf E := \mathbb R ^3 - \left\lbrace (0,0,0) \right\rbrace \)

\( \forall(x,y,z) \in \mathbb R ^3 - \left\lbrace (0,0,0) \right\rbrace \) έχω:
\begin{gather*}
\nabla \times \mathbf E = \nabla \cdot \left(
\frac{-q\vec r}{r^3}
\right) = -q\nabla \cdot \left(\frac{\vec r}{r}\right)
\underset{\text{γνωστός}}{\overset{\text{τύπος}}{=}}
-q \left(
\nabla \left(\frac{1}{r^3}\right) \cdot \vec r + \frac{1}{r^3} \nabla \cdot \vec r
\right) \\
\nabla \cdot \vec r = \left(
\pd{}{x},\pd{}{y},\pd{}{z}\cdot(x,y,z)
\right) = \pd{x}{x}+\pd{y}{y}+\pd{z}{z} = 3\\
\nabla \left(
\frac{1}{r^3}
\right)
 = \frac{\cancelto{0}{\nabla 1} \cdot r^3-1\cdot \nabla r^3}{r^6}
= \frac{-\nabla r^3}{r^6} = \frac{-3r^2\nabla r}{r^6} = \frac{-3\nabla r}{r^4}
\end{gather*}
\textbf{Αλλά} \( \nabla r = \nabla\left(x^2+y^2+z^2\right)^\frac{1}{2} = \frac{(x,y,z)}{\left(x^2+y^2+z^2\right)^\frac{1}{2}} = \frac{\vec r}{r} \).

\textbf{Τελικά}
\begin{align*}
\nabla \cdot \mathbf E &= -q \left(
\frac{-3\cdot \vec r}{r^5} \cdot \vec r + \frac{1}{r^3} \cdot 3
\right)
\\ &= -q \left(
\frac{-3}{r^5} \cdot(\vec r \cdot \vec r) + \frac{3}{r^3}
\right)
\\ &= -q \left(
\frac{-3r^2}{r^5}-\frac{3}{r^3}
\right) = 0
\end{align*}

\item
%TODO Atreas  Graph 20
Έστω \( \varSigma \) κλειστή επιφ. όπως στο σχήμα.

Αφού έχω ανωμαλία στο \( (0,0,0) \) ορίζω σφαίρα:
\[
\varSigma_1 : x^2+y^2+z^2 = \varepsilon^2
\]
(για \( \varepsilon > 0 \) τυχαίο, αλλά τέτοιο ώστε η σφαιρική μπάλα \( x^2+y^2+z^2 \leq \varepsilon^2 \) να βρίσκεται στο εσωτερικό της επιφ. \( \varSigma \))

Θεωρώ τη \( \varSigma_1 \) προς τα έξω προσανατολισμένη και εφαρμόζω γενικευμένο θεώρ. \textlatin{Gauss}.

\begin{align*}
\oiint_\varSigma \mathbf F \cdot \dif \mathbf S &=
\oiint_{x^2+y^2+z^2=\varepsilon^2} \mathbf{F} \cdot \dif \mathbf S + \cancelto{0\text{ λόγω ερωτήμ. (1)}}{\iiint_\Omega \nabla \cdot \mathbf F (x,y,z) \dif x \dif y \dif z}
\end{align*}
όπου \( \Omega \) στερεό, φραγμένο με σύνορο \( \varSigma \cup \varSigma_1 \)

\begin{align*}
\\ &= \oiint_{x^2+y^2+z^2=\varepsilon^2} \mathbf E \cdot \dif \mathbf S
\\ &\overset{\text{ορισμός}}{=} \oiint_{x^2+y^2+z^2=\varepsilon^2}
\mathbf E \cdot \mathbf n_0 \dif S
\\ &\overset{\text{τύπος}}{=} 
\left(
\iint_D \mathbf E \left(
\begin{matrix}
\text{πάνω στο} \\ \text{σύνορο}
\end{matrix}
\right) \cdot \mathbf n_0 \left( \begin{matrix}
\text{πάνω στο} \\ \text{σύνορο}
\end{matrix}
\right) \cdot \left|
\mathbf n(x,y)
\right| \dif x \dif y
\right)
\\ &\overset{\mathbf{n}_0 = \frac{\vec r}{r}}{=} \iint_{x^2+y^2 \leq \varepsilon^2} \frac{-q \cdot \vec{r}_\text{σφαίρας}}{\varepsilon^3} \cdot \frac{\vec{r}_\text{σφαίρας}}{\varepsilon} \left|
\mathbf n(x,y)
\right|\dif x \dif y
\\ &\overset{\vec{r}_\text{σφαίρας} \cdot \vec{r}_\text{σφαίρας}=\varepsilon^2}{=}
-q \iint_{x^2+y^2 \leq \varepsilon^2} \frac{\varepsilon^2}{2^4}\left|
\mathbf n(x,y)
\right| \dif x \dif y
\\ &= \frac{-q}{\varepsilon^2} \iint_{x^2+y^2 \leq \varepsilon^2} \left|
\mathbf n(x,y)
\right| \dif x \dif y
\\ &= -\frac{q}{\varepsilon^2} \iint_{x^2+y^2+z^2=\varepsilon^2} 1 \dif S
\\ &= -\frac{q}{\varepsilon^2} \text{Εμβ. σφαίρας}
\\ &= \frac{-q}{\varepsilon^2}4\pi t^2
\end{align*}

\begin{tcolorbox}
\paragraph{Περίφημος νόμος του \textlatin{Gauss}}
Η ροή του πεδίου της έντασης διαμέσου κλειστής, προς τα έξω προσανατολισμένης επιφάνειας είναι ανάλογη του φορτίου στο εσωτερικό της επιφάνειας.
\end{tcolorbox}

\item
Το πεδίο \( \mathbf E \) έντασης είναι ΚΕΝΤΡΙΚΟ ΠΕΔΙΟ, άρα από θεωρία είναι ΑΣΤΡΟΒΙΛΟ στο σύνολο
\( \mathbb R ^3 - \left\lbrace (0,0,0) \right\rbrace \).

Επειδή το \( \mathbb R ^3 - \left\lbrace (0,0,0) \right\rbrace \) είναι σύνολο ΑΠΛΑ ΣΥΝΕΚΤΙΚΟ, από θεώρημα έχουμε ότι το
\( \mathbf E \) συντηρητικό στο \( \mathbb R ^3 - \left\lbrace (0,0,0) \right\rbrace \).

\end{enumparen}

\(  E \) είναι ΣΥΝΟΛΟ \textbf{ΑΠΛΑ} συνεκτικό: \(\attnboxed{\text{Κάθε}} \) \textbf{κλειστή} καμπύλη στο
\( E \) μπορεί με \textbf{συνεχή τρόπο} να ``συσταλεί", σε \textbf{σημείο} \boxed{\text{του $E$}}
παραμένοντας ΠΑΝΤΑ στο \( E \).

\paragraph{Άσκ.}
Μα επαληθευτεί ο τύπος \textlatin{Stokes} για το πεδίο \( \mathbf F = (y,\ -2xz,\ yz^2) \)
και την (ανοικτή) επιφάνεια του παραβολοειδούς \( z=x^2+y^2 \) για \( z \leq 1 \) με τον
προσανατολισμό του σχήματος

\subparagraph{}
\[
\oint_{\underbrace{\partial \varSigma}_{\mathclap{\text{σύνορο επιφ. (ΠΡΟΣΟΧΗ! Με θετική φορά)}}}}
\mathbf F \cdot \dif \mathbf r = \iint_\varSigma \nabla \times \mathbf F \cdot \dif \mathbf S
\]


\begin{itemize}
\item Το σύνορο \( \partial \varSigma \) είναι ο κύκλος \( x^2+y^2=1 \) πάνω στο επίπεδο
z=1, με θετική φορά όπως στο σχήμα.


Όσον αφορά το \( \oint_{\partial \varSigma} \mathbf F \cdot \dif \mathbf r \) έχουμε
\[
\oint_{\partial \varSigma} \mathbf F \cdot \dif \mathbf r \overset{\text{τύπος}}{=} \int_a^b \mathbf F \left(
\mathbf{r}(t)
\right) \cdot \mathbf r'(t) \dif t
\]
Μια παραμετροποίηση του κύκλου \( x^2+y^2=1 \) στο επίπεδο \( z=1 \) είναι η:
\[
\mathbf r(t) = \left( \cos t, \sin t, 1 \right)\ \forall t \in [0,2\pi)
\]
\[
\mathbf r'(t) = \left(-\sin t,\ \cos t,0 \right)
\]
και αντικαθιστώ
\begin{align*}
\oint_{\partial \varSigma} \mathbf F \cdot \dif \mathbf r &= \int_0^{2\pi}
\left( \sin t, -2\cos t, \sin t \cdot 1^2 \right) \cdot \left(
-\sin t,\cos t, 0
\right) \dif t
\\ &= \int_0^{2\pi} -\sin^2 t - 2\cos^2 t \dif t
\\ &= \int_0^{2\pi} -1-\cos^2t \dif t = -2\pi - \int_0^{2\pi} \cos^2 t \dif t
\\ &= -2\pi - \int_0^{2\pi} 1 + \cos(2t) \dif t = -2\pi -\pi = -3\pi
\end{align*}

\item
Όσον αφορά τον όρο \( \iint_\varSigma \nabla \times \mathbf F \cdot \dif \mathbf S \)

\textbf{Παρατήρηση}: Αν \( \varSigma_1 \) είναι το τμήμα επιπέδου \( z=1 \) για κάθε \( (x,y): x^2+y^2 \leq 1 \), τότε ισχύει ότι:
\[
\iint_\varSigma \nabla \times \mathbf F \cdot \dif \mathbf S = \iint_{\varSigma_1} \nabla \times \mathbf F \cdot \dif \mathbf S
\]
με την κάθετο της \( \varSigma_1 \) να είναι όπως στο σχήμα
\begin{align*}
\nabla \times \mathbf F = \left|
\begin{matrix}
\vec i & \vec j & \vec k \\
\pd{}{x} & \pd{}{y} & \pd{}{z} \\
y & -2xz & yz^2
\end{matrix}
\right| = \left(
z+2x,\ 0, -2z-1
\right)
\end{align*}
\[
\iint_{\varSigma_1} \nabla \times \mathbf{F} \cdot \dif \mathbf S \overset{\text{τύπος}}{=} \iint_D \nabla \times \mathbf F \left(
\mathbf r(x,y)
\right) \cdot \mathbf n(x,y) \dif x \dif y
\]
όπου \( \mathbf r(x,y) \) είναι η παραμετροποίηση της \( \varSigma_1 \) και
\( D := \) προβολή της \( \varSigma_1 \) στο \( Oxy \) επίπεδο

Παραμετροπ. της \( \varSigma_1: z=1 \)
\[
\mathbf r(x,y) = \left(x,y,1\right) \ \forall (x,y):x^2+y^2 \leq 1
\]
\[
\mathbf n(x,y) = \mathbf r_x \cdot \mathbf r_y = \left|
\begin{matrix}
\vec i & \vec j & \vec k \\
1 & 0 & 0 \\
0 & 1 & 0
\end{matrix}
\right| = (0,0,1)
\]

\textbf{Τελικά}
\begin{align*}
\iint_{\varSigma} \nabla \times \mathbf F \cdot \dif \mathbf S &= \iint_{\varSigma_1} \nabla \times \mathbf{F} \cdot \dif \mathbf S
\\ &= \iint_{x^2+y^2 \leq 1} \left( 1^2+2x,\ 0,\ -2\cdot 1 \cdot 1 \right) \cdot (0,0,1) \dif x \dif y
\\ &= \iint_{x^2+y^2 \leq 1}-3\dif x \dif y = -3 \iint_{x^2+y^2 \leq 1}\dif x \dif y
\\ &= -3 \cdot \text{εμβαδόν δίσκου } x^2+y^2 =1
\\ &= -3\pi
\end{align*}


\end{itemize}




















































































\newpage

\part{Ζάχαρης}

\section{Συναρτήσεις πολλών μεταβλητών}

\[P = (x_1,\dots,x_n)  \mathbb \quad R ^n\]
\[A = (a_1,\dots,a_n)  \mathbb \quad R ^n\]
\[B = (b_1,\dots,b_n)  \mathbb \quad R ^n\]

\begin{align*}
d(A,B) = &\norm{\overrightarrow{AB}} = \sqrt{(a_1-b_1)^2 + \dots + (a_n-b_n)^2} \\
&\norm{\overrightarrow{OA}} = \sqrt{(a_1)^2 + \dots + (a_n)^2} \\
&\norm{\overrightarrow{OA}} = \sqrt{(b_1)^2 + \dots + (b_n)^2}
\end{align*}

\paragraph{Περιοχή σημείου \(Α\) ακτίνας \(\epsilon\) \(\Pi_\epsilon(A)\)}

\begin{wrapfigure}{r}{0.25\textwidth}
\begin{tikzpicture}
\filldraw[fill=green!20] (0,0) circle (7mm);
\draw[gray] (0,0) -- ++(157:7mm) node[midway,above] {$\epsilon$};
\filldraw[black] (0,0) circle (1.5pt) node[anchor=west] {$A$};
\end{tikzpicture}
\end{wrapfigure}
\[
\Pi_\epsilon(A) = \left\lbrace P \in  \mathbb R ^n : d(P,A) < \epsilon \right\rbrace
\]

\paragraph{Σημείο συσσώρευσης} \hspace{0pt} \\


\begin{wrapfigure}{l}{0.3\textwidth}
\begin{tikzpicture}
\filldraw[thick,fill=green!20] plot [smooth cycle] coordinates {
(0,0) (-0.4, 0.5) (0.6, 1) (2,0.8) ( 4,1) (4,-0.7) (3,-0.85) (2, -0.75) (1,-1)
};
\filldraw[thick,fill=white] (1.2,0) circle (4mm);
\filldraw[fill=white] (3,-0.2) circle(2pt) node[anchor=south west] {$K$};
\draw (0.3,0.5) node {$E'$};
\filldraw (2,0.8) circle(2pt) node[anchor=south west] {$A$};

\filldraw (0,2) circle(2pt) node[anchor=south west] {$I \in E$};
\filldraw (2,-2) circle(2pt) node[anchor=north west] {$P \notin E$};
\end{tikzpicture}
\vspace{-40pt}
\end{wrapfigure}

\( E = (E' - \lbrace K\rbrace ) \cup \lbrace I\rbrace \)



\begin{itemize}
\item Σημείο συσσώρευσης: \( \left( \Pi_\epsilon(P) - \left\lbrace P \right\rbrace \right) \cap E \neq \emptyset \)
\item Παράγωγο σύνολο: \(Ε'\) (τα σημεία συσσώρευσης του \(Ε\))
\item Κλειστότητα του \(Ε\): \(Ε \cup E'\)
\item Συνοριακό σημείο \(Α\): \(\forall \epsilon > 0: \Pi_\epsilon(A) \cap E \neq \emptyset\) και \(\Pi_\epsilon(A) \cap \left( \mathbb R^n - E \right) \neq \emptyset\) (το \(Ε\) αλλά και το συμπληρωματικό του ανήκουν σε κάθε περιοχή του \(Α\)).
\item Κλειστό σύνολο: Συμπεριλαμβάνει το σύνορο
\item Ανοικτό σύνολο: Δεν συμπεριλαμβάνει κανένα συνοριακό σημείο
\item Οι χώροι \(\emptyset\) και \( \mathbb R ^ n \) θεωρούνται κλειστοί και ανοικτοί.
\end{itemize}

\paragraph{Συνεκτικό σύνολο (ή συναφές)}
Κάθε δύο σημεία του συνόλου μπορούν να ενωθούν με μια γραμμή που ανήκει στο σύνολο.

\paragraph{Κυρτό σύνολο}
Κάθε δύο σημεία του συνόλου μπορούν να ενωθούν με \textit{ευθεία} γραμμή που ανήκει στο σύνολο.

\paragraph{Απλά συνεκτικό σύνολο}
Κάθε καμπύλη του συνόλου θα ανήκει μέσα στο σύνολο αν τη σφίξω.

\paragraph{} \hspace{0pt} \\
\begin{tikzpicture}
\filldraw[thick,fill=green!20] plot [smooth cycle,tension=1] coordinates {
(0,0) (3,1) (6,0) (3,-1)
};
\fill (2,0) circle (2pt);
\fill (4,0) circle (2pt);
\draw[very thick,cyan] (2,0)--(4,0);
\draw[very thick,blue] plot[smooth,tension=1] coordinates {(2,0) (2.5,0.5)  (3.5,-0.5) (4,0)};
\draw[very thick,blue] plot[smooth,tension=0.7] coordinates {(2,0) (2.5,-0.87)  (3.5,-0.87) (4,0)};
\node[anchor=north] at (current bounding box.south) {συνεκτικό, κυρτό, απλά συνεκτικό};
\end{tikzpicture}
\hspace{10pt}
\begin{tikzpicture}
\filldraw[thick,fill=green!20] plot [smooth cycle,tension=1] coordinates {
(0,0) (0.5,2) (3,0.3) (5.5,2) (6,0) (3,-0.3)
};
\fill (0.7,1.2) circle (2pt);
\fill (5.3,1.2) circle (2pt);
\draw[very thick,red] (0.7,1.2)--(5.3,1.2);
\draw[very thick,blue] plot[smooth,tension=1] coordinates {(0.7,1.2) (3,0) (5.3,1.2)};
\node[anchor=north] at (current bounding box.south) {συνεκτικό, απλά συνεκτικό};
\end{tikzpicture}

\paragraph{} \hspace{0pt} \\
%\\
\begin{tikzpicture}
%\filldraw[thick,fill=green!20] (0,0) ellipse (3 and 2);
\filldraw[thick,fill=green!20] plot [smooth cycle,tension=1.5] coordinates {
(0,0) (3,1) (6,0) (3,-1)
};
\fill (1.5,0) circle (2pt);
\fill (4.5,0) circle (2pt);
\draw[very thick,red] (1.5,0)--(4.5,0);
\filldraw[thick,fill=white!20] (3,0) circle (0.5);
\draw[dashed,red] (1.75,0)--(4.25,0);
\draw[very thick,blue] plot[smooth,tension=1] coordinates {(1.5,0) (3,-0.75) (4.5,0)};
\node[anchor=north] at (current bounding box.south) {συνεκτικό};
\end{tikzpicture}
\hspace{40pt}
\begin{tikzpicture}
\filldraw[thick,fill=green!20] (0,0.75) circle (0.5);
\filldraw[thick,fill=green!20] (0,-0.75) circle (0.5);
\draw[thick] plot[smooth,tension=1] coordinates {(0,0.25) (-0.15,0) (0,-0.25)};
\node[anchor=north] at (current bounding box.south) {συνεκτικό, απλά συνεκτικό};
\end{tikzpicture}
\hspace{40pt}
\begin{tikzpicture}
\filldraw[thick,fill=green!20] (0,0.75) circle (0.5);
\filldraw[thick,fill=green!20] (0,-0.75) circle (0.5);
\node[anchor=north] at (current bounding box.south) {απλά συνεκτικό};
\end{tikzpicture}

\paragraph{}
Η μόνη συσχέτιση που ισχύει είναι η εξής:
\begin{align*}
\text{κυρτό} &\implies \text{συνεκτικό} \\
\text{κυρτό} &\implies \text{απλά συνεκτικό}
\end{align*}

\begin{wrapfigure}{r}{0.3\textwidth}
\caption{Φραγμένο σύνολο}
\begin{tikzpicture}
\draw[->] (xyz cs:x=0) -- (xyz cs:x=2.5);
\draw[->] (xyz cs:y=0,x=0) -- (xyz cs:y=2.5,x=0);
\draw[->] (xyz cs:z=0,x=0) -- (xyz cs:z=3,x=0);

\node (0,0) [anchor=east] {\(O\)};

\filldraw[gray] (1.7,1.5) circle (25pt) node[anchor=north] {\(E\)};

\filldraw[black] (2,1.5) circle (2pt) node[anchor=west] {\(P\)};
\draw[->,blue, very thick,anchor=west] (0,0) -- (2,1.5);
\end{tikzpicture}
\end{wrapfigure}

\paragraph{Φραγμένο σύνολο} ανν \(\norm{\overrightarrow{OP}} = d(O,P)\) πεπερασμένη
\paragraph{Συμπαγές σύνολο} ανν είναι φραγμένο και περιέχει το σύνορο

\subsubsection{Ορισμός συνάρτησης}

\(E \subseteq   \mathbb R^n, \quad B \subseteq  \mathbb R \)
\[f: E \rightarrow B: z = f(x_1, \dots, x_n) \]
\[P = \left( x_1, \dots, x_n \right) \rightarrow \text{πρότυπα ή αρχέτυπα, }z\text{ εικόνες} \]


\begin{wrapfigure}{r}{0.25\textwidth}
\centering
\caption{Τρισορθογώνιο σύστημα συντεταγμένων}
\begin{tikzpicture}[scale=0.7]
\draw[->] (xyz cs:x=0) -- (xyz cs:x=2.5);
\draw[->] (xyz cs:y=0,x=0) -- (xyz cs:y=2.5,x=0);
\draw[->] (xyz cs:z=0,x=0) -- (xyz cs:z=2.5,x=0);
\end{tikzpicture}
\end{wrapfigure}
Για συνάρτηση από το \( \mathbb R ^n\), χρειάζομαι \(n+1\) άξονες. Άρα γραφικές παραστάσεις θα κάνουμε για συναρτήσεις το πολύ 2 μεταβλητών, με προοπτική παράσταση ή ισουψείς καμπύλες.



\iftrue
\begin{figure}[H]
\centering
\begin{subfigure}[b]{0.44\textwidth}
\begin{tikzpicture}[scale=0.9]
\begin{axis}
[
    view={0}{90}
]
\addplot3[
    contour gnuplot={levels={0.8, 0.4, 0.2, -0.2}}
]
{sin(deg(sqrt(x^2+y^2)))/sqrt(x^2+y^2)};
\end{axis}
\end{tikzpicture}
    \end{subfigure}
\quad
\begin{subfigure}[b]{0.44\textwidth}

\begin{tikzpicture}
\begin{axis}[
    axis lines = middle,
    colormap/cool,
]
\addplot3[
    mesh,
    samples=50,
    domain=-8:8,
]
{sin(deg(sqrt(x^2+y^2)))/sqrt(x^2+y^2)};
\end{axis}
\end{tikzpicture}
\end{subfigure}

%\caption{Τρισδιάστατη γραφική παράσταση}
\end{figure}
\fi




\paragraph{Πολυωνυμική συνάρτηση}
Περιέχει όρους της μορφής \(a x_1^{m_1} x_2^{m_2} \cdots x_n^{m_n}, \quad m_1,m_2,\dots,m_n \in  \mathbb N \).


\textit{π.χ.}
\begin{align*}
w&=3x^4y^2z^3+4x^5yz^2-7x^3yz \\
w&=f(x,y,z)
\end{align*}
\[
\mathrm{max} \left( \sum_{i=1}^{n}m_i \right) = \text{βαθμός}(f)
\]

\subparagraph{Ρητή συνάρτηση}
\[
\frac{f(P)}{g(P)} =
\frac{f(x_1,\dots,x_n)}{g(x_1,\dots,x_n)}
\quad
f,g \text{ πολυωνυμικές}
\]

\subsubsection{Όριο συνάρτησης}
\[ \lim_{(x,y) \to (x_0,y_0)} f(x,y) = \lambda\]
\begin{attnbox}{}
\paragraph{Διπλά όρια}
\[ \lim_{x \to x_0} \left( \lim_{y \to y_0} f(x,y) \right), \quad
   \lim_{y \to y_0} \left( \lim_{x \to x_0} f(x,y) \right)
\] 
\tcblower
Δεν έχουν απαραίτητα σχέση με το κανονικό όριο (και μπορεί να έχουν διαφορετική τιμή)!
Η ύπαρξη ή/και ισότητα των ορίων δεν είναι διαγνωστική για το όριο της συνάρτησης. Αν υπάρχει το \(\lambda\) και \textbf{υπάρχουν} τα παραπάνω όρια, τότε είναι ίσα με \(\lambda\). Αν τα παραπάνω όρια \textbf{υπάρχουν} και \textbf{δεν} είναι ίσα, τότε το \(\lambda\) \textbf{δεν} υπάρχει.
\begin{center}
\begin{tikzpicture}
\draw[->,thick] (0,0) -- (0,4);
\draw[->,thick] (0,0) -- (4,0);

\filldraw[very thick,fill=green!20] plot [smooth cycle] coordinates {
(0.4,0.5) (0.5,1) (0.7,2) (0.6, 3.5) (4,3.2) (2,2) (2.5,0.7)
};

\draw[gray,dashed] (3,3) -- (3,0) node[anchor=north] {\(x_0\)};
\draw[gray,dashed] (3,3) -- (0,3) node[anchor=east] {\(y_0\)};
\draw[gray,dashed] (1,1) -- (1,0) node[anchor=north] {\(x\)};
\draw[gray,dashed] (1,1) -- (0,1) node[anchor=east] {\(y\)};

\filldraw[black] (3,3) circle (2pt) node[anchor=west] {\(P_0\)};

\draw[->,blue,thick] (1,1)--(1,3);
\draw[->,blue,thick] (1,3)--(3,3);

\draw[->,orange,thick] (1,1)--(3,1);
\draw[->,orange,thick] (3,1)--(3,3);

\draw[->,thick] (1.3,1.3) to [bend left=20] (2.7,2.7);
\draw[->,thick] (1.3,2.3) to [bend left=20] (2.5,2.7);
\draw[->,thick] (1.7,3.4) to [bend right=20] (2.8,3.1);


\filldraw[black] (1,1) circle (2pt) node[anchor=west] {\(P\)};
\end{tikzpicture}
\end{center}
\end{attnbox}



\begin{infobox}{Μεθοδολογία}
\begin{align*}
w&=f(x,y) \quad \text{ή}\\
w&=f(x,y,z)
\end{align*}
\[ \lim_{P\to P_0} \]
\tcblower
\begin{enumerate}
\item Επιλέγω για την \(f(x,y)\) μια καμπύλη \(y=g(x)\) του \(Ε\) που περνά από το \(Π_0\) ή \\ επιλέγω για την \(f(x,y,z)\) μια καμπύλη \(y=g(x)\) και \(z=h(x)\) του \(Ε\) που περνά από το \(Π_0\)
\item Αντικαθιστώ και καταλήγω στον υπολογισμό του ορίου \(\lim_{x \to x_0}\)
\item Αν το αποτέλεσμα εξαρτάται από τις παραμέτρους τις καμπύλης, τότε το όριο δεν υπάρχει, ενώ αν δεν εξαρτάται, το αποτέλεσμα είναι μη διαγνωστικό.
\end{enumerate}
\end{infobox}

\begin{infobox}{Μεθοδολογία}
Για όρια ρητών συναρτήσεων \(\frac{f(P)}{g(P)}\) στο \((0,0)\):
\tcblower
\begin{enumerate}
\item Αν \(B \left[ f(P) \right] > B \left[ g(P) \right]\), μάλλον το όριο υπάρχει.
\item Αν \(B \left[ f(P) \right] \leq B \left[ g(P) \right]\), μάλλον το όριο δεν υπάρχει.
\end{enumerate}
\end{infobox}

\subsubsection{Ιδιότητες ορίων}

Αν \( \lim_{P \to P_0} f(P) = \lambda_1\) και \( \lim_{P \to P_0} g(P) = \lambda_2\) τότε:

\begin{enumparen}
\item \( \lim_{P \to P_0} f(P) \pm g(P) = \lambda_1 \pm \lambda_2
\)
\item \( \lim_{P \to P_0} f(P) \cdot g(P) = \lambda_1 \cdot \lambda_2
\)
\item \( \lim_{P \to P_0} \frac{f(P)}{g(P)} = \frac{\lambda_1}{\lambda_2}
\)
\item \( \lim_{P \to P_0} \sqrt[n]{f(P)} = \sqrt[n]{\lim_{P \to P_0} f(P)} = \sqrt[n]{\lambda_1 }
\) \quad (εφ' όσον ορίζεται)
\item \( \lim_{P \to P_0} |f(P)| = |\lim_{P \to P_0} f(P)| = |\lambda_1 |
\) \quad (\(\lambda_1 >0, f(P) > 0\))
\item \( \lim_{P \to P_0} f(P)^{g(P)} = \left[ \lim_{P \to P_0} f(P) \right] ^{\lim_{P \to P_0} g(P)}
\)
\item Κρ. παρεμβολής:
\begin{enumerate}
\item \(f(P) \leq g(P) \leq h(P) \)
\item \(\lim_{P \to P_0} f(P) = \lim_{P \to P_0} h(P) = \lambda \) 
\end{enumerate}
Τότε \(\lim_{P \to P_0} g(P) = \lambda \)

\item
\begin{enumerate}
\item \( |g(P)| \leq h(P) \)
\item \(\lim_{P \to P_0} h(P) = 0 \) 
\end{enumerate}
Τότε \(\lim_{P \to P_0} g(P) = 0\)
\item
Αν \begin{enumerate*}[label={\arabic*)},font={\bfseries}]
\item \(\lim_{P \to P_0} f(P) = 0\) και
\item η \(g\) είναι φραγμένη,
\end{enumerate*}
τότε \( \lim_{P \to P_0} f(P)g(P) = 0 \).
\end{enumparen}
Προϋπόθεση για τις παραπάνω ιδιότητες είναι να μην οδηγούμαστε σε απροσδιοριστία (π.χ. \(\frac{\infty}{\infty}\))

\subsubsection{Σύνθεση συναρτήσεων}
\begin{align*}
f &: A \subseteq \mathbb R^n \to B \subseteq \mathbb R^n \\
g &: B \to C \subseteq  \mathbb R 
\end{align*}

Ορίζω:
\[
\left( g \circ f \right)(P) = g \left( f(P) \right)
\]

Έστω \( \lim _{P \to P_0} f(P) =m\). Τότε, αν οι συναρτήσεις είναι συνεχείς στο πεδίο ορισμού, έχω:
\[
\left( g \circ f \right)(P) = g(m) = \lambda
\]

\subsubsection{Συνέχεια συνάρτησης}
Μια συνάρτηση ονομάζεται συνεχής σε ένα σημείο όταν το όριό της σε εκείνο το σημείο υπάρχει και είναι ίσο με την τιμή της εκεί.

\subsubsection{}
\[\lim_{(x,y)\to(0,0)}
\]

\begin{infobox}{Μεθοδολογία}
\[ \lim_{(x,y) \to ( \pm \infty, \pm \infty )} f(x,y) \]
\tcblower
Για \(u=\frac{1}{x} \to 0, \ v=\frac{1}{y} \to 0\),
\[
\lim_{\mathclap{(x,y) \to ( \pm \infty, \pm \infty )}} f(x,y) \; = \;
\lim_{\mathclap{(u,v) \to ( 0,0 )}} f(u,v)
\]
\end{infobox}

\subsection{Ασκήσεις}
\paragraph{(1)}
\[
\lim_{(x,y) \to (2,3)} \frac{x^2+y}{x+y^3} + \cos (xy) = \frac{7}{29} + \cos(6)
\]
\paragraph{(2)}
\[
\lim_{(x,y) \to (0,0)} \frac{xy^4}{(x^2+y^2)^2}
\]

Αντικαθιστώ με πολικές συντεταγμένες:
\[
\attnboxed{\begin{cases}x = r \cos \theta \\ y = r \sin \theta\end{cases}}
\]

\begin{align*}
\lim_{(x,y) \to (0,0)} \frac{xy^4}{(x^2+y^2)^2} \\
= \lim_{r \to 0^+} \frac{r \cos \theta \, r^4 \sin^4 \theta}{r^4} \\
= \lim_{r \to 0^+} r \cos \theta \sin^4 \theta = 0
\end{align*}


\paragraph{(3)}
\[
\lim_{(x,y) \to (0,0)} \frac{x^2y}{(x^2-y^2)^2}
\]

Θέτω \(x = \lambda \sqrt y \implies y = \frac{1}{\lambda^2} x^2 \).
\begin{align*}
&\lim_{(x,y) \to (0,0)} \frac{x^2y}{(x^2-y^2)^2} \\
&= \lim_{\lambda \to 0^+} \frac{\lambda^2 |y| \cdot y}{\left( \lambda^2 |y| + y \right)^2} \\
&= \lim_{\lambda \to 0^+} \frac{\lambda^2 y^2}{\left( \lambda^2 y + y \right)^2} \\
&= \lim_{\lambda \to 0^+} \frac{\lambda^2 y^2}{y^ 2 \left( \lambda^2 +1 \right)^2} \\
&= \frac{\lambda^2}{(\lambda^2+1)^2}
\end{align*}
Άρα το όριο δεν υπάρχει.

\paragraph{(4)}
\[
\lim_{(x,y) \to (\infty,\infty)} \frac{x+2y}{x^2+y^2}
\]

\(u=\frac{1}{x} \to 0, \ v = \frac{1}{y} \to 0\)
\begin{align*}
&\lim_{(x,y) \to (\infty,\infty)} \frac{x+2y}{x^2+y^2}
\\ &=
\lim_{(u,v) \to (0,0)} \frac{\frac{1}{u} + \frac{2}{v}}{\frac{1}{u^2}+ \frac{1}{v^2}}
\\ &=
\lim_{(u,v) \to (0,0)} \frac{u^2v^2 \left( \frac{1}{u}+\frac{2}{v} \right)}{u^2v^2 \left( \frac{1}{u^2}+\frac{1}{v^2} \right)}
\\ &=
\lim_{(u,v) \to (0,0)} \frac{uv (2u+v)}{u^2+v^2}
\end{align*}

\(u =r\cos \theta,\ v=r\sin\theta\)
\begin{align*}
\lim_{r \to 0} \frac{r\cos\theta\sin\theta (2r \cos\theta + r \sin\theta)}{r^2} &= \\
\lim_{r \to 0} r\cos\theta\sin\theta (2\cos\theta+\sin\theta) = 0
\end{align*}


\paragraph{(5)}
\[
\lim_{(x,y,z) \to (0,0,0)} \frac{x^2yz}{x^2+y^2}
\]

Αντικαθιστώ με σφαιρικές συντεταγμένες:
\[
\attnboxed{\begin{cases}x = r \sin\theta\cos\phi \\ y = r \sin \theta\sin\phi \\ z = r\cos\theta \end{cases}}
\]

\begin{align*}
&\lim_{r \to 0} \frac{r^2\sin^2\theta\cos^2\phi \, r\sin\theta\sin\phi\, r \cos\theta}{r^2} \\ &=
\lim_{r \to 0} r^2(\sin^3\theta\cos\theta\cos^3\phi\sin\phi) = 0
\end{align*}


\paragraph{(6)}
\[
\lim_{(x,y,z) \to (0,0)} \frac{x+y}{x^2+y^2}
\]

\begin{align*}
\lim_{(x,y) \to (0,0)} \frac{x+y}{x^2+y^2}
\\ &=
\lim_{x \to 0} \frac{x+\lambda x}{x^2+\lambda^2 x^2}
\\ &=
\lim \frac{x(1+\lambda)}{x^2(1+\lambda^2)}
\\ &=
\lim_{x \to 0^+} \frac{1+\lambda}{x(1+\lambda^2)}
\\ &=
\frac{1+\lambda}{0^+ (1+\lambda^2)}
\end{align*}
Ανάλογα με το πρόσημο του \((1+\lambda)\), μπορεί να το όριο να είναι \(\pm\) σε πρόσημο, άρα δεν υπάρχει.

\paragraph{}
\begin{attnbox}{}
Κανόνα \textlatin{De L' Hospital} δεν μπορώ να χρησιμοποιήσω σε συναρτήσεις πολλών μεταβλητών, παρά μόνο όταν έχω μόνο μία μεταβλητή!
\end{attnbox}

\begin{align*}
\lim_{(x,y)\to(0,0)}
\frac{1-cos \sqrt{x^2+y^2}}{x^2+y^2} &= \\
&= \lim_{r\to 0^+} \frac{1-\cos r}{r^2} \\
&= \lim_{r\to 0^+} \frac{(1-\cos r)'}{(r^2)'} \\
&= \lim_{r\to 0^+} \frac{\sin r}{2r} \\
&= \lim_{r\to 0^+} \frac{(\sin r)'}{(2r)'} \\
&= \lim_{r\to 0^+} \frac{\cos r}{2} = \frac{1}{2}
\end{align*}

\paragraph{}
\begin{align*}
\lim_{(x,y)\to (0,0)}
\left[
\frac{(2+y^3)\tan (x^3+y^3)}{x^3+y^3} +
\frac{\tan(x^5y^5)}{\tan(x^5)\tan(y^5)}
\right]
&= \\
&= 
\lim_{(x,y)\to (0,0)} (2+y^3) \cdot
\lim_{(x,y)\to (0,0)} \frac{\tan(x^3+y^3)}{x^3+y^3} +
\lim_{(x,y)\to (0,0)} \frac{\tan(x^5+y^5)}{\tan(x^5) \tan(y^5)}
\end{align*}

Αν θέσω \(x^3+y^3=u\), έχω:
\[
\lim_{(x,y)\to (0,0)} \frac{\tan(x^3+y^3)}{x^3+y^3}
= \lim_{u\to 0} \frac{\tan u}{u} =
\lim_{u\to 0} \frac{(\tan u)'}{u'} =
\lim_{u \to 0} \frac{1}{\cos^2 u} = 1
\]

\[
\lim_{(x,y)\to (0,0)} \frac{\tan(x^5+y^5)}{\tan(x^5) \tan(y^5)}
=
\lim_{(x,y)\to (0,0)} \frac{\frac{\tan(x^5+y^5)}{x^5y^5}}{\frac{\tan(x^5)}{x^5}\cdot\frac{\tan(y^5)}{y^5}}
=
\frac{
\lim_{v\to 0} \frac{\tan v}{v}
}{
\lim_{w\to 0} \frac{\tan w}{w}
\cdot
\lim_{z\to 0} \frac{\tan z}{z}
}
\quad 
\left(
\begin{matrix}
x^5y^5 &=v\\
x^5 &=w\\
y^5 &=z
\end{matrix}
\right)
\]

Άρα \(\lim \bigg( \cdots \bigg) = 3\).

\paragraph{}
\[
\lim_{(x,y) \to (0,0)} y^x \quad (y \geq 0)
\]
\subparagraph{(1) \(y=0\)}
\[
\lim_{(x,y)\to(0,0)}y^x=
\lim_{x\to 0^+} 0^x = 0
\]
\subparagraph{(2) \(y=x\)}
\begin{gather*}
\lim_{(x,y)\to(0,0)}y^x=
\lim_{x\to 0^+} x^x = 0 =
\lim_{x\to 0^+} e^{\ln x^x} = \\
\lim_{x\to 0^+} e^{x\ln x} =
e^{\lim_{x\to 0^+} \ln x^x} =
e^{\lim_{x\to 0^+} \frac{\ln x}{\frac{1}{x}}} = \\
e^{\lim_{x\to 0^+} \frac{(\ln x)'}{\left(\frac{1}{x}\right) ' }} =
e^{\lim_{x\to 0^+} \left( \frac{\frac{1}{x}}{-\frac{1}{x^2}} \right)} =  
e^{\lim_{x\to 0^+} -x} = \mathbf 1
\end{gather*}

Επομένως το ζητούμενο όριο δεν υπάρχει.

\paragraph{}
\[
\lim_{(x,y) \to(0,0)} \frac{x^2y^2}{x^2y^2+(x-y)^2}
\]

Θέτοντας \(y=\lambda x\), έχω:
\begin{align*}
\lim_{(x,y) \to(0,0)} \frac{x^2y^2}{x^2y^2+(x-y)^2}
&= \\ &=
\lim_{x \to 0} \frac{\lambda^2x^4}{\lambda^2x^4+x^2(1-\lambda)^2} \\
&=
\lim_{x \to 0} \frac{\lambda^2x^2}{\lambda^2x^2+x^2(1-\lambda)^2}
\end{align*}

Για \(\lambda=1\), γίνεται \(\lim_{x\to 0} \frac{x^2}{x^2}=1\).

Για \(\lambda=-1\), γίνεται \(\lim_{x\to 0} \frac{x^2}{x^2+4} = 0\).

Παρατηρούμε ότι για δύο διαφορετικές διαδρομές έχουμε διαφορετικό αποτέλεσμα, άρα το όριο δεν υπάρχει.

\paragraph{}
\begin{align*}
\lim_{(x,y,z)\to(0,0,0)} \frac{xy^2z^3}{x^2+y^2+z^2}
&=\\&=
\lim_{r\to0^+} \frac{r\sin\theta\cos\theta \cdot r^2\sin^2\theta\sin^2\phi \cdot r^3\cos^3\theta}{r^2} \\ &=
\lim_{r\to0^+} \underbrace{r^4}_{0} \cdot \big( \underbrace{ \cdots }_\text{φ} \big) = 0
\end{align*}

\paragraph{}
\begin{align*}
\lim_{(x,y)\to(0,0)} \frac{x+y+x^2}{x-y}
&= \\ &=
\lim_{r\to0^+} \frac{r\cos\phi+r\sin\phi+r^2\cos^2\phi}{r(\cos\phi-\sin\phi)}
\\ &= \lim
\frac{\cos\phi+\sin\phi}{\cos\phi-\sin\phi} + r
\frac{\cos^2\phi}{\cos\phi-\sin\phi}
\end{align*}
Επειδή παρατηρώ ότι υπάρχει πιθανότητα απροσδιοριστίας, θα προσπαθήσω να αποδείξω ότι δεν υπάρχει το όριο.

Θέτω \(y = \lambda x\):
\begin{align*}
\lim_{x\to0}\frac{x+\lambda x+x^2}{x-\lambda x} =
\lim_{x\to0}\frac{x(1+\lambda+x)}{x(1-\lambda)} =
\lim_{x\to0}\frac{1+\lambda+x}{1-\lambda} = \frac{1+\lambda}{1-\lambda}
\end{align*}

\paragraph{}
\begin{align*}
\lim_{(x,y)\to(0,0)} |x|^{|\frac{1}{y}|} &=
\lim_{(x,y)\to(0,0)} e^{\ln|x|^{|\frac{1}{y}|}} \\ &=
\lim_{(x,y)\to(0,0)} e^{\frac{1}{y} \ln |x|} \\ &=
e^{  \lim_{(x,y)\to(0,0)} \frac{\ln|x|}{y}   } \\ &=
e^{ \frac{\lim_{x\to0} \ln|x|  }{\lim_{y\to0} |y| }} \\ &=
e^\frac{-\infty}{0^+} \\ &= e^{-\infty} = 0
\end{align*}

\paragraph{}
\begin{align*}
\lim_{(x,y)\to (0,0)} \frac{x+y-1}{\sqrt{x}-\sqrt{1-y}} &= \\ &=
\lim_{(x,y)\to (0,0)} \frac{
(x+y-1) \left(   \sqrt{x}+\sqrt{1-y}     \right)
}{
\left(
\sqrt{x}-\sqrt{1-y}
\right)
\left(
\sqrt{x}+\sqrt{1-y}
\right)
} \\ &=
\lim_{(x,y)\to (0,0)}
\frac{
(x+y-1)\left(
\sqrt{x}+\sqrt{1-y}
\right)
}{
(\sqrt{x})^2-(\sqrt{1-y})^2
} \\ &=
\lim_{(x,y)\to(0,0)}
\frac{(x+y-1)(\sqrt{x}+\sqrt{1-y}}{|x|-|1-y|} \\ &=
\lim_{(x,y)\to(0,0)}
\frac{(x+y-1)(\sqrt{x}+\sqrt{1-y})}{x-(1-y)} = 0
\end{align*}


\paragraph{}
\begin{align*}
 \lim_{(x,y)\to(0,0)} \frac{\sin x + \sin y}{\tan(2x)+\sin y}
\end{align*}

\subparagraph{(1) \(y=x\)}
\begin{align*}
\lim_{x\to0}\frac{\sin x + \sin x}{\tan(2x)+\sin x} =
\lim_{x\to0}\frac{2\sin x}{\tan (2x) + \sin x} 
 = \lim_{x\to0}
 \frac{2\cos x}{\frac{2}{\cos^2(2x)+\cos x}}
 = \frac{2}{\frac{2}{1}+1} = \frac{2}{3}
\end{align*}
\subparagraph{(2) \(y=-x\)}
\[
\lim_{x\to0}
\frac{\sin x-\sin x}{\tan(2x)-\sin x}
= \lim_{x\to0} \frac{0}{\tan(2x)-\sin x} = 0
\]

Άρα το όριο δεν υπάρχει.

\paragraph{}
\begin{align*}
 \lim_{(x,y)\to(0,0)} \frac{\sqrt{x+y}-\sqrt{x-y}}{x}
\end{align*}
Θα βρω το πεδίο ορισμού:

\[
\begin{cases}
x+y \geq 0 &\implies y \geq -x \\
x-y \geq 0 &\implies y \leq x \\
x \neq 0
\end{cases}
\]
%TODO Zaharis Graph 02

Επειδή το \((0,1)\) είναι απομονωμένο σημείο (δεν είναι σημείο συσσώρευσης), δεν έχει νόημα ο υπολογισμός του ορίου εκεί.

\paragraph{}
\begin{align*}
 &\lim_{(x,y)\to(\infty,\infty)}
 e^{\frac{x+y}{x^2+y^2}}
 \left[
 1+ \sin \left(
 \frac{3}{|x|+|y|}
 \right)
 \right]^{|x|+|y|}
 = \\ &=
\lim_{r\to\infty}
e^{\frac{r\cos\phi+r\sin\phi}{r^2}}
\left[
 1+ \sin \left(
 \frac{3}{r\left(|\cos\phi|+|\sin\phi|\right)}
 \right)
 \right]^{r\left(|\cos\phi|+|\sin\phi|\right)}
 \\ &=
\cancelto{1}{\lim_{r\to\infty} e^{ \frac{\cos\phi+\sin\phi}{r}  }}
\cdot
\lim_{r\to\infty}
\left[
 1+ \sin \left(
 \frac{3}{r\left(|\cos\phi|+|\sin\phi|\right)}
 \right)
 \right]^{r\left(|\cos\phi|+|\sin\phi|\right)}
\end{align*}

Θέτω \(t=\frac{1}{r\left(|\cos\phi|+|\sin\phi|\right)}\), άρα το όριο γίνεται:
\begin{align*}
& \lim_{t\to0^+}
\left[
1+\sin(3t)
\right]^\frac{1}{t} \\
&=
e^{\lim_{t\to0^+} \ln \left[
1+\sin(3t)
 \right]^\frac{1}{t}}
 \\ &=
e^{\lim_{t\to0^+} \frac{\ln \left[
1+\sin(3t)\right]}{t}
 } \\
 &=
 e^{
 \lim{t\to0^+}
 \frac{3\cdot\cos(3t)}{1+\sin(3t)}
 }
\end{align*}

\paragraph{}
\begin{align*}
 \lim_{(x,y,z)\to(0,0,0)} \frac{x^2-y^2+2y^3-z}{x^2+y^2+z^2}
\end{align*}
Θέτω \(\begin{cases}y=\lambda x\\ z = \mu x\end{cases}\). Το όριο γίνεται:
\begin{align*}
& \lim_{x\to0^+}
\frac{x^2-\lambda^2x^2+2\lambda^3x^3-\mu x}{x^2+\lambda^2+\mu ^2 x^2}
\\ &=
\lim_{x\to0^+}
\frac{x(x-\lambda^2x+2\lambda^3x^2-\mu)}{x^2(1+\lambda^2+\mu^2)} \\ &=
\lim_{x\to0^+}
\frac{x(1-\lambda^2)+2\lambda^3x^2-\mu}{x(1+\lambda^2+\mu^2)} = \frac{-\mu}{0} =
\begin{cases}
-\infty &= \text{ για } \mu = 1 \\
\infty &= \text{ για } \mu = -1
\end{cases}
\end{align*}



\paragraph{}
\[
f(x,y) = \frac{xy}{x^2+y^2}+x^2y \sin \left(\frac{1}{x^2} \right)
\]
Όριο, διπλά όρια?

\begin{gather*}
 \lim_{(x,y)\to (0,0)} f(x,y) = 
  \lim_{(x,y)\to (0,0)} 
	\left[
\frac{xy}{x^2+y^2}
+ x^2y\sin \left( \frac{1}{x^2} \right)
	\right]
	=
	 \lim_{(x,y)\to (0,0)}  \frac{xy}{x^2+y^2}
+ \lim_{(x,y)\to (0,0)} 
\underbrace{x^2y}_\text{μ} \underbrace{\sin \left( \frac{1}{x^2} \right)}_\text{φ}
\\
 \lim_{r\to0^+} \frac{r\cos\theta \ r\sin\theta}{r^2}
 = \lim_{r\to0^+} \cos\theta \ \sin\theta
\end{gather*}
Εναλλακτικός τρόπος:
\[y=\lambda x,\quad  \lim_{x\to0}  \frac{\lambda x^2}{x^2(1+\lambda^2)} = \frac{\lambda}{1+\lambda^2}\]
Άρα δεν υπάρχει το όριο.

Για τα διπλά όρια:
\[
\lim_{y\to0}
\left(
\lim_{x\to0} f(x,y)
\right)
=
\lim_{y\to0} \left( \frac{0\cdot y}{0^2+y^2}+0^2y \sin \left( \frac{1}{0^2} \right) \right) = 0 = \lim_{y\to0}(0)=0
\]

\[
\lim_{x\to0}
\left(
\lim_{y\to0} f(x,y)
\right)
=
\lim_{y\to0} \left( \frac{x\cdot 0}{x^2+0^2}+x^2\cdot 0 \sin \left( \frac{x}{0^2} \right) \right) = 0 = \lim_{x\to0}(0)=0
\]
Παρατηρούμε ότι τα διπλά όρια υπάρχουν και είναι ίσα μεταξύ τους, αλλά το όριο της συνάρτησης δεν υπάρχει.

\paragraph{Άσκηση}
\[ f(x,y,z) = \frac{x\sin x + y\sin y + z\sin z}{x^2+y^2+z^2} 
\quad \mathbb R^3 -  \left\lbrace (0,0,0) \right\rbrace
\]
Τροποποίηση συνάρτησης ώστε να είναι συνεχής σε όλο το \(\mathbb R^3\), δηλαδή:
\[
f(x,y,z) = \begin{cases}
\frac{\cdots}{\cdots}, \quad &  \mathbb R ^3 -  \left\lbrace (0,0,0) \right\rbrace \\
??? & (0,0,0)
\end{cases}
\]

\begin{align*}
 & \lim_{\mathclap{(x,y,z)\to (0,0,0)} }
 \frac{x\sin x + y\sin y + z\sin z}{x^2+y^2+z^2} 
 =
  \lim_{(x,y,z)\to (0,0,0)} 
\left(
\frac{x^2}{x^2+y^2+z^2} \frac{\sin x}{x}
+\frac{y^2}{x^2+y^2+z^2} \frac{\sin y}{y}
+\frac{z^2}{x^2+y^2+z^2} \frac{\sin z}{z}
\right)
\\ &=
 \lim_{(x,y,z)\to (0,0,0)} \frac{x^2}{x^2+y^2+z^2}
\cdot \cancelto{1}{\lim_{x\to0} \frac{\sin x}{x}}
 + \lim_{(x,y,z)\to 0} \frac{y^2}{x^2+y^2+z^2}
\cdot \cancelto{1}{\lim_{y\to0} \frac{\sin y}{y}}
 + \lim_{(x,y,z)\to 0} \frac{z^2}{x^2+y^2+z^2}
\cdot \cancelto{1}{\lim_{z\to0} \frac{\sin z}{z}}
\\ &=
 \lim_{(x,y,z)\to (0,0,0)} 
 \frac{x^2+y^2+z^2}{x^2+y^2+z^2} =1 
\end{align*}

Άρα τελικά:
\[
f(x,y,z) = \begin{cases}
\frac{\cdots}{\cdots}, \quad &  \mathbb R ^3 -  \left\lbrace (0,0,0) \right\rbrace \\
1 & (0,0,0)
\end{cases}
\]

\paragraph{Άσκηση}
\[
f(x,y) = \frac{\sin x - \sin y}{\tan x - \tan y},
\quad D = \overbrace{\left[
0, \frac{\pi}{4}
\right]^2}^{\left[
0, \frac{\pi}{4}
\right] \times \left[
0, \frac{\pi}{4}
\right]} - \left\lbrace (x,y):\ x=y \right\rbrace
\]

Τροποποίηση της \(f(x,y)\) ώστε η \(f(x,y)\) να είναι συνεχής στο \(\left[
0, \frac{\pi}{4}
\right]^2\)

\begin{gather*}
\frac{\sin x - \sin y}{\frac{\sin x}{\cos x}-\frac{\sin y}{\cos y}}
= \frac{\sin x - \sin y}{\frac{\sin x \cos y - \sin y \cos x}{\cos x \cos y}}
= \frac{\sin x - \sin y}{\sin (x-y)} \cos x \cos y =\\
= \frac{\cancel{2 \sin \frac{x-y}{2}} \cos \frac{x+y}{2}}{\cancel{2\sin\frac{x-y}{2}} \cos\frac{x-y}{2}}
\cos x \cos y
= \frac{\cos\frac{x+y}{2}}{\cos\frac{x-y}{2}}\cos x \cos y
\end{gather*}

\begin{gather*}
 \lim_{(x,y)\to (x,x)}  f(x,y)
 = 
 \frac{\cos\frac{2x}{2}}{\cos 0} \cos x \cos x = \cos^3x
\end{gather*}

Άρα τελικά:
\[
f(x,y) = \begin{cases}
\frac{\sin x - \sin y}{\tan x - \tan y},\quad & \left[
0, \frac{\pi}{4}
\right]^2 - \left\lbrace (x,y):\ x=y \right\rbrace \\
\cos^3x,\quad & x=y
\end{cases}
\]

\subsection{Κατευθυνόμενη Παράγωγος}
%TODO Zaharis Graph 01
%TODO Zaharis Graph 02
\[
\pd{F(P_0)}{\vec{a}} = \nabla_{\vec{a}} f(P_0) = \vec{D}_{\vec{a}} f(P_0)
= \lim_{t\to0} \frac{\overbrace{f(P_0 +t\vec{a})}^{f(P)} - f(P_0)}{t}
= \lim_{t\to0} \frac{f(\vec{r}_{P_0}+t\vec{a})-f(\vec{r_{P_0}})}{t}
\]

\[
\frac{\Delta f}{t} = \tan \phi
\]

Μερικές παράγωγοι:
\begin{align*}
\frac{\partial f(P_0)}{\partial \vec{e_1}} &= \frac{\partial f(P_0)}{\partial x} = \lim_{t\to0} \frac{f(x_0+t,y_0)-f(x_0,y_0)}{t} \\
\frac{\partial f(P_0)}{\partial \vec{e_2}} &= \frac{\partial f(P_0)}{\partial y} = \lim_{t\to0} \frac{f(x_0,y_0+t)-f(x_0,y_0)}{t} \\
\end{align*}
(αντίστοιχα ορίζονται και για περισσότερες διαστάσεις - δε συμπεριλαμβάνεται σε αυτόν τον ορισμό ο άξονας των \(z\), αφού δεν είναι μέρος του πεδίου ορισμού)

\paragraph{Παράδειγμα}
Έστω \(f(x,y)=\frac{xy}{x^2+y^2}+x^2y\sin\left(\frac{1}{x^2}\right)\).

Τότε:
\begin{align*}
\pd{P(x,y)}{x} &= \frac{y(x^2+y^2)-xy\cdot2x}{(x^2+y^2)^2}
+ y \left[
2x\sin\left(\frac{1}{x^2}\right)
+x^2 \cdot \cos\left( \frac{1}{x^2}\right)\left( -\frac{2}{x^3} \right)
\right] \\
\pd{P(x,y)}{y} &=\frac{x(x^2+y^2)-xy\cdot2y}{(x^2+y^2)^2}+x^2\sin \left( \frac{1}{x^2}\right)
\end{align*}

\paragraph{}
Για να είναι συνεχής μια συνάρτηση στο σημείο \(P_0\), αρκεί:
\[
\begin{cases}
\exists f_x, f_y\quad \pi_\epsilon (P_0) \\
f_x,f_y\text{ πεπερασμένες}
\end{cases}
\implies\text{ΜΕΡΙΚΩΣ ΠΑΡΑΓΩΓΙΣΙΜΗ}
\]
Αν η συνάρτηση είναι μερικώς παραγωγίσιμη, υπάρχει η λύση (\textlatin{Gradient}) συνάρτησης.

\subsubsection{\textlatin{Gradient} συνάρτησης \(f(x_1,\dots,x_n)\)}
\begin{gather*}
\mathrm{grad}\,f = \nabla f(x_1,\dots,x_n) =
\left[
f_{x_1}(P)\dots f_{x_n}(P)
\right] \\
\nabla f = \left[ f_x f_y \right]
\end{gather*}

\subsubsection{}
\[
\nabla f(P_0) =
\left(
f_{x_1}(P_0),f_{x_2}(P_0),\dots,f_{x_n}(P_0)
\right)
\]

Η ύπαρξη του \(\nabla f(P_0)\) προϋποθέτει:
\begin{enumerate}
\item \(\exists f_{x_i}(P_0)\quad i=1,\dots,n\)
\item \(f_{x_i}(P_0) \rightarrow \text{ πεπερασμένη}\)
\end{enumerate}

\paragraph{Πότε η \(f\) είναι συνεχής στο \(P_0\)?}
%TODO Zaharis Graph 01

\paragraph{Πότε η \(f\) είναι παραγωγίσιμη (διαφορίσιμη) στο \(P_0\)?}
\begin{itemize}
\item \(\exists \ f_{x_i}\)
\item \(f_{x_i}\) συνεχείς
\end{itemize}

\subsection{}
\[
f'(x_0) = \lim_{x\to x_0}\frac{f(x)-f(x_0)}{x-x_0}
\implies
\lim_{x\to x_0} \left|
\frac{f(x)-f(x_0)-f'(x_0)(x-x_0)}{x-x_0}
\right| = 0
\]

%TODO Zaharis Graph 02
\[
\lim_{P\to P_0}
\frac{
\Big|
f(P)-f(P_0)-f'(P_0)(\overbrace{P-P_0}^{\vec{r_P}-\vec{r_{P_0}}})
\Big|
}{|\overrightarrow{PP_0}|}
\]

Τα διανύσματα γνωρίζουμε ότι γράφονται ως πίνακες μίας διάστασης:
\[
\vec{r_P}-\vec{r_{P_0}} =
\begin{bmatrix}
x_{1P}-x_{1P_0}\\
x_{2P}-x_{2P_0}\\
\vdots\\
x_{nP}-x_{nP_0}
\end{bmatrix}
\]

\[
f'(x_0)=\begin{bmatrix}
b_1&b_2&\cdots&b_n
\end{bmatrix}
\]

\begin{align*}
\lim_{t\to0}
\frac{
\left|
f(P_0+t\vec{a})-f(P_0)-f'(P_0)t\vec{a}
\right|
}{|t||\vec{a}|} = 0 &\implies \\
\lim_{t\to0}
\left|
\frac{f(P_0+t\vec{a})-f(P_0)-f'(P_0)t\vec{a}}{t}
\right| = 0 &\implies \\
\left|
\lim_{t\to0}
\frac{f(P_0+t\vec{a})}{t}
- f'(P_0)\vec{a}
\right| = 0 &\implies \\
\frac{\partial f(P_0)}{\partial \vec{a}} - f'(P_0)\vec{a} = 0 &\implies\\
\frac{\partial f(P_0)}{\partial \vec{a}} = f'(P_0)\cdot \vec{a}
\end{align*}

\paragraph{Συσχέτιση με \textlatin{Gradient}}
\begin{align*}
\pd{f(P_0)}{\vec{e_i}} = f'(P_0)\vec{e_1} &\implies \\
\pd{f(P_0)}{\vec{e_i}} = 
\begin{bmatrix}
b_1&\cdots&b_i&\cdots&b_n
\end{bmatrix}
\begin{bmatrix}
0\\ \vdots\\ 1  \\ \vdots \\ 0
\end{bmatrix} \rightarrow \text{ 1 μόνο γραμμή}
 &\implies \\
\boxed{\pd{f(P_0)}{x_1}=b_i}
\end{align*}

\[
f'(P_0) = \begin{bmatrix}
b_1&b_2&\cdots&b_n
\end{bmatrix}
=
\begin{bmatrix}
f_{x_1}(P_0)
&f_{x_2}(P_0)
&\cdots
&f_{x_n}(P_0)
\end{bmatrix}
= \mathbf{\nabla f(P_0)}
\]

Βέβαια η \(f'(P_0)\) ορίζεται μόνο αν οι επιμέρους παράγωγοι είναι συνεχείς!

\paragraph{Φυσική σημασία}
\begin{align*}
\left|
\pd{f(P_0)}{\vec{a}} 
\right|
= \left|
f'(P_0)\vec{a}
\right|
\implies
\left|
\pd{f(P_0)}{\vec{a}} 
\right|
= \left|
\nabla f(P_0)\cdot\vec{a}
\right|
\leq
\left|
\nabla f(P_0)
\right|
\cdot
\cancelto{1}{|\vec{a}|}
\implies \boxed{
\left|
\pd{f(P_0)}{\vec{a}}
\right|
\leq
\left|
\nabla f(P_0)
\right|}
\end{align*}

Άρα το \textlatin{gradient} προσδιορίζει το μέγιστο ρυθμό μεταβολής της συνάρτησης.

Το διάνυσμα της κατεύθυνσης όπου μεγιστοποιείται ο ρυθμός μεταβολής είναι το:
\[
\vec{a} = \frac{\nabla f(P_0)}{\left| \nabla f(P_0) \right|}
\]

\subsection{Θεώρημα Μέσης Τιμής}
\[
f:\underbrace{E}_{\mathclap{\text{κυρτό σύνολο}}} \subseteq  \mathbb R ^n \to R
\]

%TODO Zaharis Graph 03
Επάνω στην ευθεία που ενώνει τα \(P_0\) και \(P_1\), υπάρχει σημείο \(P^*\) τέτοιο ώστε:
\[
f(P_1)-f(P_0) = f'(P^*)\big( P_1-P_0\big)
\]

\subsubsection{Μερικές παράγωγοι ανώτερης τάξης}
\[
\frac{\partial^m f(P)}{\partial x_{km} \dots \partial x_{k2} \partial x_{k1}}
= f_{x_{k1}x_{k2}\dots x_{km}}(P) =
\frac{\partial}{\partial x_{km}}
\left(
\cdots
\left(
\frac{\partial}{\partial x_{k2}}
\left(
\frac{\partial f(P)}{\partial x_{k1}}
\right)
\right)
\right)
\]

\paragraph{π.χ.}
\[
f=f(x,y,z)
\]
\begin{gather*}
\pd{f}{x},\pd{f}{y},\pd{f}{z}\\
\frac{\partial^2 f}{\partial x \partial y},\
\frac{\partial^2 f}{\partial y \partial x},\
\frac{\partial^2 f}{\partial x \partial x},\
\frac{\partial^2 f}{\partial x \partial z},\
\frac{\partial^2 f}{\partial z \partial x},\
\frac{\partial^2 f}{\partial x \partial x}=\pd[2]{f}{x},\
\frac{\partial^2 f}{\partial y \partial y}=\pd[2]{f}{y},\
\end{gather*}

Η συνέχεια των μερικών παραγώγων μέχρι κάποια τάξη, επιτρέπει την αντιμετάθεση των παραγώγων.

Για παράδειγμα, αν η \(f\) έχει συνεχείς μερικές παραγώγους μέχρι και 2ης τάξης, ισχύει:
\begin{gather*}
\frac{\partial^2f}{\partial x\partial y} =
\frac{\partial^2f}{\partial y\partial x}\\
\frac{\partial^3f}{\partial x\partial y\partial z} \neq
\frac{\partial^3f}{\partial y\partial z\partial x}
\end{gather*}

Επίσης συμβολίζω:
\begin{gather*}
\frac{\partial f}{\partial x} = f_x\\
\frac{\partial^3 f(P)}{\partial x \partial y \partial z} = f_{zyx}(P)\\
\frac{\partial^3 f(P)}{\partial x^2 \partial y} = f_{yx^2}(P)
\end{gather*}
(προσοχή στην αλλαγή φοράς των \(x,\ y,\ z\)!)

Να σημειωθεί ότι η χρήση των \(\partial\) και \(f_{\dots}\) είναι καθαρά θέμα συμβολισμού.

\paragraph{Αρμονική συνάρτηση}
\(
f=f(_1,\dots,x_n)
\)

\[
f_{x_1^2}(P)+f_{x_2^2}(P)+\dots+f_{x_n^2}(P) = 0
\iff
\pd[2]{f(P)}{x_1}+\pd[2]{f(P)}{x_2}+\dots+\pd[2]{f(P)}{x_n}= 0
\]
και τα μέγιστα \& ελάχιστα της συνάρτησης λαμβάνονται πάνω στο όριο του πεδίου ορισμού της.

Ονομάζω το \(
\pd[2]{}{x_1}+\pd[2]{}{x_2}+\dots+\pd[2]{}{x_n}
\) τελεστή \textlatin{Laplace} (Λαπλασιανή), και συμβολίζω:

\[
\nabla^2 = \pd[2]{}{x_1}+\pd[2]{}{x_2}+\dots+\pd[2]{}{x_n} \rightarrow
\text{ Τελεστής \textlatin{Laplace}}
\]

Άρα ο επάνω ορισμός γίνεται:
\[
\iff \boxed{\nabla^2 f(P)} = 0
\]

\paragraph{Διαφορικό}
%TODO Zaharis Graph 04
\begin{align*}
\overbrace{\dif f(P_0)}^{\mathclap{\text{διαφορικό 1\textsuperscript{ης} τάξης ή ολικό διαφορικό}}} &= \pd{f(P_0)}{x}\dif x + \pd{f(P_0)}{y} \dif y
\quad (+\dots) \\
&=
\begin{bmatrix}
f_x(P_0)&f_y(P_0)
\end{bmatrix}
\begin{bmatrix}
\dif x\\ \dif y
\end{bmatrix}
= \nabla f(P_0) \dif P
= \nabla f(P_0) \dif \vec{r}
\end{align*}

\subparagraph{Διαφορικό 2\textsuperscript{ης} τάξης}
\begin{align*}
\dif^2 f  = \dif (\dif f) &= \dif
\left(
f_x \dif x + f_y \dif y
\right) \\
&= \pd{}{x} (f_x \dif x + f_y \dif y) \dif x
+ \pd{}{y} (f_x \dif x + f_y \dif y) \dif y
\\ &=
\left(
f_{x^2} \dif x + f_x \pd{\dif x}{x} + f_{yx} \dif y + f_y \cancel{\pd{\dif y}{x}}
\right) \dif x
+
\left(
f_{xy} \dif x + f_x \cancel{\pd{\dif x}{y}} + f_{y^2} \dif y + f_y \pd{\dif y}{y}
\right) \dif y
\\ &=
\underbrace{f_{x^2} (\dif x)^2}
+ f_x \dif^2 x
+ \underbrace{f_{yx} \dif y \dif x
+ f_{xy} \dif x \dif y}
+ \underbrace{f_{y^2} (\dif y)^2}
+ f_y \dif^2 y
%TODO
\\ & \overset{\mathclap{\text{εφόσον υπάρει συνέχεια παραγώγων}}}{=}
f_{x^2}(\dif x)^2
+2f_{xy} \dif x \dif y +
f_{y^2} (\dif y)^2
+ f_x \dif^2 x + f_y \dif^2 y
\end{align*}

Αν \(\dif x = \) σταθ. και \(\dif y = \) σταθ. τότε \(\dif^2 x =\dif^2 y = 0\).

\[
\dif^2 f = (f_x\dif x+f_y\dif y)^{(2)}
\]

Ομοίως
\[
\dif^3 f = (f_x\dif x+f_y\dif y)^{(3)} = f_{x^3} (\dif x)^3+3f_{x^2y}(\dif x)^2\dif y + 3 f_{xy} = \dif x (\dif y)^2 + f_{y^3} (\dif y)^3
\]

\subsubsection{Κριτήριο ύπαρξης ολικού διαφορικού}
\[
P(x,y) \dif x + Q(x,y) \dif y \qquad P,Q\to\text{συνεχείς μερικές παραγώγους 1\textsuperscript{ης} τάξης}
\]

\begin{align*}
f=f(x,y)\qquad \dif f &= P(x,y)\dif x + Q(x,y)\dif y & f_x(x,y) = P(x,y) \implies^{\pd{}{y}} f_{xy}=P_y
\\ \dif f &= f_x(x,y)\dif x + f_y(x,y)\dif y		 & f_y(x,y) = Q(x,y) \implies^{\pd{}{x}} f_{yx} = Q_x
\\
\implies \boxed{P_y=Q_x}
\end{align*}

\begin{align*}
\int_{A_0}^A \dif f=
\int_{(x_0,y_0)}^{(x,y)} \left[ P(x,y)\dif x + Q(x,y)\dif y \right]
\implies \\
f(A)-f(A_0) = 
\int_{(x_0,y_0)}^{(x,y_0)}\left[
P(x,y)\dif x +\cancel{Q(x,y)\dif y}
\right] + \int_{(x,y_0)}^{(x,y)}\left[
\cancel{P(x,y)\dif x} +Q(x,y)\dif y
\right]\implies \\ \boxed{\qquad \qquad
f(A)=f(A_0) + \underbrace{\int_{x_0}^x P(t,y_0) \dif t + \int_{y_0}^y Q(x,t)\dif t}_{\mathclap{\attnboxed{\text{προσοχή στα ορίσματα! γίνονται συχνά λάθη!}}}}
\qquad \qquad
}
\end{align*}

\paragraph{}
\(\vec F = (P,Q)\)
\[
\underbrace{df}_{\mathclap{\text{βαθμωτό ή αριθμητικό δυναμικό του πεδίου $\vec F$}}}
= \vec F \dif \vec r = (P,Q)\cdot (\dif x,\dif y) = P\dif x + Q\dif y
\]

\paragraph{Σε 3 διαστάσεις}
\[
P(x,y,z)\dif x + Q(x,y,z) \dif y + R(x,y,z)\dif z
\]
\(
f=f(x,y,z)
\)

\[
\begin{cases}
\dif f &= P(x,y,z)\dif x + Q(x,y,z)\dif y + R(x,y,z)\dif z\\
\dif f &= f_x(x,y,z)\dif x + f_y(x,y,z)\dif y + f_z(x,y,z)\dif z
\end{cases}
\implies
\begin{cases}
f_x&=P\\
f_y&=Q\\
f_z&=R
\end{cases}
\implies
\begin{cases}
\begin{cases}
f_{xy}&=P_y\\
f_{yx}&=Q_x
\end{cases}
&\implies P_y=Q_x\\
\begin{cases}
f_{yz}&=Q_z\\
f_{zy}&=R_y
\end{cases}
&\implies Q_z=R_y\\
\begin{cases}
f_{xz}&=P_z\\
f_{zx}&=R_x
\end{cases}
&\implies P_z=R_x
\end{cases}
\]

\begin{align*}
\begin{cases}
\dif f& = P\dif x  + Q\dif y + R\dif z = (P,Q,R)\cdot (\dif x,\dif y,\dif z)  = \vec F\cdot \dif \vec r \\
\dif f &= f_x\dif x +f_y\dif y +f_z\dif z = (f_x,f_y,f_z)(\dif x,\dif y,\dif z) = \nabla f \dif \vec r
\end{cases}
&\implies
\vec F = \nabla f \\&
\implies \nabla \times \vec F = \nabla \times \nabla f \\ &\implies \boxed{
\nabla \times \vec F = 0
}
\end{align*}

\(\vec F = (P,Q,R)\)
\begin{align*}
\nabla \times \vec F &= \left|
\begin{matrix}
\overrightarrow{e_1}&\overrightarrow{e_2}&\overrightarrow{e_3}\\
\pd{}{x}&\pd{}{y}&\pd{}{z}\\
P&Q&R
\end{matrix}
\right|
=
\overrightarrow{e_1}\left(
\pd{R}{y}-\pd{Q}{z}
\right)
-
\overrightarrow{e_2}\left(
\pd{R}{x}-\pd{P}{z}
\right)
+
\overrightarrow{e_3}\left(
\pd{Q}{x}-\pd{P}{y}
\right)\\
\nabla \times \vec F &= \vec e_1 (R_y-Q_z) - \vec e_2 (R_x-P_z) + \vec e_3 (Q_x-P_y) = 0 \implies
\begin{cases}
R_y&=Q_z\\
R_x&=P_z\\
Q_z&=P_y
\end{cases}
\end{align*}

\subparagraph{Αστρόβιλο πεδίο}
\begin{enumparen}
\item \(\vec F = \nabla f\)
\item \(\nabla \times \vec F = 0\)
\item \( \int_{A_0}^A \vec F\dif\vec r = f(A)-f(A_0)\)
\end{enumparen}

\[
\int_{A_0}^A \vec F \dif r = \int_{A_0}^A \nabla f \dif \vec r = \int_{A_0}^A \dif f = f(A)-f(A_0)
\]

Ένα αστρόβιλο πεδίο είναι και συντηρητικό όταν ο χώρος είναι απλά συνεκτικός.

\paragraph{}
%TODO Zaharis Graph 02
\begin{align*}
f(A)-f(A_0)=
\int_{A_0}^A \vec F \dif r \implies \\ f(A)=f(A_0)+\int_{A_0}^B (\cancel{P\dif x+Q\dif y} + R\dif z) +
\int_B^\Gamma (\cancel{P\dif x}+Q\dif y +\cancel{P\dif z}) + \int_\Gamma^A (P\dif x + \cancel{Q\dif y+R\dif z})
\implies \\
f(A)=f(A_0)+\int_{z_0}^z R(x_0,y_0,t)\dif t + \int_{y_0}^y Q(x_0,t,z)\dif t +\int_{x_0}^x P(t,y,z)\dif t
\end{align*}

\paragraph{Άσκηση}
\[
\underbrace{(3x^2+6xy^2)}_{P(x,y)=f_x}\dif x + \underbrace{(6x^2+4y^3)}_{Q(x,y)=f_y}\dif y
\]
Να δειχθεί ότι η έκφραση αυτή αποτελεί ολικό διαφορικό μιας συνάρτησης \(f(x,y)\), και να βρεθεί η μορφή της συναρτησης αυτής.
\begin{align*}
\begin{cases}
P=3x^2+6xy^2 &\implies P_y = 12xy\\
Q=6x^2y+4y^3 &\implies Q_x=12xy
\end{cases}
\implies P_y=Q_x
\end{align*}

\begin{align*}
f_x=3x^2+6xy^2&\implies\pd{f}{x}=3x^2+6xy^2\\ &\implies\dif f = (3x^2+6xy^2)\dif x \\ & \implies
f(x,y)=\int(3x^2+6xy^2)\dif x + g(y)
\\ & \implies f(x,y) = x^2+3x^2y^2+g(y) (1)
\end{align*}

\begin{align*}
f_y=6x^2y+4y^2&\implies\pd{f}{y}=6x^2y+4y^3\\ &\implies
6x^2y+\pd{g(y)}{y}=6x^2y+4y^3 \\ &\implies
g(y)=\int 4y^3\dif y + c \\ &\implies
g(y)=y^4+c (2)
\end{align*}

\[
(1) \text{ κ } (2) \implies \boxed{
f(x,y)=x^3+3x^2y^2+y^4+\cancel{c}
}
\]
Αν θεωρήσουμε ότι \(x=y=0 \implies \boxed{c=0}\)

\subsubsection{Συναρτηστιακή εξάρτηση}
\[
f_1(x_1,\dots,x_n),\ f_2(x_1,\dots,x_n),\dots,f_m(x_1,\dots,x_n)
\]
\(
\phi(f_1,\dots,f_n)=0
\)
\[
\text{Ιακωβιανός πίνακας } J = \left[ \begin{matrix}
\pd{f_1}{x_1}&\pd{f_1}{x_2}&\cdots&\pd{f_1}{x_n}\\
\pd{f_2}{x_1}&\pd{f_2}{x_2}&\cdots&\pd{f_2}{x_n}\\
\vdots & & \ddots &\\
\pd{f_m}{x_1}&\pd{f_m}{x_2}&\cdots&\pd{f_m}{x_n}\\
\end{matrix} \right]
\]

Για \(m>n\) πάντα υπάρχει συναρτησιακή εξάρτηση.

\[
\begin{cases}
\mathrm{rank}[J] < m \rightarrow&f_1,\dots,f_m \text{ συναρτησιακά εξαρτημένες}\\
\mathrm{rank}[J] = m \rightarrow&f_1,\dots,f_m \text{ συναρτησιακά ανεξάρτητες}
\end{cases}
\]

Ειδική περίπτωση \(\mathbf{m=n}\):
\[
\begin{cases}
|J| =0 \rightarrow&f_1,\dots,f_m \text{ συναρτησιακά εξαρτημένες}\\
|J| \neq0 \rightarrow&f_1,\dots,f_m \text{ συναρτησιακά ανεξάρτητες}
\end{cases}
\]

Αν \(m<n\):
\begin{align*}
&rank[J] \leq \min(m,n) = n <m \implies\\&
rank[J]<m \rightarrow \rightarrow&f_1,\dots,f_m \text{ συναρτησιακά εξαρτημένες}
\end{align*}

\paragraph{Άσκηση}
\[f_1=ye^x\cos z,\ f_2=ye^x\sin z,\ f_3=y^2e^{2x}\]
Να βρεθεί αν οι συναρτήσεις είναι συναρτηστιακά εξαρτημένες.

\subparagraph{}
\(\mathbf{m=n=3}\)
\begin{align*}
|J|=&\left|
\begin{matrix}
f_{1x}&f_{1y}&f_{1z}\\
f_{2x}&f_{2y}&f_{2z}\\
f_{3x}&f_{3y}&f_{3z}
\end{matrix}
\right|\\
=&\left|
\begin{matrix}
ye^x\cos z&e^x\cos z&-ye^x\sin z\\
ye^x\sin z&e^x\sin z & ye^x\cos z\\
2y^2e^{2x}&2ye^{2x}&0
\end{matrix}
\right|\\
=&\left|
\begin{matrix}
2y^2e^{2x}(e^{2x}y\cos^2 z +e^{2x}y\sin^2z)
-2ye^{2x}(e^{2x}y^2\cos^2z+e^{2x}y^2\sin^2 z)
\end{matrix}
\right|
\\=&
2y^2e^{2x}e^{2x}y\cancel{(\cos^2 z+\sin^2 z)}-2ye^{2x}e^{2x}y^2\cancel{(\cos^2 z + \sin^2 z)}=0
\end{align*}

Με το μάτι φαίνεται ότι η εξάρτηση είναι \(f_1^2+f_2^2=f_3\).


\subsection{Ασκήσεις}
\paragraph{1)}
\begin{align*}
f(x,y)=&\ln\left[\tan\left(\frac{x}{y}\right)\right]\\
g(x,y)=&x^{x^y}\\
h(x,y,z)=&\arctan\left(\frac{x+y+z}{x-y}\right)
\end{align*}
\subparagraph{}
\begin{align*}
f_x=&\frac{1}{\tan\left(\frac{x}{y}\right)}\frac{1}{\cos^2\left(\frac{x}{y}\right)}\frac{1}{y}=
\frac{1}{\frac{\sin(x/y)}{\cos(x/y)}}\frac{1}{\cos^2(x/y)}\frac{1}{y}=\frac{2}{y\cdot \sin(2x/y)}\\
f_y=&\frac{1}{\tan\left(\frac{x}{y}\right)}\frac{1}{\cos^2\left(\frac{x}{y}\right)}\left(-\frac{x}{y^2}\right)=-\frac{1}{\frac{\sin(x/y)}{\cos(x/y}}\frac{1}{\cos^2(x/y)}\frac{x}{y^2}=-\frac{2x}{y^2\sin(2x/y)}\\
g_x=&\left(x^{x^y}\right)_x=\left(e^{e^{\sin x} \cdot \ln x}\right)_x=
e^{e^{y\ln x} \cdot \ln x} \left( e^{y\ln x} \cdot \ln x\right)_x\\
=& x^{x^y} \left[
\left(e^{y\ln x}\right)_x\cdot \ln x + e^{y\ln x} \left( \ln x\right)_x
\right]
=
x^{x^y} \left(
e^{y\ln x} \cdot \frac{y}{x} \ln x + e^{y\ln x} \frac{1}{x}
\right)
\\=&
x^{x^y}\left(
x^y\frac{y}{x}\ln x+x^y\frac{1}{x}
\right)=x^{x^y}\cdot x^{y-1}\left(y\ln x+1\right) = x^{x^y+y-1}(y\ln x+1)
\\
g_y
=&
\left(
x^{x^y}
\right)_y
=
\left(
e^{e^{y\ln x}\ln x}
\right)_y
= e^{e^{y\ln x} \ln x} \left(e^{y\ln x \ln x}\right)_y
\\=&
x^{x^y}\ln x e^{y\ln x}=x^{x^y} (\ln x)^2 x^y = x^{x^y+y} (\ln x)^2
\\
h_x
=&
\frac{1}{1+\left(\frac{x+y+z}{x-y}\right)^2}
\cdot\frac{1(x-y)-(x+y+z)}{(x-y)^2}=\frac{\cancel{(x-y)^2}}{(x-y)^2+(x+y+z)^2}
\frac{-2y-z}{\cancel{(x-y)^2}}=\frac{-2y-z}{(x-y)^2+(x+y+z)^2}
\\
h_y
=&
\frac{1}{1+\left(\frac{x+y+z}{x-y}\right)^2}\cdot \frac{1(x-y)-(x+y+z)(-1)}{(x-y)^2}
= \frac{(x-y)^2}{(x-y)^2+(x+y+z)^2}\frac{2x+2}{(x-y)^2}=\frac{2x+z}{(x-y)^2+(x+y+z)^2}
\\
h_z
=&
\frac{1}{1+\left(\frac{x+y+z}{x-y}\right)^2}\frac{1}{x-y}=\frac{x-y}{(x-y)^2+(x+y+z)^2}
\end{align*}

\paragraph{2)}
\begin{gather*}
z=x^3-xy+3y^2\\
P_0=(5,\ 4) \rightarrow P=(4.8,\ 4.1)
\end{gather*}
\subparagraph{}
\begin{align*}
\dif x =& -0.2 \implies x-x_0 = -0.2 \implies x=x_0-0.2=4.8\\
\dif y =&  0.1 \implies y-y_0 =  0.1 \implies y=y_0+0.1=4.1\\
\Delta z =& z_P -z_{P_0} = (4.8^3-4.8\cdot4.1+3\cdot4.1)-(5^3-5\cdot 4 + 3 \cdot 4^2) = -4.658\\
\dif z_{P_0} =& z_x(P_0)\dif x + z_y(P_0)\dif y \\
=&(3x^2-y_0)\dif x+(-x_0+6y_0)\dif y\\
=& (3\cdot5^2-4)(-0.2)+(-5+6.4)0.1 = \mathbf{-12.3}
\end{align*}

\paragraph{3)}
\begin{gather*}
f(x,y,z)=xyz\\
P_0=(1,\ 0,\ 3) \rightarrow P_1 =(4,\ 1,\ 0)
\end{gather*}
\begin{enumparen}
\item Ρυθμός μεταβολής της $f$
\item Μέγιστος ρυθμός μεταβολής και αντίστοιχη κατεύθυνση
\item Ελάχιστος ρυθμός μεταβολής και αντίστοιχη κατεύθυνση
\end{enumparen}
\subparagraph{(α)}
\begin{gather}
\vec a = \frac{\overrightarrow{P_0P_1}}{|P_O\vec{P_1}|}=\frac{(3,1,-3)}{\sqrt{3^2+1^2+3^2}}=\left(
\frac{3}{\sqrt{19}},\frac{1}{\sqrt{19}},-\frac{3}{\sqrt{19}}
\right)\\
\pd{f(P_0)}{\vec a} = \nabla f(P_0)\vec a\\
\nabla f(P_0) = \left(f_x(P_0),f_y(P_0),f_z(P_0)\right) = (y_0z_0,x_0z_0,x_0y_0) = (0,3,0)
\\
(2)\xRightarrow{(1),\; (3)}\pd{f(P_0)}{\vec{a}} = (0,3,0)\cdot\left(
\frac{3}{\sqrt{19}},\frac{1}{\sqrt{19}},-\frac{3}{\sqrt{19}}\right)=\boxed{\frac{3}{\sqrt{19}}}
\end{gather}
\subparagraph{(β)}
\[
\vec e_{\max} = \frac{\nabla f(P_0)}{\left| \nabla f(P_0) \right|}=\frac{(0,3,0)}{3}=(0,1,0)
\]
\subparagraph{(γ)}
\[
\vec e_{\min} = \frac{-\nabla f(P_0)}{|\nabla f(P_0)|}=(0,-1,0)
\]

\paragraph{4)}
\[
f(x,y,z) = \sqrt{x^2+y^2+z^2}
\]

Να αποδείξετε ότι η $f$:
\begin{enumparen}
\item Δεν είναι αρμονική
\item Είναι διαρμονική
\end{enumparen}

\begin{attnbox}{Αρμονικές συναρτήσεις}
\begin{itemize}
\item \textbf{Αρμονική:} \(\nabla^2 f = f_{x^2}+f_{y^2}+f_{z^2} = 0\)
\item \textbf{Διαρμονική:} \(\nabla^2 \big( \nabla^2 f \big) =0\)
\end{itemize}
(\(\forall x,y,z\))
\end{attnbox}

\subparagraph{}
\begin{gather*}
f_x=\pd{f}{x}=\frac{2x}{2\sqrt{x^2+y^2+z^2}}=\frac{x}{f}, \quad f_y=\frac{y}{f},\quad f_z=\frac{z}{f}\\
f_{x^2}=(f_x)_x = \left(\frac{x}{f}\right)_x=\frac{1\cdot f - xf_x}{f^2}=\frac{f-x\cdot\frac{x}{f}}{f^2}=\frac{f^2-x^2}{f^3}\\
f_{y^2}=\frac{f^2-y^2}{f^3},\ f_{z^2}=\frac{f^2-z^2}{f^3}\\
\nabla^2 f = \frac{f^2-x^2}{f^3}+\frac{f^2-y^2}{f^3}+\frac{f^2-z^2}{f^3}=\frac{3f^2-(x^2+y^2+z^2)}{f^3}=\frac{3f^2-f^2}{f^3}=\frac{2f^2}{f^3}=\frac{2}{f}\neq0
\end{gather*}
\subparagraph{}
\begin{gather*}
\nabla^2\left(\nabla^2 f\right) = \nabla^2\left(\frac{2}{f}\right) = 2\left[
\left(\frac{1}{f}\right)_{\mathclap{x^2}}
+\left(\frac{1}{f}\right)_{\mathclap{y^2}}
+\left(\frac{1}{f}\right)_{z^2}
\right]\\
\left(\frac{1}{f}\right)_x = -\frac{f_x}{f^2}=-\frac{x}{f^3}\\
\left(\frac{1}{f}\right)_{x^2} = \left(-\frac{x}{f^3}\right)_x=-\frac{f^3-x^3f^2f_x}{f^6}=-\frac{f^3-3xf^2\frac{x}{f}}{f^6}=-\frac{f(f^2-3x^2)}{f^6}=-\frac{f^2-3x^2}{f^5}\\
\intertext{Άρα:}
\nabla^2\left(\nabla^2 f\right) 2 \left[
\left(\frac{1}{f}\right)_{x^2}
+\left(\frac{1}{f}\right)_{y^2}
+\left(\frac{1}{f}\right)_{z^2} 
\right] =\\= 2\left(
-\frac{f^2-3x^2}{f^5}-\frac{f^2-3y^2}{f^5}-\frac{f^2-3z^2}{f^5}
\right) =
-2\frac{3f^2-e(x^2+y^2+z^2)}{f^5}=-2\frac{3f^2-3f^2}{f^5}=0
\end{gather*}

\paragraph{5)}
\[
z=f(x^2+y^2)
\]
με συνεχείς παραγώγους 2\textsuperscript{ης} τάξης
\begin{enumparen}
\item ν.δ.ο \(yz_x-xz_y=0\)
\item ν.δ.ο \(y^2z_{x^2}-2xyz_{xy}+x^2z_{y^2}=xz_x+yz_y\)
\end{enumparen}

\subparagraph{}
\(z=f(v)\) όπου \(v=x^2+y^2=g(x,y) \quad \pd{z}{v}=\od{z}{v}\)
\begin{gather*}
z_x=\pd{z}{x}=\pd{z}{v}\od{v}{x}=f_v \cdot 2x = 2xf_v
\qquad
z_y=\pd{z}{y}=\pd{z}{v}\pd{v}{y}=2yf_v\\
yz_x -xz_y=y2xf_v-x2yf_v=0
\end{gather*}

\subparagraph{}
\begin{gather*}
z_{x^2}=(z_x)_x=(2xf_x)_x=2\left(f_v+x\pd{f_v}{x}\right)=2(f_v+x\pd{f_v}{v}\pd{v}{x}) = 2(f_v+xf_{y^2}2x) = 2f_v+4x^2f_{v^2}\\
z_{y^2}=(z_y)_y=(2yf_v)_y=2f_v+4y^2f_{v^2}\\
z_{xy}=(z_x)_y=(2xf_v)_y=2x(f_v)_y=2x\od{f_v}{y}=2x\pd{f_v}{v}\pd{v}{y}=2xf_{v^2}2y=4xyf_{v^2}
\intertext{Άρα:}
y^2z_{x^2}-2xyz_{xy}+x^2z_{y^2}=y^2(2f_v+4x^2f_{v^2})-2x6\cdot 4xy f_{v^2}+x^2(2f_v+4y^2f_{v^2})=\\=
2y^2f_v+4x^2y^2f_{v^2}-8x^2y^2f_{v^2}+2x^2f_v+4x^2y^2f_{v^2}=x(2xf_v)+y(2yf_v)=xz_x+yz_y
\end{gather*}

\paragraph{Άσκηση}
Δίνεται συνάρτηση $z=z(u,v)$ όπου $u=e^x\cos y$ κ $v=e^x\sin y$.

Αν η $z$ έχει συνεχείς μερικές παραγώγους 2\textsuperscript{ης} να δειχθεί ότι:
\begin{enumparen}
\item \(z_x^2+z_y^2 = (u^2+v^2)(z_u^2+z_v^2)\)
\item \(z_{x^2}+z_{y^2} = (u^2+v^2)(z_{u^2}+z_{v^2})\)
\end{enumparen}

\begin{infobox}{}
\begin{gather*}
z=z(u,v) \implies \dif z =z_u \dif u + z_v \dif v \implies \\
\left.
\begin{matrix}
u = g(x,y)\\
v=h(x,y)
\end{matrix}
\middle|
\begin{matrix}
\dif z = z_u(u_x\dif x)+z_v(v_x\dif x + v_y\dif y)\\
\dif z = z_x\dif x + z_y \dif y
\end{matrix}
\right\rbrace \implies \\
\begin{cases}
z_x &= z_uu_x+z_vv_x \\
z_y &= z_uu_y+z_vv_y
\end{cases}
\end{gather*}
\end{infobox}

\[
u=e^x\cos y \implies \begin{cases}
u_x &= e^x\cos y = u\\
u_y &= -e^x\sin y = -v
\end{cases}
\qquad
v=e^x\sin y \implies \begin{cases}
v_x &=e^x\sin y =v\\
v_y &=e^x\cos y = u
\end{cases}
\]

\begin{align*}
(1) \xRightarrow{(3),(5)} & z_x=z_u u + z_v v (7) \\
(2) \xRightarrow{(4),(6)} & z_y=z_u v + z_v u (8) \\
\end{align*}

\subparagraph{(1)}
\begin{align*}
z_x^2 + z_y^2 &= (z_u u + z_v v)^2+(-z_u v + z_v u)^2 =
\\ &=
z_u^2u^2+z_v^2v^2+euvz_uz_v+z_u^2v^2+z_v^2u^2-2uvz_uz_v \\
&=
z_u^2(u^2+v^2)+z_v^2(u^2+v^2) = (u^2+v^2)(z_u^2+z_v^2)
\end{align*}

\subparagraph{}
\begin{align}
z_{x^2} & =(z_x)_x = (z_uu+z_vv)_u u_x+(z_uu+z_vv)_vv_x \nonumber \\
&=
(z_{u^2}u+z_u+z_{vu}v)u+(z_{uv}u+z_{v^2}v+z_v)v\\
z_{y^2} &= (z_y)_y = (-z_uv +z_vu)_uu_y+(-z_uv+z_vu)_vu_y \nonumber \\
&= (-z_{u^2}v+z_{vu}u+z_v)(-v)+(-z_{uv}v-z_u+z_{v^2}u)u \\
z_{x^2}+z_{y^2} &= u(z_{u^2}u+z_u+z_{uv}v-z_{uv}v-z_u+z_{v^2}u)
+ v(z_{uv}u+z_{v^2}v+z_v+z_{u^2}v-z_{uv}u-z_v) \nonumber \\
&= u^2(z_{u^2}+z_{v^2})+v^2(z_{u^2}+z_{v^2}) \nonumber \\
&= (u^2+v^2)(z_{u^2}+z_{v^2})
\end{align}

\paragraph{Άσκηση}
\(
f(x,y)=x^2-3xy+y^2
\)

Να βρεθεί ο ρυθμός μεταβολής στο \(P_0 = (1,2) \) κατά τη μετακίνηση στο \(P_1=(3,4)\).
\begin{tikzpicture}
\draw[->,thick] (-4,0) -- (6,0);
\draw[->,thick] (0,-4) -- (0,6);

%\fill (1,2) circle{3pt} node[above] {$P_0$};
%\fill (3,24 circle{3pt} node[above] {$P_1$};
%TODO Velaki από P_0\to P_1
\end{tikzpicture}

\[
\pd{f(x_0,y_0)}{\vec a} = \nabla f(x_0,y_0)\vec a = -3\frac{1}{\sqrt{2}}+1 \frac{1}{\sqrt{2}} = \sqrt{2}
\]

\begin{align*}
\nabla f(x_0,y_0) &= \left(
f_x(1,2),f_y(1,2)
\right) = \left(
(3x^2-3_y)_{P_0}, (-3x+2y)_{P_0}
\right)
\\ &=
(3 \cdot1^2 - 3 \cdot 2, -3 \cdot 1 + 2 \cdot 2) = (-3,1)
\end{align*}

\[
\vec a = \frac{\overrightarrow{P_0P_1}}{|\overrightarrow{P_0P_1}|} = \frac{(3-1,\ 4-2)}{\sqrt{(3-1)^2+(4-2)^2}} = \frac{(2,2)}{2\sqrt{2}} = \left(
\frac{1}{\sqrt{2}}, \frac{1}{\sqrt{2}}
\right)
\]

\paragraph{Άσκηση}
\(f(x,y )\) έχει συνεχείς μερικές παραγώγους 3\textsuperscript{ης} τάξης. \newline
Αν \(\underbrace{f(x,y)}_{\mathclap{\pd[2]{f(x,y)}{x} + \pd[2]{f(x,y)}{y}=0 }}\) αρμονική ΝΔΟ \(f_x(x,y)\) είναι επίσης αρμονική.
	
	\begin{gather*}
	\pd[2]{f_x(x,y)}{x}+\pd[2]{f_x(x,y)}{y}=%
%
	\pd[2]{}{x_2} \left[
	\pd{f(x,y)}{x}
	\right] + \pd[2]{}{y} \left[
	\pd{f(x,y)}{x}
	\right] =
	\pd[3]{f(x,y)}{x}+\frac{\partial^3f(x,y)}{\partial y^2 \partial x} \\
	= \pd{}{x} \left[
	\pd[2]{f(x,y)}{x}
	\right] + \pd{}{x} \left[
	\pd[2]{f(x,y)}{y}
	\right]=
	\pd{}{x} \left[
	\pd[2]{f(x,y)}{x} + \pd[2]{f(x,y)}{y}
	\right] = \pd{0}{x} = 0
	\end{gather*}

\paragraph{Άσκηση}
Δίνεται συνάρτηση \(f(x,y)= \begin{cases}
\frac{xy}{\sqrt{x^2+y^2}} \quad & , (x,y) \neq (0,0)\\
0 \quad & , (x,y) = (0,0)
\end{cases}\)
\begin{enumparen}
\item Να υπολογιστούν οι \(f_x\) κ' \(f_y\ \forall (x,y) \in \mathbb R ^2\)
\item Είναι η \(f(x,y)\) συνεχής στο \(\mathbb{R}^2\)?
\item Είναι η \(f(x,y)\) διαφορίσιμη στο \((0,0)\)?
\end{enumparen}

\subparagraph{(1)}
Για \((x,y) \neq (0,0): f_x(x,y)=\frac{y\sqrt{x^2+y^2}-xy\frac{2x}{2\sqrt{x^2+y^2}}}{x^2+y^2}
=\frac{y(x^2+y^2)-x^2y}{\sqrt{x^2+y^2}(x^2+y^2)}=\frac{y^3}{(x^2+y^2)^\frac{3}{2}}
 \)
 
 Ομοίως θα προκύψει ότι \(
 f_y(x,y) = \frac{x^3}{(x^2+y^2)^{ ^3/_2}}
 \)
 
Για \((x,y)=(0,0) \):
\begin{align*}
f_x(x_0,y_0) &= \lim_{h\to0} \frac{f(x_0+h,y_0)-f(x_0,y_0)}{h} \implies \\
f_x(0,0) &= \lim_{h\to0} \frac{f(h,0)-\cancelto{0}{f(0,0)}}{h} \implies \\
f_x(0,0) &= \lim_{h\to0} \frac{\frac{h\cdot 0}{\sqrt{h^2+0^2}}}{h} = \lim_{h\to0}\frac{\cancel{h}\cdot 0}{\cancel{h} \sqrt{h^2+0}} \\
f_y(x_0,y_0) &= \lim_{h\to0}\frac{f(x_0,y_0+h)-\cancelto{0}{f(x_0,y_0)}}{h} \\
&= \lim_{h\to0}\frac{f(0,h)}{h} = \lim_{h\to0}\frac{\frac{0\cdot h}{\sqrt{0^2+h^2}}}{h} \\
&= \lim_{h\to0} \frac{0 \cdot \cancel{h}}{\cancel{h}\cdot \sqrt{0^2+h^2}}
\end{align*}

\subparagraph{(2α)}
\begin{align*}
f_x&=\frac{y^3}{(x^2+y^2)^\frac{3}{2}} = \frac{\cancel{r^3}\sin^3\phi}{\cancel{r^3}} \to \text{ πεπερασμένο}
\\
f_y&=\frac{x^3}{(x^2+y^2)^\frac{3}{2}} = \frac{\cancel{r^3}\cos^3\phi}{\cancel{r^3}} \to \text{ πεπερασμένο}
\end{align*}
Άρα σ/ς

\subparagraph{(2β)}
Θα ελεγθεί η συνέχεια στο \((0,0)\):
\[\lim_{(x,y)\to(0,0)} f(x,y) = \cancelto{0}{f(0,0)}\]

\begin{align*}
\lim_{(x,y)\to(0,0)}f(x,y) = \lim_{(x,y)\to(0,0)}
\frac{xy}{\sqrt{x^2+y^2}} = \lim_{r\to0^+} \frac{r^{\cancel{2}}\cos\phi\sin\phi}{\cancel{r}} = 0
\end{align*}

\subparagraph{(3α)}
\[
\lim_{(x,y)\to(0,0)}f_x(x,y)=f_x(0,0) = \lim_{(x,y)\to(0,0)}\frac{y^3}{(x^2+y^2)^{\frac{3}{2}}} = \lim_{r\to0^+}\frac{\cancel{r^3}\sin^3\phi}{\cancel{r^3}}
\]

\subparagraph{(3β)}
\[
\lim_{P\to P_0} \frac{\left|f(P)-f(P_0)-f'(P_0)(P-P_0)\right|}{|\overrightarrow{P_0P_1}|}
\]

\begin{align*}
\lim_{(x,y)\to(0,0)}
\frac{\left|f(x,y)-f(0,0)-\left(\cancelto{0}{ f_x(0,0), f_y(0,0)}\right)\cdot(x,y)\right|}{\sqrt{x^2+y^2}} &= \lim_{(x,y)\to(0,0)} \frac{\left| f(x,y) \right|}{\sqrt{x^2+y^2}} \\
&= \lim_{(x,y)\to(0,0)} \frac{\frac{|xy|}{\sqrt{x^2+y^2}}}{\sqrt{x^2+y^2}} \\
&= \lim_{(x,y)\to(0,0)} \frac{|xy|}{x^2+y^2} \\
&= \sin^3\phi
\end{align*}

\subsection{Πεπλεγμένη συνάρτηση}
\[
\Phi(x_1,x_2,\dots,x_n,y) = 0 \qquad \text{π.χ } x\cos y+ye^2+x^2y\cos z = 0 
\]
Μπορεί η \( \Phi \) να λυθεί μονοσήμαντα ως προς \( y \)?

Αν \( \exists P_0=(x_{01}.\dots,x_{0n},y_0) \) κ \( \pi_\varepsilon (P_0) \)
\begin{enumparen}
\item \( \Phi_{x_i} \quad (i=1,\dots,n),\ \Phi_y \) συνεχείς
\item \( \Phi(P_0) = 0 \)
\item \( \Phi_y(P_0) = \pd{\Phi(P_0)}{y} \neq 0  \)
\end{enumparen}

Τότε:
\begin{enumparen}
\item \(y = f(x_1,\dots,x_n)\)
\item \( y_{x_i}=-\pd{\Phi_{x_i}}{\Phi_y} \)
\end{enumparen}

%TODO Zaharis Graph 01

π.χ \( z_x = \pd{z}{x} = - \frac{\Phi_x}{\Phi_z} \)
\paragraph{Απόδ. (2)}
\begin{align*}
\Phi(x_1,\dots,x_n,y) = 0 &\implies \dif \Phi = \Phi_{x_1}\dif x_1 + \dots + 
\Phi_{x_n}\dif x_n + \Phi_y \dif y = 0\\
&\implies \Phi_{x_1}\dif x_1 + \dots + \Phi_{x_n} \dif x_n + \Phi_y(y_{x_1} \dif x_1+\dots+y_{x_n}\dif x_n) \\
&\implies (\Phi_{x_1}+\Phi_y y_{x_1}) \dif x_1 + \dots + (\Phi_{x_n}+\Phi_y y_{x_n}) \dif x_n = 0 \\
&\implies
\Phi_{x_i}+\Phi_y y_{x_i} = 0 \implies y_{x_i} = - \frac{\Phi_{x_i}}{\Phi_y}
\end{align*}

\paragraph{}

\[
\Phi(x,y,z) = 0 \implies z = f(x,y)
\]
%TODO Zaharis Graph 02

\begin{align*}
\dif \Phi &= \Phi_x(P_0)\dif x + \Phi_y(P_0)\dif y + \Phi_z(P_0) \dif z
\\ &= \nabla \Phi (P_0) \bullet \dif \vec r = 0 \implies \\
& \implies \boxed{ \nabla \Phi(P_0) \perp \dif \vec r }
\end{align*}

\paragraph{Συνέπειες}
%TODO Zaharis Graph 03
\begin{align*}
\nabla \Phi(P_0) \cdot \overrightarrow{P_0P} = 0 \implies \\
\left(
\Phi_x(P_0),\Phi_y(P_0),\Phi_z(P_0)
\right) \cdot (x-x_0,y-y_0,z-z_0) = 0 \implies \\
\Phi_x(P_0)(x-x_0) + \Phi_y(P_0)(y-y_0) + \Phi_z(P_0)(z-z_0) = 0
\end{align*}
\subparagraph{}
%TODO Zaharis Graph 04

\( \lambda \in \mathbb R  \)

\begin{align*}
\overrightarrow{OP} = \overrightarrow{OP_0} + \lambda\, \nabla \Phi(P_0) \implies \\
(x-x_0,\ y-y_0,\ z-z_0) = \lambda \left(
\Phi_x(P_0), \Phi_y(P_0),\Phi_z(P_0)
\right) \implies \\
\implies \underbrace{\begin{cases}
x &= x_0 + \lambda \Phi_x(P_0)\\
y &= y_0 + \lambda \Phi_y(P_0)\\
z &= z_0 + \lambda \Phi_z(P_0)
\end{cases}}_{\mathclap{\text{παραμετρικές εξισώσεις}}} \implies
\lambda = 
\underbrace{
	\boxed{
		\frac{x-x_0}{\Phi_x(P_0)} +
		\frac{y-y_0}{\Phi_y(P_0)} +
		\frac{z-z_0}{\Phi_z(P_0)}
		}
	}_{\mathclap{\text{Αλγεβρικές εξισώσεις}}}
\end{align*}

\subparagraph{}
\begin{align*}
&\overrightarrow{OP} = \overrightarrow{OP_0} + \overrightarrow{P_0P} \implies \overrightarrow{OP} = \overbracket{OP_0} + \lambda \vec b \\
&\implies (x-x_0,\ y-y_0,\ z-z_0) = \lambda \cdot \left|
\begin{matrix}
\overrightarrow{e_1} & \overrightarrow{e_2} & \overrightarrow{e_3} \\
\Phi_x(P_0) & \Phi_y(P_0) & \Phi_z(P_0) \\
H_x(P_0) & H_y(P_0)&H_z(P_0)
\end{matrix}
\right|
\\
&\implies
(x-x_0,y-y_0,z-z_0) = \lambda \left[
\begin{array}{rl}
&\overrightarrow{e_1} \left(\Phi_y(P_0)H_z(P_0)-\Phi_z(P_0)H_y(P_0) \right) \\
- & \overrightarrow{e_2} \left(\Phi_x(P_0)H_z(P_0)- \Phi_z(P_0)H_x(P_0) \right) \\
+ & \overrightarrow{e_3} \left( \Phi_x(P_0)H_y(P_0)-\Phi_y(P_0)H_x(P_0) \right)
\end{array}
\right]
\\ &\implies
\lambda = \frac{x-x_0}{\Phi_y(P_0)H_z(P_0)-\Phi_z(P_0)H_y(P_0)} + \frac{y-y_0}{\Phi_x(P_0)H_z(P_0)- \Phi_z(P_0)H_x(P_0)} + \frac{z-z_0}{ \Phi_x(P_0)H_y(P_0)-\Phi_y(P_0)H_x(P_0)}
\end{align*}

\paragraph{Άσκηση}
\[
x\cos y+y\cos z +z\cos x - 1=0
\]

Να εξεταστεί εάν \( y=f(x,z) \to \) λύνεται μονοσήμαντα σε περιοχή του σημείου \( P_1=(0,0) \).

Να υπολογιστούν οι \( f_x(0,0),\ f_z(0,0) \).

\subparagraph{}
\begin{enumparen}
\item \( \Phi_x = \cos y -z\sin x\)\\
\(\Phi_y = -x\sin y + \cos z \)\\
\( \Phi_z = -y\sin z + \cos x \)
\item \(\Phi(P_0)=0 \)

\(P_0 = (x_0,y_0,z_0) = (0,y_0,0) \)

\(\Phi(P_0) = \cancel{0\cos y_0} + y_0 \cancelto{1}{\cos 0} + \cancel{0\cos0} -1 =0 \implies y_0 = 1 \)
\item \( \Phi_y(P_0) \neq0 \implies -x_0\sin y_0 + \cos z_0 = 1 \neq 0 \)
\end{enumparen}

Τελικά γίνεται \boxed{y=f(x,z)}.

\begin{align*}
f_x &= y_x = \pd{y}{x} = - \frac{\Phi_x}{\Phi_y} = -\frac{\cos y_0 - z_0\sin x_0}{-x_0\sin y_0+\cos z_0} = - \cos 1
\\
f_z &= y_z = \pd{y}{z} = - \frac{\Phi_z}{\Phi_y} = - \frac{-y_0\sin z_0 + \cos x_0}{-x_0 \sin y_0 +\cos z_0} = -1
\end{align*}

\paragraph{Άσκηση}
\[
\Phi(x,y,z) = e^z+x^2y+z+5= 0
\]

Θεωρείται ότι από την \( \Phi(x,y,z) \) προκύπτει η \( z=f(x,y) \).

\( z_{xy} = ? \)

\subparagraph{}

\[
\begin{array}{l|l}
\Phi_x = 2xy    & z_x = -\frac{\Phi_x}{\Phi_z} = - 2\frac{2xy}{e^z+1} 
\\
\Phi_y = x^2    & z_y = -\frac{\Phi_y}{\Phi_z} = -\frac{x^2}{e^z+1} \\
\Phi_z = e^z+1 &
\end{array}
\]

\begin{align*}
z_{xy}&=(z_x)_y = - \left(
\frac{2xy}{e^z+1}
\right)_y = -\frac{2x(e^z+1)-2xy\cdot \  \attnboxed{ z_y}}{(e^z+1)^2}
\\ &=
-\frac{2x(e^z+1)+2xye^z\frac{x^2}{e^z+1}}{(e^z+1)^2} \\
&=
\frac{-2x(e^z+1)^2-2x^3ye^z}{(e^z+1)^3}
\end{align*}

\paragraph{Άσκηση}
Αν \( z=uv \) όπου \( u=u(x,y) \) κ \( v=v(x,y) \) \\
\( u^2+v^2-x-y=0 \) και \( u^2-v^2+3x+y=0 \) \\
να υπολογιστούν οι \( z_x \) και \( z_y \).

SOS

\subparagraph{1) Κατασκευάζω τις πεπλεγμένες συναρτήσεις}
\begin{align*}
\Phi_1(x,y,u,v) &= u^2+v^2-x-y=0 \\
\Phi_2(x,y,u,v) &= u^2-v^2+3x+y = 0
\end{align*}

\subparagraph{2)}
\begin{align*}
\dif \Phi_1 = {\Phi_1}_x\dif x +{\Phi_1}_y\dif y +{\Phi_1}_u\dif u + {\Phi_1}_v\dif v = 0
\implies \\
\dif \Phi_1 = {\Phi_1}_x\dif x +{\Phi_1}_y\dif y +{\Phi_1}_u(u_x\dif x + u_y\dif y)
+ {\Phi_1}_v(v_x\dif x+v_y\dif y) = 0 \implies \\
\dif \Phi_1 = \left(
\Phi_{1x}+\Phi_p{1u}u_x+\Phi_{1v}v_x
\right) \dif x + \left(
\Phi_{1y}+\Phi_{1u}u_y+\Phi_{1v}v_y
\right) \dif y \implies \\
\begin{cases}
\Phi_{1u}u_x+\Phi_{1v}v_x&=-\Phi_{1x} \\
\Phi_{1u}u_y+\Phi_{1v}v_y&=-\Phi_{1y} 
\end{cases}
\end{align*}

\subparagraph{3)}
Με όμοιο τρόπο καταλήγουμε στο:
\[
\begin{cases}
\Phi_{2u}u_x+\Phi_{2v}v_x&=-\Phi_{2x} \\
\Phi_{2u}u_y+\Phi_{2v}v_y&=-\Phi_{2y} 
\end{cases}
\]

\subparagraph{4) Λύση Α συστήματος}
\begin{align*}
u_x&=\frac{\left|
	\begin{matrix}
	-\Phi_{1x} & \Phi_{1v} \\
	-\Phi_{2x} & \Phi_{2v}
	\end{matrix}
	\right|
	}{\left|\begin{matrix}
	\Phi_{1u} & \Phi_{1v} \\
	\Phi_{2u} & \Phi_{2v}
	\end{matrix}
	\right|}
=  \frac{\left|
	\begin{matrix}
	-1&2v\\-3&-2v
	\end{matrix}
	\right|
}{\left|\begin{matrix}
2u&2v\\2u&-2v
\end{matrix}
\right|} = -\frac{1}{2u}
\\
v_x &= \frac{\left|
	\begin{matrix}
	\Phi_{1u} &-\Phi_{1x} \\
	\Phi_{2u} &-\Phi_{2x}
	\end{matrix}
	\right|
}{\left|\begin{matrix}
\Phi_{1u} & \Phi_{1v} \\
\Phi_{2u} & \Phi_{2v}
\end{matrix}
\right|}
=  \frac{\left|
	\begin{matrix}
	2u&1\\2u&-3
	\end{matrix}
	\right|
}{\left|\begin{matrix}
2u&2v\\2u&-2v
\end{matrix}
\right|} = -\frac{1}{2u}
\end{align*}

\subparagraph{Λύση Β-συστήματος}
\begin{align*}
u_y&=\frac{\left|
	\begin{matrix}
	-\Phi_{1y} & \Phi_{1v} \\
	-\Phi_{2y} & \Phi_{2v}
	\end{matrix}
	\right|
}{\left|\begin{matrix}
\Phi_{1u} & \Phi_{1v} \\
\Phi_{2u} & \Phi_{2v}
\end{matrix}
\right|}
 = 0
\\
v_y &= \frac{\left|
	\begin{matrix}
	\Phi_{1u} &-\Phi_{1y} \\
	\Phi_{2u} &-\Phi_{2y}
	\end{matrix}
	\right|
}{\left|\begin{matrix}
\Phi_{1u} & \Phi_{1v} \\
\Phi_{2u} & \Phi_{2v}
\end{matrix}
\right|} = \frac{1}{2v}
\end{align*}

Για να λύσω την άσκηση, μπορώ να σκεφτώ ότι \( z_x=\pd{z}{x} = \pd{uv}{x} = u\pd{v}{x}+v\pd{u}{x} \), ή:
\begin{align*}
\dif z &= z_u\dif u + z_v\dif v \implies \\
\dif z &= z_u(u_x\dif x+u_y\dif y)+z_v(v_x\dif x+v_y\dif y) \implies \\
\dif z &= (z_uu_x+z_vv_x)\dif x+(z_uu_y+z_vv_y)\dif y \implies \\
\dif z &= z_x\dif x + z_y\dif y
\end{align*}

\begin{align*}
z_x &= z_uu_x+z_vv_x = \\
    &= v \left(-\frac{1}{2u} \right)+u\frac{1}{v}=\frac{u}{v}-\frac{v}{2u}
    \\
z_y &= z_uu_y+z_vv_y = \\
&= v \cdot 0 +u\frac{1}{2v}=\frac{u}{2v}
\end{align*}

\paragraph{Άσκηση}
Επιφάνεια \( z=x^2+4y^2-2 \).\\
Ζητείται η εξίσωση του εφαπτ. επιπέδου στην επιφάνεια παράλληλου στο επίπεδο \( 2x+y-z=4 \).\\
Επίσης ζητείται το σημείο τομής \( P_0 = (x_0,y_0,z_0) \).

\subparagraph{}
Πεπλεγμένη μορφή \( \Phi(x,y,z)=x^2+4y^2-z-2=0 \).\\
Εξ. εφαπτόμενου επιπέδου:
\begin{align*}
 & \Phi_x(P_0)(x-x_0)+\Phi_y(P_0)(y-y_0)+\Phi_z(P_0)(z-z_0) =0  \\
\implies & 2x_0(x-x_0) + 8y_0(y-y_0)-(z-z_0) = 0. (1)
\end{align*}
Παρατηρώ ότι: \[
\frac{2x_0}{2} = \frac{8y_0}{1} = \frac{-1}{-1} \implies \begin{cases}
x_0 &=1 \\ y_0 &= \frac{1}{8}
\end{cases}
\]
\[
z_0=x_0^2+4y_0^2-2 \implies z_0 = -\frac{15}{16}
\]
\begin{align*}
(1) \implies & 2(x-1)+\left( y-\frac{1}{8} \right) - \left(z+\frac{15}{16} \right) =0 \\
\implies & 2x+y-z=\frac{49}{16}
\end{align*}

\paragraph{Άσκηση}
Επιφάνεια \( x^2+y^2=4z^2 \)\\
Ζητείται η εξίσωση του εφαπτόμενου επιπέδου κ της κάθετης ευθείας στην επιφάνεια στο σημείο \( P_0=(6,-8,5) \) αυτής.

\[
\Phi(x,y,z)=x^2+y^2-4z^2=0
\]

Εξίσ. εφαπτ. επιπ.: \begin{align*}
\Phi_x(P_0)(x-x_0)+
\Phi_y(P_0)(y-y_0)+
\Phi_z(P_0)(z-z_0) = 0
\\
\implies 2x_0(x-x_0)+2y_0(y-y_0)-8z_0(z-z_0) = 0 \implies \\
\implies 12(x-6)-16(y+8)-40(z-5)=0 \implies \\
\implies 3(x-6) -4(y+8) -10(z-5)=0 \implies \boxed{3x-4y-10z=0}
\end{align*}

Εξ. κάθ. ευθείας: \begin{align*}
\frac{x-x_0}{\Phi_x(P_0)}
=\frac{y-y_0}{\Phi_y(P_0)}
=\frac{z-z_0}{\Phi_z(P_0)} \implies \\
\frac{x-6}{3} = \frac{y+8}{-4} = \frac{z-5}{10}
\end{align*}

\paragraph{Άσκηση}
%TODO Zaharis 01
Να βρεθεί η εξ. της ευθείας που εφάπτεται στην καμπύλη
\(
\begin{cases}
x^2+y^2+z^2 &=14 \\x+y+z &=6
\end{cases}
\) στο σημείο \( P_0=(1,2,3) \)

όπου \( f(x,y,z)=x^2+y^2+z^2-14 = 0 \)\\
και \( g(x,y,z)=x+y+z-6=0 \)

\begin{align*}
& \overrightarrow{OP} = \overrightarrow{OP_0} +\lambda \nabla f (P_0) \times \nabla g(P_0) \\
\implies & \overrightarrow{OP} - \overrightarrow{OP_0} = \lambda \nabla f(P_0) \times \nabla g(P_0) \implies \\
\implies & (x-x_0,y-y_0,z-z_0) = \lambda \left|
\begin{matrix}
\vec{e_1} & \vec{e_2} & \vec{e_3} \\
f_x(P_0) & f_y(P_0) & f_z(P_0) \\
g_x(P_0) & g_y(P_0) & g_z(P_0)
\end{matrix}
\right| \implies (x-1,\ y-2,\ z-3 ) = \lambda \left|
\begin{matrix}
\vec{e_1} & \vec{e_2} & \vec{e_3} \\ 2&4&6\\1&1&1
\end{matrix}
\right| \\
\implies & \begin{cases}
x-1 &= \lambda \cdot (-2) \\
y-2 &= \lambda \cdot 4 \\
z-3 &= \lambda \cdot (-2)
\end{cases}
\\ \implies & \boxed{
	\frac{x-1}{-2} = \frac{y-2}{4} = \frac{z-3}{-2}
}
\end{align*}

\paragraph{}
\[
f(P) = \sum_{k=0}^\infty \frac{\dif^k f_{P_0}(P)}{k!} = \underbrace{\frac{d^0 f_{P_0}}{0!}}_{\mathclap{f(P_0)}} + \frac{\dif^1 f_{P_0}(P)}{1!} + \frac{\dif^2 f_{P_0}(P)}{2!} + \frac{\dif^3 f_{P_0}(P)}{3!} + \dots
\]

\[
\dif^k f_{P_0}(P) = \left[
f_{x_1}(P_0)(x_1-x_{10})+\dots+f_{x_n}(P_0)(x_n-x_{n0})
\right]^{(k)}
\]

\paragraph{Άσκηση}
Να αναλυθεί η συνάρτηση \( f(x,y)=\frac{1}{xy} \) γύρω από το σημείο \( P_0=(1,1) \) σε σειρά \textlatin{Taylor} χρησιμοποιώντας όρους μέχρι 3\textsuperscript{ης} τάξης.

\begin{gather*}
f(x,y) = f(x_0,y_0) + \dif f_{P_0}(P) + \frac{1}{2}\dif^2 f_{P_0}(P) + \frac{1}{6} \dif^3 f_{P_0}(P) \\
f(x_0,y_0) = 1 \\ \left.
\begin{array}{l}
\dif f_{P_0}(P)=f_x(P_0)(x-x_0)+f_y(P_0)(y-y_0) \\
f=\frac{1}{xy} \implies \begin{cases}
f_x = -\frac{1}{x^2y} \\ f_y = -\frac{1}{xy^2}
\end{cases}
\end{array} \right\rbrace \implies
\dif f_{P_0}(P)= \left(-\frac{1}{x^2y} \right)_{P_0}(x-1) + \left(
-\frac{1}{xy^2}_{P_0}(y-1)
\right) \implies \\
\implies \dif f_{P_0}(P) = -(x-1)-(y-1) = 2-x-y\\
\dif^2f_{P_0}(P) = \left[
f_x(P_0)(x-x_0)+f_y(P_0)(y-y_0)
\right]^{(2)} =
f_{x^2}(P_0)(x-x_0)^2+2f_{xy}(P_0)(x-x_0)(y-y_0)+f_{y^2}(P_0)(y-y_0) \\
f_{x^2} = (f_x)_x = \left(-\frac{1}{x^2y} \right)_x = \frac{2}{x^3y} \\
f_{xy} = (f_x)_y = \left(-\frac{1}{x^2y} \right)_y = \frac{1}{x^2y^2} \\
f_{y^2} = (f_y)_y = \left(-\frac{1}{xy^2} \right)_y = \frac{2}{xy^3} \\
\implies \dif^2 f_{P_0}(P) = \left(
\frac{2}{x^3y}_{P_0}\right)(x-x_0)^2+2\left(
\frac{1}{x^2y^2}\right)_{P_0}(x-x_0)(y-y_0)+\left(
\frac{2}{xy^3}
\right)_{P_0}(y-y_0)^2 \\
\implies \dif^2 f_{P_0}(P) = 2(x-1)^2+2(x-1)(y-1)+2(y-1)^2
\end{gather*}
\begin{align*}
\dif^3 f_{P_0}(P) &= \left[
f_x(P_0)(x-x_0)+f_y(P_0)(y-y_0)
\right]^{(3)} = \\
&= f_{x^3}(P_0)(x-x_0)^3+3f_{x^2y}(P_0)(x-x_0)^2(y-y_0)+3f_{xy^2}(P_0)(x-x_0)(y-y_0)^2 + f_{y^3}(P_0)(y-y_0)^3
\end{align*}
\begin{align*}
f_{x^3} &= (f_{x^2})_x = -\frac{6}{x^4y}\\
f_{x^2y} &= (f_{x^2})_y = -\frac{2}{x^3y^2}\\
f_{xy^2} &= (f_{xy})_y = -\frac{2}{x^2y^3}\\
f_{y^3} &= (f_{y^2})_y = -\frac{6}{xy^4}
\end{align*}
\begin{align*}
\dif^3 f_{P_0}(P) &= -6(x-1)^3+3\cdot(-2)(x-1)^2(y-1) + 3\cdot(-2)(x-1)(y-1)^2-6(y-1)^3\\
\dif^2 f_{P_0}(P) &= 2(x-1)^2+2(x-1)(y-1)+2(y-1)^2 \\
\dif f_{P_0}(P)   &= 2-x-y \\
f(P_0) &= 1
\end{align*}

Άρα:
\[
f(x,y) = 1+2-x-y+(x-1)^2+(x-1)(y-1)+(y-1)^2-(x-1)^3-(x-1)^2(y-1)-(x-1)(y-1)^2-(y-1)^3
\]

\section{Στάσιμα σημεία}
\begin{enumerate}
\item Τοπικά μέγιστα
\item Τοπικά ελάχιστα
\item Σαγματικά σημεία
\end{enumerate}

%TODO Zaharis Graph 01

Υπολογισμός: \( \nabla f(x_1,\dots,x_n) = 0 \implies \left. \begin{cases}
f_{x_1} &= 0 \\
f_{x_2} &= 0 \\
\vdots & \\
f_{x_n} &= 0
\end{cases} 
\right\rbrace P_0
\)

Χαρακτηρισμός
\(
H_f = \left[
\begin{matrix}
f_{x_1x_1} & f_{x_1x_2} & \dots & f_{x_1x_n} \\
f_{x_2x_1} & f_{x_2x_2} & \dots & f_{x_2x_n} \\
\vdots & & \ddots & \\
f_{x_nx_1} & f_{x_nx_2} & \dots & f_{x_nx_n}
\end{matrix}
\right]
\) εσσιανός

\( f_{x_ix-J} = f_{x_jx_i} \) όταν η \( f \) έχει συνεχείς μερ. παραγώγους 2\textsuperscript{ης} τάξης

\( \mathrm H_f = \mathrm H_f^T \implies \) έχει πραγματικές ιδιοτιμές

\paragraph{Κριτήριο}
(για στάσιμο σημείο)
\begin{itemize}
\item Αν \( \mathrm H_f(P_0) \) είναι θετικά ορισμένος, τότε το \( P_0 \) είναι ΤΟΠΙΚΟ ΕΛΑΧΙΣΤΟ \\
\item Αν \( \mathrm H_f(P_0) \) είναι αρνητικά ορισμένος, τότε το \( P_0 \) είναι ΤΟΠΙΚΟ ΜΕΓΙΣΤΟ
\item  Αν \( \mathrm H_f(P_0) \) είναι μικτά προσημασμένος, τότε το \( P_0 \) είναι ΣΑΓΜΑΤΙΚΟ.
\end{itemize}

\begin{enumerate}
\item Θετικά ορισμένος πίνακας \( A \):
\begin{enumerate}
\item \( uAu^T > 0 \ \forall u \in \mathbb R ^n - \left\lbrace \vec0\right\rbrace \quad \) (ορισμός)
\item Έχει μόνο θετικές ιδιοτιμές
\item Κύριες ελάσσονες ορίζουσες \( \mathrm D_k > 0,\ k=1,\cdots,n \)
\end{enumerate}

\item Αρνητικά ορισμένος πίνακας \( A \):
\begin{enumerate}
\item \( uAu^T < 0 \ \forall u \in \mathbb R ^n - \left\lbrace \vec0\right\rbrace \quad \) (ορισμός)
\item Έχει μόνο αρνητικές ιδιοτιμές
\item Κύριες ελάσσονες ορίζουσες \( \mathrm (-1)^k \mathrm D_k > 0,\ k=1,\cdots,n \)
(δηλ. \( \mathrm D_1 <0,\mathrm D_2>0,\mathrm D_3<0,\dots \))
\end{enumerate}


\item Μικτά προσημασμένος \( A \):
\begin{enumerate}
\item \( \exists u_1,u_2 \in \mathbb R ^n-\left\lbrace\vec0\right\rbrace,
\begin{array}{l}
u_1Au_1^T > 0 \\
u_2Au_2^T < 0
\end{array}
 \) (ορισμός - δεν αποκλείονται τα μηδενικά αποτελέσματα)
 \item Έχει τουλάχιστον 1 θετική και 1 αρνητική ιδιοτιμή (δεν αποκλείονται ιδιοτιμές 0)
 \item Υπό την προϋπόθεση ότι \( \mathrm D_n = \det[A] \neq 0 \) (χωρίς να αποκλείονται εσωτερικα \( \mathrm D_k \))
 
 Όταν δεν ακολουθείται στις \( \mathrm D_k \) η αλληλουχία προσήμων \( (+++++\cdots) \) ή \( (-+-+-\cdots) \) και να υπάρχει τουλάχιστον μία \( \mathrm D_k > 0 \) και μία \( \mathrm D_k < 0 \)
 
 \item Αν \( \det [A] = 0 \), 2 στοιχεία κύριας διαγωνίου ετερόσημα.
 
 \item Αν \( \det [A] \neq 0 \) και \( \mathrm{Tr}[A] = 0 \)
\end{enumerate}

\end{enumerate}

\subsection{Διερεύνηση περίπτωσης συνάρτησης 2 μεταβλητών\\\(z=f(x,y)\)}
\begin{enumerate}
\item \( \nabla f(x,y) = 0 \implies \left. \begin{cases}
f_x &= 0 \\
f_y &= 0
\end{cases} \right\rbrace P_0 \)
\\
\item
\( \mathrm H_f = \left[
\begin{matrix}
f_{xx} & f_{xy} \\
f_{yx} & f_{yy}
\end{matrix}
\right] 
\begin{array}{ll}
\to f_{xx}(P_0)>0 \text{ κ } \det\left[\mathrm H_f(P_0)\right]>0 & \to \overbrace{\mathrm H_f}^{\mathclap{\text{τοπικό ελάχιστο}}} \text{θ.ό}
\\
\to f_{xx}(P_0)<0 \text{ κ } \det\left[\mathrm H_f(P_0)\right]>0 & \to \overbrace{\mathrm H_f}^{\mathclap{\text{τοπικό μέγιστο}}} \text{α.ό}
\\
\to \det \left[\mathrm{H}_f(P_0)\right]<0 & \to \underbrace{H_f}_{\mathclap{\text{σαγματικό}}} \text{μ.π}
\\
\to \det\left[ \mathrm H_f(P_0) \right] = 0 & \to \text{δεν ξέρω}
\end{array}
\)
\end{enumerate}

\paragraph{Άσκηση}
Να βρεθούν τα στάσιμα σημεία της συνάρτησης \( f(x,y) = x^2+y^2-3xy \)
\[
    \nabla f = 0 \implies
    \begin{cases}
    f_x &= 3x^2-3y = 0 \implies y = x^2 \\
    f_y &= 3y^2-3x = 0 \implies x = y^2
    \end{cases}
\]

\( x=x^4 \implies x^4-x=0 \implies x(x^3-1) = 0 \implies \underbrace{x}_{\mathclap{x=0}}(\underbrace{x-1}_{\mathclap{x=1}})(\underbrace{x^2+x+1}_{\mathclap{\Delta < 0}}) = 0 \)

\( 
\begin{array}{ll}
x = 1\implies y=1 & \to (1,1)\\
x=0 \implies y = 0& \to (0,0)
\end{array}
 \)
 
 \( \mathrm H_f = \left[ \begin{matrix}
 f_{xx} & f_{xy} \\ f_{yx} & f_{yy} \end{matrix}
 \right] =
 \left[
 \begin{matrix}
 6x&-3\\-3&6y
 \end{matrix}
 \right]
 \)
 
 \( 
 \mathrm H_f(0,0) = \left[\begin{matrix}
 0&-3\\-3&0
 \end{matrix}\right] = -9 < 0 \to \text{μ.π} \to (0,0) \text{ σαγμ.}
  \)
  
   \( 
   \mathrm H_f(1,1) = \left[\begin{matrix}
   6&-3\\-3&6
   \end{matrix}\right] \begin{array}{l} = 27>0\\f_{xx}=6>0\end{array} \to \text{θ.ο} \to (1,1) \text{ τοπ. ελάχιστο}
   \)
   
 \paragraph{Άσκηση}
 %TODO Zaharis Graph 02
 \( v = xyz \quad x+y+z = 5  \)
 \subparagraph{}
 \( \implies z=5-x-y \)

\[
V(x,y) = xy(5-x-y) = 5xy-x^2y-xy^2 \quad \left( \begin{array}{l}
x\neq0\\y\neq0\\z\neq0
\end{array} \right.
\]

\begin{align*}
\nabla V = 0 \implies \begin{cases}
v_x &= 5y-2xy-y^2 = 0 \implies y(5-2x-y) = 0 \xRightarrow[y\neq0]{} 2x+y=5 \implies y=5-2 \\
v_y &= 5x-x^2-2xy = 0 \implies x(5-x-2y) = 0 \xRightarrow{x\neq0} x+2y=5
\end{cases}
\\ \implies x+2(5-2x)=5\implies -3x=-5\implies x = \frac{5}{3}
\\ y=5-2x=5-2\frac{5}{3}=\frac{5}{3}
\\ z = 5-x-y=\frac{5}{3}
\end{align*}

\( 
\mathrm H_f = \left[
\begin{matrix}
-2y&5-2x-2y \\
5-2x-2y&-2x
\end{matrix}
\right] = \left[
\begin{matrix}
-\frac{10}{3} & -\frac{5}{3} \\
-\frac{5}{3} & -\frac{10}{3}
\end{matrix}
\right] \rightarrow \mathrm D_1 = -\frac{10}{3} < 0 \quad \mathrm D_2=\frac{25}{3} > 0
 \)

\paragraph{Άσκηση}
\( f(x,y) = x^4-y^3-2(x-y)^2 \)
\subparagraph{}
\( \nabla f = 0 \implies
\begin{cases}
f_x &= 4x^3-4(x-y)=4x^3-4x+4y=0 \implies y=x-x^3 \\
f_y &= 4y^3+4(x-y)=4y^3+4x-4y=0 \implies y^3+x-y=0
\end{cases}
 \)
 
 \( 
 (2) \xRightarrow{(1)} \left(x-x^3\right)^3+x-\left(x-x^3\right) = 0 \implies\left(x-x^3\right)^3+x^3 = 0 \implies
 \left(x-x^3+x\right)\left[
 \left(x-x^3\right)^2-\left(x-x^3\right)x+x^2
 \right] = 0
 \implies \left(2x-x^3\right)\left(x^2+x^6-2x^4-x^2+x^4+x^2\right) = 0
 \implies x\left(2-x^2\right)\left(x^6-x^4+x^2\right)=0 \implies 
 \underbrace{x^3}_{\mathclap{x=0}}\left(
 \underbrace{2-x^2}_{\mathclap{x=\pm \sqrt{2}}}
 \right)\left(x^4-x^2+1\right) = 0
  \)
  
  \( 
  (1) \begin{array}{ll}
  \xrightarrow{x=0} & y=0 \\
  \xrightarrow{x=+\sqrt{2}} & y=-\sqrt{2} \\
  \xrightarrow{x=-\sqrt{2}} & y=+\sqrt{2} \\
  \end{array}
   \)

\( 
\mathrm H_f = \left[
\begin{matrix}
12x^2-4 & 4 \\ 4& 12y^2-4
\end{matrix}
\right]
 \)
 
 \begin{enumerate}
 \item \( P_0 = \left(\sqrt{2},-\sqrt{2}\right) \to \mathrm H_f(P_0) = 
 \left[
 \begin{matrix}
 20&4\\4&20
 \end{matrix}
 \right] \begin{array}{l}
 \mathrm D_1 > 0 \\ \mathrm{D}_2 > 0
 \end{array}\begin{array}{l}
 \mathrm H_f \to \text{θ.ο} \\
 P_0 \to \text{ τοπικό ελάχιστο}
 \end{array}
 \)
 
 \item \( P_0 = \left(-\sqrt{2},\sqrt{2}\right) \uparrow \)
 
 \item \( P_0=(0,0) \to \mathrm H_f = \left[
 \begin{matrix}
 -4&4&4&-4
 \end{matrix}
 \right] \quad \det\left[\mathrm H_f(P_0) \right] \) = 0
 
 \subparagraph{\(y=0\)}
 
 \[
 f_1(x) = x^4-2x^2
 \]
 \[
 \left. \begin{array}{r}
 f_1'(x) = 4x^3-4x \implies f_1'(0) = 0 \\
 f_1''(x) = 12x^2-4 \implies f_1''(0)-4 < 0
 \end{array} \right\rbrace \text{τοπ. \textlatin{max}}
 \]
 
 
 \subparagraph{\( y=x \)}
 \begin{align*}
 f_2(x) &= 2x^4 \\
 f_2'(x) &= 8x^3 \implies f_2'(0) = 0 \to \text{ τοπικό \textlatin{min}}
 \end{align*}
 
 Άρα είναι σαγματικό.
 \end{enumerate}
 

\paragraph{Άσκηση}
\( f(x,y,z)=x^2+y^2+z^3+3xy+3yz+3xz \)
\[
\nabla f = 0 \implies \begin{cases}
f_x &= 3x^2+3y+3z = 0 \implies x^3+y+z = 0 \quad (1) \\
f_y &= 3y^2+3x+3z = 0 \implies y^2+x+z = 0 \quad (2) \\
f_z &= 3z^2+3x+3y = 0 \implies z^2+x+y = 0 \quad (3)
\end{cases}
\]

\begin{align*}
(1) - (2) &: x^2+y-y^2 -x = 0 \implies \\ &\implies
x^2-y^2-(x-y) = 0 \implies \\ &\implies
(x-y)(x+y)-(x-y) = 0 \implies \\ &\implies
(x-y)(x+y-1) = 0 \implies \begin{cases}
x+y=1 (4) \\ \qquad \text{ή} \\ \boxed{x=y} \quad (5)
\end{cases}
\end{align*}
(3) κ (4): \( z^2+1=0\implies z^2=-1 \) ΑΤΟΠΟ

(1) κ (5): \( x^2+x+z = 0 \)

(2) κ (5): \( x^2+x+z = 0 \implies z = -x-x^2 \quad (6) \)

(3) κ (5): \( z^2+2x = 0 \quad (7) \)

\begin{align*}
(6) \text{ κ } (7) &: \left(-x-x^2\right)^2+2x = 0 \\
&\implies x^2+x^4+2x^3+2x = 0 \\
&\implies x^3(x+2)+x(x+2)= 0 \\
&\implies (x+2)(x^3+x) = 0 \\
&\implies x(x+2)(x^2+1) = 0
\end{align*}

(6) \( 
\begin{array}{ll}
\xRightarrow{x=0} & z=0 \\
\xRightarrow{x=-2} & z=-2 
\end{array}
 \)
 
 \( (0,0,0)\ \ (-2,-2,-2) \)
 
 \( 
 \mathrm H_f = \left[
 \begin{matrix}
 6x&3&3\\3&6y&3\\3&3&6z
 \end{matrix}
 \right]
  \)
  
  \begin{enumparen}
\item   \( 
  \mathrm H_f(0,0,0) = \left[
  \begin{matrix}
  0&3&3\\3&0&3\\3&3&0
  \end{matrix}
  \right] \left. \begin{array}{l}
  \to \mathrm{Tr}\left[\mathrm{H}_f(0,0,0)\right] = 0 \\
  \to \det \left[\mathrm{H}_f(0,0,0)\right] \neq 0
  \end{array} \right\rbrace \text{μικτά προσημασμένος } \implies \text{ σαγμ.}
   \)
   
   \item   \( 
   \mathrm H_f(0,0,0) = \left[
   \begin{matrix}
   -12&3&3\\3&-12&3\\3&3&-12
   \end{matrix}
   \right] \left. \begin{array}{ll}
   \mathrm D_1 = -12 < 0 \quad & -\\
   \mathrm D_2 = 135 > 0 \quad & +\\
   \mathrm D_3 = -1350 < 0 \quad & -
   \end{array} \right\rbrace \text{αρν. ορισμένος } \implies \text{ τοπικό \textlatin{max}}
   \)
  \end{enumparen}


\subsection{Υπολογισμός στασίμων σημείων συνάρτησης \( z=f(x,y) \) επάνω σε καμπύλη \( g(x,y) = 0 \)}
\begin{enumparen}
\item
Κατασκευάζουμε τη βοηθητική συνάρτηση \( \Phi(x,y,\lambda ) = f(x,y)+\lambda g(x,y) \), \( \lambda  =  \) πολ/στής \textlatin{Lagrange}
\item
Βρίσκουμε τα στάσιμα σημεία \( P_0 = (x_0,y_0,\lambda _0) \) της \( \Phi(x,y,\lambda ) \) από της σχέση \( \nabla \Phi(x,y,\lambda ) = 0 \)
\item
Κατασκευάζουμε τον εσσιανό πίνακα της \( \Phi: \mathrm H_\Phi (x,y) \) για \( \forall P_0 \)
\item
Βρίσκουμε τα μη μηδενικά διανύσματα \( (v,w) \) που ικανοποιούν τη συνθήκη \( \nabla g (x,y) \cdot (v,w) = 0 \)
\item
Κατασκευάζουμε την παράσταση \( \left[
\begin{matrix}
v & w
\end{matrix}
\right] \mathrm H_\Phi (P_0) \left[\begin{matrix}
v \\ w
\end{matrix}\right]  
\rightarrow
\begin{cases}
>0\ \forall (v,w) \quad& \text{τότε $P_0$ τοπικό \textlatin{min}} \\
<0\ \forall (v,w) \quad& \text{τότε $P_0$ τοπικό \textlatin{max}} \\
\lessgtr 0 \text{ τουλάχιστον για κάποια } (v,w) \quad& \text{τότε $P_0$ σαγμ.}
\end{cases}
 \)
\end{enumparen}

\paragraph{Άσκηση}
Να βρεθούν και να χαρακτηριστούν τα στάσιμα σημεία της συνάρτησης \( f(x,y) = x^2+y^2 \), υπό τον περιορισμό \( 5x^2+6xy+5y^2=8 \)
\subparagraph{}
\( g(x,y) = 5x^2+6xy+5y^2-8=0 \)

\( \Phi(x,y,\lambda ) = f(x,y) +\lambda g(x,y) = x^2+y^2 + \lambda \left(
5x^2+6xy+5y^2-8
\right) \qquad (1) \)

\( \nabla \Phi(x,y,\lambda ) = 0 \implies
\begin{cases}
\Phi_x &= 2x+10\lambda x+6\lambda y = 0 \quad (2) \\
\Phi_y &=  2y + 6\lambda x +10\lambda y = 0 \quad (3) \\
\Phi_\lambda &= 5x^2+6xy+5^2-8 = 0 \quad (4)
\end{cases}
 \)
 
 \( (2) \text{ κ } (3) : 2x+2y+16\lambda  x + 16\lambda  y = 0 \implies
 x+y+8\lambda (x+y) = 0 \\
 \implies (x+y)(1+8\lambda ) = 0 \implies \begin{cases}
 x &= -y \\ \text{ή} \\ \lambda &= -\frac{1}{8}
 \end{cases}
  \)
  
  \subparagraph{1η περ.}
  \( x = -y \) \\
  
  \( \left. \begin{array}{l}
  (2) : -2y-10\lambda y+ 6\lambda y= 0 \implies
  -2y-4\lambda y = 0 \implies -2y(1+2\lambda ) = 0
    \\ (3):
   2y -6\lambda y + 10\lambda y = 0 \implies 2y+4\lambda y = 0 \implies
   2y(1+2\lambda ) = 0
\end{array} \right\rbrace
y = 0 \text{ ή } \lambda = -\frac{1}{2}
    \)
    
    \( (4):5y^2-6y^2+5y^2=8 
    \implies 4y^2 = 8 \implies y^2 = 2 \implies y = \sqrt{2} \text{ ή } \sqrt{-2}
     \)
     
     \( 
     \begin{array}{l}
     A = \left( \sqrt{2},-\sqrt{2}, -\frac{1}{2} \right)  \implies f(A)=4 \\
     B = \left( -\sqrt{2}, \sqrt{2}, -\frac{1}{2} \right) \implies f(B)=4
     \end{array}
      \)
      
      \subparagraph{2η περ.}
      \( \lambda = -\frac{1}{8} \)
      
      \( 
      \begin{array}{l}
      (2): 2x - \frac{10}{8}x - \frac{6}{8}y = 0 \implies x = y 
      \quad (5)
      \\
(3): 2y - \frac{6}{8}x - \frac{10}{8}y = 0 \implies \frac{6}{8}y
-\frac{6}{8}x = 0
\end{array}
\)

\( (4) \text{ κ } (5): 5x^2+6x^2+5x^2 = 8 \implies 16x^2=8 \implies x^2=\frac{1}{2} \implies x = \pm \frac{1}{\sqrt{2}} \)

     \( 
     \begin{array}{l}
     \varGamma = \left( \frac{1}{\sqrt{2}},\frac{1}{\sqrt{2}},-\frac{1}{8} \right) \implies f(\Gamma) = 1 \\
     \varDelta = \left(-\frac{1}{\sqrt{2}}, -\frac{1}{\sqrt{2}}, -\frac{1}{8} \right) \implies f(\Delta) = 1
     \end{array}
     \)
     
     \begin{attnbox}{}
     Εδώ από τις τιμές της συνάρτησης, φαίνεται ότι τα \( \Gamma,\Delta \) είναι σημεία τοπικού \& ολικού ελαχίστου, ενώ τα \( A,B \) είναι σημεία τοπικού \& ολικού μεγίστου. Η άσκηση εδώ έχει τελειώσει. Αν προχωρούσαμε με τη μεθοδολογία θα είχαμε:
     \end{attnbox}
     
     \( \mathrm H_\Phi = (x,y) = \left[ \begin{matrix}
     \Phi_{xx} & \Phi_{xy} \\  \Phi_{yx} & \Phi_{yy}
     \end{matrix} \right] = \left[
     \begin{matrix}
     2+10\lambda & 6\lambda \\ 6\lambda & 2+10\lambda 
     \end{matrix} \qquad (6)
     \right]\)
     
     Έστω διανύσματα \( (v,w) \) που ικανοποιούν τη συνθήκη
     \begin{align*}
     \nabla g(x,y) \cdot (v,w) = 0 &\implies \left(
     10x_0+6y,\ 6x_0+10y_0
     \right) \cdot (v,w) = 0
     \\ &\implies (10x_0+6y_0)v+(6x_0+10y_0)w = 0 \qquad (7)
     \end{align*}
     
     \begin{enumerate}
     \item Σημείο \( A \): \( \mathrm H_\Phi(A) = \left[
     \begin{matrix}
     -3 & -3 \\ -3 & -3
     \end{matrix}
     \right] \)
     
     \( (7): \left(
     10\sqrt{2} - 6\sqrt{2}
     \right)v + \left( 6\sqrt{2}-10\sqrt{2}w  \right) =0 \implies \underline{  v=w } \)
     
     \( \left[\begin{matrix} v & w  \end{matrix}\right] 
     \mathrm H_\Phi (A) \left[ \begin{matrix}v \\ w \end{matrix} \right]
     = [v \; w] \left[ \begin{matrix}
     -3 & -3 \\ -3 & -3
     \end{matrix} \right]\left[ \begin{matrix}
     v w
     \end{matrix} \right] = \left[ v\; w \right] \left[
     \begin{matrix}
     -3v-3w \\ -3v-3w
     \end{matrix}
     \right]
     = (-3v-3w)v+(-3v-3w)w = -3(v+w)v-3(v+w)w=-3(v+w)^2 = -3(2v)^2=-12v^2 < 0
     \)
     
     \( A \rightarrow \) τοπικό \textlatin{max}
     
     \item Το \( B \) θα το κάνετε μόνοι σας.
     
     \item
     
     Σημείο \( \varGamma \): \( 
     \mathrm H_\Phi(\varGamma) = \left[
     \begin{matrix}
     \frac{3}{4} & -\frac{3}{4} & -\frac{3}{4} & \frac{3}{4}
     \end{matrix}
     \right]
      \)
      
      \( (7):
      \left(
      \frac{10}{\sqrt{2}}+ \frac{6}{\sqrt{2}}
      \right)v + \left(
      \frac{6}{\sqrt{2}} + \frac{10}{\sqrt{2}}
      \right)w = 0 \implies v = -w
       \)
       
       \( 
       \left[
       \begin{matrix}
       v & w
       \end{matrix}
       \right] \mathrm H_\Phi(\varGamma) \left[
       \begin{matrix}
       v \\w
       \end{matrix}
       \right] = \left[
       \begin{matrix}
       v & w
       \end{matrix}
       \right] = \left[
       \begin{matrix}
       \frac{3}{4} & -\frac{3}{4} \\ -\frac{3}{4} & \frac{3}{4}
       \end{matrix}
       \right] \left[
       \begin{matrix}
       v \\ w
       \end{matrix}
       \right] = \left(
       \frac{3}{4}v-\frac{3}{4}w
       \right) v + \left(
       -\frac{3}{4}v + \frac{3}{4}w
       \right) w = \frac{3}{4} (v-w)v - \frac{3}{4}(v-w)w = \frac{3}{4}(v-w)^2 = \frac{3}{4}(2v)^2 > 0
        \)
        
        \( \varGamma \rightarrow \) τοπικό ελάχιστο
        
        \item
        Το \( \varDelta \) αφήνεται ως άσκηση για τον αναγνώστη.
     
     \end{enumerate}
     
     \paragraph{Άσκηση}
     Να βρεθούν τα ολικά ακρότατα της \( f(x,y)=(x-2)^2+(y-8)^2+10 \) επί του χωριου \( D = \left\lbrace (x,y):x^2+y^2 \leq 1 \right\rbrace \)
     
     \begin{enumerate}
     \item Εύρεση στάσιμων σημείων εντός του \( D \ (x^2+y^2 < 1) \)
     
     \( 
     \nabla f(x,y) = 0 \implies \left\lbrace \begin{array}{ll}
     f_x = 2(x-2) = 0 &\implies x = 2\\
     f_y = 2(y-8) = 0 &\implies y = 8
     \end{array}
     \right\rbrace
      \)
      
      \item Εύρεση στάσιμων σημείων επί του \( \partial D \ (x^2+y^2=1) \)
      \( g(x,y)=x^2+y^2-1 = 0 \)
      
      \( \Phi(x,y,\lambda ) = f(x,y) = \lambda g(x,y) = (x-2)^2+(y-8)^2+10 + \lambda \left(x^2+y^2-1\right) \)
      
      \( \nabla \Phi(x,y,\lambda ) = 0 \implies \left\lbrace
      \begin{array}{l}
      \Phi_x = 2(x-2) +2\lambda x = 0  \implies x = \frac{2}{1+\lambda} (1) \\
      \Phi_y = 2(y-8)+2\lambda y = 0  \implies y = \frac{8}{1+\lambda } (2) \\
      \Phi_\lambda = x^2+y^2-1 = 0
      \end{array}
      \right\rbrace \frac{4}{(1+\lambda )^2+\frac{64}{(1+\lambda )^2}} = 1 \implies
      (1+\lambda )^2 = 68 \implies \lambda = \pm \sqrt{68} -1
       \)
      
      Για \( \lambda  = \sqrt{68}-1 \quad (1) \implies x=\frac{1}{\sqrt{17}},\ (2) \implies y = \frac{4}{\sqrt{17}} \)
      
      Για \( \lambda = -\sqrt{68}-1 \quad (1) \implies x=-\frac{1}{\sqrt{17}},\ (2) \implies y=\frac{-4}{\sqrt{17}} \)
      
      \[
      \begin{array}{lr}
      \min \leftarrow A = \left( \frac{1}{\sqrt{17}},\frac{4}{\sqrt{17}},\sqrt{68}-1 \right)
      & f(A) = 62.5 \\
      \max \leftarrow B = \left(
      -\frac{1}{\sqrt{17}},-\frac{4}{\sqrt{17}},-\sqrt{68}-1
      \right) & f(B)=95.5
      \end{array}
      \]
      
      
      
     \end{enumerate}


\paragraph{Άσκηση}
Περσινό θέμα

Να βρεθούν και να χαρακτηριστούν τα στάσιμα σημεία της \( f(x,y) = x^2+y^3-6y \)

\( \nabla f(x,y) = 0 \implies \left\lbrace
\begin{array}{ll}
f_x &= 2x = 0 \implies x = 0 \\
f_y &=3y^2-6 = 0 \implies y^2=2 \implies y = \pm \sqrt{2}
\end{array} \right\rbrace \begin{array}{l}
\nearrow A=\left(0,\sqrt{2}\right) \\
\searrow B=\left(0,-\sqrt{2}\right)
\end{array}
 \)
 
 \( 
 \mathrm{H}_f = \left[
 \begin{matrix}
 f_{xx} & f_{xy} \\ f_{yx} & f_{yy}
 \end{matrix}
 \right] = \left[
 \begin{matrix}
 2 & 0 \\ 0 & 6y
 \end{matrix}
 \right] \begin{array}{l}
 \nearrow \mathrm H_f(\underbrace{A}_{\mathclap{\text{τοπικό ελάχιστο}}}) = \left[
 \begin{matrix}
 2 & 0 \\ 0 & 6\sqrt{2}
 \end{matrix}
 \right]
 \\
 \searrow \mathrm H_f(\underbrace{B}_{\mathclap{\text{σαγματικό}}}) = \left[
 \begin{matrix}
 2 & 0 \\ 0 & -6\sqrt{2}
 \end{matrix}
 \right]
 \end{array}
  \)
  
  \paragraph{Άσκηση}
  Να βρεθούν και να χαρακτηριστούν τα στάσιμα σημεία της \( f(x,y) = x-2y \) επάνω στην καμπύλη \( x^2+y^2=4 \)
  \subparagraph{}
  \( \Phi(x,y,\lambda ) = f(x,y)+\lambda g(x,y)=x-2y+\lambda (x^2+y^2-4) \qquad (1) \)
  
  \( \nabla \Phi(x,y,\lambda ) = 0 \implies \begin{cases}
  \Phi_x = 1+2\lambda x = 0 \qquad (2) \\
  \Phi_y = -2+2\lambda y = 0 \qquad (3) \\
  \Phi_\lambda = x^2+y^2 - 4 = 0 \qquad (4)
  \end{cases} \)
  
  \( 
  \begin{array}{l}
  (2): x = -\frac{1}{2\lambda }\qquad (5) \\
  (3): y = \frac{1}{\lambda } \qquad (6)
  \end{array}
   \)
   
   \( 
   (4) \xRightarrow{(5),(6)} \frac{1}{4\lambda ^2}+\frac{1}{\lambda^2} = 4
   \implies \frac{5}{4\lambda ^2}=4 \implies \lambda ^2=\frac{5}{16} \implies \lambda = \pm \frac{\sqrt{5}}{4}
    \)
    
    \(
    \left.
    \begin{array}{l}
    \text{Για } \lambda =\frac{5}{4}\ (5): x = \frac{-2}{\sqrt{5}} \ (6):y=\frac{4}{\sqrt{5}} 
    \\
    \text{Για }  \lambda = -\frac{\sqrt{5}}{4}\ (5):x=\frac{2}{\sqrt{5}}\ (6):y = -\frac{4}{\sqrt{5}} 
    \end{array}
    \right\rbrace \begin{array}{lr}
    \rightarrow 
    A = \left(
    -\frac{2}{\sqrt{5}},\frac{4}{\sqrt{5}},\frac{\sqrt{5}}{4}\right) \rightarrow f(A) = -\frac{10}{\sqrt{5}} & \min \\
    \rightarrow B = \left(
    \frac{2}{\sqrt{5}},-\frac{4}{\sqrt{5}},-\frac{\sqrt{5}}{4}
    \right) \rightarrow f(B) = \frac{10}{\sqrt{5}} & \max
    \end{array}
    \)
    
    

\end{document}

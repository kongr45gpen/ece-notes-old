\documentclass[12pt,a4paper,titlepage,fleqn]{article}

\usepackage{amsmath}
\usepackage{amsfonts}
\usepackage{amssymb}
\usepackage{commath}
\usepackage{xcolor}
\usepackage{hyperref}
\usepackage[skins,theorems]{tcolorbox}
\usepackage{titlesec}
\usepackage{circuitikz}
\usepackage{pgfplots}
\usepackage{mathtools}
\usepackage[makeroom]{cancel}
\usepackage{mathrsfs}
\usepackage{wrapfig}
\usepackage{subcaption}
\usepackage{floatrow}
\usepackage{esint}
\usepackage{enumitem}
\usepackage{bm}
\usepackage{relsize}
\usepackage{xfrac}
\usepackage{comment}


\usetikzlibrary{arrows.meta}
\usetikzlibrary{patterns}
\usetikzlibrary{decorations.pathmorphing,patterns}
\usetikzlibrary{decorations.markings}
\usetikzlibrary{backgrounds}
\usetikzlibrary{shapes.misc}
\usetikzlibrary{shapes.multipart}

\tikzset{cross/.style={cross out, draw,
        minimum size=2*(#1-\pgflinewidth),
        inner sep=0pt, outer sep=0pt}}
\tikzset{
    mark position/.style args={#1(#2)}{
        postaction={
            decorate,
            decoration={
                markings,
                mark=at position #1 with \coordinate (#2);
            }
        }
    }
}

\usepackage[left=2cm,right=2cm,top=2cm,bottom=2cm]{geometry}

\usepackage[no-math]{fontspec}
\setmainfont{Times New Roman}
\setsansfont{Arial}
%\newfontfamily\greekfont[Script=Greek]{Linux Libertine O}
%\newfontfamily\greekfontsf[Script=Greek]{Linux Libertine O}
\usepackage{polyglossia}
\newfontfamily\greekfont[Script=Greek]{Times New Roman}
\newfontfamily\greekfontsf[Script=Greek]{Arial}
\newfontfamily\greekfonttt[Script=Greek]{Latin Modern Mono}
%\usepackage[greek]{babel}
\setdefaultlanguage{greek}
\setotherlanguage{english}
\newcommand{\textlatin}[1]{#1}
%\newcommand{\mathlarger}{}

%\usepackage[utf8]{inputenc}
%\usepackage[greek]{babel}


\usetikzlibrary{arrows.meta}
\usetikzlibrary{calc}
%\usepackage{tkz-euclide} % loads  TikZ and tkz-base
%\usetkzobj{angles} % important you want to use angles

\newlist{enumparen}{enumerate}{1}
\setlist[enumparen]{label=(\arabic*)}
\newlist{enumpar}{enumerate}{1}
\setlist[enumpar]{label=\arabic*)}

\newlist{enumgreek}{enumerate}{1}
\setlist[enumgreek]{label=\alph*.}
\newlist{enumgreekparen}{enumerate}{1}
\setlist[enumgreekparen]{label=(\alph*)}
\newlist{enumgreekpar}{enumerate}{1}
\setlist[enumgreekpar]{label=\alph*)}


\newlist{enumroman}{enumerate}{1}
\setlist[enumroman]{label=(\roman*)}

\newlist{enumlatin}{enumerate}{1}
\setlist[enumlatin]{label=(\alph*)}

\newlist{invitemize}{itemize}{1}
\setlist[invitemize]{noitemsep,label=}



\makeatletter
\let\anw@true\anw@false

%\newcommand{\attnboxed}[1]{\textcolor{red}{\fbox{\normalcolor\m@th$\displaystyle#1$}}}
\makeatother
\tcbset{highlight math style={enhanced,colframe=red,colback=white,%
        arc=0pt,boxrule=1pt,shrink tight,boxsep=1.5mm,extrude by=0.5mm}}
\newcommand{\attnboxed}[1]{\tcbhighmath[colback=red!5!white,drop fuzzy shadow,arc=0mm]{#1}}
\titleformat{\section}{\bf\Large}{Κεφάλαιο \thesection}{1em}{}
\newtcolorbox{attnbox}[1]{colback=red!5!white,%
    colframe=red!75!black,fonttitle=\bfseries,title=#1}
\newtcolorbox{infobox}[1]{colback=blue!5!white,%
    colframe=blue!75!black,fonttitle=\bfseries,title=#1}

\renewcommand{\arg}{\mathrm{Arg}\, }
\renewcommand{\Re}{\mathrm{Re}}
\renewcommand{\Im}{\mathrm{Im}}
\newcommand{\sinc}{\;\mathrm{sinc}\!}

\newif\ifhidetikz
\hidetikzfalse
%\hidetikztrue   % <---- comment/uncomment that line

\ifhidetikz

\let\oldtikzpicture\tikzpicture
\let\oldendtikzpicture\endtikzpicture

\renewenvironment{tikzpicture}{
    \tiny
    \tt
    \color{blue}
    \newcommand{\draw}{\textit{draw}}
    \newcommand{\filldraw}{\textit{filldraw}}
    %\newcommand{\x}{\textit{x}}
    %\newcommand{\p}{\textit{x}}
    \newcommand{\x1}{\textit{x1}}
    \newcommand{\y1}{\textit{y1}}
    \newcommand{\p1}{\textit{p1}}
}{
}
\newenvironment{axis}{
    \newcommand{\addplot}{\textit{addplot}}
}{
}
\fi

\newtcbtheorem[number within=section]{theorem}{Θ.}%
{colback=green!5,colframe=green!35!black,colbacktitle=green!35!black,fonttitle=\bfseries,enhanced,attach boxed title to top left={yshift=-8mm,xshift=-7mm},width=.9\textwidth,arc=.7mm}{th}
\newtcbtheorem[number within=section]{defn}{Ορισμός}%
{colback=blue!5,colframe=cyan!35!black,colbacktitle=blue!35!black,fonttitle=\bfseries,enhanced,attach boxed title to top left={yshift=-2mm,xshift=-2mm}}{def}
\newtcbtheorem[number within=section]{exercise}{Άσκηση}%
{colback=gray!3,colframe=gray!35!black,colbacktitle=gray!35!black,fonttitle=\bfseries,enhanced,attach boxed title to top left={yshift=-2mm,xshift=-2mm}}{exc}




\setmainfont{Ubuntu Light}
\setsansfont{Arial}
%\newfontfamily\greekfont[Script=Greek]{Linux Libertine O}
%\newfontfamily\greekfontsf[Script=Greek]{Linux Libertine O}
\usepackage{polyglossia}
\newfontfamily\greekfont[Script=Greek,Scale=0.9]{Ubuntu Light}

\title{Εφαρμοσμένα Μαθηματικά - Σημειώσεις}
\date{2016}
\author{\textlatin{\csuse{no\greek @numbers}\selectlanguage{english} \url{https://github.com/kongr45gpen/ece-notes}}}

\begin{document}
	\url{http://users.auth.gr/natreas} \\
	Σημειώσεις: Εγώ Κεφ. 3-4-5 \\
	Κεχαγιάς Κεφ. 1-2-6

	Βιβλία:
	\begin{itemize}
		\item Churchill - Brown (για μηχανικούς)
		\item Marjden (πιο μαθηματικό)
	\end{itemize}

	\part{Ατρέας}
	\section{Μιγαδικοί Αριθμοί}
	\textbf{Έστω} \( \mathbb C = \left\lbrace z = \overbrace{(x,y)}^{\mathclap{\text{γεωμετρική παράσταση μιγαδικού}}};\ x,y\in\mathbb R  \right\rbrace \)

	Είναι σύνολο εφοδιασμένο με τις πράξεις:
	\begin{enumgreekparen}
		\item Πρόσθεση μιγαδικών

		Αν \( z_1=(x_1,y_1) \) και \( x_2=(x_2,y_2) \), τότε:\[
		z_1+z_2 = (x_1+x_2,\ y_1+y_2)
		\]

		\item Γινόμενο \( \lambda \in \mathbb R  \) με μιγαδικό \( z \)

		Αν \( z=(x,y) \), τότε ορίζω:
		\[
		\lambda z = (\lambda x,\lambda y)
		\]

		\item \attnboxed{\text{Πολλαπλασιασμό μιγαδικών αριθμών}}

		Αν \( z_1=(x_1,y_1),\ z_2=(x_2,y_2) \), τότε ορίζω:
		\[
		z_1z_2 = \left(x_1x_2-y_1y_2,\ x_1y_2+x_2y_1\right)
		\]
	\end{enumgreekparen}

	Καλείται σύνολο των μιγαδικών αριθμών.

	\begin{itemize}
		\item Δεν μπορώ να συγκρίνω μιγαδικούς
		\item Οι γνωστές ιδιότητες των πράξεων ισχύουν στους μιγαδικούς
	\end{itemize}

	Η γεωμετρική παράσταση του \( \mathbb C \) είναι το λεγόμενο μιγαδικό επίπεδο.

	\begin{center}
	\begin{tikzpicture}[scale=2.5]
		\draw[->] (0,-1.3) -- (0,1.5);
		\draw[->] (-1.5,0) -- (1.7,0);

		\draw[dashed] (0,1) -- (1,1) -- (1,-1);
		\filldraw (1,1) circle(0.8pt) node[above right] {$z=(x,y)$} ;
		\filldraw (0,1) circle(0.6pt) node[below right] {$(0,1)=i$};
		\filldraw (1,0) circle(0.6pt);

		\draw[->] (1.4,-0.8) -- (1.4,-0.2) node[midway,right] {πραγματικός άξονας \( \Re(z) \)};

		\draw[->] (-1,0.7) -- (-0.2,0.7) node[pos=.1,below] {φανταστικός άξονας \( \Im(z) \)};

		\draw[gray,->] (0,0) -- (1,1);
		\draw[->] (.3,0) arc (0:45:.3) node[midway,right] {$\theta$};

		\draw[gray,->] (0,0) -- (1,-1);
		\draw[->] (.3,0) arc (0:-45:.3);
		\filldraw (1,-1) circle(0.8pt) node[right] {$\bar z=(x,-y)$} ;
	\end{tikzpicture}
	\end{center}

	\[
	x \in \mathbb R \xleftrightarrow{\text{1-1}} A = \left\lbrace (x,0): x \in \mathbb R  \right\rbrace
	\]

	\begin{itemize}
		\item \(
		    (x,0),(y,0) \in A \implies (x,0)+(y,0)=(x+y,0) \in A
		\)
		\item \(
		    (x,0)(y,0) = (xy,0) \in A
		\)
	\end{itemize}

	Στο εξής γράφω: \begin{align*}
	    1 &= (1,0) \\
	    x &= (x,0)
	\end{align*}

\textbf{Ορίζω}:
	\[
	\mathlarger{\mathlarger{\mathlarger{i = (0,1)}}}
	\]
	και καλείται φανταστική μονάδα του μιγαδικού επιπέδου.

	\begin{gather*}
	i^2 = (0,1)(0,1) = (0\cdot0-1\cdot1,\ 0\cdot1+1\cdot0) = (-1,0) = -1 \\
	\boxed{i^2=-1}
	\end{gather*}

	\textbf{Έτσι}:
	\begin{gather*}
	    z=(x,y) = x(1,0) + y(0,1) \\
	    \overset{x=(x,0)}{\underset{i=(0,1)}{=}} x \cdot 1 + yi \\
	    \implies \boxed{z=x+iy}
	\end{gather*}

	\[
	\mathlarger{\mathlarger{\underbrace{z=x+iy}_{\mathclap{\text{άλγεβρα}}}
			\iff \underbrace{z=(x,y)}_{\mathclap{\text{γεωμετρία}}}
			}}
	\]

	\paragraph{}
	Έστω \( z=x+iy \)
	\begin{gather}
		\overset{\text{πολικές}}{\underset{\text{του } (x,y)}{=}}
		\rho\cos\theta+i\rho\sin\theta = \nonumber
		\\ = \mathlarger{\rho(\cos\theta+i\sin\theta)} \label{eq:1}
	\end{gather}

	Έτσι, η (\ref{eq:1}) γράφεται ως:
	\begin{align*}
	z &= |z| \underbrace{(\cos\theta+i\sin\theta)} \\
	  &= |z| \cdot \mathlarger{\mathlarger{e^{i\theta}}}
	\end{align*}
	όπου στο εξής:
	\begin{align*}
	\Aboxed{e^{i\theta} = \cos\theta+i\sin\theta} \\
	\Aboxed{\text{τύπος του Euler}}
	\end{align*}

	Τελικά: \[
	\boxed{\mathlarger{\mathlarger{\mathlarger{\mathlarger{z=|z|e^{i\theta}}}}}}
	\text{ (πολική μορφή μιγαδικών)}
	\]

	\subparagraph{Σημείωση:} \( \cos\theta + i\sin\theta \)
	\begin{gather*}
	\overset{\text{σειρές}}{\underset{\text{McLaurin}}{=}} \left(
	1-\frac{\theta^2}{2!} + \frac{\theta^4}{4!} + \dots
	\right) + i \left(\theta-\frac{\theta^3}{3!}+\frac{\theta^5}{5!}-\dots\right)
	\\
	\overset{i^2=-1}{=} \left(
	1+\frac{(i\theta)^2}{2!}+\frac{(i\theta)^4}{4!}+\dots
	\right) + \left(
	i\theta+\frac{(i\theta)^3}{3!}+\frac{(i\theta)^5}{5!}+\dots
	\right)
	\\ =
	1 + (i\theta) + \frac{(i\theta)^2}{2!} + \frac{(i\theta)^3}{3!}
	+ \dots + \frac{(i\theta)^n}{n!} + \dots = \mathlarger{e^{i\theta}}
	\end{gather*}

	\begin{itemize}
		\item Ορίζω {\large πρωτεύον όρισμα} \( \mathlarger{\mathlarger{\mathrm{Arg} z}} \) (μη μηδενικού) μιγαδικού \( z \) να είναι η γωνία \( \theta \)
		που σχηματίζει ο θετικός πραγματικός ημιάξονας του \( \mathbb C \) με την
		ημιευθεία \( OA \), όπου \( A \) το σημείο της γεωμετρικής παράστασης του
		\( z=x+iy \).
	\end{itemize}

	\subparagraph{Έτσι:}
	\[
	z = |z|e^{i\arg z} \quad \text{πολική μορφή του } z
	\]

	\begin{align*}
	z_1z_2 &= |z_1|e^{i\arg z_1}|z_2|e^{i\arg z_2} \\
	\Aboxed{z_1z_2 &= |z_1||z_2|e^{i(\arg z_1 + \arg z_2)}
	}
	\end{align*}
	\begin{align*}
	\frac{z_1}{z_2} &= \frac{|z_1|}{|z_2|} \frac{e^{i\theta_1}}{e^{i\theta_2}}
	\\ &= \left| \frac{z_1}{z_2} \right| e^{i(\theta_1-\theta_2)}
	\end{align*}

	\begin{tikzpicture}[scale=2.5]
	\draw[gray,->] (0,-0.7) -- (0,2);
	\draw[gray,->] (-1.5,0) -- (1.7,0);

	\filldraw (0,0) -- ++(35:1.2) circle(0.6pt) node[above right] {$z_1$};
	\draw[->] (.3,0) arc (0:35:.3) node[midway,right] {$\theta_1$};

	\filldraw (0,0) -- ++(75:1.7) circle(0.6pt) node[above right] {$z_2$};
	\draw[->] (.6,0) arc (0:75:.6) node[pos=.8,above right] {$\theta_2$};

	\filldraw (0,0) -- ++(110:{1.2*1.7}) circle(0.6pt) node[above right] {$z_1z_2$};
	\draw[->] (1,0) arc (0:110:1) node[pos=.6,above right] {$\theta_1+\theta_2$};
	\end{tikzpicture}

	\textbf{Ιδιότητα:} \( z\bar{z} = |z|^2 \)

	\section{Μιγαδικές συναρτήσεις}
	Κάθε συνάρτηση \( f: A \subseteq \mathbb C \to \mathbb C \) καλείται μιγαδική
	συνάρτηση μιγαδικής μεταβλητής.

	\[
	f = \underbrace{f(\underbrace{z}_{\text{η μεταβλητή μιγαδικός}})}_{\text{μιγαδική συνάρτηση διότι έχει τιμή μιγαδική}}
	\]

	\paragraph{π.χ.}
	\begin{gather*}
	f(z) = z^2 \implies
	f(x+iy) = (x+iy)^2 = x^2 + (iy)^2+2x\cdot \underbrace{x^2-y^2}_{\Re(f)}+i\underbrace{(2xy)}_{\Im(f)}
	\\
	\overset{\text{γεωμετρική}}{\underset{\text{μορφή}}{=}} (x^2-y^2,\ 2xy)
	\end{gather*}
	\subparagraph{Τελικά:} \(\boxed{f(x,y)=(x^2-y^2,\ 2xy)} \quad \mathbb R^2 \to \mathbb R^2 \)

	\paragraph{π.χ.}
	\begin{gather*}
	f(z) = \frac{1}{|z|\bar{z}} \overset{z=x+iy}{=}
	\frac{1}{\sqrt{x^2+y^2}}\cdot \frac{z}{\bar{z}z} \\
	\overset{z\bar{z}=|z|^2}{=} \frac{1}{\sqrt{x^2+y^2}} \cdot \frac{z}{|z|^2}
	= \frac{x+iy}{(x^2+y^2)^{\sfrac{3}{2}}}
	\\ \overset{\text{γεωμ}}{=}
	\frac{(x,y)}{(x^2+y^2)^{\sfrac{3}{2}}}
	\overset{\vec{r} = (x,y)}{=} \boxed{\frac{\vec{r}}{|\vec{r}|^3}}
	\end{gather*}

	Κεντρικό διαν. πεδίο που θυμίζει το πεδίο Coulomb.

	\[
	\underbrace{f=f(z)}_{\mathclap{\text{μιγαδική μιγ. μεταβλ.}}} \xleftrightarrow{\quad\text{1-1}\quad}
	\begin{array}{l}
	\text{διανυσμ. πεδίο του } \mathbb R^2 \\
	F(x,y) = \left( u(x,y),\ v(x,y) \right)
	\end{array}
	\]
	όπου \( u,v \) πραγματ. συναρτ. 2 μεταβλητών

	\paragraph{Υπάρχουν} \( f:A \subseteq \mathbb R \to \mathbb C \),
	μιγαδικές πραγματικής μεταβλητής

	π.χ \begin{align*}
	f(t) &= e^{it},\ t \in (0,\pi] \\
	&= \cos t + i \sin t
	\end{align*}
	\[
	t \to (\cos t, \sin t) \quad \text{καμπύλη } x^2+y^2=\cos^2 t +\sin^2 t = 1
	\]

	\begin{tikzpicture}[scale=1.5]
	\draw[->] (0,-1.5) -- (0,1.5);
	\draw[->] (-1.5,0) -- (1.5,0);

	\draw[thick,
		decoration={markings, mark=at position 0.125 with {\arrow{>}}},
		postaction={decorate}
	] (0,0) circle (1);

	\draw (0,1) node[above right] {$f(t)=e^{it}$};
	\end{tikzpicture}

	Η γραφ. παράσταση της \( f(t)=e^{it},\ t \in (-\pi,\pi) \) είναι ο μοναδιαίος κύκλος
	κέντρου \( (0,0) \) με αντιωρολογιακή φορά.

	\[
	g(t) = 1+it, t\in \mathbb R,\ =(1,t) = (1,0)+t(0,1)
	\]
	
	\paragraph{}
	Το πεδίο ορισμού μιγαδικών συναρτήσεων μιγαδ. μεταβλητών
	υπολογίζεται ως συνήθως (με τις πραγματικές συναρτήσεις)
	ΜΕ ΚΑΠΟΙΕΣ Διαφοροποιήσεις
	
	\[
	f(z)=\frac{1}{z}
	\]
	Πρέπει ο παρον. να είναι διάφορος του μηδενός: Έτσι
	\( z \neq 0 \) Άρα Π.Ο \( = \mathbb C - \left\lbrace (0,0) \right\rbrace \)
	
    \[
    g(z) = \frac{z}{z^2+2}
    \]
    \subparagraph{Σημείωση} Η \( g \) είναι \textbf{ρητή} συνάρτηση
    (δηλ. πηλίκο δύο (μιγαδικών) πολυωνύμων).
    
    Κάθε συνάρτηση της μορφής \( 
    a_0+a_1z+\dots+a_nz^n,\ a_0,\dots,a_n \in \mathbb Z
     \) καλείται (μιγαδικό) πολυώνυμο.
     
    Πρέπει παρον. \( \neq0 \) δηλ:
    \begin{gather*}
    z^2+2=0\
        \left(
        \begin{array}{l}
        \text{\textbf{ΠΡΟΣΟΧΗ!!} Κάθε μιγαδικό} \\
        \text{πολυώνυμο βαθμού $N$ έχει} \\
        \text{ΑΚΡΙΒΩΣ $N$ ρίζες στο $\mathbb C$}
        \end{array}
        \right)
     \\ 
    z^2+2 = 0 \xRightarrow{i^2=-1} z^2-2i^2=0 \\
    \implies \left( z-\sqrt{2}i \right)\left(z+\sqrt{2}i \right)=0
    \\ \implies \boxed{z = \pm \sqrt{2}i}
    \end{gather*}
    
    \subparagraph{Τελικά} Π.Ο = \( \mathbb C - 
    \left\lbrace \pm \sqrt{2}i \right\rbrace
     \)
     
    \paragraph{}
    \[ \boxed{
    	h(z) = \arg z,\ \text{Π.Ο} = \mathbb C - \left\lbrace 0 \right\rbrace 
    } \]
    
    Για \( z=0 \) ΔΕΝ ορίζεται όρισμα, επειδή \( 0 = |0|\cdot e^{i\theta}
    \ \forall \theta
     \)
   
  \paragraph{Σημείωση}
  \( az^2+bz+c = 0 \) \\ \(\qquad a,b,c \in \mathbb C \)
  
  Λύνεται με διακρίνουσα κατά τα γνωστά.
  
  Επίσης μπορείτε να χρησιμοποιήσετε και σχήμα Horner για πολυώνυμα 
  (με πραγματικούς συντελεστές) βαθμού \( N \geq 3 \).
   
 \paragraph{}
 \begin{align*}
 a(z) &=e^z = e^{x+iy} = e^x\cdot e^{iy} \\
 &= e^x (\cos y + i \sin y) \\
 &= \left( e^x\cos y,\ e^x\sin y \right),\quad x,y\in\mathbb R 
 \end{align*}
 Ως διανυσματικό πεδίο προφανώς Π.Ο = \( \mathbb R ^2 \)
 
 Έτσι Π.Ο = \( \mathbb C \).
    
   \paragraph{}
   \begin{gather*}
   l(z) = \mathrm{Log}\. z \text{ (αντίστροφη της } e^z \text{)} \\
   \underbrace{\mathrm{Log}\. z}_{\mathclap{\text{μιγαδικός λογάριθμος}}}
   \overset{\text{ορισμός}}{:=} \ln|z| +i\arg z \\
   \text{Π.Ο} = \mathbb C - \left\lbrace 0 \right\rbrace
   \end{gather*}
   %TODO Atreas Graph 03 (small)
   \begin{align*}
   \mathrm{Log}(3) &= \ln|-3| = i\arg(-3) \\ &= \ln3+i\pi
   \end{align*}
   
    \paragraph{}
    \[
    \lambda(z) = \sin z \overset{\text{ορισμός}}{:=} \frac{e^{iz}-e^{-iz}}{2i}
    \]
    \[
    \left(
    \begin{array}{ll}
    e^{i\theta} &=\cos\theta+i\sin\theta  \quad \theta\in (-\pi,\pi] \\
    e^{-i\theta} &= \cos\theta -i\sin\theta \\[0.3pt] \hline 
    \sin\theta &= \frac{e^{i\theta}-e^{-i\theta}}{2i}
    \end{array}
    \right)
    \]
    Π.Ο = \( \mathbb C \)
    
    \begin{gather*}
    m(z) = \cos z \overset{\text{ορισμός}}{:=} \frac{e^{iz}+e^{-iz}}{2} \\
    \text{Π.Ο} = \mathbb C
    \end{gather*}
    
    Όλες οι γνωστές τριγωνομετρικές ταυτότητες ισχύουν στο \( \mathbb C \)
    όπως στο \( \mathbb R  \).
    
    \paragraph{}
    \begin{align*}
    h(z) = \sqrt[n]{z} := 
    \sqrt[n]{|z|} e^{i\frac{2k\pi+\arg z}{n}} \quad (k=0,1,\dots,n-1)
    \end{align*}
    (Η \( \sqrt[n]{a} \) ορίζεται ως το \textbf{σύνολο} όλων των λύσεων
    της εξίσωσης \( z^n=a,\quad a\in\mathbb C \) )
    \[
    \text{Π.Ο} = \mathbb C - \left\lbrace 0 \right\rbrace
    \]
    
    \subsection{Όριο/Συνέχεια\\μιγαδικών συναρτήσεων μιγαδικής μεταβλητής}
    \begin{defn*}{}
    	Έστω \( f(z)=f(x+iy)=u(x,y)+iv(x,y) \)
    	μιγ. συνάρτηση ορισμένη σε σύνολο \( A \subset \mathbb C,
    	\ z_0=x_0+iy_0 \) είναι σ.συσσ. του \( A \) και έστω \( a=a_0+ib_0 \).
    	Τότε
    	
    	\begin{gather*}
    	\lim_{z\to z_0}f(z) = a \in \mathbb C \\
    	\qquad \Updownarrow \\
    	\begin{cases}
    	\lim\limits_{(x,y)\to(x_0,y_0)} u(x,y) = a_0 \\ \qquad \text{\textbf{ΚΑΙ}} \\
    	\lim\limits_{(x,y)\to(x_0,y_0)} v(x,y) = b_0
    	\end{cases}
    	\end{gather*}
    \end{defn*}
    	\textbf{Επίσης,} αν \( z_0\in A \), τότε
    	
    	\( f \) συνεχής στο σημείο \( z_0 \)
    	\[ \Updownarrow \]
    	
    	οι συναρτήσεις \( u,v:A \subset \mathbb R^2\to\mathbb R  \)
    	είναι ΣΥΝΕΧΕΙΣ στο σημείο \( (x_0,y_0 \) (ως πραγματικές συναρτήσεις
    	δύο μεταβλητών)
    	
    \subparagraph{Έτσι:}
    \begin{align*}
    \left.
    \begin{array}{l}
    \text{οι πολυωνυμικές}\\
    \text{η εκθετική}\\
    \text{οι τριγωνομετρικές }(\sin z,\cos z)\\
    \text{οι υπερβολικές }(\mathrm{ch}\.z,\mathrm{sh}\.z)
    \end{array}
    \right\rbrace &\ \text{συνεχείς στο } \mathbb C
    \\
    \left.
    \begin{array}{l}
    \text{οι ρητές}\\
    \text{οι τριγωνομετρικές }(\tan z,\cot z)\\
    \end{array}
    \right\rbrace &\ \text{συνεχείς στο \textbf{πεδίο ορισμού τους}}
    \end{align*}
    
    Ορίζω το \( \infty \) του μιγαδικού επιπέδου να είναι το σύνολο
    σημείων που απέχουν "άπειρη" απόσταση από την αρχή των αξόνων.
    
    Το επεκτεταμένο μιγαδικό επίπεδο ορίζεται ως:
    \[
    \overline{\mathbb C} = \mathbb C \cup \left\lbrace \infty \right\rbrace,\text{ όπου:}
    \]
    \begin{align*}
    \infty+z &= \infty \quad \forall z \in \mathbb C \\
    \infty\cdot z &= \infty \quad \forall z \neq 0 \\
    \frac{z}{\infty} &= 0 \quad \forall z \neq \infty
    \end{align*}
    
    Όλες οι πράξεις του ορίου που ξέρετε ισχύουν και στους μιγαδικούς
    (αρκεί να μην εμφανίζονται οι γνωστές απροσδιόριστες μορφές):
    \[
    0\cdot\infty,\frac{\infty}{\infty},0^0,1^{\infty},\infty^0
    \]
    
    Ο κανόνας De l' Hospital ισχύει στους μιγαδικούς.
    
    \paragraph{Σημείωση:}
    \begin{gather*}
    \lim_{z\to \infty}f(z) = a \in \mathbb C \iff
    \lim_{z\to0}f\left(\frac{1}{z}\right) = a\in\mathbb C  \\
    \lim_{z\to z_0}f(z) = \infty \iff \lim_{z\to z_0}\frac{1}{f(z)} = -\\
    \lim_{z\to z_0}f(z) = 0 \iff \lim_{z\to z_0} \left|f(z)\right|=0
    \end{gather*}
    
    \begin{theorem*}{}
    	Έστω \( \arg z:\mathbb C - \left\lbrace 0 \right\rbrace
    	\to (-\pi,\pi]
    	 \)
    	 
    	Τότε η \( \arg z \) \textbf{είναι συνεχής} στο σύνολο:
    	\[
    	\mathbb C^* = \mathbb C -
    	\left\lbrace 
    	    x+iy: x \leq 0 \text{ ΚΑΙ } y = 0
    	 \right\rbrace
    	\]
    \end{theorem*}
    %TODO Atreas Graph 04
    
    Έστω \( z = x+iy \)
    %TODO Atreas Graph 05
    
    
    \[
    \arg z = \begin{cases}
    \arctan\left|\frac{y}{x}\right|, \qquad & x,y>0 \\
    \pi - \arctan\left|\frac{y}{x}\right|, \qquad & x<0,\ y>0 \\
    -\pi + \arctan\left|\frac{y}{x}\right|, \qquad & x<0,\ y<0 \\
    -\arctan\left|\frac{y}{x}\right|, \qquad & x>0,\ y<0
    \end{cases}
    \]
    
    Για \( 
    \begin{array}{ll}
    x=0,\ & \text{τότε } \arg := \frac{\pi}{2} \text{ ή } -\frac{\pi}{2}\\
    y=0,\ & \text{τότε } \arg := 0 \text{ή} \pi
    \end{array}
     \)
     
    Έστω \( z_0 = x_0 < 0 \)
    \begin{itemize}
    	\item Έστω \( z = x_0+it \quad (t>0) \)
    	
    	Για \( t\to0^+,\ z\to z_0=x_0 \), αλλά:
    	\[
    	\lim_{z\to z_0}\arg z \overset{z=x_0+it}{=}
    	\lim_{t\to0^+} \arg(x_0+it) \overset{\text{2ο τετ.}}{=}
    	\lim_{t\to0^+}\left(\pi-\arctan\left|\frac{t}{x_0}\right|\right)
    	=\pi-\arctan0=\pi
    	\]
    	\item Για \( z=x_0+it \quad (t<0) \), τότε:
    	\[
    	t\to0^-,\quad z\to z_0,\text{ και}
    	\]
    	\[
    	\lim_{z\to z_0}\arg z = \lim_{t\to0^-} \arg(x_0+it)
    	\overset{\text{3ο τετ.}}{=} -\pi+\arctan0 = -\pi
    	\]
    \end{itemize}
    Άρα το όριο στο \( z_0=x_0 \) ΔΕΝ υπάρχει, και έτσι η \( \arg z \) 
    ασυνεχής στα \( z=x_0 \) με \( x_0\leq 0 \).
    
    Αν \( \arg z \in [0,2\pi) \) πού είναι ασυνεχής;
    
	\newpage

	\part{Κεχαγιάς}
	Σπιτεργασίες λιγότερες από πέρσι, για 1 βαθμό, αφορούν μόνο το μέρος του Κεχ.
	\begin{enumerate}
		\item ΜΙΓΑΔΙΚΟΙ ΑΡΙΘΜΟΙ
		\item ΒΑΣΙΚΕΣ ΜΙΓΑΔΙΚΕΣ ΣΥΝΑΡΤΗΣΕΙΣ
		\item ΑΚΟΛΟΥΘΙΕΣ, ΣΕΙΡΕΣ
		\item ΔΥΝΑΜΟΣΕΙΡΕΣ
		\item ΑΡΜΟΝΙΚΕΣ ΣΥΝΑΡΤΗΣΕΙΣ
		\item ΔΙΑΦΟΡΙΚΕΣ ΕΞΙΣΩΣΕΙΣ με μερικές παραγώγους
	\end{enumerate}

	\setcounter{section}{0}

	\section{Μιγαδικοί αριθμοί}
	\begin{align*}
	    z = & x+iy \in \mathbb C \\
	    & x,y \in \mathbb R \qquad i^2=-1
	\end{align*}
	\begin{align*}
	z_1 &= x_1 +iy_1 \\
	z_2 &= x_2 +iy_2 \\
	z_1+z_2 &= (x_1+x_2)+i(y_1+y_2) \\
	z_1\cdot z_2 &= (x_1+iy_1)\cdot(x_2+iy_2) \\
	&= x_1x_2+iy_1y_2+ix_1y_2+ix_2y_1 \\
	&= (x_1x_2-y_1y_2)+i(x_1y_2+x_2y_1) \\
	\frac{z_1}{z_2} &= \frac{x_1+iy_1}{x_2+iy_2}
	= \frac{(x_1+iy_1)(x_2-iy_2)}{(x_2+iy^2)(x_2-iy_2)}
	\\ &= \frac{x_1x_2+y_1y_2}{x_2^2+y_2^2} + i \frac{-x_1y_2+x_2y_1}{x_2^2+y_2^2}
	\\ z &= x+iy
	\\ \bar{z} &= x-iy
	\\ \Re(z) &= x \in \mathbb R
	\\ \Im(z) &= y \in \mathbb R
	\end{align*}

	\begin{center}
		\begin{tikzpicture}[scale=2.5]
		\draw[->] (0,-0.5) -- (0,1.5) node[above right]{$y$};
		\draw[->] (-0.5,0) -- (1.7,0) node[below right]{$x$};

		\draw[gray,dashed]
			(0,1) node[above right,black] {$y$}
			-- (1,1) --
			(1,0) node[below,black] {$x$};
		\filldraw (1,1) circle(0.8pt) node[above right] {$z=x+iy$} ;

		\draw(0,0) -- (1,1) node[midway,above,sloped] {$r$};
		\draw[->] (.3,0) arc (0:45:.3) node[midway,right] {$\theta$};

		\draw (current bounding box.south) node {$r=\sqrt{x^2+y^2}$};
		\draw (current bounding box.south) node[below] {$\theta=\arctan\frac{y}{x}$};
		\end{tikzpicture}
	\end{center}
	\[
	r = \sqrt{x^2+y^2} = \sqrt{z\bar{z}} = |z| \leftarrow \text{μέτρο του } z
	\]
	γενίκευση της απόλυτης τιμής (δηλ. \( z=x \in \mathbb R,\ |z|=\sqrt{x^2}=|x| \))

	\begin{align*}
	z=x+iy &= r\cdot\cos\theta + ir\sin\theta \\
	&= r(\cos\theta+i\sin\theta) \\
	&= r\cdot e^{i\theta} \quad \text{(Euler)}
	\end{align*}
	\begin{align*}
	e^{i\theta} &= \cos\theta+i\sin\theta \text{ διότι}\\
	e^{i\theta} &= 1 + i\theta + \frac{(i\theta)^2}{2!} + \frac{(i\theta)^3}{3!}
	+ \frac{(i\theta)^4}{4!} + \dots \\
	&= \left(
	    1 - \frac{\theta^2}{2!} + \frac{\theta^4}{4!} - \dots
	\right) + i \left(
	    \theta-\frac{\theta^3}{3!} + \frac{\theta^5}{5!} - \dots
	\right)
	\\ &= \cos\theta+i\sin\theta
	\end{align*}

	Επίσης:
	\begin{align*}
	z &= x+iy \\
	&= \sqrt{x^2+y^2} \left( \frac{x}{\sqrt{x^2+y^2}}+i\frac{y}{\sqrt{x^2+y^2}} \right)
	\\ &= r \cdot (\cos\theta + i\sin\theta)
	\\ &= r \cos\theta + ir\sin\theta
	\end{align*}

	\begin{tikzpicture}[scale=2]
	\draw[->] (0,-1.5) -- (0,1.5);
	\draw[->] (-1.5,0) -- (1.5,0);

	\draw[gray,dashed] (0,1) -- (1,1) -- (1,0);
	\draw[gray,dashed] (0,-1) -- (-1,-1) -- (-1,0);

	\filldraw (1,1) circle(0.8pt) node[above right] {$z_1=1+i$} ;
	\filldraw (-1,-1) circle(0.8pt) node[above left] {$z_2=-1-i$} ;

	\draw[->,thick] (0,0) -- (1,1);
	\draw[->,thick] (0,0) -- (-1,-1);
	\draw[->] (.7,0) arc (0:45:.7);
	\draw[->] (.2,0) arc (0:225:.2);

	\end{tikzpicture}

	\begin{gather*}
		z_1=1+i = \sqrt{2}\cdot e^{i\sfrac{\pi}{4}} \\
		r_1 = \sqrt{1^2+1^2} = \sqrt{2} \\
		\theta_1 = \arctan\frac{1}{1} = \frac{\pi}{4}
	\end{gather*}
	\begin{gather*}
		z_2=-1-i=\sqrt{2}e^{i\sfrac{5\pi}{4}} = \sqrt{2}e^{i\cdot\left(
			-\sfrac{3\pi}{4} = \sqrt{2}e^{i\sfrac{13\pi}{4}}
			\right)} \\
		r_2=\sqrt{(-1)^2+(-1)^2} = \sqrt{2} \\
		\theta_2 = \arctan\frac{-1}{-1} = \frac{\pi}{4}
	\end{gather*}

	Γενικά: \( \mathlarger{-1-i=\sqrt{2}e^{i\left(
			\frac{5\pi}{4}+2k \pi
			\right)}},\quad k \in \mathbb Z \)

    \subsection{Συναρτήσεις}
    \[
    \mathlarger{\mathlarger{\mathbb C \to \mathbb R}}
    \]
    \begin{gather*}
    z=x+iy \\
    \mathrm{mod}(z)= \sqrt{x^2+y^2} \\
    \arg(z) = \begin{cases}
    \theta_0 \quad & \text{αν } z \in \text{1\textsuperscript{ο} τεταρτημόριο} \\
    \pi - \theta_0 \quad & \text{αν } z \in \text{2\textsuperscript{ο} τεταρτημόριο} \\
    \pi + \theta_0 \quad & \text{αν } z \in \text{3\textsuperscript{ο} τεταρτημόριο} \\
    2\pi - \theta_0 \quad & \text{αν } z \in \text{4\textsuperscript{ο} τεταρτημόριο}
    \end{cases} \qquad \theta_0 = \arctan\left(\left|\frac{y}{x}\right|\right) \\
    \forall z \in \mathbb C - \left\lbrace 0 \right\rbrace\ \arg(z) \in [0,2\pi)
    \end{gather*}

    Ορίζω και την πλειότιμη συνάρτηση \( \mathrm{arg}(z) = \left\lbrace
    \arg(z)+2k\pi,\ k \in \mathbb Z
     \right\rbrace \)

    \begin{align*}
    z = x+iy &= \mathrm{mod}(z) \cdot e^{i\arg(z)}
    \\ &= \mathrm{mod}(z) \cdot e^{i\left(\arg(z)+2k\pi\right)}
    \end{align*}
    \begin{align*}
    z_1 = \mathrm{mod}(z_1)e^{i\arg(z_1)}\\
    z_2 = \mathrm{mod}(z_2)e^{i\arg(z_2)}\\
    z_1z_2 = \mathrm{mod}(z_1)\mathrm{mod}(z_2)e^{i\cdot\left(
    	\arg(z_1)+\arg(z_2)
    	\right)} \\
    \arg(z_1z_2) \neq \arg(z_1)+\arg(z_2) \text{ επειδή} \\
    \arg\left(
        e^{i\frac{7\pi}{4}}e^{i\frac{7\pi}{4}}    \right) =
         \frac{7\pi}{4} + \frac{7\pi}{4} -2\pi
    \end{align*}

    Γενικά, αν \( A+B = \left\lbrace a+b: a \in A, b \in B \right\rbrace \), τότε:
    \[
    \mathrm{arg}(z_1z_2) = \mathrm{arg}(z_1) + \mathrm{arg}(z_2)
    \]
    
    Όμως:
    \begin{align*}
    \mathrm{arg}(z^z) &= \mathrm{arg}(z)+\mathrm{arg}(z) \\
    &\neq 2\mathrm{arg}(z)
    \end{align*}
    διότι:
    \begin{gather*}
    	A = \left\lbrace a_1,a_2,\dots \right\rbrace \\
    	B = \left\lbrace b_1,b_2,\dots \right\rbrace \\
    	A+B = \left\lbrace a+b: a\in A, b\in B \right\rbrace \\
    	A+A = \left\lbrace a_1+a_2:a_1,a_2\in A \right\rbrace \\
    	2A = \left\lbrace 2a:a\in A \right\rbrace \\[.3pt]
    	A = \left\lbrace 1,2,3 \right\rbrace \\
    	B = \left\lbrace 4,5 \right\rbrace \\
    	A+B = \left\lbrace a+b:a\in A, b\in B \right\rbrace
    	= \left\lbrace 1+4,1+5,2+4,2+5,3+4,3+5 \right\rbrace
    	= \left\lbrace 5,6,7,8 \right\rbrace\\
    	A+A = \left\lbrace 2,3,4,5,6 \right\rbrace \\
    	2A = \left\lbrace 2,4,6 \right\rbrace
    \end{gather*}
    
    
   	\subsection{n-οστές ρίζες}
   	\[
   	z = a^{\sfrac{1}{n}} \iff z^n=a
   	\]
   	Δηλ. ποιο \( z \) ικανοποιεί αυτή
   	\begin{align*}
   	a &= |a|e^{i\theta} \\
   	z &= re^{i\phi}
   	\end{align*}
   	\begin{align*}
   	& \left( re^{i\phi} \right)^n = |a|e^{i\theta} \\
   	\implies& r^n\cdot e ^{in\phi} = |a|e^{i\theta} \\
   	\implies& r^n\cdot(\cos n\phi+i\sin n\phi) = |a|\cdot(\cos\theta+i\sin\theta)\\
   	\implies& \begin{cases}
   	r^n=|a|\implies r =\sqrt[n]{|a|} \\
   	\left.\begin{array}{l}
   	\cos(n\phi)=\cos\theta \\
   	\sin(n\phi)=\sin\theta \implies
   	\end{array}\right| \implies
   	n\phi = \theta+2k\pi \ \in\mathbb Z \implies \phi=\frac{\theta+2k\pi}{n}
   	\end{cases} \\
   	\implies& z=a^{\sfrac{1}{n}} = \sqrt[n]{|a|}\cdot
   	e^{\sfrac{i(\theta+2k\pi)}{n}} \quad k\in\mathbb Z
   	\end{align*}
   	
   	(Όμως αρκεί να πάρω \( k\in\left\lbrace 0,1,\dots,N-1 \right\rbrace \))
   	
   	\[
   	a^{\sfrac{1}{n}} = \left\lbrace
   	\quad \sqrt[n]{|a|}e^{\sfrac{i\theta}{n}},\
   	\sqrt[n]{|a|}e^{\frac{i\theta+2\pi}{n}},\dots
   	\right.
   	\]
   	%TODO Kehagias Graph 03
   	
   	\paragraph{Παρ.} \( a^{\sfrac{1}{2}} = 1^{\sfrac{1}{2}} \)
   	\begin{gather*}
   	a=1=1\cdot e^{i0} \quad |a|=1,\theta=0\\
   	u_0 = \sqrt[2]{1}\cdot e^{i\left(\frac{0+2\cdot0\cdot\pi}{2}\right)}=e^{i0}=1 \\
   	u_1 = \sqrt[2]{1}\cdot e^{i\left(\frac{0+2\cdot\pi}{2}\right)}=e^{i\pi}=-1
   	\end{gather*}
   	%TODO Kehagias Graph 04
   	
   	\paragraph{Παρ.} \( a^{\sfrac{1}{3}}=1^{\sfrac{1}{3}}=z \)
   	\begin{gather*}
   	a=1=e^{i0},|a|=1,\theta=0 \\
   	u_0 = 1\\
   	u_1 =e^{\sfrac{i2\pi}{3}} = \frac{-1+i\sqrt{3}}{2} \\
   	u_2 =e^{\sfrac{i4\pi}{3}} = \frac{-1-i\sqrt{3}}{2}
   	\end{gather*}
   	%TODO Kehagias Graph 05
   	\subparagraph{Διαφορετικά}
   	\begin{gather*}
   	1^{\sfrac{1}{3}}=z \iff 1=z^3 \\
   	\iff z^3-1 = 0 \\
   	\iff (z-1)(z^2+z+1)=0 \\
   	\iff (z-1)\left(z+\frac{1-i\sqrt{3}}{2}\right)\left(
   	z+\frac{1+i\sqrt{3}}{2}\right) = 0
   	\end{gather*}
   	
   	\paragraph{Παρ.} \( 1^{\sfrac{1}{11}}=z \iff 1=z^{11} \)
   	\begin{gather*}
   	    \iff z^{11}-1 = 0 \\
   		\iff (z-1)(z^{11}+z^{10}+\cdots+z^1+1) = 0
   	\end{gather*}
   	\[
   	\left\lbrace u_0,u_1,\dots,u_{10} \right\rbrace
   	\]
    	
    	

    \section{Βασικές μιγαδικές συναρτήσεις}
    \( e^z,\ \log(z) \)
    \[
    e^z \overset{\text{ορισμός}}{=} e^xe^{iy}
    = e^x(\cos y+i\sin y)
    \]

    Ήξερα \( \begin{array}{l}
    e^x: \mathbb R \to \mathbb R  \\
    e^{iy}: \mathbb R \to \mathbb C
    \end{array} \).

    Τώρα η νέα συνάρτηση \( e^z: \mathbb C \to \mathbb C \) και \textbf{γενικεύει}
    τις δύο προηγούμενες συναρτήσεις.

    \paragraph{Παρ.}
    \begin{align*}
    e^{1+i}= ee^i &= e\cdot(\cos 1+i\sin 1) \\ &= e\cdot\cos1+i\cdot e\cdot \sin 1
    \\ \Re\left(e^{1+i}\right) &= e\cos1
    \\ \Im\left(e^{1+i}\right) &= e\sin1
    \end{align*}

    \paragraph{}
    \begin{align*}
    \log(e) &= 1 \\
    \log(-1) &= \log\left(e^{i(\pi+2k\pi)} \right) = i(\pi+2k\pi)
    \end{align*}
    Δηλ. η λογαριθμική συνάρτηση είναι \textbf{πλειότιμη}.
    \begin{align*}
    z &= |z|e^{i\theta} \\
    \log(z) &= \ln\left(|z|\right)+i\theta
    \end{align*}

    \paragraph{Ορίζω}
    \subparagraph{Πλειότιμη} \( \log(z) = \ln\left(|z|\right)+i\mathrm{arg}(z) \)
    \subparagraph{Μονότιμη} \( \mathrm{Log}(z) = \ln\left(|z|\right)+i\arg(z)\)
    είναι ο πρωτεύων κλάδος της πλειότιμης

    \begin{align*}
    \log(1+i) &= \log\left(
        \sqrt{2}e^{i\left(\sfrac{\pi}{4}+2k\pi \right)}
    \right) \\ &=
    \log\left(\sqrt{2}\right) + i\left( \frac{\pi}{4} +2k\pi \right)
    \end{align*}
    \begin{align*}
    \left\lbrace
    \frac{1}{2}\ln(2) + i\left( \frac{\pi}{4} +2 k \pi \right)
     \right\rbrace
    \end{align*}
    
    \subsection{}
    Από σήμερα: \( \arg(z) \in (-\pi,\pi] \)
    
    Πριν 7 ημέρες: \( \cancel{e^z=e^{x+iy}=e^x{\cos y+i\sin y}} \)
    	
    \paragraph{Σήμερα:} \( \exp(z) \overset{\text{ορ}}{=}
    1+z+\frac{z^2}{2!}+\dots=\sum_{n=0}^\infty \frac{z^n}{n!} \)
    
    \begin{theorem*}{}
    	Η \( \exp(z) \) είναι παραγωγίσιμη σε κάθε \( z\in\mathbb C  \)
    	και ικανοποιεί:
    	\begin{enumparen}
    		\item \( \forall z: \left(\exp(z)\right)' = \exp(z) \)
    		\item \( \forall z_1,z_2:\exp(z_1+z_2)=\exp(z_1)\exp(z_2) \)
    		\item \( \forall \theta\in\mathbb R
    		:\exp(i\theta)=\cos\theta+i\sin\theta \)
    	\end{enumparen}
    \end{theorem*}
    \subparagraph{Απόδ.}
    \begin{enumparen}{}
    	\item \begin{align*}
    	\left(\exp(z)\right)'
    	&= \left(1+z+\frac{z^2}{2!}+\frac{z^3}{3!}+\dots\right)'
    	\\ &= 0+1+z+\frac{z^2}{2!}+\dots = \exp(z)
    	\end{align*}
    	\item \( g(z)=\exp(z)\exp(\zeta-z) \)
    	\begin{align*}
    	\od{g}{z} &=
    	\exp(z)\exp(\zeta-z)+\exp(z)\exp(\zeta-z)(-1) = 0
    	\\ \implies g(z) &= c \implies c=g(0)=\exp(\zeta) \\
    	\implies \exp(\zeta)=g(z)=\exp(z)\exp(\zeta-z)
    	\end{align*}
    	
    	\textbf{Θέτω:} \( z=z_1,\ \zeta=z_1+z_2 \)
    	
    	\textbf{Οπότε:}
    	\[
    	\exp(z_1+z_2) = \exp(z_1)\cdot\exp(z_2)
    	\]
    	
    	\item \begin{align*}
    	\exp(i\theta) &= 1+i\theta + \frac{(i\theta)^2}{2!} + \frac{(i\theta)^3}{3!}
    	+\frac{(i\theta)^4}{4!} + \dots
    	\\ &= \left(1-\frac{\theta^2}{2!}+\frac{\theta^4}{4!}-\dots \right)
    	+ i\cdot\left(
    	\theta-\frac{\theta^3}{3!}+\dots
    	\right)
    	\\ &= \cos\theta+i\sin\theta
    	\end{align*}
    	
    	\paragraph{\( \exp(z) \quad e^z \)}
    	\begin{align*}
    	\exp(1+i)&=1+(1+i)+\frac{(1+i)^2}{2!}+\dots\\[.4pt]
    	e^{1+i} &= 1+(i+1)+\dots\\
    	\text{ή}\quad &\text{ο αρ. $e=2.718$ υψωμένος στη μιγαδική δύναμη } 1+i
    	\end{align*}
    	
    	\paragraph{}
    	\begin{theorem*}{}
    		Η \( \exp(z) \) είναι περιοδική με περίοδο \( 2\pi i \)
    	\end{theorem*}
    	\subparagraph{Απόδ.}
    	\[
    	\exp(z+2\pi i) = \exp(z)\exp(2\pi i) = \exp(z)
    	\]
    	
    	Η εικόνα του συνόλου \( A \subseteq \mathbb C  \) υπό την συνάρτηση \( f(z) \)
    	Δηλ.
    	\[ f(A)=\left\lbrace w=f(z),\ z\in A \right\rbrace \]
    	\paragraph{Παρ.} Να δειχθεί ότι \( \exp(\mathbb C ) = 
    	\mathbb C -\left\lbrace 0 \right\rbrace \)
    	
    	Διότι: έστω \( w=re^{i\phi}\in\mathbb C -\left\lbrace 0 \right\rbrace \).
    	
    	Θα βρω \( z=\rho e^{i\theta}=x+iy \) τ.ώ: \( \exp(z)=w \).
    	\begin{gather*}
    	\exp(z)=\exp(x+iy) = \exp(x)\exp(iy) \\
    	w=re^{i\phi} \\
    	\exp(x) = \left|\exp(z)\right|=|w|=r \implies \boxed{x = \ln(r)} \\[.5pt]
    	\arg\left( \exp(z) \right) = \arg(w) \\
    	\arg\left(\exp(z)\right) = \arg\left(\exp(x)\exp(iy)\right) = y \\
    	\arg(w)=\arg(re^{i\phi}) = \phi \\
        \arg\left(\exp(z)\right)=\arg(w) \implies \boxed{y=\phi}
    	\end{gather*}
    	
    	Τελικά \( z=x+iy = \ln(r)+i\phi \) ικανοποιεί \( \exp(z)=re^{i\phi}=w \).
    	Άρα \( \exp(\mathbb C)=\mathbb C-\left\lbrace 0 \right\rbrace \)
    	
    	Στην πραγματικότητα, δεν χρειάζομαι όλο το \( \mathbb C \) διότι:
    	\[
    	\exp(U) = \mathbb C-\left\lbrace 0 \right\rbrace,\quad
    	\text{όπου } U = \left\lbrace 
    	    x+iy:x\in\mathbb R,\ y\in(-\pi,\pi]
    	 \right\rbrace
    	\]
    	%TODO Kehagias Graph 07
    \end{enumparen}

\end{document}
\documentclass[11pt,a4paper,titlepage,fleqn]{article}

\usepackage{amsmath}
\usepackage{amsfonts}
\usepackage{amssymb}
\usepackage{commath}
\usepackage{xcolor}
\usepackage{hyperref}
\usepackage[skins,theorems]{tcolorbox}
\usepackage{titlesec}
\usepackage{circuitikz}
\usepackage{pgfplots}
\usepackage{mathtools}
\usepackage[makeroom]{cancel}
\usepackage{mathrsfs}
\usepackage{wrapfig}
\usepackage{subcaption}
\usepackage{floatrow}
\usepackage{esint}
\usepackage{enumitem}
\usepackage{bm}
\usepackage{relsize}
\usepackage{xfrac}
\usepackage{comment}


\usetikzlibrary{arrows.meta}
\usetikzlibrary{patterns}
\usetikzlibrary{decorations.pathmorphing,patterns}
\usetikzlibrary{decorations.markings}
\usetikzlibrary{backgrounds}
\usetikzlibrary{shapes.misc}
\usetikzlibrary{shapes.multipart}

\tikzset{cross/.style={cross out, draw,
        minimum size=2*(#1-\pgflinewidth),
        inner sep=0pt, outer sep=0pt}}
\tikzset{
    mark position/.style args={#1(#2)}{
        postaction={
            decorate,
            decoration={
                markings,
                mark=at position #1 with \coordinate (#2);
            }
        }
    }
}

\usepackage[left=2cm,right=2cm,top=2cm,bottom=2cm]{geometry}

\usepackage[no-math]{fontspec}
\setmainfont{Times New Roman}
\setsansfont{Arial}
%\newfontfamily\greekfont[Script=Greek]{Linux Libertine O}
%\newfontfamily\greekfontsf[Script=Greek]{Linux Libertine O}
\usepackage{polyglossia}
\newfontfamily\greekfont[Script=Greek]{Times New Roman}
\newfontfamily\greekfontsf[Script=Greek]{Arial}
\newfontfamily\greekfonttt[Script=Greek]{Latin Modern Mono}
%\usepackage[greek]{babel}
\setdefaultlanguage{greek}
\setotherlanguage{english}
\newcommand{\textlatin}[1]{#1}
%\newcommand{\mathlarger}{}

%\usepackage[utf8]{inputenc}
%\usepackage[greek]{babel}


\usetikzlibrary{arrows.meta}
\usetikzlibrary{calc}
%\usepackage{tkz-euclide} % loads  TikZ and tkz-base
%\usetkzobj{angles} % important you want to use angles

\newlist{enumparen}{enumerate}{1}
\setlist[enumparen]{label=(\arabic*)}
\newlist{enumpar}{enumerate}{1}
\setlist[enumpar]{label=\arabic*)}

\newlist{enumgreek}{enumerate}{1}
\setlist[enumgreek]{label=\alph*.}
\newlist{enumgreekparen}{enumerate}{1}
\setlist[enumgreekparen]{label=(\alph*)}
\newlist{enumgreekpar}{enumerate}{1}
\setlist[enumgreekpar]{label=\alph*)}


\newlist{enumroman}{enumerate}{1}
\setlist[enumroman]{label=(\roman*)}

\newlist{enumlatin}{enumerate}{1}
\setlist[enumlatin]{label=(\alph*)}

\newlist{invitemize}{itemize}{1}
\setlist[invitemize]{noitemsep,label=}



\makeatletter
\let\anw@true\anw@false

%\newcommand{\attnboxed}[1]{\textcolor{red}{\fbox{\normalcolor\m@th$\displaystyle#1$}}}
\makeatother
\tcbset{highlight math style={enhanced,colframe=red,colback=white,%
        arc=0pt,boxrule=1pt,shrink tight,boxsep=1.5mm,extrude by=0.5mm}}
\newcommand{\attnboxed}[1]{\tcbhighmath[colback=red!5!white,drop fuzzy shadow,arc=0mm]{#1}}
\titleformat{\section}{\bf\Large}{Κεφάλαιο \thesection}{1em}{}
\newtcolorbox{attnbox}[1]{colback=red!5!white,%
    colframe=red!75!black,fonttitle=\bfseries,title=#1}
\newtcolorbox{infobox}[1]{colback=blue!5!white,%
    colframe=blue!75!black,fonttitle=\bfseries,title=#1}

\renewcommand{\arg}{\mathrm{Arg}\, }
\renewcommand{\Re}{\mathrm{Re}}
\renewcommand{\Im}{\mathrm{Im}}
\newcommand{\sinc}{\;\mathrm{sinc}\!}

\newif\ifhidetikz
\hidetikzfalse
%\hidetikztrue   % <---- comment/uncomment that line

\ifhidetikz

\let\oldtikzpicture\tikzpicture
\let\oldendtikzpicture\endtikzpicture

\renewenvironment{tikzpicture}{
    \tiny
    \tt
    \color{blue}
    \newcommand{\draw}{\textit{draw}}
    \newcommand{\filldraw}{\textit{filldraw}}
    %\newcommand{\x}{\textit{x}}
    %\newcommand{\p}{\textit{x}}
    \newcommand{\x1}{\textit{x1}}
    \newcommand{\y1}{\textit{y1}}
    \newcommand{\p1}{\textit{p1}}
}{
}
\newenvironment{axis}{
    \newcommand{\addplot}{\textit{addplot}}
}{
}
\fi

\newtcbtheorem[number within=section]{theorem}{Θ.}%
{colback=green!5,colframe=green!35!black,colbacktitle=green!35!black,fonttitle=\bfseries,enhanced,attach boxed title to top left={yshift=-8mm,xshift=-7mm},width=.9\textwidth,arc=.7mm}{th}
\newtcbtheorem[number within=section]{defn}{Ορισμός}%
{colback=blue!5,colframe=cyan!35!black,colbacktitle=blue!35!black,fonttitle=\bfseries,enhanced,attach boxed title to top left={yshift=-2mm,xshift=-2mm}}{def}
\newtcbtheorem[number within=section]{exercise}{Άσκηση}%
{colback=gray!3,colframe=gray!35!black,colbacktitle=gray!35!black,fonttitle=\bfseries,enhanced,attach boxed title to top left={yshift=-2mm,xshift=-2mm}}{exc}




\setmainfont{Ubuntu Light}
\setsansfont{Arial}
%\newfontfamily\greekfont[Script=Greek]{Linux Libertine O}
%\newfontfamily\greekfontsf[Script=Greek]{Linux Libertine O}
\usepackage{polyglossia}
\newfontfamily\greekfont[Script=Greek,Scale=0.95]{Ubuntu Light}

\title{Εφαρμοσμένα Μαθηματικά - Σημειώσεις}
\date{2016}
\author{\textlatin{\csuse{no\greek @numbers}\selectlanguage{english} \url{https://github.com/kongr45gpen/ece-notes}}}

\begin{document}
	\url{http://users.auth.gr/natreas} \\
	Σημειώσεις: Εγώ Κεφ. 3-4-5 \\
	Κεχαγιάς Κεφ. 1-2-6

	Βιβλία:
	\begin{itemize}
		\item Churchill - Brown (για μηχανικούς)
		\item Marjden (πιο μαθηματικό)
	\end{itemize}

	\part{Ατρέας}
	\section{Μιγαδικοί Αριθμοί}
	\textbf{Έστω} \( \mathbb C = \left\lbrace z = \overbrace{(x,y)}^{\mathclap{\text{γεωμετρική παράσταση μιγαδικού}}};\ x,y\in\mathbb R  \right\rbrace \)

	Είναι σύνολο εφοδιασμένο με τις πράξεις:
	\begin{enumgreekparen}
		\item Πρόσθεση μιγαδικών

		Αν \( z_1=(x_1,y_1) \) και \( x_2=(x_2,y_2) \), τότε:\[
		z_1+z_2 = (x_1+x_2,\ y_1+y_2)
		\]

		\item Γινόμενο \( \lambda \in \mathbb R  \) με μιγαδικό \( z \)

		Αν \( z=(x,y) \), τότε ορίζω:
		\[
		\lambda z = (\lambda x,\lambda y)
		\]

		\item \attnboxed{\text{Πολλαπλασιασμό μιγαδικών αριθμών}}

		Αν \( z_1=(x_1,y_1),\ z_2=(x_2,y_2) \), τότε ορίζω:
		\[
		z_1z_2 = \left(x_1x_2-y_1y_2,\ x_1y_2+x_2y_1\right)
		\]
	\end{enumgreekparen}

	Καλείται σύνολο των μιγαδικών αριθμών.

	\begin{itemize}
		\item Δεν μπορώ να συγκρίνω μιγαδικούς
		\item Οι γνωστές ιδιότητες των πράξεων ισχύουν στους μιγαδικούς
	\end{itemize}

	Η γεωμετρική παράσταση του \( \mathbb C \) είναι το λεγόμενο μιγαδικό επίπεδο.

	\begin{center}
	\begin{tikzpicture}[scale=2.5]
		\draw[->] (0,-1.3) -- (0,1.5);
		\draw[->] (-1.5,0) -- (1.7,0);

		\draw[dashed] (0,1) -- (1,1) -- (1,-1);
		\filldraw (1,1) circle(0.8pt) node[above right] {$z=(x,y)$} ;
		\filldraw (0,1) circle(0.6pt) node[below right] {$(0,1)=i$};
		\filldraw (1,0) circle(0.6pt);

		\draw[->] (1.4,-0.8) -- (1.4,-0.2) node[midway,right] {πραγματικός άξονας \( \Re(z) \)};

		\draw[->] (-1,0.7) -- (-0.2,0.7) node[pos=.1,below] {φανταστικός άξονας \( \Im(z) \)};

		\draw[gray,->] (0,0) -- (1,1);
		\draw[->] (.3,0) arc (0:45:.3) node[midway,right] {$\theta$};

		\draw[gray,->] (0,0) -- (1,-1);
		\draw[->] (.3,0) arc (0:-45:.3);
		\filldraw (1,-1) circle(0.8pt) node[right] {$\bar z=(x,-y)$} ;
	\end{tikzpicture}
	\end{center}

	\[
	x \in \mathbb R \xleftrightarrow{\text{1-1}} A = \left\lbrace (x,0): x \in \mathbb R  \right\rbrace
	\]

	\begin{itemize}
		\item \(
		    (x,0),(y,0) \in A \implies (x,0)+(y,0)=(x+y,0) \in A
		\)
		\item \(
		    (x,0)(y,0) = (xy,0) \in A
		\)
	\end{itemize}

	Στο εξής γράφω: \begin{align*}
	    1 &= (1,0) \\
	    x &= (x,0)
	\end{align*}

\textbf{Ορίζω}:
	\[
	\mathlarger{\mathlarger{\mathlarger{i = (0,1)}}}
	\]
	και καλείται φανταστική μονάδα του μιγαδικού επιπέδου.

	\begin{gather*}
	i^2 = (0,1)(0,1) = (0\cdot0-1\cdot1,\ 0\cdot1+1\cdot0) = (-1,0) = -1 \\
	\boxed{i^2=-1}
	\end{gather*}

	\textbf{Έτσι}:
	\begin{gather*}
	    z=(x,y) = x(1,0) + y(0,1) \\
	    \overset{x=(x,0)}{\underset{i=(0,1)}{=}} x \cdot 1 + yi \\
	    \implies \boxed{z=x+iy}
	\end{gather*}

	\[
	\mathlarger{\mathlarger{\underbrace{z=x+iy}_{\mathclap{\text{άλγεβρα}}}
			\iff \underbrace{z=(x,y)}_{\mathclap{\text{γεωμετρία}}}
			}}
	\]

	\paragraph{}
	Έστω \( z=x+iy \)
	\begin{gather}
		\overset{\text{πολικές}}{\underset{\text{του } (x,y)}{=}}
		\rho\cos\theta+i\rho\sin\theta = \nonumber
		\\ = \mathlarger{\rho(\cos\theta+i\sin\theta)} \label{eq:1}
	\end{gather}

	Έτσι, η (\ref{eq:1}) γράφεται ως:
	\begin{align*}
	z &= |z| \underbrace{(\cos\theta+i\sin\theta)} \\
	  &= |z| \cdot \mathlarger{\mathlarger{e^{i\theta}}}
	\end{align*}
	όπου στο εξής:
	\begin{align*}
	\Aboxed{e^{i\theta} = \cos\theta+i\sin\theta} \\
	\Aboxed{\text{τύπος του Euler}}
	\end{align*}

	Τελικά: \[
	\boxed{\mathlarger{\mathlarger{\mathlarger{\mathlarger{z=|z|e^{i\theta}}}}}}
	\text{ (πολική μορφή μιγαδικών)}
	\]

	\subparagraph{Σημείωση:} \( \cos\theta + i\sin\theta \)
	\begin{gather*}
	\overset{\text{σειρές}}{\underset{\text{McLaurin}}{=}} \left(
	1-\frac{\theta^2}{2!} + \frac{\theta^4}{4!} + \dots
	\right) + i \left(\theta-\frac{\theta^3}{3!}+\frac{\theta^5}{5!}-\dots\right)
	\\
	\overset{i^2=-1}{=} \left(
	1+\frac{(i\theta)^2}{2!}+\frac{(i\theta)^4}{4!}+\dots
	\right) + \left(
	i\theta+\frac{(i\theta)^3}{3!}+\frac{(i\theta)^5}{5!}+\dots
	\right)
	\\ =
	1 + (i\theta) + \frac{(i\theta)^2}{2!} + \frac{(i\theta)^3}{3!}
	+ \dots + \frac{(i\theta)^n}{n!} + \dots = \mathlarger{e^{i\theta}}
	\end{gather*}

	\begin{itemize}
		\item Ορίζω {\large πρωτεύον όρισμα} \( \mathlarger{\mathlarger{\mathrm{Arg} z}} \) (μη μηδενικού) μιγαδικού \( z \) να είναι η γωνία \( \theta \)
		που σχηματίζει ο θετικός πραγματικός ημιάξονας του \( \mathbb C \) με την
		ημιευθεία \( OA \), όπου \( A \) το σημείο της γεωμετρικής παράστασης του
		\( z=x+iy \).
	\end{itemize}

	\subparagraph{Έτσι:}
	\[
	z = |z|e^{i\arg z} \quad \text{πολική μορφή του } z
	\]

	\begin{align*}
	z_1z_2 &= |z_1|e^{i\arg z_1}|z_2|e^{i\arg z_2} \\
	\Aboxed{z_1z_2 &= |z_1||z_2|e^{i(\arg z_1 + \arg z_2)}
	}
	\end{align*}
	\begin{align*}
	\frac{z_1}{z_2} &= \frac{|z_1|}{|z_2|} \frac{e^{i\theta_1}}{e^{i\theta_2}}
	\\ &= \left| \frac{z_1}{z_2} \right| e^{i(\theta_1-\theta_2)}
	\end{align*}

	\begin{tikzpicture}[scale=2.5]
	\draw[gray,->] (0,-0.7) -- (0,2);
	\draw[gray,->] (-1.5,0) -- (1.7,0);

	\filldraw (0,0) -- ++(35:1.2) circle(0.6pt) node[above right] {$z_1$};
	\draw[->] (.3,0) arc (0:35:.3) node[midway,right] {$\theta_1$};

	\filldraw (0,0) -- ++(75:1.7) circle(0.6pt) node[above right] {$z_2$};
	\draw[->] (.6,0) arc (0:75:.6) node[pos=.8,above right] {$\theta_2$};

	\filldraw (0,0) -- ++(110:{1.2*1.7}) circle(0.6pt) node[above right] {$z_1z_2$};
	\draw[->] (1,0) arc (0:110:1) node[pos=.6,above right] {$\theta_1+\theta_2$};
	\end{tikzpicture}

	\textbf{Ιδιότητα:} \( z\bar{z} = |z|^2 \)

	\section{Μιγαδικές συναρτήσεις}
	Κάθε συνάρτηση \( f: A \subseteq \mathbb C \to \mathbb C \) καλείται μιγαδική
	συνάρτηση μιγαδικής μεταβλητής.

	\[
	f = \underbrace{f(\underbrace{z}_{\text{η μεταβλητή μιγαδικός}})}_{\text{μιγαδική συνάρτηση διότι έχει τιμή μιγαδική}}
	\]

	\paragraph{π.χ.}
	\begin{gather*}
	f(z) = z^2 \implies
	f(x+iy) = (x+iy)^2 = x^2 + (iy)^2+2x\cdot \underbrace{x^2-y^2}_{\Re(f)}+i\underbrace{(2xy)}_{\Im(f)}
	\\
	\overset{\text{γεωμετρική}}{\underset{\text{μορφή}}{=}} (x^2-y^2,\ 2xy)
	\end{gather*}
	\subparagraph{Τελικά:} \(\boxed{f(x,y)=(x^2-y^2,\ 2xy)} \quad \mathbb R^2 \to \mathbb R^2 \)

	\paragraph{π.χ.}
	\begin{gather*}
	f(z) = \frac{1}{|z|\bar{z}} \overset{z=x+iy}{=}
	\frac{1}{\sqrt{x^2+y^2}}\cdot \frac{z}{\bar{z}z} \\
	\overset{z\bar{z}=|z|^2}{=} \frac{1}{\sqrt{x^2+y^2}} \cdot \frac{z}{|z|^2}
	= \frac{x+iy}{(x^2+y^2)^{\sfrac{3}{2}}}
	\\ \overset{\text{γεωμ}}{=}
	\frac{(x,y)}{(x^2+y^2)^{\sfrac{3}{2}}}
	\overset{\vec{r} = (x,y)}{=} \boxed{\frac{\vec{r}}{|\vec{r}|^3}}
	\end{gather*}

	Κεντρικό διαν. πεδίο που θυμίζει το πεδίο Coulomb.

	\[
	\underbrace{f=f(z)}_{\mathclap{\text{μιγαδική μιγ. μεταβλ.}}} \xleftrightarrow{\quad\text{1-1}\quad}
	\begin{array}{l}
	\text{διανυσμ. πεδίο του } \mathbb R^2 \\
	F(x,y) = \left( u(x,y),\ v(x,y) \right)
	\end{array}
	\]
	όπου \( u,v \) πραγματ. συναρτ. 2 μεταβλητών

	\paragraph{Υπάρχουν} \( f:A \subseteq \mathbb R \to \mathbb C \),
	μιγαδικές πραγματικής μεταβλητής

	π.χ \begin{align*}
	f(t) &= e^{it},\ t \in (0,\pi] \\
	&= \cos t + i \sin t
	\end{align*}
	\[
	t \to (\cos t, \sin t) \quad \text{καμπύλη } x^2+y^2=\cos^2 t +\sin^2 t = 1
	\]

	\begin{tikzpicture}[scale=1.5]
	\draw[->] (0,-1.5) -- (0,1.5);
	\draw[->] (-1.5,0) -- (1.5,0);

	\draw[thick,
		decoration={markings, mark=at position 0.125 with {\arrow{>}}},
		postaction={decorate}
	] (0,0) circle (1);

	\draw (0,1) node[above right] {$f(t)=e^{it}$};
	\end{tikzpicture}

	Η γραφ. παράσταση της \( f(t)=e^{it},\ t \in (-\pi,\pi) \) είναι ο μοναδιαίος κύκλος
	κέντρου \( (0,0) \) με αντιωρολογιακή φορά.

	\[
	g(t) = 1+it, t\in \mathbb R,\ =(1,t) = (1,0)+t(0,1)
	\]

	\newpage

	\part{Κεχαγιάς}
	Σπιτεργασίες λιγότερες από πέρσι, για 1 βαθμό, αφορούν μόνο το μέρος του Κεχ.
	\begin{enumerate}
		\item ΜΙΓΑΔΙΚΟΙ ΑΡΙΘΜΟΙ
		\item ΒΑΣΙΚΕΣ ΜΙΓΑΔΙΚΕΣ ΣΥΝΑΡΤΗΣΕΙΣ
		\item ΑΚΟΛΟΥΘΙΕΣ, ΣΕΙΡΕΣ
		\item ΔΥΝΑΜΟΣΕΙΡΕΣ
		\item ΑΡΜΟΝΙΚΕΣ ΣΥΝΑΡΤΗΣΕΙΣ
		\item ΔΙΑΦΟΡΙΚΕΣ ΕΞΙΣΩΣΕΙΣ με μερικές παραγώγους
	\end{enumerate}

	\setcounter{section}{0}

	\section{Μιγαδικοί αριθμοί}
	\begin{align*}
	    z = & x+iy \in \mathbb C \\
	    & x,y \in \mathbb R \qquad i^2=-1
	\end{align*}
	\begin{align*}
	z_1 &= x_1 +iy_1 \\
	z_2 &= x_2 +iy_2 \\
	z_1+z_2 &= (x_1+x_2)+i(y_1+y_2) \\
	z_1\cdot z_2 &= (x_1+iy_1)\cdot(x_2+iy_2) \\
	&= x_1x_2+iy_1y_2+ix_1y_2+ix_2y_1 \\
	&= (x_1x_2-y_1y_2)+i(x_1y_2+x_2y_1) \\
	\frac{z_1}{z_2} &= \frac{x_1+iy_1}{x_2+iy_2}
	= \frac{(x_1+iy_1)(x_2-iy_2)}{(x_2+iy^2)(x_2-iy_2)}
	\\ &= \frac{x_1x_2+y_1y_2}{x_2^2+y_2^2} + i \frac{-x_1y_2+x_2y_1}{x_2^2+y_2^2}
	\\ z &= x+iy
	\\ \bar{z} &= x-iy
	\\ \Re(z) &= x \in \mathbb R
	\\ \Im(z) &= y \in \mathbb R
	\end{align*}

	\begin{center}
		\begin{tikzpicture}[scale=2.5]
		\draw[->] (0,-0.5) -- (0,1.5) node[above right]{$y$};
		\draw[->] (-0.5,0) -- (1.7,0) node[below right]{$x$};

		\draw[gray,dashed]
			(0,1) node[above right,black] {$y$}
			-- (1,1) --
			(1,0) node[below,black] {$x$};
		\filldraw (1,1) circle(0.8pt) node[above right] {$z=x+iy$} ;

		\draw(0,0) -- (1,1) node[midway,above,sloped] {$r$};
		\draw[->] (.3,0) arc (0:45:.3) node[midway,right] {$\theta$};

		\draw (current bounding box.south) node {$r=\sqrt{x^2+y^2}$};
		\draw (current bounding box.south) node[below] {$\theta=\arctan\frac{y}{x}$};
		\end{tikzpicture}
	\end{center}
	\[
	r = \sqrt{x^2+y^2} = \sqrt{z\bar{z}} = |z| \leftarrow \text{μέτρο του } z
	\]
	γενίκευση της απόλυτης τιμής (δηλ. \( z=x \in \mathbb R,\ |z|=\sqrt{x^2}=|x| \))

	\begin{align*}
	z=x+iy &= r\cdot\cos\theta + ir\sin\theta \\
	&= r(\cos\theta+i\sin\theta) \\
	&= r\cdot e^{i\theta} \quad \text{(Euler)}
	\end{align*}
	\begin{align*}
	e^{i\theta} &= \cos\theta+i\sin\theta \text{ διότι}\\
	e^{i\theta} &= 1 + i\theta + \frac{(i\theta)^2}{2!} + \frac{(i\theta)^3}{3!}
	+ \frac{(i\theta)^4}{4!} + \dots \\
	&= \left(
	    1 - \frac{\theta^2}{2!} + \frac{\theta^4}{4!} - \dots
	\right) + i \left(
	    \theta-\frac{\theta^3}{3!} + \frac{\theta^5}{5!} - \dots
	\right)
	\\ &= \cos\theta+i\sin\theta
	\end{align*}

	Επίσης:
	\begin{align*}
	z &= x+iy \\
	&= \sqrt{x^2+y^2} \left( \frac{x}{\sqrt{x^2+y^2}}+i\frac{y}{\sqrt{x^2+y^2}} \right)
	\\ &= r \cdot (\cos\theta + i\sin\theta)
	\\ &= r \cos\theta + ir\sin\theta
	\end{align*}

	\begin{tikzpicture}[scale=2]
	\draw[->] (0,-1.5) -- (0,1.5);
	\draw[->] (-1.5,0) -- (1.5,0);

	\draw[gray,dashed] (0,1) -- (1,1) -- (1,0);
	\draw[gray,dashed] (0,-1) -- (-1,-1) -- (-1,0);

	\filldraw (1,1) circle(0.8pt) node[above right] {$z_1=1+i$} ;
	\filldraw (-1,-1) circle(0.8pt) node[above left] {$z_2=-1-i$} ;

	\draw[->,thick] (0,0) -- (1,1);
	\draw[->,thick] (0,0) -- (-1,-1);
	\draw[->] (.7,0) arc (0:45:.7);
	\draw[->] (.2,0) arc (0:225:.2);

	\end{tikzpicture}

	\begin{gather*}
		z_1=1+i = \sqrt{2}\cdot e^{i\sfrac{\pi}{4}} \\
		r_1 = \sqrt{1^2+1^2} = \sqrt{2} \\
		\theta_1 = \arctan\frac{1}{1} = \frac{\pi}{4}
	\end{gather*}
	\begin{gather*}
		z_2=-1-i=\sqrt{2}e^{i\sfrac{5\pi}{4}} = \sqrt{2}e^{i\cdot\left(
			-\sfrac{3\pi}{4} = \sqrt{2}e^{i\sfrac{13\pi}{4}}
			\right)} \\
		r_2=\sqrt{(-1)^2+(-1)^2} = \sqrt{2} \\
		\theta_2 = \arctan\frac{-1}{-1} = \frac{\pi}{4}
	\end{gather*}

	Γενικά: \( \mathlarger{-1-i=\sqrt{2}e^{i\left(
			\frac{5\pi}{4}+2k \pi
			\right)}},\quad k \in \mathbb Z \)

    \subsection{Συναρτήσεις}
    \[
    \mathlarger{\mathlarger{\mathbb C \to \mathbb R}}
    \]
    \begin{gather*}
    z=x+iy \\
    \mathrm{mod}(z)= \sqrt{x^2+y^2} \\
    \arg(z) = \begin{cases}
    \theta_0 \quad & \text{αν } z \in \text{1\textsuperscript{ο} τεταρτημόριο} \\
    \pi - \theta_0 \quad & \text{αν } z \in \text{2\textsuperscript{ο} τεταρτημόριο} \\
    \pi + \theta_0 \quad & \text{αν } z \in \text{3\textsuperscript{ο} τεταρτημόριο} \\
    2\pi - \theta_0 \quad & \text{αν } z \in \text{4\textsuperscript{ο} τεταρτημόριο}
    \end{cases} \qquad \theta_0 = \arctan\left(\left|\frac{y}{x}\right|\right) \\
    \forall z \in \mathbb C - \left\lbrace 0 \right\rbrace\ \arg(z) \in [0,2\pi)
    \end{gather*}

    Ορίζω και την πλειότιμη συνάρτηση \( \mathrm{arg}(z) = \left\lbrace
    \arg(z)+2k\pi,\ k \in \mathbb Z
     \right\rbrace \)

    \begin{align*}
    z = x+iy &= \mathrm{mod}(z) \cdot e^{i\arg(z)}
    \\ &= \mathrm{mod}(z) \cdot e^{i\left(\arg(z)+2k\pi\right)}
    \end{align*}
    \begin{align*}
    z_1 = \mathrm{mod}(z_1)e^{i\arg(z_1)}\\
    z_2 = \mathrm{mod}(z_2)e^{i\arg(z_2)}\\
    z_1z_2 = \mathrm{mod}(z_1)\mathrm{mod}(z_2)e^{i\cdot\left(
    	\arg(z_1)+\arg(z_2)
    	\right)} \\
    \arg(z_1z_2) \neq \arg(z_1)+\arg(z_2) \text{ επειδή} \\
    \arg\left(
        e^{i\frac{7\pi}{4}}e^{i\frac{7\pi}{4}}    \right) =
         \frac{7\pi}{4} + \frac{7\pi}{4} -2\pi
    \end{align*}

    Γενικά, αν \( A+B = \left\lbrace a+b: a \in A, b \in B \right\rbrace \), τότε:
    \[
    \mathrm{arg}(z_1z_2) = \mathrm{arg}(z_1) + \mathrm{arg}(z_2)
    \]

    \section{Βασικές μιγαδικές συναρτήσεις}
    \( e^z,\ \log(z) \)
    \[
    e^z \overset{\text{ορισμός}}{=} e^xe^{iy}
    = e^x(\cos y+i\sin y)
    \]

    Ήξερα \( \begin{array}{l}
    e^x: \mathbb R \to \mathbb R  \\
    e^{iy}: \mathbb R \to \mathbb C
    \end{array} \).

    Τώρα η νέα συνάρτηση \( e^z: \mathbb C \to \mathbb C \) και \textbf{γενικεύει}
    τις δύο προηγούμενες συναρτήσεις.

    \paragraph{Παρ.}
    \begin{align*}
    e^{1+i}= ee^i &= e\cdot(\cos 1+i\sin 1) \\ &= e\cdot\cos1+i\cdot e\cdot \sin 1
    \\ \Re\left(e^{1+i}\right) &= e\cos1
    \\ \Im\left(e^{1+i}\right) &= e\sin1
    \end{align*}

    \paragraph{}
    \begin{align*}
    \log(e) &= 1 \\
    \log(-1) &= \log\left(e^{i(\pi+2k\pi)} \right) = i(\pi+2k\pi)
    \end{align*}
    Δηλ. η λογαριθμική συνάρτηση είναι \textbf{πλειότιμη}.
    \begin{align*}
    z &= |z|e^{i\theta} \\
    \log(z) &= \ln\left(|z|\right)+i\theta
    \end{align*}

    \paragraph{Ορίζω}
    \subparagraph{Πλειότιμη} \( \log(z) = \ln\left(|z|\right)+i\mathrm{arg}(z) \)
    \subparagraph{Μονότιμη} \( \mathrm{Log}(z) = \ln\left(|z|\right)+i\arg(z)\)
    είναι ο πρωτεύων κλάδος της πλειότιμης

    \begin{align*}
    \log(1+i) &= \log\left(
        \sqrt{2}e^{i\left(\sfrac{\pi}{4}+2k\pi \right)}
    \right) \\ &=
    \log\left(\sqrt{2}\right) + i\left( \frac{\pi}{4} +2k\pi \right)
    \end{align*}
    \begin{align*}
    \left\lbrace
    \frac{1}{2}\ln(2) + i\left( \frac{\pi}{4} +2 k \pi \right)
     \right\rbrace
    \end{align*}

\end{document}